We contribute \TT, a core type theory that incorporates the core elements
of \underline{f}amily polymorphism into
\underline{M}artin-\underline{L}öf dependent \underline{t}ype \underline{t}heory~\cite{martin1982constructive},
while maintaining consistency and canonicity.

Most notably, \TT extends MLTT with what we call \emph{linkages}
(a namesake of the theoretical device through which \citet{zm2017} model
family polymorphism in an OO setting).
Linkages model families, so they are like tuples but with a twist: they
support late binding.

\TT is intended as a foundational model. So unlike our programmer-facing plugin,
typing is structural rather than nominal, and
it supports family polymorphism by providing MLTT-style typing rules that
can be used to \emph{encode} core aspects of the language supported by our
plugin, rather than building them in directly.
%
For example, \TT supports extending inductive types through
formation rules that enable the encoding of the further binding of W-types~\cite{martin1984intuitionistic} in linkages.
Also unlike our plugin, \TT does not feature automations such as
the generation of propositional equalities and partial recursors,
but it is expressive enough to encode their definitions.

\begin{figure}
\small

\renewcommand*{\arraystretch}{1.25}

\definecolor{new}{HTML}{F5EFE6}

\begin{gather*}
\begin{array}{@{}rccl@{}}
\text{Contexts} & \Gm, \Dl &\Coloneqq &
    \cdot \mid \Gm, A
    \\ 
\text{Substitutions} & \gm &\Coloneqq &
    \SubstWeak{n} \mid \SubstExt{\gm}{t} \mid \SubstComp{\gm_1}{\gm_2}
    \\
\text{Types} & A, B, T  &\Coloneqq &
    \cU \mid \cB \mid \bot \mid \top \mid \TyPi{A}{B} \mid \TySigma{A}{B} \mid \TyId{s}{t} \mid \TyS{t} \mid \El{t} \mid \sub{T}{\gm} \mid
    \\
    &&&
\tikzmarkin[disable rounded corners=true,set fill color=new,set border color=new]{New2}(0.05,-0.15)(-0.10,0.30)
    \TyWSingle{\wsig} \mid \wsigproj{j}{1}{\wsig} \mid \wsigproj{j}{2}{\wsig} \mid
    \TyLkg{\lsig} \mid \TyTkg{\lsig} \mid \lsigprojT{\lsig} \mid \lsigproj{2}{\lsig} \mid \CaseSig{A}{B}{T}
\tikzmarkend{New2}
    \\
\text{Terms} & t, s, \lkg &\Coloneqq &
    \var{n} \mid \sub{t}{\gm} \mid \codety{T} \mid \unit \mid \true \mid \false \mid \ifb{t_1}{t_2}{t_3} \mid \lam{t} \mid \app{t} \mid 
    \\ &&& \pair{t_1}{t_2} \mid \fst{t} \mid \snd{t} \mid \eqrefl{t} \mid  \Jeq{s}{t} \mid
    \\
    &&&
\tikzmarkin[disable rounded corners=true,set fill color=new,set border color=new]{New3}(0.05,-0.15)(-0.10,0.30)
    \wsingle{\wsig} \mid \wcode{\wsig} \mid \wsup{\wsig}{t_1}{t_2} \mid \LkgEmp \mid \LkgAdd{\lkg}{t} \mid \inh{h}{\lkg} \mid
    \\
    &&&
    \lkgproj{1}{\lkg} \mid \lkgproj{2}{\lkg} \mid \lsigproj{s}{\lsig} \mid \Tkg{\lkg} \mid
    \Recproj{j}{\lkg} \mid \Wrec{\lkg}{t}
\tikzmarkend{New3}
    \\
\tikzmarkin[disable rounded corners=true,set fill color=new,set border color=new]{New1}(0.05,-0.15)(-0.10,0.30)
\text{W-type signatures} & \wsig & \Coloneqq &
    \WSigEmp \mid \WSigAdd{\wsig}{A}{B} \mid \sub{\wsig}{\gm} \mid \WSigSub{\wsig}
\tikzmarkend{New1}
    \\
\tikzmarkin[disable rounded corners=true,set fill color=new,set border color=new]{New5}(0.05,-0.15)(-0.10,0.30)
\text{Linkage signatures} & \lsig & \Coloneqq &
    \LSigEmp \mid \LSigAdd{\lsig}{\seal}{T} \mid \lsigproj{1}{\lsig} \mid \RecSig{\wsig}{T} \mid \sub{\lsig}{\gm}
\tikzmarkend{New5}
    \\
\tikzmarkin[disable rounded corners=true,set fill color=new,set border color=new]{New4}(0.05,-0.15)(-0.10,0.30)
\text{Linkage transformers} & h & \Coloneqq &
    \InhId \mid \InhExt{h}{t} \mid \InhOv{h}{t} \mid \InhInh{h} \mid \InhNest{h}{h'}
\tikzmarkend{New4}
\end{array}
\end{gather*}

\begin{mathpar}
\judgebox{\goodCtx{\Gm}{i}}

\judgebox{\goodSub{\Gm}{\gm}{\Dl}}

\judgebox{\goodType{\Gm}{T}{}}

\judgebox{\goodTerm{\Gm}{t}{T}}

\judgebox{\goodWSig{\Gm}{\wsig}{n}}

\judgebox{\goodSig{\Gm}{\lsig}{n} }

\judgebox{\goodInh{\Gm}{h}{\lsig_1}{\lsig_2}}
\\

\Rule[name=sub/id]{
    \goodCtx{\Gm}{i}
}{
    \goodSub{\Gm}{\SubstWeak{0}}{\Gm}
}

\Rule[name=sub/weaken]{
    \goodSub{\Gm}{\SubstWeak{n}}{\Dl}
    \\
    \goodType{\Gm}{A}{}
}{
    \goodSub{\Gm, A}{\SubstWeak{n+1}}{\Dl}
}

\Rule[name=sub/ext]{
    \goodSub{\Gm}{\gm}{\Dl}
    \\
    \goodTerm{\Gm}{t}{\sub{A}{\gm}}
    \\
    \goodType{\Dl}{A}{}
}{
    \goodSub{\Gm}{\SubstExt{\gm}{t}}{\Dl,A}
}

\Rule[name=ty/sub]{
    \goodType{\Dl}{T}{j}
    \\
    \goodSub{\Gm}{\gm}{\Dl}
}{
    \goodType{\Gm}{T[\gm]}{j}
}

\Rule[name=tm/sub]{
    \goodTerm{\Dl}{t}{T}
    \\
    \goodSub{\Gm}{\gm}{\Dl}
}{
    \goodTerm{\Gm}{\sub{t}{\gm}}{\sub{T}{\gm}}
}

\Rule[name=tm/var]{
    \goodCtx{\Gm, A_n, ..., A_1, A_0}{}
}{
    \goodSub{\Gm, A_n, ..., A_1, A_0}{\var{n}}{\sub{A_n}{\SubstWeak{n+1}}}
}

\Rule[name=tm/lam]{
    \goodType{\Gm}{A}{}
    \\
    \goodTerm{\Gm, A}{t}{B}
}{
    \goodTerm{\Gm}{\lam{t}}{\TyPi {A} {B}}
}

\Rule[name=tm/app]{
    \goodTerm{\Gm}{t}{\TyPi {A} {B}}
}{
    \goodTerm{\Gm, A}{\app{t}}{B}
}

\Rule[name=tm/pair]{
    \goodTerm{\Gm}{t_1}{A} 
    \\
    \goodTerm{\Gm}{t_2}{\sub{B}{\SubstExt{\SubstWeak{0}}{t_1}}}
}{
    \goodTerm{\Gm}{(t_1,t_2)}{\TySigma{A}{B}}
}

\Rule[name=tm/proj]{
    \goodTerm{\Gm}{t}{\TySigma A B}
}{
    \goodTerm{\Gm}{\fst{t}}{A}
    \\
    \goodTerm{\Gm}{\snd{t}}{\sub{B}{\SubstExt{\SubstWeak{0}}{\fst{t}}}}
}

\Rule[name=ty/el]{
    \goodTerm{\Gm}{t}{\cU}
}{
    \goodType{\Gm}{\El{t}}{j}
}

\Rule[name=tm/c]{
    \goodType{\Gm}{T}{j}
}{
    \goodTerm{\Gm}{\codety{T}}{\cU}
}

\Rule[name=tm/eq/c]{
    \goodTerm{\Gm}{t}{\cU}
}{
    \goodTerm{\Gm}{\codety{\El t} \equiv t}{\cU}
}

\Rule[name=ty/eq/el]{
    \goodType{\Gm}{T}{j}
}{
    \goodType{\Gm}{\El{\codety T} \equiv T}{}
}

\Rule[name=tm/w]{
    \goodWSig{\Gm}{\wsig}{n}
}{
    \goodTerm{\Gm}{\wsingle{\wsig}}{\TyWSingle{\wsig}}
    \\
    \goodTerm{\Gm}{\wcode{\wsig}}{\cU}
}

\Rule[name=wsig/empty]{
}{
    \goodWSig{\Gm}{\WSigEmp}{0}
}

\Rule[name=tm/wsup]{
    \goodWSig{\Gm}{\wsig}{n}
    \\
    \goodTerm{\Gm}{t_1}{\wsigproj{j}{1}{\wsig}}
    \\
    \goodTerm{\Gm, \sub{\wsigproj{j}{2}{\wsig}}{\SubstExt{\SubstWeak{0}}{t_1}}}{t_2}{\El{\wcode{\wsig}}}
    \\
    j \in \{1, ..., n\}
}{
    \goodTerm{\Gm}{\wsup{\wsig}{t_1}{t_2}}{\El{\wcode{\wsig}}}
}

\Rule[name=wsig/add]{
    \goodWSig{\Gm}{\wsig}{n}
    \\
    \goodType{\Gm}{A}{i}
    \\
    \goodType{\Gm, A}{B}{i}
}{
    \goodWSig{\Gm}{\WSigAdd{\wsig}{A}{B}}{n+1}
}

\Rule[name=tm/wrec]{
    \goodTerm{\Gm}{\lkg}{\TyLkg{\RecSig{\wsig}{T}}}
    \\
    \goodTerm{\Gm}{t}{\El{\wcode{\wsig}}}
}{
    \goodTerm{\Gm}{\Wrec{\lkg}{t}}{T}
}

\Rule[name=ty/casety]{
    \goodType{\Gm}{A}{}
    \\
    \goodType{\Gm, A}{B}{}
    \\
    \goodType{\Gm}{T}{}
}{
    \goodType{\Gm}{
        \CaseSig{A}{B}{R}
        \equiv
        \TyPi{A}{\TyPi {\TyPi{B}{\sub{T}{\SubstWeak{2}}}} {\sub{T}{\SubstWeak{2}}}}
    }{}
}

\Rule[name=lsig/add]{
    \goodSig{\Gm}{\lsig}{n} 
    \\
    \goodTerm{\Gm,\TyTkg{\lsig}}{\seal}{\sub{A}{\SubstWeak{1}}}
    \\
    \goodType{\Gm, A}{T}{}
}{
    \goodSig{\Gm}{\LSigAdd{\lsig}{\seal}{T}}{n+1}
}

\end{mathpar}

\caption{Syntax and selected typing rules of \TT}
\label{fig:typing-selected}
\end{figure}

\noindentparagraph{MLTT.}

\cref{fig:typing-selected} presents the syntax of \TT.
The new forms (relative to MLTT) are highlighted.

As is standard in MLTT presentations, we use explicit substitutions~\cite{substcalculus,abadi1989subst}:
substitutions~$\gm$ and their applications (e.g., \sub{T}{\gm}) are part
of the syntax rather than meta-operations.
%
Variables are represented by de Bruijn indices:
\var{n} is the variable bound by the $n$-th closest enclosing binder.
For example, $\lambda x.\,\lambda y.\,x$ is $\lam{(\lam{\var{1}})}$.
%
Formation rules for substitutions have\EDJ{It seems to me G |- g : D is a judgement not a rule. Rules usually have two parts (premises + conclusion). Judgment is a standard terminology invented by Martin Lof (https://ncatlab.org/nlab/show/judgment see reference here). I will suggest saying 'Formation rules for substitution lead to a judgement G |- g : D' sth like this. In pen-and-paper, the formation rule (corresponding to QIIT well-typed-ness of judgement) is called presuppose (can be found in Modal MLTT paper) } the form 
$\goodSub{\Gm}{\gm}{\Dl}$, which says that applying~$\gm$ to terms
valid in the context~$\Dl$ yields terms valid in~$\Gm$. 
%
The two main forms of substitutions are weakening and extension:
\sub{t}{\SubstWeak{n}} introduces $n$ free variables into the context of $t$, and
\sub{t}{\SubstExt{\gm}{t'}} substitutes~$t'$ for~$\var{0}$ in~$t$ and
then applies~$\gm$.
For example, rule \ruleref{tm/proj} states that if $t$ is a dependent pair
of type $\TySigma{A}{B}$, then $\snd{t}$ has type $\sub{B}{\SubstExt{\SubstWeak{0}}{\fst{t}}}$.

%Note that we are actually using predicative universe, so we actually have different levels of universes $\cU_i$. 
%We omit universe levels in our latex presentation for conciseness and readability.


\newpage

To formalize our metatheory, we follow the formulation of~\citet{altkap2016}, which provides a declarative and intrinsic style of the predicative \textit{Martin-Lof Type Theory} (MLTT)~\cite{martin1982constructive},
using \textbf{Quotient Inductive Inductive Type}(QIIT)\footnote{Inductive Inductive Type(IIT) is a generalization of mutual inductive type where (indexed-)types \mintinline{Coq}{A : Type, B : A -> Type} can be defined mutually. QIIT further enhances IIT by allowing equality constructor (i.e. mathematical quotient).}, \textbf{explicit substitution} and \textbf{debruijn indices} as the main tool. 
Generally speaking, we \textbf{use each QIIT type to represent each kind of judgements, and thus a term will represent a derivation of a judgement}. The classical formulation of dependent type theory requires a lot of \textit{presupposition} and quotient afterwards, causing a lot of duplicacy. Both issues can be concisely solved by using QIIT expressing intrinsically typed syntax instead. What's more, working \textbf{in a type-theoretical setting} using QIIT makes checking of the pen-and-paper proof easier. However, we will follow conventional notation and thus the reader might not notice the usage of QIIT explicitly.\EDJ{We might be able to remove QIIT in the main text totally. Let's try to do it later.}

$$
\Rule[name=Subst]
{\goodType{\Delta}{T}{j} 
  \quad {\goodSub{\Gamma}{\gamma}{\Delta}}}
{\goodType{\Gamma}{T[\gamma]}{j}},
\Rule[]
{\goodTerm{\Delta}{t}{T}
  \quad {\goodSub{\Gamma}{\gamma}{\Delta}}}
{\goodTerm{\Gamma}{t[\gamma]}{T[\gamma]}}
\quad
\Rule[name=Dbj Indices]
{}
{\goodSub{\Gamma, A}{p^1 \gamma}{\Gamma}},
\Rule[]
{}
{\goodSub{\Gamma, A}{"var"_0}{A[p^1]}}
$$

% Explicit substitution
Instead of meta-level substitution, we use \textbf{explicit substitution}, i.e., substitution itself is a expression in the program and will be done in the runtime. Thus substitution has its own judgement $\goodSub{\Gamma}{\gamma}{\Delta}$. We can apply substitution to types and terms\footnote{The type judgement is the form $\goodType{\Gamma}{T}{}$. The term judgement is the form $\goodTerm{\Gamma}{t}{T}$} in MLTT via \ruleref{Subst}.  Note how substitution will change the context---we suggest intuitvely understanding context as holes of the term/types and substitution as filling those holes, where the substitution itself has holes. After filling the holes, then the new context is about the holes of the substitution. We have an identity substitution $\goodSub{\Gamma}{"id"}{\Gamma}$.
% Debruijn Indices
Contrary to named variables, \textbf{debruijn indices} refer to the variable in the context like operating a stack. We will use "var"$_i$ indicating the $i$-th variable (from right to left, $i$ starts with 0) in the context.  On the other hand, we can also get the rest part of the stack, or ``subpart of "id"'' using $\goodSub{\Gamma, A}{p^1 \gamma}{\Gamma}$.  For example, we can get the very first variable in the context $\Gamma, x : T \vdash x : T$ by $\goodTerm{\Gamma, T}{var_0}{T[p^1]}$. Note here $T$ is a type in $\Gamma$, but now shifted into $\goodType{\Gamma, T}{T[p^1]}{}$ using $p^1$. Similarly, to get the second and prior variables, we use "var"$_i$ and $p^{i+1}$ correspondingly. 


Now, we will mainly focus on the newly introduced typing rules, and illustrate their connection to the plugin implementation. For those curious about the complete meta-theory formulation, we refer to our appendices or \citet{altkap2016, kaposi2017type} or maybe those not using QIIT \cite{coquand2018canonicity, sterling2019algebraic}. Our metatheory is formulated using QIIT in an extensional setting.
%\YZ{
%    Is extensionality in conflict with the claim in §3 that "subst tm_unit x t" and "tm_unit" are
%    not definitionally equal but propositionally equal in family STLC?
%}\EDJreply{No, TLDR: it is a different level. \\ 
%There are two type theories, one is Agda-like (= Latex = Conventional math), the other is the one formulated in QIIT. This sentence is saying that the Agda-like one is extensional MLTT (i.e. we have function extensionality and etc here in Latex writing directly). \\ But the one you are asking about is in the metatheory in QIIT: if we have an identity type $Id x y$ (propositionally equal) can we have a judgemetnal equality for the two terms in QIIT $x \equiv y$. \\
%Apparently, we don't setup our QIIT formulation in this way (I thought we need, but now I don't think we need it. Our QIIT formulation is purely in intensional MLTT now).\\ 
%But even if we do setup like that, I will not say a conflict is happening (because it is to our advantage if we have judgemental equal in metatheory). I will say it is a bit off as to how Coq is formulated (Coq is closer to an intensional MLTT, so no judgemental equal only propositional one). 
%}

% linkage, signature and compilation and seal
$$
\Rule[name=lsig/ept]
{""}
{\goodSig{\Gm}{\LSigEmp}{0}}
\quad
\Rule[name=ty/lkg]
{\goodSig{\Gm}{\lsig}{n}}
{\goodType{\Gm}{\TyLkg{\lsig}}{i}
\\ \goodType{\Gm}{\TyTkg{\lsig}}{}}
\quad
\Rule[name=Lnkg Compile]
{ \goodTerm{\Gamma}{o}{\cL \sigma} 
}
{
  \goodTerm{\Gamma}{\Tkg {o}}{\cC \sigma}
}
$$
$$
\Rule[name=Sig Add]
{\goodSig{\Gamma}{\sigma}{n} 
 \quad \goodType{\Gamma}{A}{}
 \quad \goodSeal{\Gamma}{f}{\sigma}{A}
 \quad \goodType{\Gamma, A}{T}{i}}
{\goodSig{\Gamma}{(\nu^+ \ \sigma \ \{f\} \ T)}{n+1}}
\quad
\Biggl( \boxed{\goodSeal{\Gamma}{s}{\sigma}{A}} 
:= \boxed{\goodTerm{\Gamma, \cC\sigma}{s}{A[p^1]}} \Biggr)
$$
$$
\Rule[name=Ept Lnkg]
{}
{\goodTerm{\Gamma}{\mu\cdot}{\cL \nu\cdot}}
\quad
\Rule[name=Lnkg Add]
{ \goodTerm{\Gamma}{o}{\cL \sigma} 
\quad \goodType{\Gamma}{A}{}
\quad  {\goodSeal{\Gamma}{f}{\sigma}{A}}
 \quad \boxed{\goodTerm{\Gamma, A}{t}{T}}
}
{\goodTerm{\Gamma}{(\mu^+ \ o \ \{f\} \ t)}{\cL(\nu^+ \ \sigma \ \{f\} \ T)}}
$$
We start with signature (in a brand new judgement $\goodSig{\Gamma}{\sigma}{n}$). A signature is a list of types contained by a \textit{linkage}\footnote{Following the convention by \citet{zm2017}, we name the meta-theoretical correspondents of \textit{family} as \textit{linkage}.}, tracking the length as superscript.   We will use $\{a : T_1, b : T_2, ..\}$ as a shorthand for signature with type $T_1, T_2, ..$ for readability. 
\newpage 
Adding new types into signature \ruleref{Sig Add} requires a \textit{sealing judgement} $\goodSeal{\Gamma}{f}{\sigma}{A}$, which is an alias of a term judgement. Sealing is corresponding \textit{to indicating whether each field is overridable(extensible) or non-overridable (thus definitionally exposed) in the Coq Plugin}. 
Concretely, we encode our STLC example to show the motivation of the sealing.
%\YZ{Is sealing used only for late-binding the signatures of inductive types? Do I understand correctly that late binding of other kinds of fields (like subst) is achieved by the boxed premise in rule "LNKG ADD"?}\EDJreply{All late-bound are achieved by ``selective-abstraction'' imposed by sealing. For example, recursion will be consequently abstracted because of the abstraction of the inductive type it recurse on. 
%}\EDJreply{Sorry, I see what you mean. Yes. the box part is important for late-binding. But the functionality of sealing is always there. They are both required to achieve overriding. A illustrative example is, if the type is $S(\bot)$, then even with the boxed premise, it is still not overridable because the type itself restricts all the possibilities.}\YZreply{
%  By "selective abstraction", do you mean in the metatheory the programmer chooses what and how to abstract?
%}\EDJreply{Yes.}

% For example, if a field is overridable, we abstract that to its type; if a field has exposed its definition, we abstract it into the singleton type containing it instead.  Sealing $\goodSeal{\Gamma}{f}{\sigma}{A}$ is useful when the surrounding interface is $\sigma$ and we want to construct the new field based on $A$.

% We can now point out how "Overridable" and "pins" is modeled. Despite the fact we have a complicated dynamic of invisibility between overridable and normal fields in the plugin, every field can be elaborated into the (pseudo)-syntax form \\ \mintinline{Coq}{Field hiding {x, y, z ...} newfield := t...}  where every former field x,y,z,...  hide its definition when type-check t, and remaining former fields have their definition exposed in t.
% This above elaborated "hiding" syntax directly corresponding to the $f$ in $(\mu^+ ~o~\{f\}~t)$, where each fields decides the abstraction of the prior.


\begin{figure}[!htb]
  \lstset{
      basicstyle=\fontsize{8}{8.5}\ttfamily,
  % numbers=left,
  }
  
  \begin{minipage}{\textwidth}
  \begin{multicols}{3}
  

  \definecolor{codecomment-color}{HTML}{0DA3FF}
  
  \begin{lstlisting}
  Family STLC. 
   FInductive ty (* no hiding *)

   FInductive tm (* hiding ty *) 
   
   FRecursion subst
     Case ... (* hiding ty tm *)
     ...
   End subst (* hiding ty *) 
   

   Field t1 := (tvar 0) 
   (* hiding ty tm subst *) 


   Field t2 := (tapp t1 t1)
   (* hiding ty tm subst *)

  ...
  \end{lstlisting}
  
  % \makeline[0pt]{Parser-exmp-before-start}{Parser-exmp-before-end}[codecomment-color!50]
  
  \columnbreak
  % \definecolor{codecomment-color}{HTML}{5D030F}
  
  \begin{lstlisting}

    (* s₁  : {} | {} *)
    - o₁  = μ+ {} s₁ ⋆ : 𝕃{ty : 𝕎τ₁}
    (* s₂  : {ty : 𝕎τ₁} | [ty : 𝕌; bool : ty; ...] *) 
    - o₂  = μ+ o₁ s₂ ⋆ : 𝕃{ty : 𝕎τ₁; tm : 𝕎τ₂}
    (* s₃₁  : {ty : 𝕎τ₁; tm : 𝕎τ₂} 
            | [ty : 𝕌; ... ; tm : 𝕌; tvar : tm; ...] *)
    - o₃₁  = μ+ o₂ s₃ (μ+ ..) : 𝕃{ty : 𝕎τ₁; tm : 𝕎τ₂; subst' : 𝕃(RecSig)}
    (* s₃ : {ty : 𝕎(τ₁); tm : 𝕎τ₂; subst' : 𝕃(RecSig)} 
          | [ty : 𝕌; ... ; tm : 𝕎τ₂; subst' : 𝕃(RecSig)] *)
    - o₃ = μ+ o₃₁ s₃ (λ (Wrec ..)) : 𝕃{ty : 𝕎τ₁; tm : 𝕎τ₂; subst' : 𝕃(RecSig); subst : tm → nat → tm → tm}
    
    (* s₄ : {ty : 𝕎τ₁; ... subst : tm → nat → tm → tm } 
          | [ty : 𝕌; ... ; tm : 𝕌; subst : tm → nat → tm → tm] *)
    - o₄ = μ+ o₃ s₄ .. : 𝕃{ty : 𝕎τ₁; ... t₁ : tm}
  
    (* s₅ : {ty : 𝕎τ₁; tm : 𝕎τ₂; subst : 𝕃(RecSig); t1 : tm} 
          | [ty : 𝕌; tm : 𝕌; subst : 𝕃(RecSig); t1 : tm] *)
    - o₅ = μ+ o₄ s₅ : 𝕃{ ... }
  \end{lstlisting}
  
  \columnbreak
  

  % \makeline[.5\textwidth+9pt]{Parser-exmp-after-start}{Parser-exmp-after-end}[codecomment-color!50]
  
  \end{multicols}
  \end{minipage}
  \caption{Sealing and Linkage}\label{fig:sealing+linkage}
  \end{figure}

In \cref{fig:sealing+linkage}, the "hiding" comment illustrates the late-bound prior fields for each fields. For "tm", "ty" is late-bound and thus extending "ty" won't cause disruption on the inheritance of "tm". For "subst" itself, the definition of "tm" is not hidden to achieve exhaustiveness checking. However, for each case handler inside "subst", "tm" is late-bound and thus each handler is inheritable. 

Each hiding comment corresponds to the textual-aligned sealing derivation on RHS. We use surface syntax for eligibility: curly brackets indiate a signature (omitting sealing at type level for readability) and square bracket indicate a nested sigma type, corresponding to sealing judgement $\goodSeal{}{s}{\sigma}{A}$.

For "tm", "ty" is hidden and thus the sealing "s₂" will make "ty" into an arbitrary type "𝕌" and expose its constructor "bool", "arrow", and etc. Then when constructing "o₂", the newly added type "tm" as "⋆ : 𝕎τ₂" is constructed in the context where "ty : 𝕌", ignoring its original eliminator. (Because eliminator is only possible when "ty : 𝕎τ₁"). On the other hand, for "subst", only "ty" is hidden and thus the sealing "s₃" leave "tm : 𝕎τ₂" untouched and "Wrec" becomes possible when constructing "o₃".

Basically, the hidden prior fields---inductive type are sealed into arbitrary type with ``constructors only'', where the recursion is constructed without sealing the inductive type.  Generally speaking, each extensible inductive type are hidden for all the other fields except for the recursion on itself. Furthermore, normal fields (e.g. "t1, t2") can expose its definition to all in meta-theory by using singleton type. (e.g.  saying "t1 : S(tvar 0)" instead of the current "t1 : tm"); and then overridable fields are hidden for normal fields (by sealing "t1 : S(tvar 0)" into "t1 : tm"). Generally speaking, anything with explicit "hiding" decoration can be understood as sealing. 

% Intuitively, sealing corresponds to the following surface syntax\\
% \mintinline{Coq}{Field hiding {x, y, z ...} newfield := t...} where every former field "x","y","z",...  hide its definition when type-check "t", and remaining former fields have their definition exposed for "t". Thus, "x","y","z", ... can be overridden but other fields have to stay the same if we want to inherit "t". In meta-theory, we control the sealing $f$ so that "x","y","z",... only has type exposed and remaining fields are definitionally exposed using singleton type.


% The reader should feel this subtle difference compared to our current plugin implementation---$(\mu^+ ~o~\{f\}~t)$ reads as \textit{``$t$ is a new well-typed field when we abstract the \textbf{prior} fields using $f$''}. In other words, every field in meta-theory has the right to choose how they perceive and abstract the prior fields. While in the plugin, "Overridable" keyword will decide how the current defining field is percived by all the following fields.


% Given a signature $\sigma$, we have the linkage type $\cL\sigma$ inhabited by linkages; and the \textit{compilation type} $\cC\sigma$ of the linkage type. The compilation type corresponds to how we compile a context into a \mintinline{Coq}{Module Type} in the plugin implementation---in metatheory, $\cC\sigma$ will be a nested sigma type. For example, $\cC\{a : \cU, b : a \to \bot \} = \Sigma(a : \cU).\Sigma (b : a \to \bot). \top$.

Due to the definition of sealing, we can see the importance of compilation type, as the result of a sealing is usually a (nested) sigma type instead of a linkage type. For example, $\TyLkg{\{a : S(\bot), b : a \to \bot\}}$ can only be sealed into $\TyTkg{\{a : \cU, b : a \to \bot\}}$ but never $\TyLkg{\{a : \cU, b : a \to \bot\}}$. Because the later cannot be inhabitted---by projecting the second field of the $ t : \TyLkg{\{a : \cU, b : a \to \bot\}}$, we can get a proof of bottom type.

% The difference between compilation type and linkage types lies in their introduction rules. The compilation types are just syntactic sugar for sigma types. However, the intro rules for the linkage types gives it the power of late-binding (also the fundation of inheritance and overriding). More concretely, the boxed premise in \ruleref{Lnkg Add} is making $t$ parametric on $A$, contrary to how in sigma type, the $t$ will be about a concrete $a : A$. This parametricity makes $t$ inheritable and the earlier stuff about $A$ late-bound.

% The sealing judgement provides a witness $f$ that a context $\Gamma, \cC\sigma$ can be abstracted into an \textit{interface} $\Gamma, A$ for a type $\goodType{\Gamma}{A}{}$. For example, sealing judgement can abstract an inductive type with constructors into an arbitrary type with functions (but without eliminators). Then any later fields $\goodTerm{\Gamma, A}{t}{T}$ relies only on the \textit{interface} $A$ can be inherited to other contexts (e.g. using extended inductive type), as long as the targeting contexts can be abstracted into the surrounding \textit{interface} for $t$ (i.e. $\Gamma, A$)  as well.

% The importance of the compilation type lies in its difference from the linkage type. First we have a singleton type $"s A" : S(T) \iff "A" \equiv T$, i.e. a type of one single term.  We can see that we have a term $x : \cC\{A : S(\bot), B : A \to \bot \} \vdash h_1 : \cC\{A : \cU, B:A \to \bot \}$ which is equivalent to $x : \Sigma (A: S(\bot)). A \to \bot \vdash h_1 : \Sigma (A: \cU). A \to \bot$. This can be considered as using seal to do an abstraction on $A:S(\bot)$.

% But we can never have a term $x : \cL\{a : S(\bot), b : a \to \bot \} \vdash h_2 : \cL\{a : \cU, b : a \to \bot \}$, as we know $\cL\{a : S(\bot), b : a \to \bot \}$ is inhabited
% and that leads to a close term of $\cL\{a : \cU, b : a \to \bot \}$ due to the nature of late-binding of $\cL\sigma$---once we override $a$ with $\top$ we will have inconsistency using $b$. This example illustrates the programming difference between the sigma type and linkage type.  

% However, this form of abstracting a type is very important when programming in our metatheory---as later we will show inductive type of different constructors locates in different type (universe), and to make later fields inheritable we need to work on an interface of abstracted type. Thus the compilation type is indispensable.

% Looking closely, the sealing judgement helps to model most machinaries about abstraction in plugin implementation.

% Inductive type and recursor and handlers

$$
\Rule[]
{}
{\goodWSig{\Gamma}{w\cdot}{0}}
\quad
\Rule[name=WSig-Add]
{\goodWSig{\Gamma}{\tau}{n}
  \quad \goodType{\Gamma}{A}{i}
  \quad \goodType{\Gamma, A}{B}{i}}
{\goodWSig{\Gamma}{w^+ \  \tau \  A \  B}{n+1}}
\quad
\Rule[name=Ind-Univ]
{\goodWSig{\Gamma}{\tau}{n}}
{\goodType{\Gamma}{\bW \tau}{i}
}
$$
$$
\Rule[name=Ind-Term]
{\goodTerm{\Gamma}{T}{\bW \tau}
  \quad \goodTerm{\Gamma}{a}{\pi^j_1\tau}
  \quad \goodTerm{\Gamma, \pi^j_2\tau[(id, a)]}{b}{El~("W"~T)[p^1]}}
{\goodTerm{\Gamma}{"Wsup"~j~T~a~b}{El~("W"~T)} }
$$
$$
\Rule[name=Hdler]
{\goodType{\Gamma}{A}{}
\quad \goodType{\Gamma, A}{B}{}
\quad \goodType{\Gamma}{R}{}
}
{\goodType{\Gamma}{"RecSig1"~A~B~R \equiv \Pi A\; \Pi (\Pi B (R[p^2]))\; R[p^2]}{}}
\quad 
\Rule[name=Hdlers]
{\goodWSig{\Gamma}{\tau}{n}
\quad \goodType{\Gamma}{R}{}
}
{\goodSig{\Gamma}{"RecSig"~\tau~R}{n}}
$$
$$
\Rule[name=Hdler-Proj]
{ \goodWSig{\Gamma}{\tau}{N}
\quad \goodTerm{\Gamma}{o}{\cL("RecSig"~\tau~R)}
\quad j < N
}
{\goodTerm{\Gamma}{"prjR"~j~o}{("RecSig1"~(\pi_1^j \tau)~(\pi_2^j \tau)~R)}}
$$
$$
\Rule[name=Rec-Constr]
{ 
  % \goodWSig{\Gamma}{\tau}{N}
\quad \goodTerm{\Gamma}{T}{\bW \tau}
% \quad \goodType{\Gamma}{R}{}
\quad \goodTerm{\Gamma}{r}{\cL("RecSig"~\tau~R)}
\quad \goodTerm{\Gamma}{w}{El~("W"~T)}
}
{\goodTerm{\Gamma}{"Wrec"~r~w}{R}}
\YZ{I suppose Rec-Constr is only for exhaustivity checking? If so, then would it be more intuitive to not include the w in the term? And I suppose the programmer will want to use sealing to hide its type, or otherwise it would fail to normalize for lack of a beta rule?}\EDJreply{ 1. Yes. But not include w will cause $"Wrec"~r$ be a function type. we are not supposed to add a new intro-rule for function type(or for any other type), this is bad meta-theory design. \\
  2. It is actually the other way around -- if things are abstracted, then no beta rule can apply. More concretely, $T : \bW\tau$ is abstracted into $T' : \cU$, then this later one we cannot use $"Wrec"$ on it for sure. But once we have $\{T : \bW\tau, "rec" : \cL(RecSig~\tau)\}$ we are okay, and this, after wrap "rec" with an lambda abstraction, we will have $\cC\{T : \bW\tau, "rec" : \cL(RecSig~\tau~R), "recf" : T -> R\}$ as compilation type. Then with sealing abstraction, we will have $\Sigma(T : U, \Sigma ("rec" : ?, \Sigma("ref" : T -> R)))$ 
}\YZreply{
  What happens if the programmer forgets to seal "rec"? I suppose rec
  could still be inherited into a derived family. But it would not be
  useful if the inductive type is extended there, because rec is
  applicable only to the inductive type defined in the base family.
}\EDJreply{
  You are right. Since "rec" has type "\cL(RecSig~\tau~R)", after inheritance, it is basically the same type "rec : \cL(RecSig~\tau[k]~R[k])" for some "k" stands for the upcasting. but the most important $\tau[k]$ has an  unchanged shape (like the number of constructors) so it is still a recursor for inductive type of signature $\tau$. Once the parent inductive type $\bW\tau$ is extended $\bW\tau_2$ in the children family, with enriched signature $\tau_2$, then because of the signature inconsistency with $\tau[k]$, this old recursor is not useful for this new inductive type.    \\
  In other words, this recursor linkage "rec" can be used for any inductive type with signature $\tau$.
}
%\YZreply{
%  So there IS a beta rule for recursor (but you don't show it), right?
%}\EDJreply{
%  Yes. 
%  % https://github.com/DKXXXL/Redstone/blob/46c2420982e1fac50138924cdd7e152786cd38c8/Sketch/Extensible-Module-1/Syntax.agda#L589 \\ 
%  The only reference is here, but I commented out this url. The reduction applies to $"Wrec" \_ \_ ("Wcstr")$. \\
%  We also have canonicity (i.e. a closed term of boolean reduces to true or false), so there is definitely a beta rule for inductive type.
%}\YZreply{I'd add typeset this beta rule}\EDJreply{I just realize you use the words normalize. I will say currently this system cannot ``reduce'' any open term. But can only ``reduce'' close term. \\ 
%When the programmer program on this meta-theory, they don't need canonicity or normalization at all (they write out a chain of  judgemental equality by themselves). If they program on a proof assistant, then our proof assistant will need normalization to do type checking.
%
$$
$$
\Rule[name=Rec-Beta]
{\goodTerm{\Gamma}{h}{\cL("RecSig"~\tau~R)} 
\\ \goodTerm{\Gamma}{T}{\bW\tau}}
{\goodTerm{\Gamma}{"Wrec"~h~("Wsup"~j~T~a~b) \equiv 
("app"~(("app"~("prjR"~j~h))[("id", a)]))[("id", "lam" ("Wrec"~h[p^1]~b))]
}{R}}\EDJ{"app" and "lam" are lambda application and abstraction}
$$


Our formulation of (extensible) inductive type is close to W-type (\ruleref{WSig-Add} requires a pair of types $A \vdash B$ as well), but with mulitple constructors. Thus we have a brand new judgement ${\goodWSig{\Gamma}{\tau}{n}}$ indicating signature of inductive type by its constructors. Unconventionally, a signature can induce a "universe" of types $\bW\tau$---in other words, $T : \bW\tau$ will be the inductive type instead of $\bW\tau$ itself. "Wsup" is used to construct a specific term of a given inductive type.


When programming via meta-theory, we usually seal the inductive type, only expose its constructors (as a mundane function member) and hide the eliminator by abstracting $T : \bW\tau$ into $"W"~T : \cU$. Then the later fields---those only rely on this interface with $\cU$---can be inherited to other contexts (e.g. using enriched inductive type), as long as the targeting contexts can be abstracted into this {interface} again. 

Similar to plugin implementation, we delegate handlers (and their reuse) for recursion to linkages (and their inheritance). We use "RecSig1" to compute the type of a single handler and "RecSig" for the complete family of handlers for all constructors. \ruleref{Rec-Constr} uses "RecSig" to achieve exhaustiveness checking on elimination. The partial recursors and the computational axioms are treated as normal fields as expected, since we know they are provable in the plugin implementation.

$$
\Rule[name=Inh]
{\goodInh{\Gamma}{h}{\sigma_1}{\sigma_2}
\quad \goodTerm{\Gamma}{l}{\cL \sigma_1}
}
{\goodTerm{\Gamma}{("inh" \ h \ l)}{\cL \sigma_2}},
\Rule[name=Inh-Id]
{}
{\goodInh{\Gamma}{"inhid"}{\sigma}{\sigma}},
\Rule[name=Inh-Ext]
{\goodInh{\Gamma}{h}{\sigma_1}{\sigma_2}
  \quad \goodTerm{\Gamma, A_2}{t}{T}}
{\goodInh{\Gamma}{"inhext" \ h \ t}{\sigma_1}{(\nu^+ \  \sigma_2\  T)}}
$$

% Inhertance judgement
In the plugin implementation, "Family"s are second class objects like Coq's "Module". What's more, a "Family" (with or without "extends" clause) can be considered as a standalone piece of code trying to \textit{inherit} an existent (but maybe empty) family. Thus we model them using \textit{inheritance judgement} $\goodInh{\Gamma}{h}{\sigma_1}{\sigma_2}$ instead of considering them as first class terms. The only way to use $h$ is by \ruleref{Inh} and transforming a linkage of signature $\sigma_1$ to another linkage of $\sigma_2$. Inheritance judgement can be inductively constructed via inheriting operations, overriding operations, extending operations and etc, corresponding to "Inherit", "Override" and "Extend" command  in our plugin implementation.


In this way, we will consider the program in our metatheory similar to those vanilla MLTT program, except there are meta variables denoting inheirtance judgement (i.e. "Family"), of which the only usage is to apply \ruleref{Inh}. We don't yet consider non-trivial equality between derivations of inheritance judgements.


\subsection{Syntactic Translation that removes Linkages}
To further provide the intuition why our core calculus is sound, we provide a syntactic translation from our core calculus to the subpart where linkages are absent. This can justify the \textit{consistency} and \textit{canonicity} modulo the quirky (but apparent) formulation of our inductive type. To be complete, we also provide a consistency and canonicity proof for our complete core calculus in the appendix.

\newcommand{\denotesT}[1]{{{\llbracket {#1} \rrbracket}_T}}
\newcommand{\Sigr}[2]{{ "Sig"^r~{#1}~{#2} }}

Thanks to QIIT formulation, syntactic translation becomes a function between QIIT type, which by definition respects all the judgemental equalities. We will now sketch this translation $\denotesT{\_}$, which keeps most parts untouched except for the linkages related. We omit most of the inheritance translation.



\begin{align*}
  \text{We first define a new type }& \goodType{\Gamma}{\Sigr{\Gamma}{n}}{}  \text{ when } \goodCtx{\Gamma}{}  \text{ is well-formed and } n \in \mathbb{N} \\
  \denotesT{\goodSig{\Gamma}{\_}{n}},\ &({\Sigr{\Gamma}{n}}), \text{ and } \ \denotesT{\TyTkg {-}} \text{ are mutually recursive, } \\
  & \text{defined by induction on the signature length} \\  
  \denotesT{\goodSig{\Gamma}{\_}{n}} &= (\goodTerm{\Gamma}{\_}{\Sigr{\Gamma}{n}}) \\ 
  &\text{, and thus } \denotesT{\goodSig{\Gamma}{\sigma}{n}} \text{ defined} \iff \goodTerm{\Gamma}{\denotesT{\sigma}}{\Sigr{\Gamma}{n}} \\ 
  {\Sigr{\Gamma}{(n+1)}} &= 
    \Sigma~(\Sigr{\Gamma}{n})
          ~(\Sigma~\cU
                  ~(\Sigma~(\Pi(\TyTkg {("var"_0[p^1])})(El~"var"_0[p^1]))~\cU)) \\
  \Sigr{\Gamma}{0} &= \top \\
  \denotesT{\goodType{\Gamma}{\TyTkg {\sigma}}{}}, \denotesT{\goodType{\Gamma}{\TyLkg{\sigma}}{}} &\text{ defines upon } \denotesT{\sigma} 
  \text{ and inductively on the signature length} \\
  \denotesT{\goodType{\Gamma}{\TyTkg {\sigma}}{}} &= \denotesT{\goodType{\Gamma}{\TyLkg{\sigma}}{}} = \top \quad
      \text{given } \denotesT{\goodSig{\Gamma}{\sigma}{0}} \\ 
  \denotesT{\goodType{\Gamma}{\TyTkg {\sigma}}{}} &= 
    \Sigma~\denotesT{\TyTkg {(\lsigproj{1}{\sigma})}}~(El \ (\app{("pjr"^3~\denotesT{\sigma})})[(p^1, \app{(\fst{("pjr"^2~{\denotesT{\sigma}})})})]) \\
  \denotesT{\goodType{\Gamma}{\TyLkg{\sigma}}{}} &=
  \denotesT{\TyLkg{(\lsigproj{1}{\sigma})}} \times \Pi(El~(\fst{(\snd{\denotesT{\sigma}})}))(El~(\app{("pjr"^3~\denotesT{\sigma})} )) \\
  &\text{given } \denotesT{\goodSig{\Gamma}{\sigma}{n+1}} \\
  \denotesT{\goodSig{\Gamma}{\LSigAdd{\sigma}{f}{T}}{n+1}} &= (\denotesT{\sigma}, \codety{A}, \lam{f}, \lam{(\codety{T})}) \\ 
  \denotesT{\goodTerm{\Gamma}{\mu^+~m~t}{\TyLkg{(\LSigAdd{\sigma}{s}{T})}}} &= (\denotesT{m}, \lam{t}) \\ 
  \denotesT{\goodTerm{\Gamma}{\Tkg{m}}{\TyTkg {\sigma}}} & \text{ is defined upon } \denotesT{m} \text{ and inductively on the signature length} \\ 
  \denotesT{\goodTerm{\Gamma}{\Tkg {m}}{\TyTkg {\sigma}}} &= 
  (\denotesT{\Tkg {o}}, t[(\SubstExt{p^1}{f})][(\SubstExt{"id"}{\denotesT{\Tkg {o}}})]) \\
  & \text{given } \denotesT{\goodSig{\Gamma}{\sigma}{n+1}}, \text{ where } o = \lsigproj{1}{m}, t = \app{(\snd{\denotesT{m}})}, f = p_f\nu~\sigma \\ 
  \denotesT{\goodTerm{\Gamma}{\Tkg {m}}{\TyTkg {\sigma}}} &= () \quad \text{given } \denotesT{\goodSig{\Gamma}{\sigma}{0}}
\end{align*}
\begin{align*}
  \denotesT{\goodInh{\Gamma}{\_}{\sigma_1}{\sigma_2}} &= \goodTerm{\Gamma, \TyLkg{\sigma_1}}{\_}{\TyLkg{\sigma_2[p^1]}} \\ \text{ and thus } \denotesT{\goodInh{\Gamma}{h}{\sigma_1}{\sigma_2}} &\iff  \goodTerm{\Gamma, \TyLkg{\sigma_1}}{h}{\TyLkg{\sigma_2[p^1]}} \\
  \denotesT{\goodInh{\Gamma}{"inhinh"~h~T~\uparrow^s}{(\LSigAdd {\sigma_1} {s_1}{T})}{(\LSigAdd {\sigma_2} {s_2} {T[(p^1, \uparrow^s)]})}} &= \mu^+~(h[(p^1,\lkgproj{1}{"var"_0})])\\ &\quad \quad \{f_2[p^1]\}~(\lkgproj{2}{"var"_0})[(p^1, \uparrow^s[{p^1}^{\uparrow}])] \\
\end{align*}



