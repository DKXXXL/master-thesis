To formalize our metatheory, we follow the formulation of~\citet{altkap2016}, which provides a declarative and intrinsic style of the predicative \textit{Martin-Lof Type Theory} (MLTT)~\cite{martin1982constructive},
using \textbf{Quotient Inductive Inductive Type}(QIIT)\footnote{Inductive Inductive Type(IIT) is a generalization of mutual inductive type where (indexed-)types \mintinline{Coq}{A : Type, B : A -> Type} can be defined mutually. QIIT further enhances IIT by allowing equality constructor (i.e. mathematical quotient).}, \textbf{explicit substitution} and \textbf{debruijn indices} as the main tool. 
Generally speaking, we \textbf{use each QIIT type to represent each kind of judgements, and thus a term will represent a derivation of a judgement}. The classical formulation of dependent type theory requires a lot of \textit{presupposition} and quotient afterwards, causing a lot of duplicacy. Both issues can be concisely solved by using QIIT expressing intrinsically typed syntax instead. What's more, working \textbf{in a type-theoretical setting} using QIIT makes checking of the pen-and-paper proof easier. However, we will follow conventional notation and thus the reader might not notice the usage of QIIT explicitly.\EDJ{We might be able to remove QIIT in the main text totally. Let's try to do it later.}


% Explicit substitution
Instead of meta-level substitution, we use explicit substitution, i.e., substitution itself is a expression in the program and will be done in the runtime. Thus substitution has its own typing judgement.
% Debruijn Indices
Contrary to named variables, debruijn indices use number to indicate the variable in the context. For example, \mintinline{Agda}{}


Now, we will mainly focus on the new typing rules that we introduced. We refer the those curious about the complete formulation to our appendices or \citet{altkap2016, kaposi2017type} or maybe those not using QIIT \cite{coquand2018canonicity, sterling2019algebraic}. Our metatheory is in an extensional setting. 

% linkage, signature

% Inductive type and recursor and handlers

% Inhertance
