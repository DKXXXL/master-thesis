
With the growing practice of mechanizing language metatheories,
it has become ever more pressing that interactive theorem provers 
make it easy to write reusable, extensible code and proofs.
%
This thesis presents a novel language design geared towards extensible metatheory
mechanization in a proof assistant.
The new design achieves reuse and extensibility via a form of family
polymorphism, an object-oriented idea, that allows code and
proofs to be polymorphic to their enclosing families.
Our development addresses technical challenges that arise
from the underlying language of a proof assistant being simultaneously
functional, dependently typed, a logic, and an interactive tool.
%
Our results include
\begin{enumerate*}
\item a prototypical implementation of the language design as a Coq plugin,
\item a dependent type theory capturing the essence of the language mechanism
      and its consistency and canonicity results,
%\EDJ{We don't yet have normalization though}
and 
\item case studies showing how the new expressiveness naturally addresses real
programming challenges in metatheory mechanization.
\end{enumerate*}

% \ifreport
% \smallskip

% \noindent
% This technical report supplements \citet{fpop-pldi2023}.
% \fi