Integrating family polymorphism into a proof assistant presents challenges
far beyond those found in an object-oriented setting, %\cite{ncm2004,vc-calculus-2006,zm2017},
as the underlying programming language is simultaneously
functional, dependently typed, a logic, and an interactive tool.

\noindentparagraph{%
\headinglabel{C1}{chg:extensible-v-exhaustive}.\,
Extensible inductive types vs.\ exhaustive inductive reasoning.\rD{induction reasoning changes to inductive reasoning}
}

Inductive types, generalizing algebraic data types found in
functional languages, are a central feature of any proof assistant
in use for mechanizing language metatheories.
They offer a means to define abstract syntax and inference rules.
But unfortunately, inductive types are closed to extension by design.

A family-polymorphism design could potentially support extensible
inductive types, by allowing a \emph{derived family} to add new constructors
to inductive types inherited from a \emph{base family}.
Such a feature would be useful for extending mechanized languages.
As an example, \cref{fig:stlc-nonmechanized} shows an STLC extension,
the mechanization of which would be made easier by extensible inductive types.
Code would be organized into two families, with the
derived family inheriting constructors from the base family (e.g., the
ellipsis under ``Typing rules'') and adding new constructors to model
the syntax and semantics of a fixpoint construct.

\begin{figure}

\BeforeBeginEnvironment{mathpar}{%
  \vspace{-1ex}
}

\fontsize{8.0}{8.8}\selectfont
\def\MathparLineskip{\lineskip=0.1ex}

\fcolorbox{black}{black!10}{
\begin{minipage}{0.42\textwidth}

\noindent%
Types \quad
$ \tau \ \Coloneqq\ \mathbb{1} \bnf  \tau_1 \to \tau_2 $
\medskip

Terms \quad
$ e \ \Coloneqq\ x \bnf () \bnf \lambda x.\,e \bnf e_1~e_2 $
\medskip

Typing rules
\begin{mathpar}
\Rule{ \Gamma(x)=\tau }{ \Gamma \vdash x : \tau }

\Rule{ \Gamma, x:\tau_1 \vdash e : \tau_2 }{ \Gamma \vdash \lambda x.\,e : \tau_1 \to \tau_2 }

\Rule{}{ \Gamma \vdash () : \mathbb{1} }
\quad
\Rule{
  \Gamma \vdash e_1 : \tau_1 \to \tau_2
  \quad
  \Gamma \vdash e_2 : \tau_1
}{
  \Gamma \vdash e_1~e_2 : \tau_2
}

\end{mathpar}

Values
\begin{mathpar}
\Rule{}{\textit{Val}(())}

\Rule{}{\textit{Val}(\lambda x.\,e)}
\end{mathpar}

Substitution
\begin{mathpar}
  \subst{y}{e}{x} = \begin{cases} y, & \text{if } x \neq y \\ e, & \text{if } x = y \end{cases}

  \subst{()}{e}{x} = ()

  \subst{(\lambda y.\,e')}{e}{x} = \begin{cases} \lambda y.\,\subst{e'}{e}{x}, & \text{if } x \neq y \\ \lambda y.\,e', & \text{if } x = y \end{cases}

  \subst{(e_1~e_2)}{e}{x} = \subst{e_1}{e}{x}~\subst{e_2}{e}{x}
\end{mathpar}

Reduction rules
\begin{mathpar}
\Rule{ e_1 \longrightarrow e_1' }{ e_1~e_2 \longrightarrow e_1'~e_2 }

\Rule{ \textit{Val}(e_1) \quad e_2 \longrightarrow e_2' }{ e_1~e_2 \longrightarrow e_1~e_2' }

\Rule{\textit{Val}(e_2)}{ (\lambda x.\,e_1)~e_2 \longrightarrow \subst{e_2}{e_1}{x} }

\end{mathpar}

\begin{tabular}{@{}l@{\qquad}l@{}}
Substitution lemma & Preservation theorem \smallskip\\
Progress theorem & Type-safety theorem
\end{tabular}%}

\end{minipage}
}%
\raisebox{20pt}{%
%\tikz \draw [arrows = {-Latex[width'=0pt .5, length=10pt, fill=white]}] (0,0) -- (1,0);
\tikz \draw [
  line width = .5pt,
  arrows = {-Stealth[inset=0pt, length=5pt, fill=white, angle'=60, scale=2]}
] (1.2,0) -- (0,0);%
%https://tikz.dev/tikz-arrows
\fcolorbox{black}{black!10}{
\begin{minipage}{0.36\textwidth}

\noindent%
Types \quad $ \tau \ \Coloneqq\ \cdots $ \textit{(no change)}
\medskip

Terms \quad $ e \ \Coloneqq\ \cdots \bnf \mathrm{fix}\,x.\,e $
\medskip

Typing rules
\begin{mathpar}
\cdots

\Rule{
  \Gamma, x : \tau \vdash e : \tau
}{
  \Gamma \vdash \mathrm{fix}\,x.\,e : \tau
}

\end{mathpar}

Values $\quad \cdots$ \textit{(no change)}
\medskip

Substitution $\quad \cdots$ \textit{(no change)}
\medskip

Reduction rules
\begin{mathpar}
\cdots

\Rule{}{ \mathrm{fix}.\,x.\,e \longrightarrow \subst{e}{\mathrm{fix}.\,x.\,e}{x} }

\end{mathpar}

\begin{tabular}{@{}l@{\ \ }l@{}}
Substitution lemma   & \textit{(a new case required)} \smallskip\\
Preservation theorem & \textit{(a new case required)} \smallskip\\
Progress theorem     & \textit{(a new case required)} \smallskip\\
Type-safety theorem  & \textit{(no change)}
\end{tabular}

\end{minipage}
}%
}

\caption{%
  STLC metatheories (left) and its extension with fixpoints (right).
}
\label{fig:stlc-nonmechanized}
\end{figure}


However, there is a tension between extensibility of inductive types
and exhaustivity of inductive reasoning.\footnote{%
The tension reflects a duality between variants and records.
With record-like language constructs (e.g., OO classes),
it is safe for an extension to add new fields:
existing fields can still be projected.
But variant-like constructs (e.g., inductive types) do not automatically
enjoy safe, modular addition of constructors:
existing pattern matches could become non-exhaustive.}
In \cref{fig:stlc-nonmechanized}, all the lemmas and theorems, as well as
the substitution function, require induction (i.e., elimination of inductive types).
A language design must enforce that induction remains exhaustive in the
face of the new constructors in the derived family.
For modularity, the type system should do so without requiring
redefinition or rechecking of those cases already handled by the base
family.

\noindentparagraph{%
\headinglabel{C2}{chg:latebind-v-defeq}.\,
Late binding vs.\ definitional equality.%
}

Family polymorphism enables modular reuse via late binding:
the code of a base family can be reused by a derived family,
because fields referenced by that code have meanings polymorphic to the
enclosing family.

This flexibility, however, prevents \emph{definitional equality}
that one takes for granted when programming in a proof assistant.
In \cref{fig:stlc-nonmechanized}, a proof assistant supporting family
polymorphism cannot unfold references to the substitution function into its definition (i.e., a pattern match against four cases), %Review D : l135
as a derived family may modify the definition of the function by adding
new cases.
Without the ability to unfold the substitution function, how can the
programmer even prove the substitution lemma?
%
The problem is compounded by the occasional need in derived families to override fields,
which is potentially at odds with being able to use equalities over the fields.

%Common to family-polymorphism mechanisms in the OO realm is the ability
%for a derived family to override fields defined in the base family.
%As we show later, this expressiveness is also useful for metatheory
%mechanization.
%However, it is in conflict with dependent typing, wherein type checking
%often involves unfolding definitions.
%For example, in an STLC mechanization, the substitution lemma is checked
%against the definition of the substitution function.
%If a derived family were allowed to arbitrarily override the
%substitution function, then the proof of the substitution lemma, when
%inherited, would be ill-formed (even in the absence of any inductive-type extensions).
%A language design should only allow those overridings that are safe.

\noindentparagraph{%
\headinglabel{C3}{chg:self-v-consistency}.\,
Self reference vs.\ logical consistency.%
}

The language-theoretic essence of late binding is self reference;
inheritance and family polymorphism are mechanisms for
incrementally modifying self-referential definitions~\cite{cook1990inheritance}.
%the implicit self parameterization is how they achieve late binding and
%thus offer extensibility.
However, self reference could easily lead to divergence.
Divergence is not a concern for the design of ordinary OO or
functional languages, but it would mean logical inconsistency---and
hence unsoundness!---for a language aimed at logical reasoning.
A family-polymorphism design must tame self reference to guarantee consistency.

\noindentparagraph{%
\headinglabel{C4}{chg:tooling}.\,
User experience and system implementation.%
}

Interactive theorem proving and tactic programming are central to
a typical programming experience with a proof assistant.
A language design integrating family polymorphism should be compatible
with these forms of programming.
In particular, it should be possible in our system to
incrementally navigate through vernacular commands and, moreover, construct proofs
with common tactics while getting instant feedback on the proof state,
even in the middle of a family definition.
Last but not least, in addition to proving theorems,
it should be possible for terms defined with families to possess computational content.

\ifShowOldWriting

\newpage

We want the metatheory mechanization to allow adding new
function/proving new lemmas, and adding new constructors for inductive
type at the same time  (Expression Problem~\cite{wadler-ep}). Coq
support the former but not the later---when an existent inductive type
is extended with new constructor, we have to modify the pattern matching
in the existent functions. This motivates us to add inheritance/Family
Polymorphism into Coq as a solution to the Expression Problem. 

\begin{figure}[!htb]
  \begin{minipage}[t]{0.47\linewidth}
\begin{lstlisting}[language=Coq,  escapeinside={@}{@}]
Family STLC.
Inductive ty : Set :=
  | unit : ty
  | arrow : ty -> ty -> ty.
Inductive term : Set := 
  | tm_var : id -> term 
  | tm_abs : id -> term -> term 
  | tunit : term ...
Fixpoint subst 
  : id -> term -> term := ...
Inductive has_type 
  : context -> term -> type := ...
Proposition subst_lemma :
  has_type (G; x ↦ A) t T ->
  has_type G u A ->
  has_type G (subst x u t) T.
Inductive step : term -> term -> Prop 
  := ...
(* ... and more, end with type safety *)
EndFamily.
\end{lstlisting}
  \end{minipage}
  \begin{minipage}[t]{0.47\linewidth}
\begin{lstlisting}[language=Coq,  escapeinside={@}{@}]
Family STLC_bool extends STLC.
Extend Inductive ty : Set :=
  | bool : ty.

Extend Inductive term : Set := 
  | tt : term | ff : term 
  | tif : term -> term -> term -> term

Extend Fixpoint subst (* Need to handle new term *)
  : id -> term -> term := ...
Extend Inductive has_type (* .. and new ty too *)
  : context -> term -> type := ...
Extend Proposition subst_lemma :
  ... (* Need to prove extra cases *)


Extend Inductive step : term -> term -> Prop 
  := ... (* Need to expand this binary relation *)
(* ... and more extension *)
EndFamily.
\end{lstlisting}
  \end{minipage}
  \caption{Example STLC and its extension}\label{fig:STLC-example}
\end{figure}

Refering to the syntax of \citet{zm2017}, we will come up the above
example. However, to add inheritance into a dependent type system like
Coq is not simple and there are various challenges and tension we need
to resolve. 


\textlabel{Challenge (1)}{chg:extensible-inductive-type}~\textbf{Extensible Inductive Type vs. Exhaustiveness of Inductive Reasoning}.
Unlike most general purposed programming languages, Coq, as a dependent
type theory, is a total language and thus each pattern matching cases
must fully ``cover'' an inductive type. This process is
``\textit{interactive}'' for inheritance---when inhertiance extend the
inductive type in the children family but the user forgets to extend the
corresponding inductive reasoning, our compiler need to signal an error
correspondingly, or better---guide the user to fill the incomplete
inductive reasoning. For example, in \cref{fig:STLC-example} when "term"
is extended with "tt" in the children family "STLC_bool", our compiler
need to make sure "subst" and "subst_lemma" are also extended correctly
otherwise signaling an error.
Additionally, our compiler must support direct inheritance (and
overriding) of the old case handlers for "tm_var", "tm_abs" and etc, to avoid
the boilerplate code. 


\textlabel{Challenge (2)}{chg:definition-relevant-reasoning}~\textbf{Definition-awared Reasoning vs. Overridability}.
In family polymorphism for general purposed programming language,
overriding is allowed to happen anywhere. But the problem becomes more
subtle in the case of dependent type since the later fields can have a
type dependent on the former fields.  Consider a small example where we
have consecutive fields \mintinline{Coq}{{.. a : nat := 1; a-def : Eq a
1 := eq_refl; ..}}, should we support override "a"? If we do support it,
how can our compiler detect that "a-def" is not inheritable and signal
an error when such thing happens? 

A similar example is about "subst" and "subst_lemma" in
\cref{fig:STLC-example}. For "subst_lemma" to hold throughout two
families, we shouldnot allow overriding on the definition of "subst".
For example, we should still stick with the fact that "subst tunit x y =
tunit", as this fact is used when proving "subst_lemma" and we want this
lemma to be inherited to the children family.

Note we cannot simply just abandon overriding because we do want it sometimes---say we want a monotonic function \mintinline{Coq}{{.. f := .. ; p : monotonic f; ..}}. The proof of "p" likely requires the concrete definition of "f" but in the future we might want to override "f" and "p" together to swap the definition.

These three examples both show the incompatibility between exposing the
concrete definition of a field for reasoning (this second example even
involves the definition of fixpoint and pattern matching) and
overridability. We need to be careful about this tension when designing
the language.

\textlabel{Challenge (3)}{chg:consistency}~\textbf{Consistency vs. Self Parametrization}.
The meta\-theoretical formulation in~\citet{zm2017} for family
polymorphism is very natural as they consider an extension of the family
as the concatenation of a new linkage. However, this formulation is not
directly adaptable for dependent type theory because it can easily
encode non-termination sabotaging consistency by the presence of "self".
Informally, in the orignal formulation, we can override the "f" in
family $\{"f" := id; "g" := "call self.f"\}$ with "call self.g",
resulting the mutual calling of "f" and "g" thus a deadloop. This
conflict between self parametrization and consistency requests changing
the original formulation of family and inheritance for dependent type
theory.


\textlabel{Challenge (4)}{chg:software-engineering}~\textbf{Engineering and Ecosystem}.
Not only for the metatheory, we also want our implementation of family
to have good compatibility with the rest of Coq. For example, we
shouldn't compromise \textbf{tactic programming} as it is part of the
idiomatic Coq programming experience. Additionally, the interactive development
style requires \textbf{incremental type checking}---our design has to be
able to type check each Coq Vernacular command once inputted. What's
more, we want our family ``term'' to be incorporated into Coq \textbf{without
breaking its other functionality}. For example, the user may want to
project the family member anywhere inside an arbitrary Gallina term,
that may later be computed and normalized. However, families themselves
can be second class citizens just like \citet{zm2017}.
%\YZ{I was thinking
%if the paper should start by talking about the functional encoding of
%inheritance (namely exposing a self parameter to allow late binding, and
%later taking the fixpoint of the extensible module when it needs to be
%accessed outside the family).  I think most readers will benefit from a
%recap of this encoding.\\
%Coq is total but that doesn’t prevent Coq from having a Fixpoint keyword (plus you proved your language is consistent).
%}\EDJreply{Feel free to add these. I don't have many knowledge about this part.}

% \subsection{Design Space}

% When adapting Family Polymorphism into dependent types, we choose to
% focus only on the essence of family (and inheritance)
% structure in \citet{zm2017}, and thus a lot of unrelated features, like
% Interface, will be removed. In this case, it will look like module with
% late-binding. Inheritance can be considered as code and proof reuse mechanism for module. We also
% need it to have good compatibility with inductive types, because we
% don't want to retain the idiomatic Coq programming experience
% %\YZ{By 'reasoning power of Coq', I suppose you mean the expressive power of Gallina?}\EDJreply{I think the sentence last time was not really making sense because using Church Encoding of inductive doesn't sabotage the expressiveness. I rephrase it now, please check.}
% (and its tactic programming). 

% Let's start with a basic example---STLC and its extension in
% \cref{fig:STLC-example}---to consider what kind of features are required
% and how much of them can be supported by family polymorphism.
% add the example code here





% First and foremost, we don't want to throw away some basic good
% properties of Coq. \textlabel{Req~(0A)}{chg:software-engineering}~\textbf{we
% want to retain incremental (modular) type checking}. Notice that, the
% modularity here in Coq is a bit different from other languages, because
% Coq supports \textit{interactive} theorem proving, so we actually need
% \textit{statement-wise} incremental type checking, not only for avoiding
% re\-compilation, but also to enable immediate feedback and incremental
% type-checking for the Coq developer. We don't want our family to ruin
% this conventional routine of interactive development.
% \textlabel{Req~(0B)}{chg:software-engineeringb}~\textbf{we also want to keep the
% computational ability of Coq when using families}. Coq, based on constructive
% logic, can consider proof as programs. 
% We don't want our Family facility to ruin this when incorporated with 
% the remaining parts of Coq: 
% the developers should be able to project fields of families 
% as first-class value just like how they can project fields of Coq's Module.

% We want to reason about fields in a family.
% \textlabel{Req (1)}{chg:definition-relevant-reasoning}:\textbf{ we want to be able to
% reason about fields in a family}, just like in a module, where a field
% can reason about its former fields. More generally, we want to allow a
% later field to be type-dependent on the former fields. In the example of
% \cref{fig:STLC-example}, "subst", "has_type", "subst_lemma" all require
% this feature.

% we want extensible inductive type
% \textlabel{Req (2)}{chg:extensible-inductive-type}:\textbf{ we need extensible
% inductive types}, 
% to extend "term" and "ty" in \cref{fig:STLC-example}.
% Extensible inductive types are not supported by \citet{zm2017}, so
% we need further consideration about it on both implementation and
% metatheory.

% \textlabel{Req (3)}{chg:extensible-inductive-type}: \textbf{we also want extensible
% pattern matching and induction reasoning}, to extend "subst" in
% \cref{fig:STLC-example}.
% There are actually two kinds of ``pattern matching'', one for data and
% the other for induction reasoning, i.e., one uses the eliminator to
% \mintinline{Coq}{Set} or \mintinline{Coq}{Type}, and the other uses the
% eliminator to \mintinline{Coq}{Prop}. Luckily, in this setting,
% extensible pattern matching can be easily expressed with family
% extension---we just aggregate all case handlers of pattern matching
% into one family (as a bunch of functions), and then one family can
% encode one pattern matching, and family inheritance can express adding
% case handlers. Then we just need to introduce a primitive that will
% ``wrap'' that family into a recursive function. Induction reasoning can
% be handled in an almost identical way. 

% we want to be able to reason pattern matching as well
% However, there is still difference between data recursion and induction
% reasoning, because the former one is \textit{proof-relevant}. This
% difference leads to another issue: \textlabel{Req (4)}{chg:definition-relevant-reasoning}:
% \textbf{we need to reason about (the computation about) extensible
% pattern matching}, just like "subst_lemma" in \cref{fig:STLC-example}.
% When proving "subst_lemma", we have to know "subst" is invariant on
% "tt", i.e., \mintinline{Coq}{(subst i x tt) = tt}. This kind of
% information requires exposing \textit{computational rules} from the
% recursors.


% we want tactic programming and certain level of automation
% Finally, \textlabel{Req (5)}{chg:software-engineering}: \textbf{we desire tactic
% programming} to relieve us from the hassle of direct manipulation of proof terms.

\newpage

\fi