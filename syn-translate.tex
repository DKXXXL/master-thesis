We present a translation from the \TT syntax to the fragment of \TT without
linkage signatures or linkages.
The translation echoes how the \Lang plugin is implemented (\cref{sec:coqimpl})
as a translation to Coq without facilities for family polymorphism.
%In \Lang, families are compiled to Coq modules; similarly, in this appendix
%we translate linkages in \TT to dependent tuples.
The translation is type-preserving by construction, as we work in an
intrinsically typed setting.
%The described compilation strategy can be considered as
%a \textit{type preserving syntactic translation}, a function mapping from our syntactic
%model(typed Abstract Syntax Tree(AST)) to a ``subpart''\footnote{Not exactly subpart---the only
%difference lies in "Wrec". In our metatheory, it is constructed
%using linkage; in ``subpart'', it will be using $\Sigma$-types (translation
%of the linkage).} of our syntactic model(typed AST) \textit{without
%linkages}.

%This ``subpart'' is the standard MLTT with our
%(unconventional) formulation of W-types,
%and recursors that use $\Sigma$-types to aggregate handlers.
%
%Basically, we can compile away (the derivation of) judgement $\goodSig{\Gamma}{v}{n}$,
%the type $\cL$, $\cP$, and related terms, by transforming them into
%corresponding $\Sigma$-types.
 

% \newcommand{\denotesT}[1]{{{\llbracket {#1} \rrbracket}_T}}
% \newcommand{\Sigr}[2]{{ "Sig"^r~{#1}~{#2} }}

We define the translation $\denotesT{}$ below; the fragment of the syntax
irrelevant to linkages is translated using the identity function and is thus elided.
When the context~$\Gm$ is clear, we use $\denotesT{T}$ to mean $\denotesT{\goodType{\Gamma}{T}{}}$.

We will first define two helpers
\begin{itemize}
  \item a new type $\goodType{\Gamma}{\Sigr{\Gamma}{n}{j}}{}$  when
  $\goodCtx{\Gamma}{}$  is well-formed and $n \in \mathbb{N}$
  such that $\goodTerm{\Gm}{\denotesT{\goodSigU{j}{\Gamma}{\lsig}{n}}}{\Sigr{\Gm}{n}{j}}$
  \item and a function $\mathscr{P} : \{\lsig' : \goodTerm{\Gm}{\lsig'}{\Sigr{\Gm}{n}{j}} \} \to \{ T : \goodType{\Gm}{T}{} \}$ map one syntax term of $\Sigr{\Gm}{n}{j}$ into a type.
\end{itemize}
mutually inductively on $n$, the signature length.

\begin{align*}
  {\Sigr{\Gamma}{(n+1)}{j}} &= 
    \Sigma~(\Sigr{\Gamma}{n}{j})
          ~(\Sigma~\cU_j \\
          & \quad \quad ~(\Sigma~(\TyPi{{\mathscr{P}({\sub{\var{0}}{\SubstWeak{1}}})}}{\El{\sub{\var{0}}{\SubstWeak{1}}}}) \\
          & \quad \quad \quad \TyPi{\El{\sub{\var{0}}{\SubstWeak{1}}}}{\cU_j})) \\
  \Sigr{\Gamma}{0}{j} &= \top \\
  \mathscr{P}(\lsig') &=  
    \Sigma~(\fst{\lsig'})~(\El{\app{"snd"^3~\lsig'}[(\SubstWeak{1}, \app{\fst{("snd"^2~{\lsig'})}})]}) \\ & \text{ when } \goodTerm{\Gm}{\lsig'}{\Sigr{\Gm}{n+1}{j}} \\
  \mathscr{P}(\lsig') &=  \top \quad \text{ when } \goodTerm{\Gm}{\lsig'}{\Sigr{\Gm}{0}{j}}  \\
\end{align*}

% We make $({\Sigr{\Gamma}{n}{}})$ and 

% $\denotesT{\TyTkg {-}}$ mutually recursive,
% defined by induction on $n$ and the signature length.
% \denotesT{\goodSig{\Gamma}{\_}{n}} &= (\goodTerm{\Gamma}{\_}{\Sigr{\Gamma}{n}}) \\ 
% &\text{, and thus } \denotesT{\goodSig{\Gamma}{\sigma}{n}} \text{ defined} \iff \goodTerm{\Gamma}{\denotesT{\sigma}}{\Sigr{\Gamma}{n}} \\

With the above two helpers, we can define denotation $\denotesT{}$.

\begin{align*}
  \denotesT{\goodSigU{j}{\Gamma}{\lsig}{n}} &\in \{t \,|\, \goodTerm{\Gm}{t}{\Sigr{\Gm}{n}{j}} \} \\
  {\goodType{\Gamma}{\TyTkg{\lsig}}{}} &= \mathscr{P}(\denotesT{\lsig}) \\
  \denotesT{\goodType{\Gamma}{\TyLkg{\sigma}}{}} &\text{ defines upon } \denotesT{\lsig} 
  \text{ and inductively on the signature length} \\
  \denotesT{\goodType{\Gamma}{\TyLkg{\sigma}}{}} &=
  \denotesT{\TyLkg{\lsigproj{1}{\sigma}}} \times \Pi(\El{\fst{(\snd{\denotesT{\sigma}})}})(\El{\app{"snd"^3~\denotesT{\sigma}}}) \\
  &\text{given } \denotesT{\goodSig{\Gamma}{\sigma}{n+1}} \\
  \denotesT{\goodType{\Gamma}{\TyLkg{\sigma}}{}} &= \top \quad
  \text{given } \denotesT{\goodSig{\Gamma}{\sigma}{0}} \\
  \denotesT{\goodSigU{j}{\Gamma}{\LSigEmp}{0}} &= \unit \\
  \denotesT{\goodSigU{j}{\Gamma}{\LSigAdd{\sigma}{f}{T}}{n+1}} &= (\denotesT{\sigma}, \codety{A}, \lam{f}, \lam{\codety{T}}) \\
  \denotesT{\goodTerm{\Gamma}{\LkgEmp}{\TyLkg{\LSigEmp}}} &= \unit \\
  \denotesT{\goodTerm{\Gamma}{\mu^+~m~t}{\TyLkg{\LSigAdd{\sigma}{s}{T}}}} &= (\denotesT{m}, \lam{t}) \\
  \denotesT{\goodTerm{\Gamma}{\Tkg{m}}{\TyTkg {\sigma}}} & \text{ is defined upon } \denotesT{m} \text{ and inductively on the signature length} \\
  \denotesT{\goodTerm{\Gamma}{\Tkg {m}}{\TyTkg {\sigma}}} &= () \quad \text{given } \denotesT{\goodSig{\Gamma}{\sigma}{0}} \\
  \denotesT{\goodTerm{\Gamma}{\Tkg {m}}{\TyTkg {\sigma}}} &= 
  (\denotesT{\Tkg {o}}, t[(\SubstExt{p^1}{f})][(\SubstExt{"id"}{\denotesT{\Tkg {o}}})]) \\
  & \text{given } \denotesT{\goodSig{\Gamma}{\sigma}{n+1}}, \text{ where } o = \lsigproj{1}{m}, t = \app{\snd{\denotesT{m}}}, f = p_f\nu~\sigma 
\end{align*}
% \begin{align*}
%   \denotesT{\goodInh{\Gamma}{\_}{\sigma_1}{\sigma_2}} &= \goodTerm{\Gamma, \TyLkg{\sigma_1}}{\_}{\TyLkg{\sigma_2[p^1]}} \\ \text{ and thus } \denotesT{\goodInh{\Gamma}{h}{\sigma_1}{\sigma_2}} &\iff  \goodTerm{\Gamma, \TyLkg{\sigma_1}}{h}{\TyLkg{\sigma_2[p^1]}} \\
%   \denotesT{\goodInh{\Gamma}{"inhinh"~h~T~\uparrow^s}{(\LSigAdd {\sigma_1} {s_1}{T})}{(\LSigAdd {\sigma_2} {s_2} {T[(p^1, \uparrow^s)]})}} &= \mu^+~(h[(p^1,\lkgproj{1}{"var"_0})])\\ &\quad \quad \{f_2[p^1]\}~(\lkgproj{2}{"var"_0})[(p^1, \uparrow^s[{p^1}^{\uparrow}])] \\
% \end{align*}



The main idea is that
%$\goodSig{\Gamma}{\nu^+ \_ }{n+1}$
a linkage $\goodSig{\Gamma}{\mu^+ \_}{n+1}$
is translated to a non-dependent tuple while introducing explicit universal
quantification to the second component of the tuple;
the universal quantification achieves late binding.
%These two definitions decide the syntactic translation for
%$\goodSig{\Gamma}{\_}{n}$ and $\cL$.
The translation for $\cP$ is given by the relevant $\beta$-rules.

% We use the local morphism\cite{abbott2003category} (the same one for encoding the Sealing
% judgements) to translate inheritance judgement. This is mainly due to
% the existence of \ruleref{Inh} rule. 
% Here we only typeset the overriding
% and inheritance rule, but all of the translation for inh judgment are
% induced by the $\beta$-rule of \ruleref{Inh}.

We omit validating the equational rules ($\beta$, $\eta$, and substitution) here.
Note that, when we mutually recursively define the type
${\Sigr{\Gamma}{n}{}}$ (of signatures $\denotesT{\goodSig{\Gamma}{t}{n}}$), and the function
$\denotesT{\cP} = \mathscr{P}$ above, we actually have to prove the two substitution laws
$\sub{(\TyTkg{\lsig})}{\gm}\equiv \TyTkg{\sub{\lsig}{\gm}}$ and $\sub{(\Sigr{\Delta}{n}{})}{\gm}
\equiv \Sigr{\Gamma}{n}{}$ together. 
% The details (of all the equational rules) can be found in the appendices written in (pseudo-)Agda form. We encourage the reader to read that (pseudo-)Agda-style proof because it is more friendly to type theorists, easy to do type checking and
% validate the proof and it is clearer on the mathematical object each
% translation uses.


We have constructed a model for the \TT syntax using
only the linkage-irrelevant fragment of the syntax.
%It is also
%possible to compile our $\goodWSig{\Gamma}{w}{n}$ and W-type into a more
%conventional formulation, but the recursor will always require a sigma
%type as handler and thus the elimination rule for our W-type is still
%non-conventional.
%
%This translation justifies the intuition -- a linkage is actually a
%module \textbf{with universal quantification wrapping its fields to
%achieve late binding}. 
%
We would have consistency and canonicity for \TT immediately, if we could assume
the consistency and canonicity of the linkage-irrelevant fragment.
However, because our formulation of W-types is unconventional,
we choose not to take such an assumption for granted and choose to directly prove
consistency and canonicity for \TT.


