The implementation described in \cref{sec:coqimpl} can be formalized in our metatheory. The described compilation strategy can be considered as a \textit{syntactical translation}, a function mapping from our syntactical (QIIT-)model to a subpart of our syntactical (QIIT-)model \textit{without linkages}.\footnote{Here, syntactical QIIT-model can be roughly understood as just AST, with equational rules acting like quotient. So this mapping is mapping AST to AST, respecting quotient/judgemental equality.} This subpart is the standard MLTT with $\goodWSig{}{}{}$, a (non-conventional) W-type and a recursor that using sigma type to aggregate the handlers. 

Basically, we can compile away the judgement $\goodSig{\Gamma}{v}{n}$, the type $\cL$ and $\cC$, and related terms, by transforming them into the corresponding sigma type.

 

\newcommand{\denotesT}[1]{{{\llbracket {#1} \rrbracket}_T}}
\newcommand{\Sigr}[2]{{ "Sig"^r~{#1}~{#2} }}

We use $\denotesT{}$ as the translation. This translation is invariant under the subpart. Since we are in an intrinsic setting, everything here is well-typed (well-formed)including this translation. Thus $\denotesT{T}$ means the same as $\denotesT{\goodType{\Gamma}{T}{}}$ (i.e. every type is tracking its context). In other words, every derivation of any judgement will track its ``context''. This will be clearer in the (pseudo-)Agda formalization.

\begin{align*}
  \text{We will define a new type }& \goodType{\Gamma}{\Sigr{\Gamma}{n}}{} \text{ when } \goodCtx{\Gamma}{} \text{ and } n \in \mathbb{N} \\
  \denotesT{\goodSig{\Gamma}{\_}{n}},\ &({\Sigr{\Gamma}{n}}), \text{ and } \ \denotesT{\cC} \text{ are mutually recursively defined together, } \\
  & \text{and they define inductively on the signature length} \\  
  \denotesT{\goodSig{\Gamma}{\_}{n}} &= (\goodTerm{\Gamma}{\_}{\Sigr{\Gamma}{n}}) \text{, and thus } \denotesT{\goodSig{\Gamma}{\sigma}{n}} \iff \goodTerm{\Gamma}{\sigma}{\Sigr{\Gamma}{n}} \\ 
  {\Sigr{\Gamma}{(n+1)}} &= 
    \Sigma~(\Sigr{\Gamma}{n})
          ~(\Sigma~\cU
                  ~(\Sigma~(\Pi(\cC (\pi_2[\pi_1]))(El~\pi_2[\pi_1]))~\cU)) \\
  \Sigr{\Gamma}{0} &= \top \\
  \denotesT{\goodType{\Gamma}{\cC \sigma}{}} &\text{ defines upon } \denotesT{\sigma} \\
  \denotesT{\goodType{\Gamma}{\cC \sigma}{}} &= 
    \Sigma~(\denotesT{\cC ("pjl"~\sigma)})~(El \ ("app"~("pjr"^3~\sigma))[(\pi_1, "app"~("pjl"~("pjr"^2~t)))]) \\
      &\text{given } \denotesT{\goodSig{\Gamma}{\sigma}{n+1}} \\
  \denotesT{\goodType{\Gamma}{\cC \sigma}{}} &= \top \quad
      \text{given } \denotesT{\goodSig{\Gamma}{\sigma}{0}} \\ 
  \denotesT{\goodType{\Gamma}{\cL\sigma}{}} & \text{ is defined upon } \denotesT{\sigma} \\
  & \text{ and inductively on the signature length} \\ 
  \denotesT{\goodType{\Gamma}{\cL\sigma}{}} &=
  \denotesT{\cL("pjl"~\sigma)} \times \Pi(El~("pjl"~("pjr"~\sigma)))(El~("app"~("pjr"^3~\sigma))) \\
  &\text{given } \denotesT{\goodSig{\Gamma}{\sigma}{n+1}} \\
  \denotesT{\goodType{\Gamma}{\cL\sigma}{}} &= \top \quad \text{given } \denotesT{\goodSig{\Gamma}{\sigma}{0}} \\
  \denotesT{\goodSig{\Gamma}{\nu^+~\sigma~f~T}{n+1}} &= (\sigma, "c"~A, "lam"~f, "lam"~("c"~T)) \\ 
  \denotesT{\goodTerm{\Gamma}{\mu^+~m~t}{\cL(\nu^+~\sigma~T)}} &= (m, "lam"~t) \\ 
  \denotesT{\goodTerm{\Gamma}{\cCt m}{\cC\sigma}} & \text{ is defined upon } \denotesT{m}\\
  & \text{ and inductively on the signature length} \\ 
  \denotesT{\goodTerm{\Gamma}{\cCt m}{\cC\sigma}} &= 
  (\denotesT{\cCt o}, t["sf"~f][("id", \denotesT{\cCt o})]) \\
  & \text{given } \denotesT{\goodSig{\Gamma}{\sigma}{n+1}}, \text{ where } o = "pjl"~m, t = "app"~("pjr"~m), f = p_f\nu~\sigma \\ 
  \denotesT{\goodTerm{\Gamma}{\cCt m}{\cC \sigma}} &= () \quad \text{given } \denotesT{\goodSig{\Gamma}{\sigma}{0}}
\end{align*}
\begin{align*}
  &\denotesT{\goodSig{\Gamma}{p_1\nu~\sigma}{n}} = "pjl"~\sigma 
  &&\denotesT{\goodType{\Gamma}{p_1\nu'~\sigma}{}} = El~("pjl"~("pjr"~\sigma)) \\
  &\denotesT{\goodSeal{\Gamma}{p_f\nu~\sigma}{p_1\nu~\sigma}{p_1\nu'~\sigma}} = "app"~("pjl"~("pjr"^2~\sigma)) 
  &&\denotesT{\goodSig{\Gamma}{p_2\nu~\sigma}{n}} = El~("app"~("pjr"^3~\sigma)) \\ 
  &\denotesT{\goodTerm{\Gamma}{p_1\mu~o}{\cL(p_1\nu\sigma)}} = "pjl"~o 
  &&\denotesT{\goodTerm{\Gamma, p_1\nu'~\sigma}{p_2\mu'~o}{p_2\nu~\sigma}} = "app"~("pjr"~o)
\end{align*}
\begin{align*}
  \denotesT{\goodInh{\Gamma}{\_}{\sigma_1}{\sigma_2}} &= \goodTerm{\Gamma, \cL\sigma_1}{\_}{\cL\sigma_2[\pi_1]} \\ \text{ and thus } \denotesT{\goodInh{\Gamma}{h}{\sigma_1}{\sigma_2}} &\iff  \goodTerm{\Gamma, \cL\sigma_1}{h}{\cL\sigma_2[\pi_1]} \\
  \denotesT{\goodInh{\Gamma}{"inhinh"~h~T~\uparrow^s}{(\nu^+ \  \sigma_1 \  T)}{(\nu^+ \  \sigma_2 \  T[(\pi_1, \uparrow^s)])}} &= \mu^+~(h[(\pi_1,\pi_1\mu~\pi_2)])\\ &\quad \quad \{f_2[\pi_1]\}~(p_2\mu'~\pi_2)[(\pi_1, \uparrow^s[\pi_1^{\uparrow}])] \\
  \denotesT{\goodInh{\Gamma}{"inhov"~h~t}{(\nu^+ \  \sigma_1 \  T_1)}{(\nu^+ \  \sigma_2\  T_2)}} &= \mu^+ (h[(\pi_1, p_1\mu~\pi_2)])~\{f_2[\pi_1]\}~t[\pi_1^{\uparrow}]
\end{align*}

The main idea of this syntactical translation is that, we make sure $\goodSig{\Gamma}{\nu^+ \_ }{n+1}$ and $\goodSig{\Gamma}{\mu^+ \_}{n+1}$ wrapped into a dependent pair. These two definitions decide the syntactical translation for $\goodSig{\Gamma}{\_}{n}$ and $\cL$. The syntactical translation for $\cC$ and $\cCt$ are decided by their $\beta$-rules. 

We use the local morphism (the same one for encoding the Sealing judgements) to translate inheritance judgement. This is mainly due to the existence of \ruleref{Inh} rule. Here we only typeset the overriding and inheritance rule, but all of the translation for inh judgment are induced by the $\beta$-rule of \ruleref{Inh}.

We omit all the equational rules ($\beta,\eta$ and substitution) here. However, for coherence, when we mutual recursively define $\denotesT{\goodSig{\Gamma}{\_}{n}}$, ${\Sigr{\Gamma}{n}} $, and $\denotesT{\cC}$ above, we actually have to prove the substitution laws $(\cC\sigma)[\tau]\equiv \cC(\sigma[\tau])$ and $\Sigr{\Delta}{n}[\tau] \equiv \Sigr{\Gamma}{n}$ together. The details (of all the equational rules) can be found in the appendices written in (pseudo-)Agda form. We encourage the reader to read that (pseudo-)Agda-style proof because it is more friendly to type theorists, easy to do type checking and validate the proof and it is clearer on the mathematical object each translation uses.


Ultimately, we have constructed a model for our QIIT-syntax using only subparts of our QIIT-syntax -- without these linkages. This also justify the intuition -- a linkage is actually a module \textit{with universal quantification wrapping its fields to achieve late binding}. 

It is also possible to compile away $\goodWSig{\Gamma}{w}{n}$ but the recursor will always require a sigma type as handler and thus the elimination rule for our W-type is still non-conventional.

After syntactical translation, we can directly have consistency and canonicity \textit{if we assume the consistency and canonicity of the subpart}. However, due to our non-conventional formulation of the inductive type, it is better we don't impose such assumption and directly give a consistency and canonicity proof.

% the link for syntactical translation
% https://drive.google.com/file/d/1pMqn8DS4T4jiCubzk3HgMANjrfKpcGlI/view?usp=sharing
