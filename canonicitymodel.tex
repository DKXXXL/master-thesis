Now we prove canonicity (and consistency) for \TT using a logical-relations model.
We follow the reducibility argument of
\citet{kaposi2019gluing}, \citet{coquand2018canonicity}, and \citet{sterling2019algebraic}
to construct our model.
The meta\-languages of these prior models are based on QIITs, categories with
families \cite{dybjer1995internal}, and the generalized algebraic theory
\cite{cartmell1986generalised}, respectively.
Without exposing the reader to too many technical details, our meta\-language
should be understood as an instance of any of the above logical frameworks---the
difference is that quotienting is manual in our formulation, whereas it is
automatic with the logical frameworks.
%Our model works directly on the earlier introduced syntax for readability\footnote{We still
%take quotient, so ultimately our syntax should be considered as an instance of
%any of the above logical framework, without exposing the reader too many
%technical details}.

We state the canonicity theorem first:

\begin{theorem}[Canonicity]
\label{thm:canonicity-appendix}
  If $\goodTerm{\cdot}{t}{\cB}$, then either $\goodTerm{\cdot}{t \equiv \true}{\cB}$ or $\goodTerm{\cdot}{t \equiv \false}{\cB}$.
\end{theorem}

\noindent
Canonicity is a key criterion for a dependent type theory to be considered as
a programming language or as a computational foundation for mathematics.
%\footnote{See nlab explanation: \href{https://ncatlab.org/nlab/show/canonical+form}{Canonical Form}}
% We can even argue that, if this theorem is proven in a computable meta\-logic, 
% then by the Curry–Howard correspondence, the canonicity theorem provides a big-step
% interpreter for closed terms of the boolean type.
%\YZ{Can we say something about
%strong/weak normalization?}\EDJreply{
%  No. \\
%  1. In the declarative system, we cannot talk about strong/weak normalization (because we don't have operational semantic, if you recall the definition of strong/weak normalization, it based on reduction/path of steps/op semantic). So any review says strong/weak normalization to you, please stop them. \\ 
%  2. In declarative system, we only say ``normalization'' directly, which means, we construct a function that \\
%      for arbitrary class of (judgemental-equivalent) terms, it will return one syntax representation (called normal-form) for each class \\
%    (apparently most people will make the normal form  without beta redex, however, interestingly, they may do eta expansion instead of eta reduction because theoretically easier) \\ 
%  3. So translation won't work because it compiles away the normal form of linkage (the $\mu^+$). This part requires a direct proof which construct normal form and neutral form for our calculus, and a normalization model from which we extract a runnable interpreter for open terms \\
%  4. Of course we can conjecture we have that normalization function \\
%  5. Also we do have close term normalization, which is canonicity. Or more operationally speaking, we have termination (a runnable interpreter) for close term (since QIIT is runnable, the canoncity model itself is the runnable interpreter), \\
%  but we don't have termination (a runnable interpreter) for open terms.  \\ 
%  So earlier you say ``normalize'' a recursion on an inductive type, that is no doable -- if the programmer wants to program on meta-theory, they will have to manually write out the a chain of judgemental equality (with a bunch of beta rules inside) by themselves. 
%}

First, we need the mathematical setup to interpret universe levels, following \citet{sterling2019algebraic}:

\newcommand{\Set}[1]{\ensuremath{\texttt{Set}_{#1}}}
\newtheorem{assumption}{Assumption}[section]

\begin{assumption}[Set-theoretic Universe Assumption] We assume an infinite hierarchy of Grothendieck universes $\Set{i}$ for $i \in \mathbb{N}$ in our ambient meta\-logic.
\end{assumption}

We can roughly consider each Grothendieck universe $\Set{i}$ as the $\Set{i}$ in Agda: 
\begin{itemize}
  \item Each $\Set{i}$ is closed under dependent function types and dependent
  pair types. For example, later, for our interpretation of dependent function
  types, when we have $\denotesC{A}, \denotesC{B} \in \Set{i}$, we will have
  $\denotesC{\TyPi{A}{B}} \in \Set{i}$.
  \item The universe hierarchy is cumulative, as $\Set{i} \in \Set{i+1}$ and $\Set{i} \subseteq \Set{i+1}$.
  \item Thus, if $\denotesC{A} \in \Set{i}, \denotesC{B} \in \Set{j}$, we will
  have $\denotesC{\TyPi{A}{B}} \in \Set{i} \cup \Set{j} = \Set{i \lcup j}$.
\end{itemize}


Like most logical-relations proofs, we interpret each judgment and inductively
interpret each syntax piece. We are working in an intrinsic setting; thus, even if
we omit contexts for brevity, the syntax piece is still well-typed. However,
unlike most logical-relations proofs, our logical-relations model is proof-relevant,
which is essential for modeling universes
properly~\cite{coquand2018canonicity}.
Our canonicity model for the base MLTT fragment is close to the ones
in \citet{coquand2018canonicity} and \citet{sterling2019algebraic} in that it
utilizes the facilitates of the ambient meta\-logic; for example, we use dependent functions and
dependent tuples in the ambient meta\-logic to model dependent functions and tuples in \TT.
% We will omit the treatment of the universe and the verification of most equality for simplicity (and the treatment of the universe will not affect our model). Interested audience can refer to \citet{kaposi2019gluing, coquand2018canonicity,sterling2019algebraic} for a complete technical treatements.

\begin{align*}
  \denotesC{\goodCtx{\Gm}{k}} \text{ is a function }&: \setc{\gm}{\goodSub{\cdot}{\gm}{\Gm}} \to \Set{k} \quad \text{ (i.e., sets indexed by closed substitution)}\\
  % &\text{and we define } \goodCtxS{\Gm}{i} :\iff \denotesC{\goodCtx{\Gm}{i}} \text{ is defined} \\
  \denotesC{\goodType{\Gm}{T}{j}} \text{ is a dependent function }&: \prod_{\goodSub{\cdot}{\gm}{\Gm}}\prod_{\gm' \in \denotesC{\goodCtx{\Gm}{}}(\gm)}\setc{t}{\goodTerm{\cdot}{t}{\sub{T}{\gm}}}\to\Set{j} \\
  % &\text{and we define } \goodTypeS{\Gm}{T}{i} : \iff \denotesC{\goodType{\Gm}{T}{i}} \text{ is defined } \\ 
  \denotesC{\goodSub{\Gm}{\delta}{\Delta}} \text{ is a dependent function }&: \prod_{\goodSub{\cdot}{\gm}{\Gm}} \prod_{\gm' \in \denotesC{\goodCtx{\Gm}{}}(\gm)} \denotesC{\goodCtx{\Dl}{}}(\delta \circ \gm) \\
  % &\text{and we define } \goodSubC{\Gm}{\gm}{\Delta} :\iff \denotesC{\goodSub{\Gm}{\gm}{\Delta}} \text{ is defined} \\
  \denotesC{\goodTerm{\Gm}{t}{T}} \text{ is a dependent function}&: \prod_{\goodSub{\cdot}{\gm}{\Gm}} \prod_{\gm' \in \denotesC{\goodCtx{\Gm}{}}(\gm)}\denotesC{\goodType{\Gm}{T}{}}(\gm)(\gm')(\sub{t}{\gm})\\
  \denotesC{\goodType{\Gm}{\sub{T}{\sigma}}{}}(\gm)(\gm')(t) &= \denotesC{T}(\sigma \circ \gm)(\denotes{\sigma}(\gm)(\gm'))(t)\\
  \denotesC{\goodType{\Gm}{\top}{}}(\gm)(\gm')(t) &= \{\star \} \quad \text{  a singleton set } \\
  \denotesC{\goodType{\Gm}{\bot}{}}(\gm)(\gm')(t) &= \emptyset \\
  \denotesC{\goodType{\Gm}{\cB}{}}(\gm)(\gm')(t) &= 
    %\YZ{Is this more accurate than 'disjoint union'?}\EDJreply{It is accurate. It is actually the way Coquand write it. But I find this way weird type-theoretically---at the very end, when we want to prove a closed term t is either true or false, we will use this cB. But defining in this Coquand way, type-theoretically, we cannot extract a proof term t = true. (i.e. there is no proof term stored here!). I know my above sentence is a bit awkward but I just find this Coquand way of saying very...counter-intuitive.}
    \begin{cases}
      \{\star^1\} & \text{if } t \equiv \true \\
      \{\star^2\} & \text{if } t \equiv \false \\
      \emptyset & \text{otherwise }
    \end{cases}\\
  \denotesC{\goodType{\Gm}{\TyId{a}{b}}{}}(\gm)(\gm')(t) &= 
  \begin{cases}
    \{\star\} & \text{if } t \equiv \eqrefl{\sub{a}{\gm}} \text{ and } \sub{a}{\gm} \equiv \sub{b}{\gm}\\
    \emptyset & \text{otherwise }
  \end{cases}\\
  \denotesC{\goodTerm{\Gm}{\Jeq{w}{t}}{\sub{C}{\SubstExt{\SubstExt{id}{v}}{t} }}}(\gm)(\gm') &= \denotesC{w}(\gm)(\gm') \quad \quad \text{ given } \goodTerm{\Gm}{t}{\TyId{u}{v}} \\ 
  & \text{Since $\denotesC{t}(\gm)(\gm')$ witnesses } \sub{t}{\gm} \equiv \eqrefl{\sub{u}{\gm}} \text{ and } \sub{u}{\gm} \equiv \sub{v}{\gm} \\
  \denotesC{\goodType{\Gm}{\TyPi{A}{B}}{}}(\gm)(\gm')(t) &= \prod_{\goodTerm{\cdot}{u}{\sub{A}{\gm}}} \ \ \prod_{u' \in \denotesC{A}(\gm)(\gm')(u)} \denotesC{B}(\SubstExt{\gm}{u})((\gm', u'))(\sub{\app{t}}{\SubstExt{"id"}{u}})\\
  \denotesC{\goodType{\Gm}{\TySigma{A}{B}}{}}(\gm)(\gm')(t) &= \sum_{u' \in \denotesC{A}(\gm)(\gm')(\fst{t})} \denotesC{B}(\SubstExt{\gm}{\fst{t}})((\gm', u'))(\snd{t}) \\
  \denotesC{\goodCtx{\cdot}{}}(\gm) &= \{\star\}  \\ 
  \denotesC{\goodCtx{\Gm, T}{}}(\gm_t) &= \setc{ (\gamma' , t') }{ \gm' \in \denotesC{\goodCtx{\Gm}{}}(\pi_1 \gm_t), t' \in \denotesC{\goodType{\cdot}{\sub{T}{\gm}}{}}(\pi_1 \gm_t)(\gm')(\pi_2 \gm_t)  } \\
  \denotesC{\goodSub{\Gm}{\pi_1 \delta}{\Dl}}(\gm)(\gm') &= \denotesC{\delta}(\gm)(\gm')[0] \quad \text{ get the first element of the tuple} \\
  \denotesC{\goodTerm{\Gm}{\pi_2 \delta}{\Dl}}(\gm)(\gm') &= \denotesC{\delta}(\gm)(\gm')[1] \quad \text{ get the second element of the tuple} \\
  \denotesC{\goodSub{\Gm}{\SubstExt{\delta}{t}}{\Dl}}(\gm)(\gm') &= (\denotesC{\delta}(\gm)(\gm'),\denotesC{t}(\gm)(\gm')) \\ 
  \denotesC{\goodSub{\Gm}{"id"}{\Gm}}(\gm)(\gm') &= \gm' \\
  \denotesC{\goodSub{\Gm}{\delta_1 \circ \delta_2}{\Dl}}(\gm)(\gm') &= \denotesC{\delta_1}(\SubstComp{\delta_2}{\gm})(\denotesC{\delta_2}(\gm)(\gm')) \\
  \denotesC{\goodTerm{\Gm}{\sub{t}{\sigma}}{\sub{T}{\sigma}}}(\gm)(\gm') &= \denotesC{t}(\sigma \circ \gm)(\denotesC{\sigma}(\gm)(\gm')) \\
  \denotesC{\goodTerm{\Gm}{\unit}{\top}}(\gm)(\gm') &= \star\\
  \denotesC{\goodTerm{\Gm}{\true}{\cB}}(\gm)(\gm') &= \star^1 \\
  \denotesC{\goodTerm{\Gm}{\false}{\cB}}(\gm)(\gm') &= \star^2 \\
  \denotesC{\goodTerm{\Gm}{\ifb{c}{a}{b}}{T}}(\gm)(\gm') &= 
  \begin{cases}
    \denotesC{a}(\gm)(\gm') \quad \text{ if } \denotes{c}(\gm)(\gm') = \star^1 \\
    \denotesC{b}(\gm)(\gm') \quad \text{ if } \denotes{c}(\gm)(\gm') = \star^2
  \end{cases}\\
  % \denotesC{a}(\gm)(\gm') \quad \text{ if } \denotes{c}(\gm)(\gm') = \star^1 \\
  % \denotesC{\goodTerm{\Gm}{\ifb{c}{a}{b}}{T}}(\gm)(\gm') &= \denotesC{b}(\gm)(\gm') \quad \text{ if } \denotes{c}(\gm)(\gm') = \star^2 \\
  \denotesC{\goodTerm{\Gm}{\eqrefl{t}}{\TyId{t}{t}}}(\gm)(\gm') &= \star\\
  \denotesC{\goodTerm{\Gm}{\lam{t}}{\TyPi{A}{B}}}(\gm)(\gm') &= \lambda u \lambda u' . \denotesC{t}(\SubstExt{\gm}{u}) (\gm' , u') \\
  \denotesC{\goodTerm{\Gm}{\app{t}}{B}}(\gm)(\gm') &= \denotesC{t}(\pi_1 \gm)(\gm'[0])(\pi_2 \gm)(\gm'[1]) \\
  \denotesC{\goodTerm{\Gm}{\pair{a}{b}}{\TySigma{A}{B}}}(\gm)(\gm') &= (\denotesC{a}(\gm)(\gm') , \denotesC{b}(\gm)(\gm') )\\
  \denotesC{\goodTerm{\Gm}{\fst{t}}{T}}(\gm)(\gm') &= \denotesC{t}(\gm)(\gm')[0]  \quad \text{ extract the first element in the tuple }\\
  \denotesC{\goodTerm{\Gm}{\snd{t}}{T}}(\gm)(\gm') &= \denotesC{t}(\gm)(\gm')[1]\\
  \denotesC{\goodType{\Gm}{\cU_j}{j+1}}(\gm)(\gm')(T) &= \setc{t} { \goodTerm{\cdot}{t}{\El{T}} }  \to \Set{j} \\
  \denotesC{\goodTerm{\Gm}{\codety{T}}{\cU_j}}(\gm)(\gm') &= \denotesC{T}(\gm)(\gm') \\ 
  \denotesC{\goodType{\Gm}{\El{T}}{j}}(\gm)(\gm')(t) &= \denotesC{T}(\gm)(\gm')(t) \\
\end{align*}

\newcommand{\Glued}[1]{\ensuremath{{#1}^\bullet}}
\newcommand{\GluedPi}[2]{\ensuremath{\Pi^\bullet({#1},{#2})}}
\newcommand{\GSubstExt}[2]{\ensuremath{{#1},^\bullet{#2}}}
\newcommand{\Gpair}[2]{\ensuremath{({#1},^\bullet{#2})}}
\newcommand{\Gfst}[1]{\ensuremath{\texttt{fst}^\bullet~{#1}}}
\newcommand{\Gsnd}[1]{\ensuremath{\texttt{snd}^\bullet~{#1}}}
\newcommand{\Gsub}[2]{\ensuremath{{#1}\!\left[{#2}\right]^\bullet}}
\newcommand{\Glam}[1]{\ensuremath{\lambda^\bullet({#1})}}
\newcommand{\Gapp}[1]{\ensuremath{\texttt{app}^\bullet({#1})}}
\newcommand{\GSubstWeak}[1]{\ensuremath{(\texttt{p}^\bullet)^{#1}}}
\newcommand{\GLSigAdd}[3]{\ensuremath{\nu^{+\bullet}({#1},{#2},{#3})}}
\newcommand{\GCaseSig}[3]{\ensuremath{\texttt{CaseTy}^\bullet({#1},{#2},{#3})}}
\newcommand{\Gmodel}[1]{\ensuremath{{({#1})}^c}}
\newcommand{\GEl}[1]{\ensuremath{\Glued{\texttt{El}}({#1})}}
\newcommand{\Gwcode}[1]{\ensuremath{\Glued{\texttt{W}}({#1})}}
\newcommand{\GSubstId}{\ensuremath{\Glued{"id"}}}
\newcommand{\Gwsigproj}[3]{\ensuremath{\Glued{{\texttt{w}\pi^{#1}_{\texttt{#2}}}}({#3})}}
\newcommand{\GRecproj}[2]{\ensuremath{\Glued{{\texttt{R}\pi^{#1}}}({#2})}}



% What is perhaps unintuitive about the proof-relevant logical-relations model is the
% treatment of propositions.
%The model treats propositions in a special way : \EDJ{Without this first sentence, this paragraph seems unmotivated.}\YZreply{Why is it 'difficult'? Is there a more appropriate adjective than 'difficult'?}\EDJreply{Don't you find difficult? At first sight, I cannot understand why this sentence, proposed by Sterling in two of his paper, can be valid mathematically. It only makes sense to me until I am in a type-theoretic world (by Kaposi, since Eq is an inductive type now) (or maybe topos-theoretically where categorical logic will encode equality using data like equalizer). What about changing the phrase to "eccentric" or "weird"?}
% A proposition is encoded as a sub\-singleton. For
% example, $\setc{\star^1}{\text{ when } t \equiv \true} \uplus \setc{\star^2}{\text{ when } t \equiv \false}$ represents a disjoint union of two
% sub\-singleton sets, where the first one is either a singleton if we have a proof of
% $t \equiv \true $ or an empty set otherwise.
% In other words,
% $\star^1$ witnesses a proof that $t \equiv \true$.
Here, $\star$, $\star^1$, $\star^2$ are just some arbitrary fixed elements.  


Given the above model for MLTT, there should be a function ${\Pi}^c$ such
that ${\Pi}^c({\denotesC{A}},{\denotesC{B}}) = \denotesC{\TyPi{A}{B}}$. 
Type-theoretically speaking, this ${\Pi}^c$ uses the \emph{internal dependent
function type} of
the above model. We hope to
use this function when defining the logical-relations model for the rest of \TT.
However, such a function $\Pi^c$ is not yet possible because the definition
$\denotesC{\TyPi{A}{B}}$ is not based solely on $\denotesC{A}$ and
$\denotesC{B}$,
but also on the syntax $\goodType{\Gm}{A}{}$ and $\goodType{\Gm, A}{B}{}$. 

Thus, we define a new denotation $\denotesCC{S} \coloneq (S, \denotesC{S})$
that also returns the syntax piece $S$.\footnote{This is also called
glued interpretation in \citet{sterling2019algebraic}.}
Thus, we can have a
function $\Glued{\Pi}$ such that $\GluedPi{\denotesCC{A}}{\denotesCC{B}} =
\denotesCC{\TyPi{A}{B}}$ now that the syntax is available.
Similarly, there are functions $\Glued{\Sigma}$, $\Gpair{a}{b}$, and
$\GSubstExt{\gm}{t}$ for dependent pair types, dependent pairs, and
substitution extension (and more for other constructions).
Furthermore, given $\Glued{S}$ (i.e., the syntax and its semantic interpretation),
we use $\Gmodel{\Glued{S}}$ to mean the latter of the two.%\YZ{why not just $\denotesC{S}$?}\EDJreply{Using superscript C is shorter and more precise -- we are not "reconstructing the C model from the syntax, but taken out the C model directly from the pair". Look at the canonicity model where superscript c is used, if change to $\denotesC{-}$ on them, it is not really well-founded recursion, technically speaking (only equivalent to one).}
% These will be useful in our \TT model because as syntactic translation hinted, \TT reuses a lot from MLTT. What's more, during the construction of the model for \TT, we will use $\Glued{(\cdot)}$ when we can.

\newcommand{\GTy}[2]{\ensuremath{\Glued{\texttt{Ty}}_{#1}~{#2}}}
\newcommand{\GCon}[1]{\ensuremath{\Glued{\texttt{Con}}_{#1}}}
\newcommand{\GTm}[2]{\ensuremath{\Glued{\texttt{Tm}}~{#1}~{#2}}}
\newcommand{\GSub}[2]{\ensuremath{\Glued{\texttt{Sub}}~{#1}~{#2}}}
\newcommand{\GMWSig}[3]{\ensuremath{{\texttt{WSig}^C}_{#1}^{#3}~{#2}}}
\newcommand{\GWSig}[3]{\ensuremath{{\Glued{\texttt{WSig}}}_{#1}^{#3}~{#2}}}


We need more \emph{internal} type-theoretic constructions:

We define $\GCon{k} \coloneq \sum_{\goodCtx{\Gm}{k}} \{\gm : \goodSub{\cdot}{\gm}{\Gm} \} \to \Set{k}$ 


and $\GTy{j}{\Glued{\Gm}} \coloneq \sum_{\goodType{\Gm}{T}{j}} \prod_{\goodSub{\cdot}{\gm}{\Gm}}\prod_{\gm' \in \Gmodel{\Glued{\Gm}}(\gm)}\{t : \goodTerm{\cdot}{t}{\sub{T}{\gm}}\}\to\Set{j} $ for $\Glued{\Gm} \in \GCon{k}$

and $\GTm{\Glued{\Gm}}{\Glued{T}} \coloneq \sum_{\goodTerm{\Gm}{t}{T}} \prod_{\goodSub{\cdot}{\gm}{\Gm}} \prod_{\gm' \in\Gmodel{\Glued{\Gm}}} \Gmodel{\Glued{T}}(\gm)(\gm')(\sub{t}{\gm})$

and $\GSub{\Glued{\Gm}}{\Glued{\Dl}} \coloneq \sum_{\goodSub{\Gm}{\delta}{\Dl}} \prod_{\goodSub{\cdot}{\gm}{\Gm}} \prod_{\gm' \in \Gmodel{\Glued{\Gm}}(\gm)} \Gmodel{\Glued{\Dl}}(\delta \circ \gm) $

These four sets are collecting the glued interpretation.
For each well-formed type $\goodType{\Gm}{T}{j}$, we have its denotation $\denotesCC{T} \in \GTy{j}{\denotesCC{\Gm}}$;
for each well-typed term $\goodTerm{\Gm}{t}{T}$, we have its denotation $\denotesCC{t} \in \GTm{\denotesCC{\Gm}}{\denotesCC{T}}$;
etc.


Notice that
$\prod_{\goodSub{\cdot}{\gm}{\Gm}} \prod_{\gm' \in \Gmodel{\Glued{\Gm}}(\gm)}$
is part of $\GTy{\!}{\!}$, $\GTm{\!}{\!}$, and $\GSub{\!}{\!}$.
A useful fact about them is : given a pair of arbitrary $\goodSub{\cdot}{\gm}{\Gm}$
and $\gm' \in \Gmodel{\Glued{\Gm}}(\gm)$, we can consider $(\gm, \gm')$ as an
element of $\GSub{\Glued{\cdot}}{\Glued{\Gm}}$, and vice versa.
Thus, we consider the pair $(\gm,
\gm')$ the equivalent form of an element $\Glued{\gm} \in \GSub{\Glued{\cdot}}{\Glued{\Gm}}$.


Then we define $\GMWSig{j}{\Glued{\Gm}}{n} \coloneq "Vector"^n~\sum_{\Glued{A}
\in \GTy{j}{\Glued{\Gm}}} \GTy{j}{(\GSubstExt{\Glued{\Gm}}{\Glued{A}})}$ for
$\Glued{\Gm} \in \GCon{k}$, a length-$n$ list of pairs of types.
As before, we can define glued interpretation $\GWSig{j}{\Glued{\Gm}}{n}
\coloneq \{ \wsig : \goodWSig{\cdot}{\wsig}{n} \} \times
\GMWSig{j}{\Glued{\Gm}}{n}$. This will be useful when we interpret judgments for
W-type signatures.


Now we can extend the model above to include \TT constructs.
%We still start with the judgments.
%Since linkage transformers are just syntactic sugar, we will not interpret it.
The key idea of this extension is similar to that of the syntactic translation:
we interpret linkage types using the canonicity model of sigma types (developed in the prior work).
We use the inductive facility of the ambient meta\-logic to justify our W-types.

\newcommand{\CWmodel}{\ensuremath{\mathit{W}^C}}
\newcommand{\CWsup}{\ensuremath{\mathit{W^Csup}}}
\newcommand{\CWrec}{\ensuremath{\mathit{W^Crec}}}

\begin{align*}
  \denotesC{\goodSig{\Gm}{\lsig}{n}} & \text{ is a list of 3-tuple of length $n$  } \\
  % & \text{ and we define  } {\goodSigC{\Gm}{\lsig}{n}} :\iff \denotesC{\goodSig{\Gm}{\lsig}{n}} \text{ is defined} \\ 
  \denotesC{\goodWSigU{j}{\Gm}{\wsig}{n}} &: \GMWSig{j}{\denotesCC{\Gm}}{n} \\ 
    &\text{ i.e., a list of 2-tuple of length $n$ } \\
  % & \text{ and we define } \goodWSigC{\Gm}{\wsig}{n} :\iff \denotesC{\goodWSig{\Gm}{\tau}{n}} \text{ is defined} \\
  \text{we define }\TyLkg{}, \TyTkg{} & \text{ inductively on its input signature, by defining } \mathit{L},\Glued{\mathit{P}} \text{ first}  \\ 
  \mathit{L}("nil") &=  \denotesC{\top}  \\
  \mathit{L}((A,s,T) :: tl)(\gm)(\gm')(t) &= {\mathit{L}(tl)}(\gm)(\gm')(\lkgproj{1}{t}) \times \Gmodel{\GluedPi{A}{T}}(\gm)(\gm')(\lam{\lkgproj{2}{t}})   \\
  \Glued{\mathit{P}}("nil") &= ( \TyTkg{\LSigEmp} , \denotesC{\top} )\\
  \Glued{\mathit{P}}((A,s,T) :: tl) &= \Glued{\Sigma}(\Glued{\mathit{P}}(tl), \Gsub{T}{\GSubstExt{\Glued{\pi_1}}{s}}) \quad \text{(doing substitution on $T$)} \\
  \denotesC{\goodType{\Gm}{\TyLkg{\lsig}}{}} &= {\mathit{L}(\denotesC{\lsig})} \\
  \denotesC{\goodType{\Gm}{\TyTkg{\lsig}}{}} &= \Gmodel{\Glued{\mathit{P}}(\denotesC{\lsig})} \quad \text{ discard syntax info}  \\
  % 
  \denotesC{\goodSig{\Gm}{\LSigEmp}{0}} &= "nil"\\ 
  \denotesC{\goodSig{\Gm}{\LSigAdd{\lsig}{s}{T}}{n+1}} &= (\denotesCC{A}, \denotesCC{s}, \denotesCC{T})::\denotesC{\lsig} \quad \text{ given } \goodTerm{\Gm, \TyTkg{\lsig}}{s}{A} \\ 
  \denotesC{\goodSig{\Gm}{\sub{\lsig}{\gm}}{n}} & \text{ is done by point-wise/component-wise substitution} \\
  \denotesC{\goodSig{\Gm}{\lsigproj{1}{\lsig}}{n}} &= "tl"~\denotesC{\lsig}\\ 
  \denotesC{\goodType{\Gm}{\lsigprojT{\lsig}}{}} &= \Gmodel{("hd"~\denotesC{\lsig})[0]} \quad \text{ take the first element in the tuple,...}\\ 
  \denotesC{\goodTerm{\Gm, \TyTkg{\lsigproj{1}{\lsig}}}{\lsigproj{s}{\lsig}}{\lsigprojT{\lsig}}} &= \Gmodel{("hd"~\denotesC{\lsig})[1]} \quad \text{  ... and discard syntax}\\ 
  \denotesC{\goodType{\Gm, \lsigprojT{\lsig}}{\lsigproj{2}{\lsig}}{}} &= \Gmodel{("hd"~\denotesC{\lsig})[2]}\\
  \denotesC{\goodTerm{\Gm}{\LkgEmp}{\TyLkg{\LSigEmp}}} &= \denotesC{\unit} \\
  \denotesC{{\LkgAdd{\lkg}{t}}}(\gm)(\gm') &= (\denotesC{\lkg}(\gm)(\gm') , \Gmodel{\Glam{\denotesCC{t}}}(\gm)(\gm')) \\
  \denotesC{\goodTerm{\Gm}{\Tkg{\lkg}}{\TyTkg{\lsig}}} &= \denotesC{\unit}
  \quad \text{ when \goodSig{\Gm}{\lsig}{0}} \\
  \denotesC{\goodTerm{\Gm}{\Tkg{\lkg}}{\TyTkg{\lsig}}} &= \Gmodel{\Gpair{\Glued{\ell'}}{\Gsub{\Gsub{\denotesC{\lkgproj{2}{\lkg}}}{\GSubstExt{\Glued{\GSubstWeak{1}}}{s}}}{\GSubstExt{\Glued{("id")}}{ \Glued{\ell'} }}}}
  \\ &\text{ where } \Glued{\ell'} = \denotesCC{\Tkg{\lkgproj{1}{\lkg}}} 
  \\ &\text{ when } \denotesC{\goodSig{\Gm}{\lsig}{n+1}} = (A, s, T)::\_ \\
  \denotesC{\goodTerm{\Gm}{\lkgproj{1}{\lkg}}{\TyLkg{\lsig}}}(\gm)(\gm') &= \denotesC{\ell}(\gm)(\gm')[0] \quad \text{ take the first element in the tuple} \\
  \denotesC{\goodTerm{\Gm, \lsigprojT{\lsig}}{\lkgproj{2}{\lkg}}{T}}(\gm_+)(\gm_+') &= ((\denotesC{\ell}(\pi_1{\gm_+})(\gm_+'[0]))[1])(\pi_2{\gm_+})(\gm_+'[1]) \\
  \denotesC{\goodWSig{\Gm}{\WSigEmp}{0}} &= "nil" \\
  \denotesC{\goodWSig{\Gm}{\WSigAdd{\wsig}{A}{B}}{n+1}} &= (\denotesCC{A} , \denotesCC{B}) :: \denotesC{\wsig}\\
  \denotesC{\goodWSig{\Gm}{\sub{\wsig}{\gm}}{n}} & \text{ is done by point-wise/component-wise substitution} \\
  \denotesC{\goodType{\Gm}{\wsigproj{j}{1}{\wsig}}{}} &= \Gmodel{(\text{ $j$-th element of }~\denotesC{\wsig})[0]}\\
  \denotesC{\goodType{\Gm, \wsigproj{j}{1}{\wsig}}{\wsigproj{j}{2}{\wsig}}{}} &= \Gmodel{(\text{ $j$-th element of }~\denotesC{\wsig})[1]}\\
  \denotesC{\goodWSig{\Gm}{\WSigSub{\wsig}}{n}} &= "tl"~\denotesC{\wsig}\\
  \denotesC{\goodTerm{\Gm}{\wcode{\wsig}}{\cU}}(\gm)(\gm')(t) &= \CWmodel~(\Gsub{\denotesCC{\wsig}}{\Glued{\gm}})~t \\
  \denotesC{\goodTerm{\Gm}{\Wsup{i}{\wsig}{a}{b}}{\El{\wcode{\wsig}}}}(\gm)(\gm') &= \CWsup~i~(\Gsub{\denotesCC{a}}{\Glued{\gm}})~(\Gsub{\denotesCC{b}}{{\Glued{\gm^\uparrow}}}) \\
  % CsTy~A~B~R &= {\GluedPi{A}{\GluedPi {\GluedPi{B}{\Gsub{R}{\GSubstWeak{2}}}} {\Gsub{R}{\GSubstWeak{2}}}}} \\
  \denotesC{\goodType{\Gm}{\CaseSig{A}{B}{R}}{}} &= 
  \Gmodel{\GluedPi{\denotesC{A}}{\GluedPi {\GluedPi{\denotesC{B}}{\Gsub{\denotesC{R}}{\GSubstWeak{2}}}} {\Gsub{\denotesC{R}}{\GSubstWeak{2}}}}} \\
  % CsTy~\denotesC{A}~\denotesC{B}~\denotesC{R}\\
  \text{we define } \RecSig{}{} &\text{ by induction on the signature} \\
  RS~"nil"~R &= \denotesCC{\LSigEmp} \\
  RS~((A, B) :: tl)~R &= \GLSigAdd{RS~tl~R}{\Glued{\pi_2}}{\GCaseSig{A}{B}{R}} \\
  \denotesC{\goodType{\Gm}{\RecSig{\wsig}{"R"}}{}} &= \Gmodel{RS~\denotesC{\tau}~\denotesC{"R"}} \\
  \denotesC{\Recproj{j}{\lkg}} &= \text{ take the $j$-th field from } \lkg 
\end{align*}
\begin{align*}
  \denotesC{\goodTerm{\Gm}{\Wrec{\wsig}{\lkg}{t}}{T}}(\gm)(\gm') = \CWrec \quad &\Gsub{\denotesCC{\wsig}}{\Glued{\gm}} \\ 
  & (\lambda w. \denotesC{"R"}(\gm)(\gm')(\Wrec{\sub{\wsig}{\gm}}{\sub{\lkg}{\gm}}{w}))\\
  &f^r\\ &\sub{t}{\gm}\\ 
  &(\denotesC{t}(\gm)(\gm')) 
\end{align*}
\begin{align*}  
  \text{ where } \quad f^r~j~\Glued{a}~b~b^C &= 
  \texttt{let}~\Glued{\rho}  \in \GTm{(\Glued{\cdot} \Glued{,} \Gsub{\Glued{B}}{\GSubstExt{\Glued{"id"}}{\Glued{a}}})}{(\Gsub{\Gsub{\Glued{R}}{\Glued{\gm}}}{\GSubstWeak{1}})} 
  \\
  &\texttt{ s.t.}~\Glued{\rho} \coloneq (\Wrec{\wsig}{\sub{\lkg}{\SubstComp{\gm}{\SubstWeak{1}}}}{b},b^C)~\texttt{ in } \\
  & \Gmodel{\Gsub{\Gapp{\Gsub{\Gapp{\GRecproj{j}{\sub{\denotesCC{\lkg}}{\Glued{\gm}}}}}{\GSubstExt{\GSubstId}{\Glued{a}}}}}{\GSubstExt{\GSubstId}{\Glam{\Glued{\rho}}}}}(\epsilon)(\star) \\
  & \quad 
   \text{ given } \goodTerm{\Gm}{\lkg}{\RecSig{\wsig}{"R"}} \\
   & \quad \text{ where }\Glued{R} = \denotesCC{"R"}, \Glued{B} = \Gwsigproj{j}{2}{\Glued{\wsig}}
\end{align*}

The semantic interpretations above are defined together with the following inductively defined indexed set $\CWmodel$ (with only one constructor $\CWsup$)
%
\begin{align*}
  "Inductive"~\CWmodel &: (\Glued{\wsig} \in \GWSig{i}{\denotesCC{\cdot}}{N}) \to \{ t : \goodTerm{\cdot}{t}{\El{\wcode{\wsig}}} \} \to \Set{i + 1} \quad "where" \\
    \CWsup &: j < N \to \Glued{a} \in \GTm{\Glued{\cdot}}{\wsigproj{j}{1}{\Glued{\wsig}}} \\
    & \to \Glued{b} \in (\GTm{(\Glued{\cdot} \Glued{,} \Gsub{\wsigproj{j}{2}{\Glued{\wsig}}}{\SubstExt{\GSubstId}{\Glued{a}}}) }{\Gsub{\GEl{\Gwcode{\Glued{\wsig}}}}{\GSubstWeak{1}}}) \to \CWmodel~\Glued{\wsig}~\Wsup{j}{\wsig}{a}{b}
\end{align*}
and its eliminator $\CWrec$
\begin{align*}
  \CWrec \  &: (\Glued{\wsig} \in \GWSig{i}{\denotesCC{\cdot}}{N}) \to 
    (P : \{ t : \goodTerm{\cdot}{t}{\El{\wcode{\wsig}}} \} \to \Set{k})  \\
    & \to 
    [j < N \to  \Glued{a} \in \GTm{\Glued{\cdot}}{\Gwsigproj{j}{1}{\Glued{\wsig}}} \\ 
    & \quad \quad \to 
    \goodTerm{({\cdot} {,} \sub{\wsigproj{j}{2}{{\wsig}}}{\SubstExt{"id"}{{a}}}) }{b}{\sub{\El{\wcode{\wsig}}}{\SubstWeak{1}}} \\ 
    & \quad \quad \to b^C : (\Glued{\gm} \in \GSub{\Glued{\cdot}}{(\Glued{\cdot} \Glued{,} \sub{\Gwsigproj{j}{2}{\Glued{\wsig}}}{\GSubstExt{\GSubstId}{\Glued{a}}})} \to P~(\sub{b}{ \gm })) \\
    & \quad \quad \to P~(\Wsup{j}{\wsig}{a}{b}) ] \quad \text{ (square bracket here just to make things readable)} \\
    & \to \goodTerm{\cdot}{t}{\El{\wcode{\wsig}}} \to \CWmodel~\Glued{\wsig}~t \to P~t \\
  \CWrec \  & \Glued{\wsig}~P~f~t~(\CWsup~\Glued{a}~\Glued{b}) = f~\Glued{a}~b~(\lambda \Glued{\gm}. \CWrec~P~f~(\sub{b}{\Gmodel{\Glued{\gm}}})~(b^C~\Glued{\gm}))
\end{align*}

Note that in $\CWsup$, the $\Glued{b}$ uses the definition of
$\denotesC{\wcode{\wsig}}$, which after unfolding, recursively references $\CWmodel$
in a strictly positive position. %So it is a well-founded definition.
We do not distinguish $(b, b^C)$ and $\Glued{b}$ for simplicity. Again, we omit validating the equational rules ($\beta$, $\eta$, and substitution) here.

We state the fundamental property of the logical-relations model.
%in our proof relevant model, it is admitting that our model is really a model,
%and our syntax is the initial model so there is map from our syntax to our
%canonicity model:
\begin{theorem}[Fundamental Property]
  If\/ $\goodTerm{\Gm}{t}{T}$, then its semantic interpretation is a dependent function such that
  $\denotesC{t} : \prod_{\goodSub{\cdot}{\gm}{\Gm}} \prod_{\gm' \in \denotesC{\goodCtx{\Gm}{}}(\gm)}\denotesC{\goodType{\Gm}{T}{}}(\gm)(\gm')(\sub{t}{\gm})$.
\end{theorem}

The first consequence of this model is the consistency of \TT---we cannot derive
$\goodTerm{\cdot}{t}{\bot}$. Otherwise, we would have an element in the empty set,
$\denotesC{\goodTerm{\cdot}{t}{\bot}}(\SubstEmp)(\star) \in
\denotesC{\bot}(\SubstEmp)(\star)(\sub{t}{\gm}) = \emptyset$, a contradiction.

\begin{theorem}[Consistency]
  The typing judgment $\goodTerm{\cdot}{t}{\bot}$ is not derivable for any term~$t$.
\end{theorem}

Next, with the logical-relations model, we can map an arbitrary closed boolean
term $\goodTerm{\cdot}{t}{\cB}$ to get the result
$\denotesC{\goodTerm{\cdot}{t}{\cB}}(\SubstEmp)(\star) = \star^1$ or $\star^2$,
witnessing the proof of $t \equiv \true \text{ or } t \equiv\false$ by the definition of our model, arriving at
\cref{thm:canonicity-appendix}.


Further, with the help of eta rules, we have the following canonical forms.
\begin{theorem}[Canonical Forms]\hfill
  \begin{itemize}
    \item If $\goodTerm{\cdot}{t}{\El{\wcode{\tau}}}$ and $\goodWSig{\cdot}{\tau}{n}$, then $\goodTerm{\cdot}{t \equiv \Wsup{j}{\tau}{a}{b}}{\El{\wcode{\tau}}}$ for some $\goodTerm{\cdot}{a}{A}$, $\goodTerm{B[("id", a)]}{b}{\El{\wcode{\tau}}}$, and $j < n$
    \item If $\goodTerm{\cdot}{t}{\mathbb{B}}$ then $\goodTerm{\cdot}{t \equiv "tt"}{\mathbb{B}}$ or $\goodTerm{\cdot}{t \equiv "ff"}{\mathbb{B}}$ 
    \item If $\goodTerm{\cdot}{t}{\TyLkg{\lsig}}$ with $\goodSig{\cdot}{\sigma}{n}$, then $\goodTerm{\cdot}{t \equiv \LkgAdd{o}{t}}{\TyLkg{\lsig}}$ 
      for some $\goodTerm{\cdot}{o}{\TyLkg{\lkgproj{1}{\sigma}}}$ and $\goodTerm{\lsigproj{2}{\sigma}}{t}{\lkgproj{2}{\sigma}}$
    \item If $\goodTerm{\cdot}{t}{\TySigma A B}$ then $\goodTerm{\cdot}{t \equiv (a, b)}{\TySigma A B}$ with $\goodTerm{\cdot}{a}{A}$ and $\goodTerm{\cdot}{b}{B[("id", a)]}$
    \footnote{We emphasize the last one because $\Tkg{\ell}$ is a dependent pair}
  \end{itemize}
\end{theorem}

% the canonical form of inductive type, linkage and boolean 