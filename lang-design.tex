When adapting Family Polymorphism into dependent type, we choose to tailor and focus on only the essence of family (and inheritance) structure in \citet{zm2017}, and thus a lot of unrelated feature, like Interface, will be removed. In this case, it will look like module with late-binding and inheritance as code and proof reuse for module. We also need it to have a good compatibility with Inductive Type, because we don't want to lose too much reasoning power of Coq (and its tactic programming). 

Let's start with the basic example---STLC and its extension in \cref{fig:STLC-example}---to consider what kind of features are required and how much of them can be supported by family polymorphism.
% add the example code here

\begin{figure}
  \begin{minipage}[t]{0.47\linewidth}
\begin{minted}[fontsize=\footnotesize]{Coq}
Family STLC.
Inductive ty : Set :=
  | unit : ty
  | arrow : ty -> ty -> ty.
Inducitive term : Set := 
  | var : id -> term 
  | lam : id -> term -> term ...
Fixpoint subst 
  : id -> term -> term := ...
Inductive has_type 
  : context -> term -> type := ...
Proposition subst_lemma :
  has_type (Γ; x ↦ A) t T ->
  has_type Γ u A ->
  has_type Γ (subst x u t) T.
Inducitve step : term -> term -> Prop 
  := ...
(* ... and more, end with type safety *)
EndFamily.
\end{minted}
  \end{minipage}
  \begin{minipage}[t]{0.47\linewidth}
\begin{minted}[fontsize=\footnotesize]{Coq}
Family STLC_bool extends STLC.
Extend Inductive ty : Set :=
  | bool : ty.

Extend Inducitive term : Set := 
  | tt : term | ff : term 
  | tif : term -> term -> term -> term
Extend Fixpoint subst (* Need to handle new term *)
  : id -> term -> term := ...
Extend Inductive has_type (* .. and new ty too *)
  : context -> term -> type := ...
Extend Proposition subst_lemma :
  ... (* Need to prove extra cases *)


Extend Inducitve step : term -> term -> Prop 
  := ... (* Need to expand this binary relation *)
(* ... and more extension *)
EndFamily.
\end{minted}
  \end{minipage}
  \caption{Example STLC and its extension}\label{fig:STLC-example}
\end{figure}


% We want to reason about fields in a family.
\textlabel{Req (1)}{langdesign-req1}:\textbf{ we want to be able to reason about fields in a family}, just like in a module, where a field can reason about its former fields. More generally, we want it possible for a later field to have type dependent on the former fields. In the example of \cref{fig:STLC-example}, "subst", "has_type", "subst_lemma" all require this feature.

% we want extensible inductive type
\textlabel{Req (2)}{langdesign-req2}:\textbf{ we need extensible inductive type}, to extend "term" and "ty" in \cref{fig:STLC-example}. The extension of inductive type is not supported by \citet{zm2017} thus we need further consideration about it on both implementation and metatheory. Additionally, \textlabel{Req (3)}{langdesign-req3}:\textbf{we also want extensible pattern matching and induction reasoning}, to extend "subst" in \cref{fig:STLC-example}. There are actually two kinds of ``pattern matching'', one for data and the other for induction reasoning, i.e. one use the eliminator to \mintinline{Coq}{Set} or \mintinline{Coq}{Type}, and the other uses the eliminator to \mintinline{Coq}{Prop}. Luckily, in this setting, extensible pattern matching can be easily expressed with family extension -- we just aggregate all case handlers of pattern matching into one family (as a bunch of functions), and then one family can encode one pattern matching, and family inheritance can express adding case handlers. Then we just need to introduce a primitive that will ``wrap'' that family into a recursive function. Induction reasoning can be handled in an almost identical way. 

% we want to be able to reason pattern matching as well
However, there is still difference between data recursion and induction reasoning, because the former one is \textit{proof-relevant}. This difference leads to another issue: \textlabel{Req (4)}{langdesign-req4}:\textbf{we need to reason about (the computation about) extensible pattern matching}, just like "subst_lemma" in \cref{fig:STLC-example}. When proving "subst_lemma", we have to know "subst" is invariant on "tt", i.e. \mintinline{Coq}{(subst i x tt) = tt}. This kind of information requires \textit{computational rules} from the recursors.
% we want tactic programming and certain level of automation
Finally, \textlabel{Req (5)}{langdesign-req5}:\textbf{we desire tactic programming} to relieve us from the hassle of direct manipulation of proof terms.


