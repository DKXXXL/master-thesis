\subsection{Challenges}

We want the metatheory mechanization to allow adding new
function/proving new lemmas, and adding new constructors for inductive
type at the same time  (Expression Problem~\cite{wadler-ep}). Coq
support the former but not the later---when an existent inductive type
is extended with new constructor, we have to modify the pattern matching
in the existent functions. This motivates us to add inheritance/Family
Polymorphism into Coq as a solution to the Expression Problem. 

\begin{figure}[!htb]
  \begin{minipage}[t]{0.47\linewidth}
\begin{lstlisting}[language=Coq,  escapeinside={@}{@}]
Family STLC.
Inductive ty : Set :=
  | unit : ty
  | arrow : ty -> ty -> ty.
Inductive term : Set := 
  | var : id -> term 
  | lam : id -> term -> term 
  | tunit : term ...
Fixpoint subst 
  : id -> term -> term := ...
Inductive has_type 
  : context -> term -> type := ...
Proposition subst_lemma :
  has_type (G; x ↦ A) t T ->
  has_type G u A ->
  has_type G (subst x u t) T.
Inductive step : term -> term -> Prop 
  := ...
(* ... and more, end with type safety *)
EndFamily.
\end{lstlisting}
  \end{minipage}
  \begin{minipage}[t]{0.47\linewidth}
\begin{lstlisting}[language=Coq,  escapeinside={@}{@}]
Family STLC_bool extends STLC.
Extend Inductive ty : Set :=
  | bool : ty.

Extend Inductive term : Set := 
  | tt : term | ff : term 
  | tif : term -> term -> term -> term

Extend Fixpoint subst (* Need to handle new term *)
  : id -> term -> term := ...
Extend Inductive has_type (* .. and new ty too *)
  : context -> term -> type := ...
Extend Proposition subst_lemma :
  ... (* Need to prove extra cases *)


Extend Inductive step : term -> term -> Prop 
  := ... (* Need to expand this binary relation *)
(* ... and more extension *)
EndFamily.
\end{lstlisting}
  \end{minipage}
  \caption{Example STLC and its extension}\label{fig:STLC-example}
\end{figure}

Refering to the syntax of \citet{zm2017}, we will come up the above
example. However, to add inheritance into a dependent type system like
Coq is not simple and there are various challenges and tension we need
to resolve. 


\textlabel{Challenge (1)}{chg:extensible-inductive-type}~\textbf{Extensible Inductive Type vs. Exhaustiveness of Inductive Reasoning}.
Unlike most general purposed programming languages, Coq, as a dependent
type theory, is a total language and thus each pattern matching cases
must fully ``cover'' an inductive type. This process is
``\textit{interactive}'' for inheritance---when inhertiance extend the
inductive type in the children family but the user forgets to extend the
corresponding inductive reasoning, our compiler need to signal an error
correspondingly, or better---guide the user to fill the incomplete
inductive reasoning. For example, in \cref{fig:STLC-example} when "term"
is extended with "tt" in the children family "STLC_bool", our compiler
need to make sure "subst" and "subst_lemma" are also extended correctly
otherwise signaling an error.
Additionally, our compiler must support direct inheritance (and
overriding) of the old case handlers for "var", "lam" and etc, to avoid
the boilerplate code. 


\textlabel{Challenge (2)}{chg:definition-relevant-reasoning}~\textbf{Definition-awared Reasoning vs. Overridability}.
In family polymorphism for general purposed programming language,
overriding is allowed to happen anywhere. But the problem becomes more
subtle in the case of dependent type since the later fields can have a
type dependent on the former fields.  Consider a small example where we
have consecutive fields \mintinline{Coq}{{.. a : nat := 1; a-def : Eq a
1 := eq_refl; ..}}, should we support override "a"? If we do support it,
how can our compiler detect that "a-def" is not inheritable and signal
an error when such thing happens? 

A similar example is about "subst" and "subst_lemma" in
\cref{fig:STLC-example}. For "subst_lemma" to hold throughout two
families, we shouldnot allow overriding on the definition of "subst".
For example, we should still stick with the fact that "subst tunit x y =
tunit", as this fact is used when proving "subst_lemma" and we want this
lemma to be inherited to the children family.

These two examples both show the incompatibility between exposing the
concrete definition of a field for reasoning (this second example even
involves the definition of fixpoint and pattern matching) and
overridability. We need to be careful about this tension when designing
the language.\YZ{
  The two examples seem to suggest there should not be any overriding allowed at all.
  Is there an example where you actually want the ability to override?
}

\textlabel{Challenge (3)}{chg:consistency}~\textbf{Consistency vs. Self Parametrization}.
The meta\-theoretical formulation in~\citet{zm2017} for family
polymorphism is very natural as they consider an extension of the family
as the concatenation of a new linkage. However, this formulation is not
directly adaptable for dependent type theory because it can easily
encode non-termination sabotaging consistency by the presence of "self".
Informally, in the orignal formulation, we can override the "f" in
family $\{"f" := id; "g" := "call self.f"\}$ with "call self.g",
resulting the mutual calling of "f" and "g" thus a deadloop. This
conflict between self parametrization and consistency requests changing
the original formulation of family and inheritance for dependent type
theory.


\textlabel{Challenge (4)}{chg:software-engineering}~\textbf{Engineering and Ecosystem}.
Not only for the metatheory, we also want our implementation of family
to have good compatibility with the rest of Coq. For example, we
shouldn't compromise \textbf{tactic programming} as it is part of the
idiomatic Coq programming experience. Additionally, the interactive development
style requires \textbf{incremental type checking}---our design has to be
able to type check each Coq Vernacular command once inputted. What's
more, we want our family ``term'' to be incorporated into Coq \textbf{without
breaking its other functionality}. For example, the user may want to
project the family member anywhere inside an arbitrary Gallina term,
that may later be computed and normalized. However, families themselves
can be second class citizens just like \citet{zm2017}.\YZ{I was thinking
if the paper should start by talking about the functional encoding of
inheritance (namely exposing a self parameter to allow late binding, and
later taking the fixpoint of the extensible module when it needs to be
accessed outside the family).  I think most readers will benefit from a
recap of this encoding.\\
Coq is total but that doesn’t prevent Coq from having a Fixpoint keyword (plus you proved your language is consistent).
}

% \subsection{Design Space}

% When adapting Family Polymorphism into dependent types, we choose to
% focus only on the essence of family (and inheritance)
% structure in \citet{zm2017}, and thus a lot of unrelated features, like
% Interface, will be removed. In this case, it will look like module with
% late-binding. Inheritance can be considered as code and proof reuse mechanism for module. We also
% need it to have good compatibility with inductive types, because we
% don't want to retain the idiomatic Coq programming experience
% %\YZ{By 'reasoning power of Coq', I suppose you mean the expressive power of Gallina?}\EDJreply{I think the sentence last time was not really making sense because using Church Encoding of inductive doesn't sabotage the expressiveness. I rephrase it now, please check.}
% (and its tactic programming). 

% Let's start with a basic example---STLC and its extension in
% \cref{fig:STLC-example}---to consider what kind of features are required
% and how much of them can be supported by family polymorphism.
% add the example code here





% First and foremost, we don't want to throw away some basic good
% properties of Coq. \textlabel{Req~(0A)}{chg:software-engineering}~\textbf{we
% want to retain incremental (modular) type checking}. Notice that, the
% modularity here in Coq is a bit different from other languages, because
% Coq supports \textit{interactive} theorem proving, so we actually need
% \textit{statement-wise} incremental type checking, not only for avoiding
% re\-compilation, but also to enable immediate feedback and incremental
% type-checking for the Coq developer. We don't want our family to ruin
% this conventional routine of interactive development.
% \textlabel{Req~(0B)}{chg:software-engineeringb}~\textbf{we also want to keep the
% computational ability of Coq when using families}. Coq, based on constructive
% logic, can consider proof as programs. 
% We don't want our Family facility to ruin this when incorporated with 
% the remaining parts of Coq: 
% the developers should be able to project fields of families 
% as first-class value just like how they can project fields of Coq's Module.

% We want to reason about fields in a family.
% \textlabel{Req (1)}{chg:definition-relevant-reasoning}:\textbf{ we want to be able to
% reason about fields in a family}, just like in a module, where a field
% can reason about its former fields. More generally, we want to allow a
% later field to be type-dependent on the former fields. In the example of
% \cref{fig:STLC-example}, "subst", "has_type", "subst_lemma" all require
% this feature.

% we want extensible inductive type
% \textlabel{Req (2)}{chg:extensible-inductive-type}:\textbf{ we need extensible
% inductive types}, 
% to extend "term" and "ty" in \cref{fig:STLC-example}.
% Extensible inductive types are not supported by \citet{zm2017}, so
% we need further consideration about it on both implementation and
% metatheory.

% \textlabel{Req (3)}{chg:extensible-inductive-type}: \textbf{we also want extensible
% pattern matching and induction reasoning}, to extend "subst" in
% \cref{fig:STLC-example}.
% There are actually two kinds of ``pattern matching'', one for data and
% the other for induction reasoning, i.e., one uses the eliminator to
% \mintinline{Coq}{Set} or \mintinline{Coq}{Type}, and the other uses the
% eliminator to \mintinline{Coq}{Prop}. Luckily, in this setting,
% extensible pattern matching can be easily expressed with family
% extension---we just aggregate all case handlers of pattern matching
% into one family (as a bunch of functions), and then one family can
% encode one pattern matching, and family inheritance can express adding
% case handlers. Then we just need to introduce a primitive that will
% ``wrap'' that family into a recursive function. Induction reasoning can
% be handled in an almost identical way. 

% we want to be able to reason pattern matching as well
% However, there is still difference between data recursion and induction
% reasoning, because the former one is \textit{proof-relevant}. This
% difference leads to another issue: \textlabel{Req (4)}{chg:definition-relevant-reasoning}:
% \textbf{we need to reason about (the computation about) extensible
% pattern matching}, just like "subst_lemma" in \cref{fig:STLC-example}.
% When proving "subst_lemma", we have to know "subst" is invariant on
% "tt", i.e., \mintinline{Coq}{(subst i x tt) = tt}. This kind of
% information requires exposing \textit{computational rules} from the
% recursors.
\YZ{Seems like what's missing here is to
convince the reader why exposing the definition of subst could be a
problem for family polymorphism.}\EDJreply{Actually I don't see why I need to emphasize this "as a problem". This is just a requirement like any other requirements earlier. I don't know how to show its difficulty at this early stage.}\YZreply{There is apparently a tension between the need to expose definitional equality and the ability to override. Looking at the bigger picture, I think this section reads better if phrased as discussing challenges posed by adding family polymorphism to a proof assistant. I have some initial ideas that I need to discuss with you.}\EDJreply{Design Space section is rewritten into Challenge now.}

% we want tactic programming and certain level of automation
% Finally, \textlabel{Req (5)}{chg:software-engineering}: \textbf{we desire tactic
% programming} to relieve us from the hassle of direct manipulation of proof terms.
