In this appendix, we sketch how to use the ``library'' of linkage
transformers in \TT to inductively construct a linkage that models a
derived family of \lsti{STLC}.

We use \lsti{STLCBool} as an example. \lsti{STLCBool} extends
\lsti{STLC} with boolean values (\lsti{tm_true} and \lsti{tm_false})
and if-then-else expressions.
The left column of the table below shows initial code excerpted from
this family.
Each cell in the last column defines a linkage transformer~$h_i$
inductively constructed from the linkage transformers $h_0$, ..., $h_{i-1}$
using one of the introduction forms.
The goal is to eventually construct a linkage transformer $h_n$ representing
the entire family \lsti{STLCBool}.

Of note is the two grayed rows. They are constructing a linkage transformer~$h_\beta$
containing the case handlers for \lsti{subst}.
This $h_\beta$ is then appended to $h_3$ as a nested linkage transformer.

\medskip

\if 0
% Please add the following required packages to your document preamble:
% \usepackage[table,xcdraw]{xcolor}
% If you use beamer only pass "xcolor=table" option, i.e. \documentclass[xcolor=table]{beamer}
\begin{centered}
  \small
\renewcommand*{\arraystretch}{1.35}
\begin{tabular}{|l|l|l|l|}
  \hline
  \rowcolor[HTML]{FFFC9E} 
  surface-syntax program & $i$ & $\lsig_i = \LSigAdd{\lsig_{i-1}}{s_i}{T_i}$ & $\lsig_i' = \LSigAdd{\lsig_{i-1}'}{s_i'}{T_i'}$ \\ \hline
  \lsti!Family STLCBool...!   & 0        & $\{\}$                                     & $\{\}$                                     \\ \cline{2-4} 
  \lsti!FInductive tm += !      & 1        & $\{"tm" : \TyS{\wcode{\wsig_{\texttt{tm}}}}\}$         & $\{"tm" : \TyS{\wcode{\wsig_{\texttt{tm}}'}}\}$        \\ \cline{2-4} 
  \codecomment{existent constr.}    & 2        & $\{..; "tm_var": \El{"tm"}, .. \}$         & $\{..;"tm_var": \El{"tm"},.. \}$           \\ \cline{2-4} 
  \lsti! .. | tm_true | .. !   & 3        & $\lsig_2$                                  & $\{..,"tm_true":\El{"tm"} \}$              \\ \cline{2-4} 
  \rowcolor[HTML]{CDCDCD} 
  \lsti!  FRecursion subst!       &          &                                            &                                            \\ \cline{2-4} 
  \rowcolor[HTML]{CDCDCD} 
  \lsti!  (* Inherit ... *)!      & $\alpha$ & $\RecSig{\wsig_{\texttt{tm}}}{"tm"}$                   & $\RecSig{\wsig_{\texttt{tm}}}{"tm"}$                   \\ \cline{2-4} 
  \rowcolor[HTML]{CDCDCD} 
  \lsti!  Case tm_true!          & $\beta$  & $\lsig_{\alpha}$                          & $\{.. "tm_true": .. \}$                       \\ \cline{2-4} 
  \rowcolor[HTML]{FFFFFF} 
  \lsti!  End subst!              & 4        & $\{.. "subst'" : \TyLkg{\lsig_\alpha}\}$   & $\{.. "subst'" : \TyLkg{\lsig_\beta'}\}$   \\ \cline{2-4} 
                            & 5        & $\{.. "subst" : \El{"tm"} \to id \to ..\}$ & $\{.. "subst" : \El{"tm"} \to id \to ..\}$ \\ \cline{2-4} 
  \lsti!(* Do the same on ty*)! & 6        & ...                                        & ...                                        \\ \cline{2-4} 
  \lsti!(* Inherit env *)!      & 7        & $\{.. "env" : id \to "option"~"ty"\}$      & $\{.. "env" : id \to "option"~"ty"\}$      \\ \hline
  \end{tabular}
\end{centered}

\fi

% Please add the following required packages to your document preamble:
% \usepackage[table,xcdraw]{xcolor}
% If you use beamer only pass "xcolor=table" option, i.e. \documentclass[xcolor=table]{beamer}
\begin{centered}
\small
\renewcommand*{\arraystretch}{1.25}
\begin{tabular}{|l|l|l|}
\hline
\rowcolor[HTML]{FFFC9E} 
surface-syntax program & $i$      & $\goodInh{\cdot}{h_i}{\lsig_i}{\lsig_i'}$             \\ \hline
\lsti!Family STLCBool!   & 0        & $\InhId$                                             \\ \cline{2-3} 
\lsti!FInductive tm += !       & 1        & $\InhOv{h_0}{\wcode{\wsig_{\texttt{tm}}'}}$                      \\ \cline{2-3} 
\codecomment{existing constructors}   & 2        & $\InhOv{h_1}{\Wsup{\wsig_{\texttt{tm}}'}{\top}{\bot}}$           \\ \cline{2-3} 
\lsti!| tm_true |!\ \;\dadada    & 3        & $\InhExt{h_2}{\Wsup{\wsig_{\texttt{tm}}'}{\top}{\bot}}$          \\ \cline{2-3} 
\rowcolor[HTML]{CDCDCD} 
\lsti!FRecursion subst!\ \;\dadada\ \;\lsti!+=! & $\alpha$ & $\InhId$ \\ \cline{2-3} 
\rowcolor[HTML]{CDCDCD} 
\lsti!  Case tm_true := !\dadada. \lsti! Case !\dadada & $\beta$  & $\InhExt{h_\alpha}{...}$ \\ \cline{2-3} 
\rowcolor[HTML]{FFFFFF} 
\lsti!End subst.!              & 4        & $\InhNest{h_3}{h_\beta}$
  %with $\uparrow^s_3$
  \\ \cline{2-3} 
                          & 5        & $\InhOv{h_4}{\lambda t. \Wrec{\wsig_{\texttt{tm}}'}{t}{"substCases"}}$ \\ \cline{2-3} 
\lsti!FInductive ty += !\dadada. & 6        & ...                                                  \\ \cline{2-3} 
\codecomment{Inherit "env"}   & 7        & $\InhInh{h_6}$
  %with $\uparrow^s_6$
  \\ \cline{2-3} 
\codecomment{Inherit "empty"}   & 8        & $\InhInh{h_7}$
  \\ \cline{2-3} 
\dadada   & \dadada & \dadada
  \\ \hline
\end{tabular}
\end{centered}


\if 0

We take \lsti!Family STLC_bool! extending \lsti!STLC! as an example, based on the example \cref{fig:stlc-linkage-typing} and \cref{fig:stlc-mechanized}. We list out the surface syntax and their translation into \TT. In our figure, $\lsig_i$ stands for the parent signature (the signature before the transformation); and $\lsig_i'$ stands for the signature after the transformation. We omit the information for $s_i$ and $A_i$


When sketching signature $\lsig$, we use curly bracket as a shorthand for signature. Because every $\lsig_i$ is always constructed from $\lsig_{i-1}$ so in the figure we omit the previous fields in each signature.  One exception are the rows indexed by $\alpha$ and $\beta$. These two rows are describing a nested family for case handlers, and $\lsig_\beta'$ is based on $\lsig_\alpha'$ and $\lsig_\beta = \lsig_\alpha$. The other exception is $\lsig_3$, which is directly $\lsig_2$ since the parent doesn't have those new constructor. Note for $\lsig_2$ and $\lsig_2'$, there are actually multiple constructors inserted into the signature (e.g. "tm_lam" and "tm_app") which we omitted here.


Our final target is to construct a term $\goodInh{\cdot}{?}{\lsig_7}{\lsig_7'}$ using the four linkage transformers' rules. There are two upcasting operations $\goodTerm{A_3'}{\uparrow^s_3}{A_3}$ and $\goodTerm{A_6'}{\uparrow^s_6}{A_6}$ (omitting debruijn indices here) for checking the compatibility between the omitted $A_i$ and $A_i'$.

\fi