\begin{abstract}

With the growing practice of mechanizing language metatheories,
it has become ever more pressing that interactive theorem provers 
make it easy to write reusable, extensible code and proofs.
%
This paper presents a language design useful for extensible metatheory
mechanization in a proof assistant.
The new design achieves reuse and extensibility via a form of family
polymorphism, a seemingly object-oriented idea, that allows code and
proofs to be polymorphic to their enclosing families.
Our development addresses significant technical challenges that arise
from the underlying language of a proof assistant being simultaneously
functional, dependently typed, a logic, and an interactive tool.
%
Our results include
\begin{enumerate*}
\item a prototypical implementation of the language design as a Coq plugin,
\item a formalization of the language design as a constructive type theory with inductive types,
\item proofs of strong guarantees afforded by the type theory,
and 
\item case studies showing how the new expressiveness naturally addresses real
programming challenges in metatheory mechanization.
\end{enumerate*}
\end{abstract}

\maketitle

% lstlisting coq style (inspired from a file of Assia Mahboubi)
% https://tex.stackexchange.com/questions/434523/coq-syntax-highlighting
\section{Introduction}
\label{sec:intro}
% mention expression problem and family polymorphism solving it
% proof engineering, not being focused, fampoly on proof engineering
% contribution

It is hard to write modular, extensible code.
It is even harder to write modular, extensible proofs.
While there exist many linguistic solutions to code reuse and
extensibility in object-oriented and functional languages,
linguistic solutions are lacking in proof assistants.

Meanwhile, language designers and researchers are increasingly
using proof assistants to mechanize meta\-theories, building on each
other's work.
Therefore, it has become ever more important that proof assistants
make it easy to write reusable, extensible code and proofs.

% Concept 
% Contribution

\section{Design Requirements and Challenges}
% what feature we want
% what problem/formalization we want to achieve
% Before/After Comparison of examples (with/without Fampoly)
\label{sec:background+challenge}
Integrating family polymorphism into a proof assistant presents challenges
far beyond those found in an object-oriented setting~\cite{ncm2004,vc-calculus-2006,zm2017},
because the underlying programming language is simultaneously
functional, dependently typed, a logic, and an interactive tool.

\noindentparagraph{%
\headinglabel{C1}{chg:extensible-v-exhaustive}.\,
Extensible inductive types vs.\ exhaustive induction reasoning.%
}

Inductive types, generalizing algebraic data types found in
functional languages, are a central feature of any proof assistant
in use for mechanizing language metatheories.
They offer a means to define abstract syntax and inference rules.
But unfortunately, inductive types are closed to extension by design.

A family-polymorphism design could potentially support extensible
inductive types, by allowing a \emph{derived family} to add new constructors
to inductive types inherited from a \emph{base family}.
Such a feature would be useful for extending mechanized languages.
For example, \cref{fig:stlc-nonmechanized} shows the part of an STLC extension that could
potentially be mechanized using such extensible inductive types: a derived family
inherits constructors of inductive types from the base family (e.g., the ellipsis under ``Typing rules'')
and adds new constructors to model the new language feature.

\begin{figure}

\newcommand{\headerfont}[1]{\textbf{#1}}

\BeforeBeginEnvironment{mathpar}{%
  \vspace{-.8ex}
}

\fontsize{8.0}{8.8}\selectfont
\def\MathparLineskip{\lineskip=0.1ex}

\fcolorbox{black}{black!10}{
\begin{minipage}{0.42\textwidth}

\noindent%
\headerfont{Types} \hspace{1.0em}
$ T \ \Coloneqq\ \mathbb{1} \bnf  T_1 \to T_2 $
\medskip

\headerfont{Terms} \hspace{0.85em}
$ t \ \Coloneqq\ () \bnf x \bnf \lambda x.\,t \bnf t_1~t_2 $
\medskip

\headerfont{Typing rules}
\begin{mathpar}
\Rule{}{ \Gamma \vdash () : \mathbb{1} }

\Rule{ \Gamma(x)=T }{ \Gamma \vdash x : T }

\Rule{ \Gamma, x:T_1 \vdash t : T_2 }{ \Gamma \vdash \lambda x.\,t : T_1 \to T_2 }

\Rule{
  \Gamma \vdash t_1 : T_1 \to T_2
  \quad
  \Gamma \vdash t_2 : T_1
}{
  \Gamma \vdash t_1~t_2 : T_2
}

\end{mathpar}

\headerfont{Value forms}
\begin{mathpar}
\Rule{}{\textit{Val}(())}

\Rule{}{\textit{Val}(\lambda x.\,t)}
\end{mathpar}

\headerfont{Substitution function}
\begin{mathpar}
  \subst{()}{t}{x} \defeq ()

  \subst{y}{t}{x} \defeq \begin{cases} y, & \text{if } x \neq y \\ e, & \text{if } x = y \end{cases}

  \subst{(\lambda y.\,t')}{t}{x} \defeq \begin{cases} \lambda y.\,\subst{t'}{t}{x}, & \text{if } x \neq y \\ \lambda y.\,t', & \text{if } x = y \end{cases}

  \subst{(t_1~t_2)}{t}{x} \defeq \subst{t_1}{t}{x}~\subst{t_2}{t}{x}
\end{mathpar}

\headerfont{Reduction rules}
\begin{mathpar}
\Rule{ t_1 \longrightarrow t_1' }{ t_1~t_2 \longrightarrow t_1'~t_2 }

\Rule{ \textit{Val}(t_1) \quad t_2 \longrightarrow t_2' }{ t_1~t_2 \longrightarrow t_1~t_2' }

\Rule{\textit{Val}(t_2)}{ (\lambda x.\,t_1)~t_2 \longrightarrow \subst{t_2}{t_1}{x} }

\end{mathpar}

\begin{tabular}{@{}l@{\quad}l@{}}
\headerfont{Weakening lemma} & \headerfont{Substitution lemma} \smallskip\\
\headerfont{Preservation theorem} & \headerfont{Progress theorem} \smallskip\\
\headerfont{Type-safety theorem}
\end{tabular}%}

\end{minipage}
}%
\raisebox{26pt}{%
%\tikz \draw [arrows = {-Latex[width'=0pt .5, length=10pt, fill=white]}] (0,0) -- (1,0);
\tikz \draw [
  line width = .5pt,
  arrows = {-Stealth[inset=0pt, length=5pt, fill=white, angle'=60, scale=2]}
] (1.2,0) -- (0,0);%
%https://tikz.dev/tikz-arrows
\fcolorbox{black}{black!10}{
\begin{minipage}{0.39\textwidth}

\noindent%
\headerfont{Types} \hspace{1.0em}
$ T \ \Coloneqq\ \cdots $ \textit{(no change)}
\medskip

\headerfont{Terms} \hspace{0.85em}
$ t \ \Coloneqq\ \cdots \bnf \mathrm{fix}\,x.\,t $
\medskip

\headerfont{Typing rules}
\begin{mathpar}
\cdots

\Rule{
  \Gamma, x : T \vdash t : T
}{
  \Gamma \vdash \mathrm{fix}\,x.\,t : T
}

\end{mathpar}

\headerfont{Value forms}
\begin{mathpar}
\cdots \textit{ (no change)}

\end{mathpar}

\headerfont{Substitution function}
\begin{mathpar}
{\raisebox{-8pt}{$\cdots$}}\medskip

\subst{(\mathrm{fix}\,y.\,t')}{t}{x} \defeq \begin{cases} \mathrm{fix}\,y.\,\subst{t'}{t}{x}, & \text{if } x \neq y \\ \mathrm{fix}\,y.\,t', & \text{if } x = y \end{cases}

\end{mathpar}

\headerfont{Reduction rules}
\begin{mathpar}
\cdots

\Rule{}{ \mathrm{fix}\,x.\,t \longrightarrow \subst{t}{\mathrm{fix}\,x.\,t}{x} }

\end{mathpar}

\begin{tabular}{@{}l@{\ \ }l@{}}
\headerfont{Weakening lemma}      & \textit{(a new case required)} \smallskip\\
\headerfont{Substitution lemma}   & \textit{(a new case required)} \smallskip\\
\headerfont{Preservation theorem} & \textit{(a new case required)} \smallskip\\
\headerfont{Progress theorem}     & \textit{(a new case required)} \smallskip\\
\headerfont{Type-safety theorem}  & \textit{(no change)}
\end{tabular}

\end{minipage}
}%

}

\vspace{-42pt}
\hfill
\begin{minipage}{.48\textwidth}
\fontsize{9.0}{10}\selectfont

Mechanization of
types, terms, typing rules, value forms, and reduction rules is via
inductive types.\smallskip

Mechanization of
lemmas, theorems, and the substitution function is by
induction over inductive types.
\end{minipage}

\caption{%
  STLC metatheories (left) and its extension with fixpoints (right).
}
\label{fig:stlc-nonmechanized}
\end{figure}


However, there is a tension between extensibility of inductive types
and exhaustivity of induction reasoning.\footnote{%
The tension reflects a duality between variants and records.
For record-like language constructs (e.g., classes and objects), it is
safe for an extension to add more fields: existing fields can still be
projected.
But variant-like constructs (e.g., inductive types) do not automatically
enjoy safe, modular addition of constructors:
existing pattern matches could potentially become non-exhaustive.}
In \cref{fig:stlc-nonmechanized}, all the lemmas and theorems, as well as
the substitution function, require induction reasoning.
A language design must enforce that induction remains exhaustive in the
face of the new constructors in the derived family.
For modularity, it should do so without requiring
redefinition or rechecking of those cases already handled by the base
family.

\noindentparagraph{%
\headinglabel{C2}{chg:override-v-dependent}.\,
Overriding vs.\ dependent typing.%
}

Common to many family-polymorphism mechanisms is the ability for a
derived family to override fields defined in the base family.
As we show later, this expressiveness is also useful for metatheory
mechanization.
However, it is in conflict with dependent typing, wherein type checking
often involves unfolding definitions.
For example, in an STLC mechanization, the substitution lemma is checked
against the definition of the substitution function.
If a derived family were allowed to arbitrarily override the
substitution function, then the proof of the substitution lemma, when
inherited, would be ill-formed (even in the absence of any inductive-type extensions).
A language design should only allow those overridings that are safe.

\noindentparagraph{%
\headinglabel{C3}{chg:self-v-consistency}.\,
Self reference vs.\ logical consistency.%
}

Inheritance and family polymorphism are mechanisms for
incrementally modifying self-referential definitions~\cite{cook1990inheritance}:
the implicit self parameterization is how they achieve late binding and
thus offer extensibility.
However, self reference could easily lead to divergence.
Divergence is not a concern for the design of ordinary OO or
functional languages, but it would mean logical inconsistency---and
hence unsoundness---for a language aimed at logical reasoning!
A family-polymorphism design must tame self reference to guarantee consistency.

\noindentparagraph{%
\headinglabel{C4}{chg:tooling}.\,
User experience and tooling.%
}

Interactive theorem proving and tactic programming are
part of the idiomatic programming experience in a proof assistant.
A language design integrating family polymorphism should be compatible
with these programming styles.
In particular, it should be possible in our tool to
incrementally navigate through vernacular commands and, moreover, construct proofs
with common tactics while getting instant feedback on the proof state,
even in the middle of a family definition.
In addition, it should remain possible to perform computation in the extended Gallina language.

\ifShowOldWriting

\newpage

We want the metatheory mechanization to allow adding new
function/proving new lemmas, and adding new constructors for inductive
type at the same time  (Expression Problem~\cite{wadler-ep}). Coq
support the former but not the later---when an existent inductive type
is extended with new constructor, we have to modify the pattern matching
in the existent functions. This motivates us to add inheritance/Family
Polymorphism into Coq as a solution to the Expression Problem. 

\begin{figure}[!htb]
  \begin{minipage}[t]{0.47\linewidth}
\begin{lstlisting}[language=Coq,  escapeinside={@}{@}]
Family STLC.
Inductive ty : Set :=
  | unit : ty
  | arrow : ty -> ty -> ty.
Inductive term : Set := 
  | tm_var : id -> term 
  | tm_abs : id -> term -> term 
  | tunit : term ...
Fixpoint subst 
  : id -> term -> term := ...
Inductive has_type 
  : context -> term -> type := ...
Proposition subst_lemma :
  has_type (G; x ↦ A) t T ->
  has_type G u A ->
  has_type G (subst x u t) T.
Inductive step : term -> term -> Prop 
  := ...
(* ... and more, end with type safety *)
EndFamily.
\end{lstlisting}
  \end{minipage}
  \begin{minipage}[t]{0.47\linewidth}
\begin{lstlisting}[language=Coq,  escapeinside={@}{@}]
Family STLC_bool extends STLC.
Extend Inductive ty : Set :=
  | bool : ty.

Extend Inductive term : Set := 
  | tt : term | ff : term 
  | tif : term -> term -> term -> term

Extend Fixpoint subst (* Need to handle new term *)
  : id -> term -> term := ...
Extend Inductive has_type (* .. and new ty too *)
  : context -> term -> type := ...
Extend Proposition subst_lemma :
  ... (* Need to prove extra cases *)


Extend Inductive step : term -> term -> Prop 
  := ... (* Need to expand this binary relation *)
(* ... and more extension *)
EndFamily.
\end{lstlisting}
  \end{minipage}
  \caption{Example STLC and its extension}\label{fig:STLC-example}
\end{figure}

Refering to the syntax of \citet{zm2017}, we will come up the above
example. However, to add inheritance into a dependent type system like
Coq is not simple and there are various challenges and tension we need
to resolve. 


\textlabel{Challenge (1)}{chg:extensible-inductive-type}~\textbf{Extensible Inductive Type vs. Exhaustiveness of Inductive Reasoning}.
Unlike most general purposed programming languages, Coq, as a dependent
type theory, is a total language and thus each pattern matching cases
must fully ``cover'' an inductive type. This process is
``\textit{interactive}'' for inheritance---when inhertiance extend the
inductive type in the children family but the user forgets to extend the
corresponding inductive reasoning, our compiler need to signal an error
correspondingly, or better---guide the user to fill the incomplete
inductive reasoning. For example, in \cref{fig:STLC-example} when "term"
is extended with "tt" in the children family "STLC_bool", our compiler
need to make sure "subst" and "subst_lemma" are also extended correctly
otherwise signaling an error.
Additionally, our compiler must support direct inheritance (and
overriding) of the old case handlers for "tm_var", "tm_abs" and etc, to avoid
the boilerplate code. 


\textlabel{Challenge (2)}{chg:definition-relevant-reasoning}~\textbf{Definition-awared Reasoning vs. Overridability}.
In family polymorphism for general purposed programming language,
overriding is allowed to happen anywhere. But the problem becomes more
subtle in the case of dependent type since the later fields can have a
type dependent on the former fields.  Consider a small example where we
have consecutive fields \mintinline{Coq}{{.. a : nat := 1; a-def : Eq a
1 := eq_refl; ..}}, should we support override "a"? If we do support it,
how can our compiler detect that "a-def" is not inheritable and signal
an error when such thing happens? 

A similar example is about "subst" and "subst_lemma" in
\cref{fig:STLC-example}. For "subst_lemma" to hold throughout two
families, we shouldnot allow overriding on the definition of "subst".
For example, we should still stick with the fact that "subst tunit x y =
tunit", as this fact is used when proving "subst_lemma" and we want this
lemma to be inherited to the children family.

Note we cannot simply just abandon overriding because we do want it sometimes---say we want a monotonic function \mintinline{Coq}{{.. f := .. ; p : monotonic f; ..}}. The proof of "p" likely requires the concrete definition of "f" but in the future we might want to override "f" and "p" together to swap the definition.

These three examples both show the incompatibility between exposing the
concrete definition of a field for reasoning (this second example even
involves the definition of fixpoint and pattern matching) and
overridability. We need to be careful about this tension when designing
the language.

\textlabel{Challenge (3)}{chg:consistency}~\textbf{Consistency vs. Self Parametrization}.
The meta\-theoretical formulation in~\citet{zm2017} for family
polymorphism is very natural as they consider an extension of the family
as the concatenation of a new linkage. However, this formulation is not
directly adaptable for dependent type theory because it can easily
encode non-termination sabotaging consistency by the presence of "self".
Informally, in the orignal formulation, we can override the "f" in
family $\{"f" := id; "g" := "call self.f"\}$ with "call self.g",
resulting the mutual calling of "f" and "g" thus a deadloop. This
conflict between self parametrization and consistency requests changing
the original formulation of family and inheritance for dependent type
theory.


\textlabel{Challenge (4)}{chg:software-engineering}~\textbf{Engineering and Ecosystem}.
Not only for the metatheory, we also want our implementation of family
to have good compatibility with the rest of Coq. For example, we
shouldn't compromise \textbf{tactic programming} as it is part of the
idiomatic Coq programming experience. Additionally, the interactive development
style requires \textbf{incremental type checking}---our design has to be
able to type check each Coq Vernacular command once inputted. What's
more, we want our family ``term'' to be incorporated into Coq \textbf{without
breaking its other functionality}. For example, the user may want to
project the family member anywhere inside an arbitrary Gallina term,
that may later be computed and normalized. However, families themselves
can be second class citizens just like \citet{zm2017}.
%\YZ{I was thinking
%if the paper should start by talking about the functional encoding of
%inheritance (namely exposing a self parameter to allow late binding, and
%later taking the fixpoint of the extensible module when it needs to be
%accessed outside the family).  I think most readers will benefit from a
%recap of this encoding.\\
%Coq is total but that doesn’t prevent Coq from having a Fixpoint keyword (plus you proved your language is consistent).
%}\EDJreply{Feel free to add these. I don't have many knowledge about this part.}

% \subsection{Design Space}

% When adapting Family Polymorphism into dependent types, we choose to
% focus only on the essence of family (and inheritance)
% structure in \citet{zm2017}, and thus a lot of unrelated features, like
% Interface, will be removed. In this case, it will look like module with
% late-binding. Inheritance can be considered as code and proof reuse mechanism for module. We also
% need it to have good compatibility with inductive types, because we
% don't want to retain the idiomatic Coq programming experience
% %\YZ{By 'reasoning power of Coq', I suppose you mean the expressive power of Gallina?}\EDJreply{I think the sentence last time was not really making sense because using Church Encoding of inductive doesn't sabotage the expressiveness. I rephrase it now, please check.}
% (and its tactic programming). 

% Let's start with a basic example---STLC and its extension in
% \cref{fig:STLC-example}---to consider what kind of features are required
% and how much of them can be supported by family polymorphism.
% add the example code here





% First and foremost, we don't want to throw away some basic good
% properties of Coq. \textlabel{Req~(0A)}{chg:software-engineering}~\textbf{we
% want to retain incremental (modular) type checking}. Notice that, the
% modularity here in Coq is a bit different from other languages, because
% Coq supports \textit{interactive} theorem proving, so we actually need
% \textit{statement-wise} incremental type checking, not only for avoiding
% re\-compilation, but also to enable immediate feedback and incremental
% type-checking for the Coq developer. We don't want our family to ruin
% this conventional routine of interactive development.
% \textlabel{Req~(0B)}{chg:software-engineeringb}~\textbf{we also want to keep the
% computational ability of Coq when using families}. Coq, based on constructive
% logic, can consider proof as programs. 
% We don't want our Family facility to ruin this when incorporated with 
% the remaining parts of Coq: 
% the developers should be able to project fields of families 
% as first-class value just like how they can project fields of Coq's Module.

% We want to reason about fields in a family.
% \textlabel{Req (1)}{chg:definition-relevant-reasoning}:\textbf{ we want to be able to
% reason about fields in a family}, just like in a module, where a field
% can reason about its former fields. More generally, we want to allow a
% later field to be type-dependent on the former fields. In the example of
% \cref{fig:STLC-example}, "subst", "has_type", "subst_lemma" all require
% this feature.

% we want extensible inductive type
% \textlabel{Req (2)}{chg:extensible-inductive-type}:\textbf{ we need extensible
% inductive types}, 
% to extend "term" and "ty" in \cref{fig:STLC-example}.
% Extensible inductive types are not supported by \citet{zm2017}, so
% we need further consideration about it on both implementation and
% metatheory.

% \textlabel{Req (3)}{chg:extensible-inductive-type}: \textbf{we also want extensible
% pattern matching and induction reasoning}, to extend "subst" in
% \cref{fig:STLC-example}.
% There are actually two kinds of ``pattern matching'', one for data and
% the other for induction reasoning, i.e., one uses the eliminator to
% \mintinline{Coq}{Set} or \mintinline{Coq}{Type}, and the other uses the
% eliminator to \mintinline{Coq}{Prop}. Luckily, in this setting,
% extensible pattern matching can be easily expressed with family
% extension---we just aggregate all case handlers of pattern matching
% into one family (as a bunch of functions), and then one family can
% encode one pattern matching, and family inheritance can express adding
% case handlers. Then we just need to introduce a primitive that will
% ``wrap'' that family into a recursive function. Induction reasoning can
% be handled in an almost identical way. 

% we want to be able to reason pattern matching as well
% However, there is still difference between data recursion and induction
% reasoning, because the former one is \textit{proof-relevant}. This
% difference leads to another issue: \textlabel{Req (4)}{chg:definition-relevant-reasoning}:
% \textbf{we need to reason about (the computation about) extensible
% pattern matching}, just like "subst_lemma" in \cref{fig:STLC-example}.
% When proving "subst_lemma", we have to know "subst" is invariant on
% "tt", i.e., \mintinline{Coq}{(subst i x tt) = tt}. This kind of
% information requires exposing \textit{computational rules} from the
% recursors.


% we want tactic programming and certain level of automation
% Finally, \textlabel{Req (5)}{chg:software-engineering}: \textbf{we desire tactic
% programming} to relieve us from the hassle of direct manipulation of proof terms.

\newpage

\fi

\section{Language Design}
% we need to make this section before Metatheory
% because a lot of our formulation in the metatheory
%   is inspired by the implementation
\label{sec:lang-design}
When adapting Family Polymorphism into dependent type, we choose to shrink and focus on only the essence of family (and inheritance) structure in \citet{zm2017}, and thus a lot of unrelated feature, like Interface, will be tailored. In this case, it will look like module with late-binding and inheritance as code reuse for module. 

We also need it to have a good compatibility with Inductive Type, because we don't want to lose too much reasoning power of Coq. 



\section{Compiling Family Polymorphism to Parameterized Modules}
% we need to make this section before Metatheory
% because a lot of our formulation in the metatheory
%   is inspired by the implementation
\label{sec:coqimpl}

% \subsection{Language Design}


% \subsection{Plugin Implementation}

In this section, we describe how we can implement the proposed Vernacular commands.

After considering the pros and cons, we decide to implement a Coq Plugin where we can add new Vernacular commands and translate each new command into a bunch of Coq commands (the orignal surface Vernacular command) on the fly, instead of modifying the code-base of Coq. 

Despite possible difficulties for future maintenance, this approach has
various advantages: 1. it is the easiest and the most accessible way to
prototype as it relieves us the necessity of familarity of Coq base,
especially for the implementation of the module and functor; 2. we have
a clear definition of trusted-base---the whole Coq; 3. it is easy to
debug---we just need to check the translated commands; 4. it can be
well-incorporated with the existent tools like VSCoq; 5. it is more
accessible for the interested audience who can then easily adapt our
plugin and give a try---otherwise they have to download and re-build the
whole customized Coq; 6. it is more stable because the surface syntax of
Coq should be stable across different versions; 7. still, this plugin
can capture the key ingredients of implementing family polymorphism
inside Coq and act as a reference for guiding an appropriate
implementation of Family Polymorphism inside all sorts of proof
assistants.\YZ{Does the Coq implementation perform type checking to
prevent illegal uses fam poly?}\EDJreply{Yes. Because I define feature of fampoly as a direct translation into Coq feature.  The fampoly is like a shorthand into multiple Coq commands. \\
Do you find translation from my current metatheory into MLTT (without linkage) helpful? \\
Actually I am also not sure what this question is aiming towards. \\
Are you complaining there is no formal specification of my fampoly feature in paper and thus there is no proof my Coq surface syntax is related to metatheory? If so, I thought our Coq plugin is only advertising how powerful Fampoly is to solve expression problem in mechanized proving. I can decouple the relationship between our plugin and our metatheory if you find the gap is really there... And I can only claim metatheory is about a partial result on incorporating fampoly with dependent type, Coq plugin is using two example to illustrate how mechanized proving can get benefit and our Coq plugin is only inspired by our metatheory (and not claim any relationship). Doing so would require rearrange the metatheory section before this section. \\
But I don't think the gap is there at all because if you want all possible gaps to disappear, then the action of using debruijn is wrong because we need to prove the one using debruijn and the surface syntax that is not using debruijn are ``equivalent''. What's more, module is not sigma type because Module is not a term in Coq and thus there is no formalism of Module at all. Let alone I don't see the plausibility of mechanizing dependent type without debruijn. \\ 
Please comment back so that I know what corresponding modification is needed on the paper. 
}\YZreply{My original question was more directed towards the fact
that currently type checking is deferred to the Coq base. Generally,
when one uses a statically typed language as a compilation target, they
also want to have type checking at the source level, (1) because there
could be type errors at the source level that could not otherwise be
caught at the target level and (2) because it allows better error messages.
So it needs to be addressed why deferring type checking to Coq is OK.}\EDJreply{Good point. I think (1) is largely mediated by the fact the semantic of family is very close to that of the module. And the supporting feature is simple enough that there won't appear some problems like type error at source-level but not target level(2) is happening for sure but some basic error message is almost "transparent" i.e. the error-messages from Coq can be directly interpreted for the user. But my justification is solely empirical and as you know the empirical study for this paper is not really strong enough to support anything.}


\textbf{Family compiled into module.} The main idea of our plugin is to support family by compiling
family components into Coq modules, family ``types'' into module types, and the context of a given term
into parameters of the module. We need to faithfully
reflect the \textit{late binding} nature of families: for each defining
field $ .. \cL\sigma \vdash t : T $, we need to compile $t$ using
``universal-quantifier-wrapped term'' : $.. \vdash \lambda t : \forall
\cL \sigma. T$. However, instead of using Coq's universal quantifier, we
choose module and module parameters to achieve this wrapping---each
family type in the context will be one module parameter
with the compiled module type, and thus each field of the family will
compile into a functor parametrized by its context. Doing so we can also
get rapid feedback from type checking when the users are defining each
field interactively with Coq.
% Insert one pseudo-code example for explanation of the mechanism
\begin{figure}[!htb]
  \begin{minipage}[t]{0.30\linewidth}
\begin{lstlisting}[language=Coq, escapeinside={@}{@}]
Family A.
Field a : nat := 1.
Field b : a = a := eq_refl.
Final Field a' : nat := 1. 
EndFamily.
\end{lstlisting}
  \end{minipage}%\YZ{Can the self's be omitted?}\EDJreply{Yes. But in our current plugin implementation it is not omitted. I think altering example is fine. If you think some small distance between example and the current implementation is fine, please comment back and I will remove all the self_}\YZreply{If omitting the self's poses no technical challenges, then I'd say it's fine to also omit them in this figure.}\EDJreply{Omitting self_ is easy and not much engineering effort is required. But omitting self_ will lead to `A.a' instead of `a' actually. So if we really want `a' as you last time mentioned, then a resolution is required and more engineering effort is required. But I don't see any technical difficulties. I have removed all the self, please check.}
  \begin{minipage}[t]{0.35\linewidth}
\begin{minted}[fontsize=\footnotesize]{Coq}
Module a_4 (self_A: EmptySig).
Definition a : nat := 1. End a_4.
Module Type a_5 (self_A: EmptySig).
Parameter (a : nat). End a_5.
Module Type a_6 := a_5 EmptyMod.
\end{minted}
  \end{minipage}
  \begin{minipage}[t]{0.3\linewidth}
\begin{minted}[fontsize=\footnotesize]{Coq}
Module b_7 (self_A: a_6).
Definition b 
  : self_A.a = self_A.a 
  := eq_refl. End b_7.
  
Module A. (* Final Aggregation *)
Include a_4. Include b_7. End A.
\end{minted}
  \end{minipage}
  \caption{Example Code and Plugin Translation}\label{fig:plugin-example1}
\end{figure}


% Concrete details of compilation of a family term data structure
Taking \cref{fig:plugin-example1} as an example: immediately after the user inputs the line "Field a : nat := ..", our plugin will
translate this statement into the "Module a_4", "Module Type a_5, a_6"
three components.
{The "EmptySig" is just an empty module type.} Then,
these modules and module types will be generated and type-checked by Coq
immediately. The compiled "Module a_4" is then part of the context of
"Field b" (as the module type of "self_A") for "b" to refer to. The
internal representation of family "A" is the list of these
compiled modules and module types. We can achieve incremental
checking as in \ref{chg:software-engineering}; overriding is also directly supported---for example, if we
want to override "Field a", we simply replace that compiled module with
the new module inside the list. This compiled module type serves as an
abstraction that enables overriding. 



This is achieved by a complete compilation from a family to a module,
via aggregating the list of the modules in the internal representation:
the plugin will repeatedly using Coq's "Include" command as the "Module
A" as \cref{fig:plugin-example1} illustrates.
References to family "A" outside "A" will be compiled to references to
this Coq module "A".

Note that, in our example, we include two
functors without arguments---this is due to a peculiarity of Coq's
"Include": uninstantiated parameters will be automatically
instantiated by the appropriate fields in the surrounding defining
module.
For example, "b_7" requires a module of type "a_6". Inside module "A",
when "Include a_4" is done, we will have a field \mintinline{Coq}{a : nat}
inside the current surrounding "A". Then when we
\mintinline{Coq}{Include b_7} without specifying the module parameter,
Coq will try to find fields in the surrounding module (i.e., module
"A") to instantiate \mintinline{Coq}{self_A: a_6}. This is satisfied by
the \mintinline{Coq}{a : nat} that is included earlier.
The resulting compiled module is just like mundane Coq's module and 
thus maintains all the computational mechanism from Coq, fulfilling
requirements in \ref{chg:software-engineering}.\YZ{
  Do I understand correctly that an alternative to individually
  compiling family members to modules is to compile a family as a whole,
  but this alternative would not satisfy the "incremental type checking"
  or "instant feedback" requirement?
}\YZreply{
  Looks like individually compiling family members addresses not only
  this requirement, but also the "self vs. consistency" challenge.
}



% explaining inheritance
\textbf{Implementing Inheritance.}
To simplify the implementation, we can consider all families as
inheritance and the standalone families are extending the empty family.
Thus, we need to deal with only inheritance during interactive theorem proving.


Simiarly, we
use a list to encode the three kinds of inheritance data in the
implementation. The inheritance data will include the
information of (1) the ``type'' of the parent family, (2) a list of
operation indicating how each field of the parent will be dealt with
(either inherited or overriden), (3) the operations indicating newly
extended fields, and (4) the ``type'' of the children family.
The surface syntax for the three kinds of inheritance
are demonstrated in the exemplar \cref{fig:plugin-example3}. The
implementation of overriding, inheritance, and extension is achieved by
simply manipulating the internal representation of a family
correspondingly---we can swap the module in the list for overriding,
retain the module in the list for inheritance, and add new module into
the list for extension. The compilation from a family (internal
representation) to a module still acts the same.

Though inheritance is solely a first-order data rather
than a ``mapping'', it is still possible to \text{mixin} two inheritance
data---by carefully ``mix'' the internal list, we can compose the
inheritance. However, a good definition of mixin is still under
investigation. We will show one practical example of \textit{mixin}
later.\YZ{How about making mixin composition one of your Req's? Also
does any example in the paper illustrate how to do this?}\EDJreply{mixin
is not really formalized. There is no example illustrate it. The only
example is the STLC example at the very end and I just kind of
``mention'' it. I think here I just want to emphasize, inheritance as
first order data is still possible to support composition.}

Our implementation also makes sure the first attempt of defining "c"
failed successfully in \cref{fig:plugin-example3}---proving "c"
would require a concrete definition of "a", and our plugin rejects
\mintinline{Coq}{c : a = 1 := eq_refl} in a context where we
only know \mintinline{Coq}{a : nat}.



\textbf{Implementing "Final" and "Sealed Family".}
We implement "Final" by
exposing the whole definition directly into the compiled module type.
Doing so will prohibit overriding. Say, "bop", declared "Final" in
\cref{fig:plugin-example3}, is not overridable.
We require the users to manual decorate field as `Final' so that the
plugin will proceed with this special treatment.
%\YZ{This para is confusing: which kind of equality is used, propositional or judgmental? Where does 'Final' show up in the example?}\EDJreply{I rephrase it and remove the part on talking about meta-theory. Now everything is about judgemental so I don't have to emphasize judgemental. I wanted to make the example small but I guess I can add one `Final' statement into our example. Don't resolve it until I add it into Figure 2.}

% \textbf{Special Fields: Overridable/Sealed Family Simulating Sigma Type.} 

The implementation of sealed family is not so different from that of the normal family, but we need to deal with the compiled module type. The compiled module type is the aggregation of the type instead of the concrete definition of the fields.
For example, the compiled module type of
"MonotonicF" in \cref{fig:plugin-example3} will be "{f : nat -> nat; p : monotonic f}". Thus any future overriding family will always have these two fields "f, p" with the corresponding type.\YZ{Can an overriding, sealed family define new fields not in its parent?}\EDJreply{I don't find this a problem but I didn't support it in my plugin. }


% Basically, {a : int -> int; property : a is monotonic}, in this case
%   we may want to override "a" using different computation, but we still wnat
%   this refinement/property




\textbf{Special Fields: ``Extensible'' Inductive Type and Defining Recursor.}
We support extensible inductive types in Coq mentioned in
\ref{chg:extensible-inductive-type}, but we will not be aiming at the
research of its formal semantic.\footnote{Because to give a satisfactory
formal definition and semantic for extensible inductive type is not
something this paper aims at, and obviously requires more theoretical
research effort}
%\YZ{I do consider that your plugin supports extensible inductive types. It's just that they are implemented as a plugin, rather than in the Coq core.}\EDJreply{I rephrased it. It is more like I don't think this is really the final form of extensible inductive type that everyone is happy about}

% \begin{figure}[!htb]\YZ{Change this one into STLC example, to see the layout}
%   \begin{minipage}[t]{0.32\linewidth}
% \begin{lstlisting}[language=Coq,  escapeinside={@}{@}]
% Family B.
%   FInductive b : Set 
%     := tt : b | ff : b.
%   Family neg_handler. 
%     Field tt : self_B.b 
%       := self_B.ff.
%     Field ff : self_B.b 
%       := self_B.tt.
%   EndFamily.
%   FRecursor neg 
%     about self_B.b 
%     motive (fun _ => self_B.b)
%     using self_B.neg_handler
%     by _rec.
%   Field example := 
%     self_B.neg self_B.tt. 
% EndFamily.
% Family B2 extends B.
%   Extend FInductive b : Set 
%     := uu : b.
%   Extend Family neg_handler.
%     Field uu : self_B2.b 
%       := self_B2.uu.
%   EndFamily. 
% EndFamily.
% \end{lstlisting}
%   \end{minipage}
%   \begin{minipage}[t]{0.65\linewidth}
% \begin{minted}[fontsize=\footnotesize,escapeinside=@@]{Coq}
%   (* Abstraction for Inductive Type *)
% Module Type b_3 (self_B: EmptySig_4).
% Parameter (b : Set). Parameter (tt ff : b). End b_3.      
% Module Type b_6 := b_3 EmptyMod.
%   (* Recursor to Set for b *)
% Module b_rec_12 (self_B: b_6).
% Definition __recursor_type_b_rec :=
%   forall P : self_B.b -> Set,
%   P self_B.tt -> P self_B.ff 
%   -> forall __i : self_B.b, P __i. End b_rec_12. 
%   (* Field B.neg_handler.ff *)
% Module ff_24 (self_B: b_6) 
%              (self_neg_handler: tt_23 self_B).
% Definition ff : self_B.b := self_B.tt. End ff_24.
%   (* Intermediate Module solely for type checking *)
% Module v_33_34 (self_B: neg_handler_26).
% Include b_rec_12 self_B.
% Parameter (recursor_for_type_checking : __recursor_type_b_rec).
% Definition term_for_type_checking :=
%   recursor_for_type_checking (fun _ : self_B.b => self_B.b)
%     self_B.neg_handler.tt
%     self_B.neg_handler.ff.  
% End v_33_34.
% \end{minted}
%   \end{minipage}
% \caption{Example Code for Inductive Type}\label{fig:plugin-example2}
% \end{figure}


\begin{figure}[!htb]
  \begin{minipage}[t]{0.32\linewidth}
\begin{lstlisting}[language=Coq,  escapeinside={@}{@}]
Family STLC.
(* ... *)
  Family subst_internal. 
  Final Field tm_var : ...
:= fun s x t => 
    if (eqb x s) 
    then t else (tm_var s).
  Final Field tm_abs : ...
:= fun s (b : tm)   
   (recb : id → tm → tm) 
   x t => 
    if (eqb x s) 
    then (tm_abs s body)
    else (tm_abs s (recb x t)).
(* ... *)
  EndFamily.
  FRecursor subst 
    about tm 
    motive (fun _ => id → tm → tm)
    using subst_internal
    by _rec.
(*
Field test := (subst (tm_var 0) 0 (tm_var 0)).
*)
(* ... *)
EndFamily.
Family STLC_bool extends STLC.
(* ... *)
Extend FInductive tm : Set :=
  | tm_true : tm ...

Extend Family subst_internal.
Final Field tm_true : ...
  := fun x t => tm_true.
(* ... *)
EndFamily. 
Inherits subst.
(* ... *)
EndFamily.
\end{lstlisting}
  \end{minipage}
  \begin{minipage}[t]{0.65\linewidth}
\begin{minted}[fontsize=\footnotesize,escapeinside=@@]{Coq}

Module Type STLC_409.
  Parameter (tm : Set).
  Parameter (tm_var : id → tm).
  Parameter (tm_abs : id → tm → tm).
  (* ... and all the fields defined before
  and the abstraction for inductive type ...*)
End STLC_409.
(* Collect eliminator to Set for tm *)
Module tm_rec_425 (STLC: STLC_409).
  Definition __recursor_type_tm_rec :=
    forall P : STLC.tm -> Set,
    (forall n : id, P (STLC.tm_var n)) ->
    (forall (n : id) (i : STLC.tm),
    P i -> P (STLC.tm_abs n i)) ->
    ... -> forall i : STLC.tm, P i.
End tm_rec_425.
(* Compiled Field STLC.subst_internal.tm_var *)
Module tm_var_411 (STLC: STLC_409)
  (subst_internal: EmptySig).
  Definition tm_var :
    forall (s : id) (x : id) (t : STLC.tm), STLC.tm :=
    fun s x t ⇒ if eqb x s then t else STLC.var s.
End tm_var_411. (* ... and more *)
(* Aggregate into one subst_internal*)
Module subst_internal_410 (STLC: STLC_409). 
  Include tm_var_411 STLC. (* And other cases ... *)
End subst_internal_410.
Module v_33_34 (STLC: STLC_409).
(* Intermediate Module solely for type checking *)
  Include tm_rec_425 STLC.
  Parameter (recursor_for_type_checking : __recursor_type_tm_rec).
  Definition term_for_type_checking :=
    recursor_for_type_checking (fun _  ⇒ id → STLC.tm → STLC.tm)
      subst_internal_410.tm_var
      subst_internal_410.tm_abs ... .
End v_33_34.
\end{minted}
  \end{minipage}
\caption{Exemplar STLC Code, especially about Inductive Type}\label{fig:plugin-example2}
\end{figure}



% We start with the programming interface. We use "FInductive" to define an extensible inductive type and in the children family we use "Extend Finductive ..." to extend the type with new constructors. In \cref{fig:plugin-example2}, we make the example of boolean and three-valued boolean. 



% Recall, to construct a recursor, we need to create for each
% constructor a handler (function) and aggregate them into a family (e.g. "subst_internal" in \cref{fig:plugin-example2}). We use "FRecursor" to declare a recursor (e.g. "subst") by specifying the motive, the aiming inductive type and the handler family. Once recursor is created it will look like any other function fields (e.g. "subst"). Here in \cref{fig:plugin-example2} we can apply the function "subst" to "tm_var 0" as a new field. The inheritance of recursor is mainly delegated by the inheritance of the handler family---if the user extend with new constructor (e.g. "tm_true"), then the inherited family (e.g. "subst_handler") will need to extend correspondingly (e.g. "subst_internal.tm_true"). If not, the plugin will error with ``Non-exhaustiveness Pattern'' when inheriting the field "subst".\EDJ{This paragraph is duplicating but I am not sure how to make it concise.}



Let's look at the exemplar \cref{fig:plugin-example2}, the example of
"subst" on STLC. When we define the extensible inductive type "tm", all
the following fields (e.g., "subst_internal.tm_var") can only know "tm"
is a type (of "Set") and there are at least two constructors for it,
just like what we declare in ``the abstracted interface''---the compiled
module type "STLC_409". We cannot export
%\YZ{export or put?}\EDJreply{Ok.
%I just figure out how my wording is confusing. put is absolutely correct
%but by put I want to mean ``export''. I think I use ``export'' in
%several places...}
the eliminator of the inductive type "tm" to "STLC_409".
Otherwise all the subsequent fields (e.g. "subst_internal.tm_var" and
the corresponding "tm_var_411") would not be able to be inherited to the
context where "tm" is extended with a third constructor, \textbf{since
the mere existence of the recursor will assert that there are only two
constructors} in "STLC_bool", not \textbf{extensible} at all!

However, we later still need the concrete eliminator on "tm" for doing
exhaustiveness checking (as in \ref{chg:extensible-inductive-type}),
thus we will store the corresponding info somewhere.  Here we show one
exemplar extraction of the eliminator to \mintinline{Coq}{Set} as module
"tm_rec_425". This concrete eliminator will be used by our "FRecursor".
For example, in \cref{fig:plugin-example2}, "FRecursor" command will
feed Coq a module "v_33_34" for doing type-checking with specified
handlers (e.g. "neg_handler") during compilation---what it does is
basically
assert the existence of the recursor and apply to see if handlers can
pass the type checking by Coq.\YZ{
  Sounds like we can still claim the plugin is doing type checking.
  It's just that the checking is done by generating some Coq code and
  then having Coq check it.
}
Note that, at this point of the example,
"FRecursor" will only do type check. The compilation of "FRecursor" into
a component of the compiled module only happens when closing the whole
family, by inserting the following into the compiled module: 
\begin{minted}{Coq}
Definition subst := tm_rec (λ _ ⇒ id → tm → tm) subst_internal.tm_var ...
\end{minted}
where "tm_rec" and "tm" are the corresponding recursor and
the (vanilla) inductive type in that compiled module. Thus a "FRecursor"
field is unlike other fields, even though it is exposed as a mundane
function for the following fields in the programming interface.\YZ{
  Why is this special treatment needed for compiling FRecursor?
}

The plugin actually realizes the "Extend FInductive" via overriding---%
when extending an inductive type $A$ with new constructor $c$, our
plugin will generate a new (syntactic) inductive definition $A'$ with
this new $c$ on the fly and feed it to Coq. But we must be careful on
the exposing data in the family type (the compiled module type) for the
following field. In other words, we must be careful on what interface
(context) the following fields are defined based on. 

Importantly, the recursor field (e.g. "subst") is not actually
inheritable, because it is not compiled like any other fields.
Recall how other fields are compiled into reusable pieces of
(parametrized) modules; a recursor field is not compiled like that.
For all the children family, we have to reconstruct and override the
"FRecursor" field with a ``new'' recursor using (maybe the same) handler
family.  This is affordable since each aggregated recursor handler
doesn't need to be rechecked---look at "STLC_bool.neg_handler", there
are only  new handlers for "tm_true" and the handlers (mundane fields)
from the parent "B.subst_internal.tm_var" have already been type-checked
and inherited (since they only rely on ``the abstracted interface''
module "STLC_409"). 
Thanks to family inheritance, their corresponding compiled module from the parents---for example, 
the module "tm_var_411"---doesn't require another type-check and is
re-used during the type-checking and compilation of "STLC_bool.subst".
For example, for the new "subst", when compiling family "STLC_bool" into
a module, at the appropriate place our plugin will insert
\begin{minted}{Coq}
Definition subst := tm_rec (λ _ ⇒ id → tm → tm) subst_internal.tm_var ...
\end{minted}
where "tm_rec" and "tm" are the corresponding recursor and the (vanilla
but extended and modified) inductive type in that compiled module. Note
that, even though "tm_rec" and "tm" has the same name as before, they
are in the newly-compiled module so they refer to totally different
things. 





Of course, in the implementation, our plugin will ``inherit'' the
recursor "neg" in the surface,
but what the plugin actually does is doing a second exhaustiveness
checking in children family like mentioned above, with the ``same'' motive and the ``same'' handler family.
%\YZ{A source of confusion in this section is that the reader often has to guess who performs the actions: is it the programmer or the Coq plugin?}\EDJreply{Good idea. Let me reorgnaize this section to ``first programming interface'', then ``plugin implementation''}



 

% Insert one pseudo-code example for explanation of the mechanism
% Use the natural number example




% Explain why the module type need extra care

% explain how recursor is constructed
% explain we have the incremental-checking for any recursor 

% explain ftheorem wrapping this complicated recursor
This design of decoupling of the case handlers and exhaustiveness checking handles the semantic well, however, it brings a lot of boilerplate code---we need to create an extra (handler) family, manually specifying the type of each case handlers, and then using "FRecursor" to ``tie the knot'' and construct a recursor field from the handlers.  
We thus provide a command "FTheorem" to avoid these boilerplate in programming interface when using tactic programming. 
This command (1) can avoid specifying the handlers family and the type of each case handlers when writing induction like above, (2) invoke
proof interaction mode and thus allow tactic programming, (3) and is also open to extension like "FRecursor".\YZ{example?}\EDJreply{FTheorem is really just a short hand for "FRecursor" + handler family. Nothing more conceptual here. If the reader is interested in the concrete syntax, I think I need to make the two examples avaliable in the appendix.} We expect this "FTheorem" to be used in theorem proving but not general programming. 

To summarize, (1) to make each recursor
handler inheritable, we need to seal an abstraction around the inductive
type (e.g., module "b_3"); but to construct a recursor (and carry out the
exhaustiveness checking), we need to break this abstraction and see the
concrete definition of the inductive type (e.g., module "b_rec_12"). Our
meta-theory need to handle these two seemingly contradicting ideas
simultaneously when solving \ref{chg:extensible-inductive-type} . (2) \ref{chg:extensible-inductive-type} is also resolved by this \textit{decoupling} of the syntax of implementation of recursor handlers and exhaustiveness checking. The former is handled by family inheritance and thus avoiding the boilerplate code; and the non-inheritance of the compiled recursor ensures the exhaustiveness checking happens for every recursor.  


% Mention still needing total recursor, for exhaustiveness checking

\textbf{Hack: Global Reasoning.}\YZ{This is more like an escape hatch. Escape from family poly into vanilla Coq. Maybe these two hacks should be grouped into one section}
% explain the places using ``Closing Fact''
% 1. No need to export partial recursor
% 2. Computational rule for the total recursor
Unfortunately, there are cases an extensible proof brings more boilerplate code. For example, in STLC, we expect
``\mintinline{Coq}{~ value (tm_app x y)}'' to hold in all future
extensions of STLC---applications will never be considered as values
by any extension.
We know the correct way to prove the proposition
\mintinline{Coq}{value_not_app : value (tm_app x y) -> False.} is to use
"FRecursor" to inductively reason "value (tm_app x y)". However, this
means that every time we extend "value" with new value forms, we will
need to extend corresponding inductive cases for "app_not_value". We can
imagine the newly extended proof to be boring case analysis that the
newly added value forms are not "tm_app". What's worse, similar
scenario can happen on other statements like \mintinline{Coq}{~ value
(tm_if cond x y)}. Everything here is extended correctly but we end up
having boilerplate code, \textbf{semantically}.
\begin{figure}[!htb]\YZ{Make this part concise. (Just say this is a continuation of the earlier STLC example)}\EDJreply{Done}
\begin{lstlisting}[language=Coq,  escapeinside={@}{@}]
Family STLC.
  FInductive tm : Set := ... 
  FInductive value : tm -> Prop := ... 
  Closing Fact value_not_app : forall x y, ~ value (tm_app x y) 
      by { intros x y H; inversion H; eauto }.
  Fail Ltac inv := 
    match goal with 
    | [h : value (tm_app _ _) |- ] => destruct (value_not_app _ _ h)
    (* Fail: value, tm_app, value_not_app unfound *)
    end. 
  MetaData tactic1.
    Ltac inv := (* ... the same definition as above ... *)
  EndMetaData.
  (* ... STLC example continue ... *)
EndFamily.
\end{lstlisting}  
\caption{Example for Global Reasoning and MetaData, using in STLC example}\label{fig:plugin-example-global-reasoning-meta-data}
\end{figure}


Our proposal is to allow global reasoning by providing "Closing Fact" command. Ultimately, "Closing Fact" is asserting a constraint upon all the
possible extension of the surrounding family and claiming that a proof script
can solve the constraints.\YZ{This is a good one-liner that summarizes the intended usage scenarios of "Closing Fact". Should probably bring it out earlier.}\EDJreply{I put it here as you suggested but I think this one-liner at here will cause more confusion because it is too abstract.}
The assertion will only be verified when the whole family is closed and compiled into a module.

% We provide "Closing Fact" as a command to
% carry out this global reasoning, attached with a proof script and proof
% term. This "Closing Fact" will look like an assertion during the
% construction of the enclosing family. This assertion will only be
% verified when the whole family is closed and compiled into a module,
% with the help of the attached proof script and proof terms.

Here we show the example to prove "value_not_app" in
\cref{fig:plugin-example-global-reasoning-meta-data}, the script inside
curly brackets after "by" is the attached proof script that will be
executed upon "EndFamily", which triggers the compilation of the family
being defined.
Since we are dealing with a concrete (vanilla) inductive type during compilation,
we can use Coq's "inversion" tactic to complete this proof.
This "Closing Fact", as an assertion, will have no problem being
inherited, but the proof script will be executed every time a derived
family is compiled.

The downside is that "Closing Fact" cannot immediately check the correctness of the
attached proof term/proof script because this verification only happens when we
finished defining a family, and thus a bit harder to handle.\YZ{'harder' in what way? I suppose you mean the "Fail Ltac" in Figure 5?}\EDJreply{The proof script cannot be run immediately to reply the user whether the proof script is even correct. This is bad engineering to some extent. In those bad cases, because error only happens at the the very end ``EndFamily'' , the user will need to go back to the place of Closing Fact to adjust the proof script, or to adjust other parts. I am not sure if early check is possible to be done (engineering-wise) I think it is possible (imagine I just do an early compilation, i.e. only compile part of the family to the point closing fact is used) but it is a big engineering difficulty for me.}
We expect it to be used only sporadically for maintainability, and as an alternative to using "FRecursor". This is called global reasoning because this is a post-hoc reasoning when the family as a \textit{whole} is compiled into a module.



% However, to verify and prove the recursor and partial recursor and their
% computational rules (as in \ref{chg:definition-relevant-reasoning}), our plugin has to
% compile the extensible family into a non-extensible Coq Module and define
% the recursors by instantiating the content of Coq module with concrete
% inductive types. We provide "Closing Fact" as a command to carry out this global reasoning, 

% This style of ``Global Reasoning'' says that we can prove the
% corresponding theorem only when we know the ``full picture'', in particular,
% all the constructors of an inductive type. 
% We provide "Closing Fact" as
% a command to carry out this global reasoning

%\YZ{Are recursors and partial recursors automatically generated and proved by the plugin, or must they be asserted and proved by the programmer?}\EDJreply{(1) Closing Fact is not only used to generate recursor and partial recursor. I think the intro of this section is just giving people wrong impression. So I rewrite the whole section now. (2)  Recursors and Partial recursors are possible to be automatically generated but my plugin didn't implement it. Current stage is that user only need to assert them. Nobody has to prove it (plugin will prove it). Here when I say the user has to provide proof script, it means for arbitrary proposition if the user want to carry out global reasoning.}

% ---we can attach to "Closing
% Fact" either (proof) terms or Coq's proof script, then only when
% compilation to module happens, is the term/proof script
% type-checked to see if a "Closing Fact" statement can be proved. Note
% that, "Closing Fact" cannot immediately check the correctness of the
% term/proof script because this verification only happens when we
% finished defining a family, and thus a bit harder to handle.

% This "Closing Fact" also can ease us from the necessity to let plugin
% generate proof for recursors and partial recursors and their
% computational rules (as in \ref{chg:definition-relevant-reasoning}), and thus making
% plugin development easier.
%\YZ{This reads like a bad excuse: the programmer does not care about if the plugin development is easy. Question is if it makes the programmer's life easier.}\EDJreply{You are right. It does sound like a bad excuse. I rewrite the whole section now. Closing Fact is initially used when the user wants global reasoning instead of extensible proof by induction, for example, the inversion lemma. \\ It just happens that they can be used by me so that I can postpone generation of these rule plugin (which is hard for me to do). This paragraph is just me being honest that my plugin is not mature enough to generate everything.}
% For example, the computation rule for a
% concrete recursor like "B.neg" and partial recursors in
% \cref{fig:plugin-example2} can be directly asserted by using "Closing
% Fact" and their proofs verified later during compilation.
%\YZ{What is the alternative if not using "Closing Fact"?}\EDJreply{(1) For partial recursor and recursor, that should be generated by plugin (2) For general global reasoning, I don't know. There must be some other ways in the literature. }\YZ{Still not entirely clear to me what is gained and what is lost by using "Closing Fact".}\EDJreply{Please check.}
% This ``Global Reasoning'' will be again useful in later examples.

\textbf{Hack: ``Meta-data''.} Due to our implementation of the family,
the family data is invisible to Coq internal and thus causes some
inconvenience. For example, in
\cref{fig:plugin-example-global-reasoning-meta-data} the first attempt
of defining tactic "inv" fails. Because the earlier defining "tm",
"value", and "value_not_app" are currently just data structure in the
plugin and not visible for Coq internal (only visible after the family
is compiled into module). 

Thus to refer to them, we have to scope the defining tactic with
"MetaData" and our plugin will set up the appropriate environment. After
a "MetaData" block is closed, the plugin will aggregate the content into
the family being defined so that they can also be used when defining
the following fields of the family.\YZ{It reads like this MetaData wrapping could be automatically inserted, no?}\EDJreply{Not sure. This auto-insertion will practically look like hijack the original Coq command. I am not sure this kind of hijacking can happen.}
This can fulfill \ref{chg:software-engineering} and define new customized tactic.


% To fulfill \ref{chg:software-engineering}, we need somehow to aggregate tactic inside compiled module type, due to the implementation of our family---when defining every field of 

% For example, 

%   For example, currently it is not possible to define a
% vanilla inductive type dependent on our ``extensible'' inductive type
% due to the invisibility, unless we compile the whole family. Similarly,
% customized tactic expressions as required by \ref{chg:software-engineering} cannot
% use lemmas or theorems in the fields of the current defining family as
% well.

% To remedy this, we support a "MetaData" command. This command
% bundles any original Coq primitive (in the current family
% context) into the compiled module type, and makes them visible to the
% following defining field. Doing so, we can define customized tactic as
% in \ref{chg:software-engineering} that uses lemmas proved in the family.


% Following needs a rewrite

\textbf{Propositional Partial Recursor.} Notice that, for both "FRecursor" and "Closing Fact", they are not necessarily ``\textit{conservative extension}''---they will possibly disrupt arbitrary extension of inductive type. For example, for \cref{fig:STLC-example}, once we proved \mintinline{Coq}{var_not_value : ∀ i, ~ value (tm_var i)} in the "Family STLC" either using "FRecursor" or "Closing", then any children family of "STLC" cannot extend a new constructor \mintinline{Coq}{vvar: ∀ i, value (tm_var i)}. 

This raise an interesting question: what propositions are conservative extension? For example, apparently we know injectiveness of constructor, (i.e. \mintinline{Coq}{STLC.tt ≠ STLC.ff}) must be conservative because it holds for all extension of the inductive type "STLC.tm". But is there a \textbf{``best'' proposition} (i.e. the proposition that everything the proposition derives are exactly conservative extension)?

Since injectiveness of the constructors are derived by recursor in vanilla inductive type, an educated guess would be related to recursor. But we know the original recursor won't work because the mere existence of the original recursor will break extensibility. Thus we need to modify it.
The simplest idea is to allow partiality---i.e., returning
\mintinline{Coq}{option T} instead of the original \mintinline{Coq}{T}.

We call this \textit{propositional partial recursor}.\YZ{What is propositional about partial recursors? I suppose it has to do with the computational axioms?}\EDJreply{Yes. I want to emphasize the equality used by computational axiom is not judgemental equality. So Coq cannot automatically compute/reduce when a partial recursor applied to a term. This reduction can only be semi-automatically done by using [rewrite tactic and the computational axiom].}
For example, the propositional partial recursor for \mintinline{Coq}{B.b} is\YZ{Mention that partial recursors can be automatically generated}\EDJreply{Added below please check.}

\begin{minted}[escapeinside=@@]{Coq}
(* Inside Family B *)
FInductive b : Set := tt : b | ff : b.
b_prec : forall R, option R -> option R -> b -> option R.@\YZ{Not clear what it takes to define/prove partial recursors. The reader needs to be told if they require breaking abstraction like recursors do.}\EDJreply{What do you mean? recursor doesn't break abstraction but break extensibility. Below mentioned partial recursor doesn't break extensibility}@
\end{minted}
with appropriate computational axioms. This partial recursor differs from original recursor only by decorating an "option" with the return type thus possible to be automatically generated (and the case is similar for those computational axiom). The partiality doesn't break
extensibility (i.e. all future extensions of "b" can support this
"b_prec") because "b_prec" can just return "None" when bump into extended constructor in the future. 

The first thing to check is that, if partial recursor is theoretically enough for injectiveness of the constructor. The rough idea to achieve this
is to use "b_prec" to reflect our special inductive type into vanilla
inductive type, and use the original "injection" and "discriminate"
tactics. For example, we want to prove \mintinline{Coq}{B.tt ≠
B.ff}, we simply 
\begin{minted}{Coq}
Definition reflect : B.b → option bool 
                   := b_prec (fun _ => bool) (Some true) (Some false).
Definition tt_neq_ff : B.tt = B.ff → False.
(* Because B.tt = B.ff → reflect B.tt = reflect B.ff 
      → Some true = Some false    (by computational axiom)
      → False           (by discriminate of bool and injection of Some) *)
\end{minted}


Even though we cannot prove propositional partial recursor is \textit{the best},
the above "reflect" function implies that propositional partial
recursor is a good \textit{extensional characterization} of (non-indexed) extensible inductive type (with certain constraints)---because we can ``embed'' vanilla inductive types into
types supporting a partial recursor: "reflect" is actually a left
inverse of an injection "bool → self_B.b". A bit more formally, and
restricting our focus to non-indexed inductive types:

\begin{theorem}\label{thm:prec-complete} Given a list of $n$ pairs of types $\{ x : "A"_i \vdash "B"_i(x) \}_{i}$ s.t.


  \begin{itemize}
    \item \textlabel{(Detectable Partiality)}{prop:detectable-partiality} for each pair $x : A_i \vdash B_i(x)$,
    \begin{minted}{Coq}
      Axiom detectable:
      forall {T} {a} (f: Bᵢ a -> option T),
        {forall x, f x <> None} + {exists x, f x = None}.
    \end{minted} 
    \item Define $"C"$ as the inductive type using this list, i.e.,
    \begin{minted}{Coq}
Inductive C : Set := ... | cᵢ : forall (x : Aᵢ), (Bᵢ x -> C) -> C | ...
    \end{minted}
    \item Assume an arbitrary type \mintinline{Coq}{D : Set} with
    \begin{itemize}
      \item $n$ \textbf{functions} \mintinline{Coq}{dᵢ : forall (x : Aᵢ), (Bᵢ x -> D) -> D}  
      \item a partial recursor
      \begin{minted}[escapeinside=@@]{Coq}
prec : forall (R : Set), ..,  @\YZ{Can R be indexed?}\EDJreply{This part actually currently incorrect and needs an extra constraint. Let me fix other parts first}@
  (rᵢ : (forall (x : Aᵢ), (Bᵢ x -> option R) -> option R)), .., 
      D -> option R
      \end{minted}
      \item and $n$ (propositional) computational axioms: for all $i$, 
      \begin{minted}{Coq}
prec {R} r₁ r₂ .. rₙ (dᵢ aᵢ bᵢ) 
    = rᵢ aᵢ (fun x => prec {R} r₁ r₂ .. rₙ (bᵢ x))
      \end{minted}
    % where \mintinline{Coq}{(lift rᵢ) : (forall (x : Aᵢ), (Bᵢ x -> option R) -> option R)} is defined as expected
    \end{itemize}
  \end{itemize}
  Then there exists an \textbf{embedding} from "C" to "D".
  More concretely, we can have \mintinline{Coq}{inj : C -> D} and 
  \mintinline{Coq}{linv : D -> option C} defined using the eliminator of
  "C" and "prec" of "D" (both acting like identity) such that
  \begin{minted}[escapeinside=@@]{Coq}
    forall c : C, linv (inj c) = some c@\YZ{Does 'linv' correpsond to 'reflect' above? If so, then the reader should be explicitly reminded of this fact rather than having to guess it.}\EDJreply{Added below. I mentioned left inverse 'linv' is the generalization}@
  \end{minted}
\end{theorem}
The reason we have this weird looking \ref{prop:detectable-partiality} is because the construction of "linv"---at one point, using "prec" to construct "linv", we need to prove \mintinline{Coq}{∀ a, (Bᵢ a → option C) → option C}, generally not provable, only if we have \mintinline{Coq}{option (Bᵢ a → C)} as an argument instead. This \ref{prop:detectable-partiality} is exactly transforming former to the later: recall that, according to the semantic of inductive type, \mintinline{Coq}{(Bᵢ a → option C)} is actually the result of using "linv" on the substructure of "D". This function points out each of the translation result from "D" to "C" and some of them are "None" because of failing. For us, we only care about if \textbf{any} of the substructure of "D" is failed to translate---if any one substructure fails, then the whole transltion fails and thus we should have "None" as return. 

\ref{prop:detectable-partiality} is derivable for all the finitary branching inductive type\footnote{By simply enumerating through the finite domain "Bᵢ a"}---that includes all of the examples shown in this paper. \ref{prop:detectable-partiality} is not supported when using function (infinitary branching) in the constructors, for example, the case of using \textit{higher order abstract syntax} to represent functions in "tm". In those cases, we still have the fact that partial recursor is a conservative extension, but we cannot show that partial recursor is the extensional characterization any more。


We emphasize that in \cref{thm:prec-complete}, each function $"d"_i$ can
actually be considered as a constructor, because they can be reflected
to real constructors of "C" using the left inverse "linv" (directly generalization of "reflect : self_B.b → option bool" of the above). Thus, \cref{thm:prec-complete}
implies that (1) every future extension of the inductive type can
support this partial recursion (trivially); (2) every type supporting
this partial recursor with its computational axioms at least supports
these constructors because of the embedding. In other words, \textit{a
type (at least) supports these constructors if and only if this type
supports the corresponding partial recursor (with the computational
axioms)}.  Thus, we can argue that the partial recursor gives a sound and
complete extensional characterization of all the extension of a given (non-indexed)
inductive type.\YZ{What about indexed inductive types?}\EDJreply{Done. I also have a formalization of this proof now, for indexed type and dependent eliminator. Though the formalization is about single constructor.}

We can support partial dependent eliminator as well, but partial dependent eliminator can deduce "prec" and our "prec" is enough to contruct this left inverse for (non-indexed) inductive type. 
\begin{minted}{Coq}
  pdelim : ∀ (P : D -> Set), .., 
  (rᵢ : (∀ (x : Aᵢ) (w : Bᵢ x -> D), (∀ (b : Bᵢ x), option (P (w b))) 
    -> option (P (dᵢ x w)))),  .., ∀ (d : D) -> option (P d)
\end{minted}
To generalize to indexed inductive type, we need to base on indexed W type~\cite{martin1982constructive, morris2009indexed,jashug2017} and dependent eliminator is unavoidable. Please refer to the supplementary material and appendix for a formalization of the left inverse in the context of indexed inductive type.   


\section{\TT: A Core Dependent Type Theory}
\label{sec:metatheory2}
We contribute \TT, a core type theory that incorporates the core elements
of \underline{f}a\underline{m}i\underline{l}y polymorphism into
\underline{M}artin-\underline{L}öf dependent \underline{t}ype \underline{t}heory~\cite{martin1982constructive},
while maintaining consistency and canonicity.

Most notably, \TT extends MLTT with what we call \emph{linkages}
(a namesake of the theoretical device through which \citet{zm2017} model
family polymorphism in an OO setting).
Linkages model families, so they are like tuples but with a twist: they
support late binding.

\TT is intended as a foundational model. So unlike our programmer-facing plugin,
typing is structural rather than nominal, and
it supports family polymorphism by providing MLTT-style typing rules that
can be used to \emph{encode} core aspects of the language supported by our
plugin, rather than building them in directly.
%
For example, \TT supports extending inductive types through
formation rules that enable the encoding of the further binding of W-types~\cite{martin1984intuitionistic} in linkages.
Also unlike our plugin, \TT does not feature automations such as
the generation of propositional equalities and partial recursors,
but it is expressive enough to encode their definitions.

\begin{figure}
\small

\renewcommand*{\arraystretch}{1.25}

\definecolor{new}{HTML}{F5EFE6}

\begin{gather*}
\begin{array}{@{}rccl@{}}
\text{Contexts} & \Gm, \Dl &\Coloneqq &
    \cdot \mid \Gm, A
    \\ 
\text{Substitutions} & \gm &\Coloneqq &
    \SubstWeak{n} \mid \SubstExt{\gm}{t} \mid \SubstComp{\gm_1}{\gm_2}
    \\
\text{Types} & A, B, T  &\Coloneqq &
    \cU \mid \cB \mid \bot \mid \top \mid \TyPi{A}{B} \mid \TySigma{A}{B} \mid \El{t} \mid \sub{T}{\gm} \mid
    \\
    &&&
\tikzmarkin[disable rounded corners=true,set fill color=new,set border color=new]{New2}(0.05,-0.15)(-0.10,0.30)
    \TyWSingle{\wsig} \mid \wsigproj{j}{1}{\wsig} \mid \wsigproj{j}{2}{\wsig} \mid
    \TyLkg{\lsig} \mid \TyTkg{\lsig} \mid \lsigprojT{\lsig} \mid \lsigproj{2}{\lsig} \mid \CaseSig{A}{B}{T}
\tikzmarkend{New2}
    \\
\text{Terms} & t, s, \lkg &\Coloneqq &
    \var{n} \mid \true \mid \false \mid \codety{T} \mid \sub{t}{\gm} \mid \lam{t} \mid \app{t} \mid \pair{t_1}{t_2} \mid \fst{t} \mid \snd{t} \mid
    \\
    &&&
\tikzmarkin[disable rounded corners=true,set fill color=new,set border color=new]{New3}(0.05,-0.15)(-0.10,0.30)
    \wsingle{\wsig} \mid \wcode{\wsig} \mid \wsup{\wsig}{t_1}{t_2} \mid \LkgEmp \mid \LkgAdd{\lkg}{t} \mid \inh{h}{\lkg} \mid
    \\
    &&&
    \lkgproj{1}{\lkg} \mid \lkgproj{2}{\lkg} \mid \lsigproj{s}{\lsig} \mid \Tkg{\lkg} \mid
    \Recproj{j}{\lkg} \mid \Wrec{\lkg}{t}
\tikzmarkend{New3}
    \\
\tikzmarkin[disable rounded corners=true,set fill color=new,set border color=new]{New1}(0.05,-0.15)(-0.10,0.30)
\text{W-type signatures} & \wsig & \Coloneqq &
    \WSigEmp \mid \WSigAdd{\wsig}{A}{B} \mid \sub{\wsig}{\gm} \mid \WSigSub{\wsig}
\tikzmarkend{New1}
    \\
\tikzmarkin[disable rounded corners=true,set fill color=new,set border color=new]{New5}(0.05,-0.15)(-0.10,0.30)
\text{Linkage signatures} & \lsig & \Coloneqq &
    \LSigEmp \mid \LSigAdd{\lsig}{\seal}{T} \mid \lsigproj{1}{\lsig} \mid \RecSig{\wsig}{T} \mid \sub{\lsig}{\gm}
\tikzmarkend{New5}
    \\
\tikzmarkin[disable rounded corners=true,set fill color=new,set border color=new]{New4}(0.05,-0.15)(-0.10,0.30)
\text{Linkage transformers} & h & \Coloneqq &
    \InhId \mid \InhExt{h}{t} \mid \InhOv{h}{t} \mid \InhInh{h} \mid \InhNest{h}{h'}
\tikzmarkend{New4}
\end{array}
\end{gather*}

\begin{mathpar}
\judgebox{\goodCtx{\Gm}{i}}

\judgebox{\goodSub{\Gm}{\gm}{\Dl}}

\judgebox{\goodType{\Gm}{T}{}}

\judgebox{\goodTerm{\Gm}{t}{T}}

\judgebox{\goodWSig{\Gm}{\wsig}{n}}

\judgebox{\goodSig{\Gm}{\lsig}{n} }

\judgebox{\goodInh{\Gm}{h}{\lsig_1}{\lsig_2}}
\\

\Rule[name=sub/id]{
    \goodCtx{\Gm}{i}
}{
    \goodSub{\Gm}{\SubstWeak{0}}{\Gm}
}

\Rule[name=sub/weaken]{
    \goodSub{\Gm}{\SubstWeak{n}}{\Dl}
    \\
    \goodType{\Gm}{A}{}
}{
    \goodSub{\Gm, A}{\SubstWeak{n+1}}{\Dl}
}

\Rule[name=sub/ext]{
    \goodSub{\Gm}{\gm}{\Dl}
    \\
    \goodTerm{\Gm}{t}{\sub{A}{\gm}}
    \\
    \goodType{\Dl}{A}{}
}{
    \goodSub{\Gm}{\SubstExt{\gm}{t}}{\Dl,A}
}

\Rule[name=ty/sub]{
    \goodType{\Dl}{T}{j}
    \\
    \goodSub{\Gm}{\gm}{\Dl}
}{
    \goodType{\Gm}{T[\gm]}{j}
}

\Rule[name=tm/sub]{
    \goodTerm{\Dl}{t}{T}
    \\
    \goodSub{\Gm}{\gm}{\Dl}
}{
    \goodTerm{\Gm}{\sub{t}{\gm}}{\sub{T}{\gm}}
}

\Rule[name=tm/var]{
    \goodCtx{\Gm, A_n, ..., A_1, A_0}{}
}{
    \goodSub{\Gm, A_n, ..., A_1, A_0}{\var{n}}{\sub{A_n}{\SubstWeak{n+1}}}
}

\Rule[name=tm/lam]{
    \goodType{\Gm}{A}{}
    \\
    \goodTerm{\Gm, A}{t}{B}
}{
    \goodTerm{\Gm}{\lam{t}}{\TyPi {A} {B}}
}

\Rule[name=tm/app]{
    \goodTerm{\Gm}{t}{\TyPi {A} {B}}
}{
    \goodTerm{\Gm, A}{\app{t}}{B}
}

\Rule[name=tm/pair]{
    \goodTerm{\Gm}{t_1}{A} 
    \\
    \goodTerm{\Gm}{t_2}{\sub{B}{\SubstExt{\SubstWeak{0}}{t_1}}}
}{
    \goodTerm{\Gm}{(t_1,t_2)}{\TySigma{A}{B}}
}

\Rule[name=tm/proj]{
    \goodTerm{\Gm}{t}{\TySigma A B}
}{
    \goodTerm{\Gm}{\fst{t}}{A}
    \\
    \goodTerm{\Gm}{\snd{t}}{\sub{B}{\SubstExt{\SubstWeak{0}}{\fst{t}}}}
}

\Rule[name=ty/el]{
    \goodTerm{\Gm}{t}{\cU}
}{
    \goodType{\Gm}{\El{t}}{j}
}

\Rule[name=tm/c]{
    \goodType{\Gm}{T}{j}
}{
    \goodTerm{\Gm}{\codety{T}}{\cU}
}

\Rule[name=tm/eq/c]{
    \goodTerm{\Gm}{t}{\cU}
}{
    \goodTerm{\Gm}{\codety{\El t} \equiv t}{\cU}
}

\Rule[name=ty/eq/el]{
    \goodType{\Gm}{T}{j}
}{
    \goodType{\Gm}{\El{\codety T} \equiv T}{}
}

\Rule[name=tm/w]{
    \goodWSig{\Gm}{\wsig}{n}
}{
    \goodTerm{\Gm}{\wsingle{\wsig}}{\TyWSingle{\wsig}}
    \\
    \goodTerm{\Gm}{\wcode{\wsig}}{\cU}
}

\Rule[name=wsig/empty]{
}{
    \goodWSig{\Gm}{\WSigEmp}{0}
}

\Rule[name=tm/wsup]{
    \goodWSig{\Gm}{\wsig}{n}
    \\
    \goodTerm{\Gm}{t_1}{\wsigproj{j}{1}{\wsig}}
    \\
    \goodTerm{\Gm, \sub{\wsigproj{j}{2}{\wsig}}{\SubstExt{\SubstWeak{0}}{t_1}}}{t_2}{\El{\wcode{\wsig}}}
    \\
    j \in \{1, ..., n\}
}{
    \goodTerm{\Gm}{\wsup{\wsig}{t_1}{t_2}}{\El{\wcode{\wsig}}}
}

\Rule[name=wsig/add]{
    \goodWSig{\Gm}{\wsig}{n}
    \\
    \goodType{\Gm}{A}{i}
    \\
    \goodType{\Gm, A}{B}{i}
}{
    \goodWSig{\Gm}{\WSigAdd{\wsig}{A}{B}}{n+1}
}

\Rule[name=tm/wrec]{
    \goodTerm{\Gm}{\lkg}{\TyLkg{\RecSig{\wsig}{T}}}
    \\
    \goodTerm{\Gm}{t}{\El{\wcode{\wsig}}}
}{
    \goodTerm{\Gm}{\Wrec{\lkg}{t}}{T}
}

\Rule[name=ty/casety]{
    \goodType{\Gm}{A}{}
    \\
    \goodType{\Gm, A}{B}{}
    \\
    \goodType{\Gm}{T}{}
}{
    \goodType{\Gm}{
        \CaseSig{A}{B}{R}
        \equiv
        \TyPi{A}{\TyPi {\TyPi{B}{\sub{T}{\SubstWeak{2}}}} {\sub{T}{\SubstWeak{2}}}}
    }{}
}

\Rule[name=lsig/add]{
    \goodSig{\Gm}{\lsig}{n} 
    \\
    \goodTerm{\Gm,\TyTkg{\lsig}}{\seal}{\sub{A}{\SubstWeak{1}}}
    \\
    \goodType{\Gm, A}{T}{}
}{
    \goodSig{\Gm}{\LSigAdd{\lsig}{\seal}{T}}{n+1}
}

\end{mathpar}

\caption{Syntax and selected typing rules of \TT}
\label{fig:typing-selected}
\end{figure}

\noindentparagraph{MLTT.}

\cref{fig:typing-selected} presents the syntax and selected typing rules of \TT.
The new forms (relative to MLTT) are highlighted.
We briefly review the base MLTT fragment first.

As is standard in MLTT presentations, we use explicit substitutions~\cite{substcalculus,abadi1989subst}:
substitutions~$\gm$ and their applications (e.g., \sub{T}{\gm}) are part
of the syntax rather than meta-operations.
%
Variables are represented by de Bruijn indices:
\var{n} is the variable bound by the $n$-th closest enclosing binder.
For example, $\lambda x.\,\lambda y.\,x$ is $\lam{\lam{\var{1}}}$.
%
Typing judgments for substitutions have the form 
%\EDJ{It seems to me G |- g : D is a judgement not a rule. Rules usually have two parts (premises + conclusion). Judgment is a standard terminology invented by Martin Lof (https://ncatlab.org/nlab/show/judgment see reference here). I will suggest saying 'Formation rules for substitution lead to a judgement G |- g : D' sth like this. In pen-and-paper, the formation rule (corresponding to QIIT well-typed-ness of judgement) is called presuppose (can be found in Modal MLTT paper) }
$\goodSub{\Gm}{\gm}{\Dl}$.
The idea is that applying~$\gm$ to terms valid in the context~$\Dl$
yields terms valid in~$\Gm$ (e.g., rule \ruleref{tm/sub}).
%
The two main forms of substitutions are weakening and extension:
\sub{t}{\SubstWeak{n}} introduces $n$ free variables into the context of $t$, and
\sub{t}{\SubstExt{\gm}{t'}} substitutes~$t'$ for~$\var{0}$ in~$t$ and
then applies~$\gm$.
For example, rule \ruleref{tm/proj} states that if $t$ is a dependent pair
of type $\TySigma{A}{B}$, then $\snd{t}$ has type $\sub{B}{\SubstExt{\SubstWeak{0}}{\fst{t}}}$.

We use Tarski-style universes~\cite{hofmann1997syntax}: a universe~$\cU$
is inhabited by the \emph{codes} of types,
with $\El{t}$ decoding~$t$ (rule \ruleref{ty/el})
and $\codety{T}$ encoding type~$T$ (rule \ruleref{tm/c}).
MLTT has an infinite hierarchy of universes;
we omit universe levels in the presentation for conciseness.

\noindentparagraph{Introducing and eliminating inductive types.}

W-types~\cite{martin1984intuitionistic} are a standard way to model
inductive types in MLTT.
A constructor of an inductive type~$T$ is given by a pair of types
$\goodType{\Gm}{A}{}$ and $\goodType{\Gm,A}{B}{}$ (rule \ruleref{wsig/add}):
given two arguments $\goodTerm{\Gm}{t_1}{A}$ and
$\goodTerm{\Gm,\sub{B}{\SubstExt{\SubstWeak{0}}{t_1}}}{t_2}{T}$,
it constructs a term of type~$T$ (rule \ruleref{tm/wsup}).
As the rules suggest, our formulation additionally supports
\emph{W-type signatures}.
A well formed W-type signature \goodWSig{\Gm}{\wsig}{n} is
composed of $n$ pairs of types, each of which represents a constructor
(rules \ruleref{wsig/empty} and \ruleref{wsig/add}).

W-types are eliminated with the form $\Wrec{\wsig}{\lkg}{t}$, where $t$ is of a
W-type $\El{\wcode{\wsig}}$, and $\lkg$ is essentially an $n$-tuple of
case handlers for the $n$ constructors in~$\wsig$ (\ruleref{tm/wrec}).
Each case handler has a type of the form $\CaseSig{A}{B}{T}$, where
$T$ is the motive of the recursion (\ruleref{ty/casety}).
This collection of case handlers then encodes those defined or inherited
by an \lsti{FRecursion} command in our plugin.
We choose to type it with a linkage type $\TyLkg{\RecSig{\wsig}{T}}$
to avoid introducing $n$-tuples, which linkages generalize.

Rules \ruleref{tm/wsup} and \ruleref{tm/wrec} require knowing~$\wsig$:
the W-type is exhaustively generated by its constructors, and
its elimination must exhaustively handle all the constructors in
its signature.

In contrast, $\wsig$ should be hidden from the typing context of any
term that does not invoke $\Wsup{\wsig}{\cdot}{\cdot}$ or $\Wrec{\wsig}{\cdot}{\cdot}$, so that
the term can be reused---without being rechecked---for a different
W-type signature~$\wsig'$ that extends~$\wsig$ with additional
constructors.
Moreover, the typing of the term should be made polymorphic to
the definitions of those fields that invoke $\texttt{Wsup}$ or $\texttt{Wrec}$,
so that the term can be reused---without being rechecked---when those fields
are overridden to support the extended signature~$\wsig'$.



\newpage

To formalize our metatheory, we follow the formulation of~\citet{altkap2016}, which provides a declarative and intrinsic style of the predicative \textit{Martin-Lof Type Theory} (MLTT)~\cite{martin1982constructive},
using \textbf{Quotient Inductive Inductive Type}(QIIT)\footnote{Inductive Inductive Type(IIT) is a generalization of mutual inductive type where (indexed-)types \mintinline{Coq}{A : Type, B : A -> Type} can be defined mutually. QIIT further enhances IIT by allowing equality constructor (i.e. mathematical quotient).}, \textbf{explicit substitution} and \textbf{debruijn indices} as the main tool. 
Generally speaking, we \textbf{use each QIIT type to represent each kind of judgements, and thus a term will represent a derivation of a judgement}. The classical formulation of dependent type theory requires a lot of \textit{presupposition} and quotient afterwards, causing a lot of duplicacy. Both issues can be concisely solved by using QIIT expressing intrinsically typed syntax instead. What's more, working \textbf{in a type-theoretical setting} using QIIT makes checking of the pen-and-paper proof easier. However, we will follow conventional notation and thus the reader might not notice the usage of QIIT explicitly.\EDJ{We might be able to remove QIIT in the main text totally. Let's try to do it later.}

$$
\Rule[name=Subst]
{\goodType{\Delta}{T}{j} 
  \quad {\goodSub{\Gamma}{\gamma}{\Delta}}}
{\goodType{\Gamma}{T[\gamma]}{j}},
\Rule[]
{\goodTerm{\Delta}{t}{T}
  \quad {\goodSub{\Gamma}{\gamma}{\Delta}}}
{\goodTerm{\Gamma}{t[\gamma]}{T[\gamma]}}
\quad
\Rule[name=Dbj Indices]
{}
{\goodSub{\Gamma, A}{p^1 \gamma}{\Gamma}},
\Rule[]
{}
{\goodSub{\Gamma, A}{"var"_0}{A[p^1]}}
$$

% Explicit substitution
Instead of meta-level substitution, we use \textbf{explicit substitution}, i.e., substitution itself is a expression in the program and will be done in the runtime. Thus substitution has its own judgement $\goodSub{\Gamma}{\gamma}{\Delta}$. We can apply substitution to types and terms\footnote{The type judgement is the form $\goodType{\Gamma}{T}{}$. The term judgement is the form $\goodTerm{\Gamma}{t}{T}$} in MLTT via \ruleref{Subst}.  Note how substitution will change the context---we suggest intuitvely understanding context as holes of the term/types and substitution as filling those holes, where the substitution itself has holes. After filling the holes, then the new context is about the holes of the substitution. We have an identity substitution $\goodSub{\Gamma}{"id"}{\Gamma}$.
% Debruijn Indices
Contrary to named variables, \textbf{debruijn indices} refer to the variable in the context like operating a stack. We will use "var"$_i$ indicating the $i$-th variable (from right to left, $i$ starts with 0) in the context.  On the other hand, we can also get the rest part of the stack, or ``subpart of "id"'' using $\goodSub{\Gamma, A}{p^1 \gamma}{\Gamma}$.  For example, we can get the very first variable in the context $\Gamma, x : T \vdash x : T$ by $\goodTerm{\Gamma, T}{var_0}{T[p^1]}$. Note here $T$ is a type in $\Gamma$, but now shifted into $\goodType{\Gamma, T}{T[p^1]}{}$ using $p^1$. Similarly, to get the second and prior variables, we use "var"$_i$ and $p^{i+1}$ correspondingly. 


Now, we will mainly focus on the newly introduced typing rules, and illustrate their connection to the plugin implementation. For those curious about the complete meta-theory formulation, we refer to our appendices or \citet{altkap2016, kaposi2017type} or maybe those not using QIIT \cite{coquand2018canonicity, sterling2019algebraic}. Our metatheory is formulated using QIIT in an extensional setting.

$$
\Rule[name=wsig/ept]
{""}
{\goodWSig{\Gm}{\WSigEmp}{0}}
\quad 
\Rule[name=wsigeq/ept/eta]
{\goodWSig{\Gm}{\wsig}{0}}
{\goodWSig{\Gm}{\wsig \equiv \WSigEmp}{0}}
\quad
\Rule[name=wsig/add]
{\goodWSig{\Gm}{\wsig}{n}
  \quad \goodType{\Gm}{A}{i}
  \quad \goodType{\Gm, A}{B}{i}}
{\goodWSig{\Gm}{\WSigAdd{\wsig}{A}{B}}{n+1}}
\quad
\Rule[name=wsig/universe]
{\goodWSig{\Gm}{\wsig}{n}}
{\goodType{\Gm}{\TyWSingle {\wsig}}{i}
\\ \goodTerm{\Gm}{\wsingle{\wsig}}{\TyWSingle{\wsig}}
}
$$
$$
\Rule[name=ind/cstr]
{{\goodWSig{\Gm}{\wsig}{n}}
  \\ \goodTerm{\Gm}{a}{\wsigproj{j}{1}{\wsig}}
  \\ \goodTerm{\Gm, \sub{\wsigproj{j}{2}{\wsig}}{(id, a)}}{b}{\El{(\wcode{\wsig})}}}
{\goodTerm{\Gm}{\Wsup{T}{a}{b}}{\El{(\wcode{\wsig})}} }
$$
$$
\Rule[name=ty/casesig]
{\goodType{\Gm}{A}{}
\quad \goodType{\Gm, A}{B}{}
\quad \goodType{\Gm}{R}{}
}
{\goodType{\Gm}{\CaseSig{A}{B}{R} \equiv (\TyPi {A} {(\TyPi {(\TyPi {B} {(\sub{R}{ \SubstComp {\pi_1}{\pi_1}})})} {\sub{R}{\SubstComp{\pi_1}{\pi_1}}})} )}{}}
\quad 
\Rule[name=Hdlers]
{\goodWSig{\Gamma}{\tau}{n}
\quad \goodType{\Gamma}{R}{}
}
{\goodSig{\Gamma}{"RecSig"~\tau~R}{n}}
$$
$$
\Rule[name=reclkg/proj]
{ \goodWSig{\Gm}{\wsig}{N}
\\ \goodTerm{\Gm}{o}{\TyLkg{(\RecSig{\wsig}{R})}}
\\ j < N
}
{\goodTerm{\Gm}{\Recproj{j}{o}}{\sub{(\CaseSig{(\pi_1^j \wsig)}{(\pi_2^j \wsig)}{R})}{\pi_1}}}
$$
$$
\Rule[name=tm/wrec]
{ \goodWSig{\Gm}{\wsig}{N}
\\ \goodType{\Gm}{R}{}
\\ \goodTerm{\Gm}{r}{\TyLkg{(\RecSig{\wsig}{R})}}
\\ \goodTerm{\Gm}{w}{\El{(\wcode{\wsig})}}
}
{\goodTerm{\Gm}{\Wrec{\wsig}{w}{r}}{R}}
\YZ{I suppose Rec-Constr is only for exhaustivity checking? If so, then would it be more intuitive to not include the w in the term? And I suppose the programmer will want to use sealing to hide its type, or otherwise it would fail to normalize for lack of a beta rule?}\EDJreply{ 1. Yes. But not include w will cause $"Wrec"~r$ be a function type. we are not supposed to add a new intro-rule for function type(or for any other type), this is bad meta-theory design. \\
  2. It is actually the other way around -- if things are abstracted, then no beta rule can apply. More concretely, $T : \bW\tau$ is abstracted into $T' : \cU$, then this later one we cannot use $"Wrec"$ on it for sure. But once we have $\{T : \bW\tau, "rec" : \cL(RecSig~\tau)\}$ we are okay, and this, after wrap "rec" with an lambda abstraction, we will have $\cC\{T : \bW\tau, "rec" : \cL(RecSig~\tau~R), "recf" : T -> R\}$ as compilation type. Then with sealing abstraction, we will have $\Sigma(T : U, \Sigma ("rec" : ?, \Sigma("ref" : T -> R)))$ 
}\YZreply{
  What happens if the programmer forgets to seal "rec"? I suppose rec
  could still be inherited into a derived family. But it would not be
  useful if the inductive type is extended there, because rec is
  applicable only to the inductive type defined in the base family.
}\EDJreply{
  You are right. Since "rec" has type "\cL(RecSig~\tau~R)", after inheritance, it is basically the same type "rec : \cL(RecSig~\tau[k]~R[k])" for some "k" stands for the upcasting. but the most important $\tau[k]$ has an  unchanged shape (like the number of constructors) so it is still a recursor for inductive type of signature $\tau$. Once the parent inductive type $\bW\tau$ is extended $\bW\tau_2$ in the children family, with enriched signature $\tau_2$, then because of the signature inconsistency with $\tau[k]$, this old recursor is not useful for this new inductive type.    \\
  In other words, this recursor linkage "rec" can be used for any inductive type with signature $\tau$.
}
%\YZreply{
%  So there IS a beta rule for recursor (but you don't show it), right?
%}\EDJreply{
%  Yes. 
%  % https://github.com/DKXXXL/Redstone/blob/46c2420982e1fac50138924cdd7e152786cd38c8/Sketch/Extensible-Module-1/Syntax.agda#L589 \\ 
%  The only reference is here, but I commented out this url. The reduction applies to $"Wrec" \_ \_ ("Wcstr")$. \\
%  We also have canonicity (i.e. a closed term of boolean reduces to true or false), so there is definitely a beta rule for inductive type.
%}\YZreply{I'd add typeset this beta rule}\EDJreply{I just realize you use the words normalize. I will say currently this system cannot ``reduce'' any open term. But can only ``reduce'' close term. \\ 
%When the programmer program on this meta-theory, they don't need canonicity or normalization at all (they write out a chain of  judgemental equality by themselves). If they program on a proof assistant, then our proof assistant will need normalization to do type checking.
%
$$
$$
\Rule[name=tmeq/rec/beta]
{\goodTerm{\Gamma}{h}{\cL(\RecSig{\tau}{R})} 
\\ \goodTerm{\Gamma}{T}{\TyWSingle{\tau}}}
{\goodTerm{\Gamma}{\Wrec{\wsig}{h}{(\Wsup{\wsig}{a}{b})} \equiv 
(\app{((\app{(\Recproj{j}{h})} )[("id", a)])} )[("id", \lam{(\Wrec{\wsig}{h[\pi_1]}{b})})]
}{R}}\EDJ{"app" and "lam" are lambda application and abstraction}
$$


We start with the formulation of (extensible) inductive type, which is close to W-type (\ruleref{WSig-Add} requires a pair of types $A \vdash B$ as well), but with mulitple constructors. Thus we have a brand new judgement ${\goodWSig{\Gamma}{\tau}{n}}$ indicating signature of inductive type by its constructors. Unconventionally, a signature can induce a "universe" of types $\bW\tau$---in other words, $T : \bW\tau$ will be the inductive type instead of $\bW\tau$ itself. "Wsup" is used to construct a specific term of a given inductive type.


% When programming via meta-theory, we usually seal the inductive type, only expose its constructors (as a mundane function member) and hide the eliminator by abstracting $T : \bW\tau$ into $"W"~T : \cU$. Then the following fields---those only rely on this interface with $\cU$---can be inherited to other contexts (e.g. using enriched inductive type), as long as the targeting contexts can be abstracted into this {interface} again. 

Similar to plugin implementation, we delegate handlers (and their reuse) for recursion to linkages (and their inheritance). We use "RecSig1" to compute the type of a single handler and "RecSig" for the complete family of handlers for all constructors. \ruleref{Rec-Constr} uses "RecSig" to achieve exhaustiveness checking on elimination. The partial recursors and the computational axioms are treated as normal fields as expected, since we know they are provable in the plugin implementation. 

To introduce the concrete programming on inductive type in metatheory, we need to first introduce the formulation of families.


%\YZ{
%    Is extensionality in conflict with the claim in §3 that "subst tm_unit x t" and "tm_unit" are
%    not definitionally equal but propositionally equal in family STLC?
%}\EDJreply{No, TLDR: it is a different level. \\ 
%There are two type theories, one is Agda-like (= Latex = Conventional math), the other is the one formulated in QIIT. This sentence is saying that the Agda-like one is extensional MLTT (i.e. we have function extensionality and etc here in Latex writing directly). \\ But the one you are asking about is in the metatheory in QIIT: if we have an identity type $Id x y$ (propositionally equal) can we have a judgemetnal equality for the two terms in QIIT $x \equiv y$. \\
%Apparently, we don't setup our QIIT formulation in this way (I thought we need, but now I don't think we need it. Our QIIT formulation is purely in intensional MLTT now).\\ 
%But even if we do setup like that, I will not say a conflict is happening (because it is to our advantage if we have judgemental equal in metatheory). I will say it is a bit off as to how Coq is formulated (Coq is closer to an intensional MLTT, so no judgemental equal only propositional one). 
%}

% linkage, signature and compilation and seal
$$
\Rule[name=lsig/ept]
{""}
{\goodSig{\Gm}{\LSigEmp}{0}}
\quad
\Rule[name=ty/lkg]
{\goodSig{\Gm}{\lsig}{n}}
{\goodType{\Gm}{\TyLkg{\lsig}}{i}
\\ \goodType{\Gm}{\TyTkg{\lsig}}{}}
\quad
\Rule[name=lnkg/pkg]
{ \goodTerm{\Gm}{o}{\TyLkg {\lsig}} 
}
{
  \goodTerm{\Gm}{\Tkg {o}}{\TyTkg{\lsig}}
}
$$
$$
\Rule[name=lsig/add]
{\goodSig{\Gm}{\lsig}{n} 
 \quad \goodType{\Gm}{A}{}
 \quad \goodSeal{\Gm}{\seal}{\lsig}{A}
 \quad \goodType{\Gm, A}{T}{i}}
{\goodSig{\Gm}{(\LSigAdd {\lsig}{\seal}{T})}{n+1}}
\quad
\Biggl( \boxed{\goodSeal{\Gamma}{s}{\sigma}{A}} 
:= \boxed{\goodTerm{\Gamma, \cC\sigma}{s}{A[p^1]}} \Biggr)
$$
$$
\Rule[name=lnkg/ept]
{""}
{\goodTerm{\Gm}{\LkgEmp}{\TyLkg {\LSigEmp}}  
}
\quad
\Rule[name=lkg/add]
{ \goodTerm{\Gm}{o}{\TyLkg {\lsig}} 
\\ \goodType{\Gm}{A}{}
\\  \goodSeal{\Gm}{\seal}{\lsig}{A} 
\\  \boxed{\goodTerm{\Gm, A}{t}{T}}
}
{\goodTerm{\Gm}{(\LkgAdd {o} {t})}{\TyLkg{\LSigAdd {\lsig} {\seal} {T}}}
}
$$
$$
\Rule[name=lkg/proj]
{\goodTerm{\Gm}{o}{\TyLkg{\lsig}}}
{\goodTerm{\Gm}{\lkgproj{1} {o}}{\TyLkg {(\lsigproj{1} {\lsig})}}
\\ \boxed{\goodTerm{\Gm, \lsigprojT {\lsig}}{\lkgproj{2}{o}}{\lsigproj{2}{\lsig}}}
}
$$

We start with signature (in a brand new judgement $\goodSig{\Gamma}{\sigma}{n}$). A signature is a list of types contained by a \textit{linkage}\footnote{Following the convention by \citet{zm2017}, we name the meta-theoretical correspondents of \textit{family} as \textit{linkage}.}, tracking the length as superscript.   We will use $\{a : T_1, b : T_2, ..\}$ as a shorthand for signature with type $T_1, T_2, ..$ for readability. 

\ruleref{lkg/proj} indicates how linkage captures the idea of \textit{code reuse}, as $\lkgproj{2}{-}$ will lead to the boxed term that is parametric on the abstracted interface of linkage $\lsigprojT{\lsig}$ . To code reuse, we simply project out a field from the parent and add back to the children family.
\newpage 
Adding new types into signature \ruleref{Sig Add} requires a \textit{sealing judgement} $\goodSeal{\Gamma}{f}{\sigma}{A}$, which is an alias of a term judgement. Sealing is corresponding \textit{to indicating whether each field is overridable(extensible) or non-overridable (thus definitionally exposed) in the Coq Plugin}. 
Concretely, we encode our STLC example to show the motivation of the sealing.
%\YZ{Is sealing used only for late-binding the signatures of inductive types? Do I understand correctly that late binding of other kinds of fields (like subst) is achieved by the boxed premise in rule "LNKG ADD"?}\EDJreply{All late-bound are achieved by ``selective-abstraction'' imposed by sealing. For example, recursion will be consequently abstracted because of the abstraction of the inductive type it recurse on. 
%}\EDJreply{Sorry, I see what you mean. Yes. the box part is important for late-binding. But the functionality of sealing is always there. They are both required to achieve overriding. A illustrative example is, if the type is $S(\bot)$, then even with the boxed premise, it is still not overridable because the type itself restricts all the possibilities.}\YZreply{
%  By "selective abstraction", do you mean in the metatheory the programmer chooses what and how to abstract?
%}\EDJreply{Yes.}

% For example, if a field is overridable, we abstract that to its type; if a field has exposed its definition, we abstract it into the singleton type containing it instead.  Sealing $\goodSeal{\Gamma}{f}{\sigma}{A}$ is useful when the surrounding interface is $\sigma$ and we want to construct the new field based on $A$.

% We can now point out how "Overridable" and "pins" is modeled. Despite the fact we have a complicated dynamic of invisibility between overridable and normal fields in the plugin, every field can be elaborated into the (pseudo)-syntax form \\ \mintinline{Coq}{Field hiding {x, y, z ...} newfield := t...}  where every former field x,y,z,...  hide its definition when type-check t, and remaining former fields have their definition exposed in t.
% This above elaborated "hiding" syntax directly corresponding to the $f$ in $(\mu^+ ~o~\{f\}~t)$, where each fields decides the abstraction of the prior.


\begin{figure}[!htb]
  \lstset{
      basicstyle=\fontsize{8}{8.5}\ttfamily,
  % numbers=left,
  }
  
  \begin{minipage}{\textwidth}
  \begin{multicols}{3}
  

  \definecolor{codecomment-color}{HTML}{0DA3FF}
  
  \begin{lstlisting}
  Family STLC. 
   FInductive ty (* no hiding *)

   FInductive tm (* hiding ty *) 
   
   FRecursion subst
     Case ... (* hiding ty tm *)
     ...
   End subst (* hiding ty *) 
   

   Field t1 := (tvar 0) 
   (* hiding ty tm subst *) 


   Field t2 := (tapp t1 t1)
   (* hiding ty tm subst *)

  ...
  \end{lstlisting}
  
  % \makeline[0pt]{Parser-exmp-before-start}{Parser-exmp-before-end}[codecomment-color!50]
  
  \columnbreak
  % \definecolor{codecomment-color}{HTML}{5D030F}
  
  \begin{lstlisting}

    (* s[@₁@]  : {} | {} *)
    - o[@₁@]  = μ+ {} s[@₁@] ⋆ : 𝕃{ty : 𝕎τ[@₁@]}
    (* s[@₂@]  : {ty : 𝕎τ[@₁@]} | [ty : 𝕌; bool : ty; ...] *) 
    - o[@₂@]  = μ+ o[@₁@] s[@₂@] ⋆ : 𝕃{ty : 𝕎τ[@₁@]; tm : 𝕎τ[@₂@]}
    (* s[@₃@][@₁@]  : {ty : 𝕎τ[@₁@]; tm : 𝕎τ[@₂@]}
            | [ty : 𝕌; ... ; tm : 𝕌; tvar : tm; ...] *)
    - o[@₃@][@₁@]  = μ+ o[@₂@] s[@₃@] ([@μ@]+ ..) : 𝕃{ty : 𝕎τ[@₁@]; tm : 𝕎τ[@₂@]; subst' : 𝕃(RecSig)}
    (* s[@₃@] : {ty : 𝕎τ[@₁@]; tm : 𝕎τ[@₂@]; subst' : 𝕃(RecSig)} 
          | [ty : 𝕌; ... ; tm : 𝕎τ[@₂@]; subst' : 𝕃(RecSig)] *)
    - o[@₃@] = μ+ o[@₃@][@₁@] s[@₃@] (λ (Wrec ..)) : 𝕃{ty : 𝕎τ[@₁@]; tm : 𝕎τ[@₂@]; subst' : 𝕃(RecSig); subst : tm → nat → tm → tm}
    
    (* s[@₄@] : {ty : 𝕎τ[@₁@]; ... subst : tm → nat → tm → tm } 
          | [.. ; tm : 𝕌; tvar : nat → tm; .. subst : tm → nat → tm → tm] *)
    - o[@₄@] = μ+ o[@₃@] s[@₄@] [@("tvar"~0)@] : 𝕃{ty : 𝕎τ[@₁@]; ... t[@₁@] : [@\TyS{"tvar"~0}@]}
  
    (* s[@₅@] : {ty : 𝕎τ[@₁@]; tm : 𝕎τ[@₂@]; subst : 𝕃(RecSig); t1 : [@\TyS{"tvar"~0}@]} 
          | [ty : 𝕌; tm : 𝕌; subst : 𝕃(RecSig); t1 : [@\TyS{"tvar"~0}@]] *)
    - o[@₅@] = μ+ o[@₄@] s[@₅@] .. : 𝕃{ ... }
  \end{lstlisting}
  
  \columnbreak
  

  % \makeline[.5\textwidth+9pt]{Parser-exmp-after-start}{Parser-exmp-after-end}[codecomment-color!50]
  
  \end{multicols}
  \end{minipage}
  \caption{Sealing and Linkage}\label{fig:sealing+linkage}
  \end{figure}

In \cref{fig:sealing+linkage}, the "hiding" comment illustrates the late-bound prior fields for each fields.   For "tm", "ty" is late-bound and thus extending "ty" won't cause disruption on the inheritance of "tm". For "subst" itself, the definition of "tm"  cannot hidden for the sake of exhaustiveness checking. However, for each case handler inside "subst", "tm" is late-bound and thus each handler is inheritable. 

Each hiding comment corresponds to the textual-aligned sealing derivation on RHS. For legibility, curly brackets indicate a signature (omitting sealing at type level for readability) and square bracket indicate a nested $\Sigma$-type, corresponding to sealing judgment $\goodSeal{}{s}{\sigma}{A}$.

For "tm", "ty" is hidden and thus the sealing "s₂" will make "ty" into an arbitrary type "𝕌" and expose its constructor "bool", "arrow", and etc. Then when constructing "o₂", the newly added field "tm" as "⋆ : 𝕎τ₂" is constructed in the context where "ty : 𝕌", ignoring its original eliminator. (Because eliminator is only possible when "ty : 𝕎τ₁"). On the other hand, for "subst", only "ty" is hidden and thus the sealing "s₃" leave "tm : 𝕎τ₂" untouched and "Wrec" becomes possible when constructing "o₃".

Basically, the hidden prior fields---inductive type are sealed into arbitrary type with ``constructors only'', where the recursion is constructed without sealing the inductive type.  Generally speaking, each extensible inductive type are hidden for all the other fields except for the recursion on itself. 

We choose to expose the definition of $t_1$ to all the following fields including $t_2$ by using singleton type $\TyS{-}$. If other fields, say $t_3$, want to hide it, just seal $t_1$ into type "tm"; and then overriding $t_1$ will not disrupt the inheritablity of $t_3$. But such overriding will definitely disrupt the inheritablity of $t_2$ as $s_5$ doesn't hide $t_1$. Generally speaking, any field that has a clear "hiding" semantic on the prior fields can be understood as sealing. All of our plugin commands have clear "hiding" semantics.

% Intuitively, sealing corresponds to the following surface syntax\\
% \mintinline{Coq}{Field hiding {x, y, z ...} newfield := t...} where every former field "x","y","z",...  hide its definition when type-check "t", and remaining former fields have their definition exposed for "t". Thus, "x","y","z", ... can be overridden but other fields have to stay the same if we want to inherit "t". In meta-theory, we control the sealing $f$ so that "x","y","z",... only has type exposed and remaining fields are definitionally exposed using singleton type.


% The reader should feel this subtle difference compared to our current plugin implementation---$(\mu^+ ~o~\{f\}~t)$ reads as \textit{``$t$ is a new well-typed field when we abstract the \textbf{prior} fields using $f$''}. In other words, every field in meta-theory has the right to choose how they perceive and abstract the prior fields. While in the plugin, "Overridable" keyword will decide how the current defining field is percived by all the following fields.


% Given a signature $\sigma$, we have the linkage type $\cL\sigma$ inhabited by linkages; and the \textit{compilation type} $\cC\sigma$ of the linkage type. The compilation type corresponds to how we compile a context into a \mintinline{Coq}{Module Type} in the plugin implementation---in metatheory, $\cC\sigma$ will be a nested sigma type. For example, $\cC\{a : \cU, b : a \to \bot \} = \Sigma(a : \cU).\Sigma (b : a \to \bot). \top$.

Due to the definition of sealing, we can see the importance of compilation type, as the result of a sealing is usually a (nested) $\Sigma$-type instead of a linkage type. For example, $\TyLkg{\{a : S(\bot), b : a \to \bot\}}$ can only be sealed into $\TyTkg{\{a : \cU, b : a \to \bot\}}$ but never $\TyLkg{\{a : \cU, b : a \to \bot\}}$. Because the latter cannot be inhabitted---by projecting the second field of the $ t : \TyLkg{\{a : \cU, b : a \to \bot\}}$, we can get a proof of bottom type. For simplicity, we make sealing a term in the context of compilation type instead of linkage type.

% The difference between compilation type and linkage types lies in their introduction rules. The compilation types are just syntactic sugar for sigma types. However, the intro rules for the linkage types gives it the power of late-binding (also the fundation of inheritance and overriding). More concretely, the boxed premise in \ruleref{Lnkg Add} is making $t$ parametric on $A$, contrary to how in sigma type, the $t$ will be about a concrete $a : A$. This parametricity makes $t$ inheritable and the earlier stuff about $A$ late-bound.

% The sealing judgement provides a witness $f$ that a context $\Gamma, \cC\sigma$ can be abstracted into an \textit{interface} $\Gamma, A$ for a type $\goodType{\Gamma}{A}{}$. For example, sealing judgement can abstract an inductive type with constructors into an arbitrary type with functions (but without eliminators). Then any later fields $\goodTerm{\Gamma, A}{t}{T}$ relies only on the \textit{interface} $A$ can be inherited to other contexts (e.g. using extended inductive type), as long as the targeting contexts can be abstracted into the surrounding \textit{interface} for $t$ (i.e. $\Gamma, A$)  as well.

% The importance of the compilation type lies in its difference from the linkage type. First we have a singleton type $"s A" : S(T) \iff "A" \equiv T$, i.e. a type of one single term.  We can see that we have a term $x : \cC\{A : S(\bot), B : A \to \bot \} \vdash h_1 : \cC\{A : \cU, B:A \to \bot \}$ which is equivalent to $x : \Sigma (A: S(\bot)). A \to \bot \vdash h_1 : \Sigma (A: \cU). A \to \bot$. This can be considered as using seal to do an abstraction on $A:S(\bot)$.

% But we can never have a term $x : \cL\{a : S(\bot), b : a \to \bot \} \vdash h_2 : \cL\{a : \cU, b : a \to \bot \}$, as we know $\cL\{a : S(\bot), b : a \to \bot \}$ is inhabited
% and that leads to a close term of $\cL\{a : \cU, b : a \to \bot \}$ due to the nature of late-binding of $\cL\sigma$---once we override $a$ with $\top$ we will have inconsistency using $b$. This example illustrates the programming difference between the sigma type and linkage type.  

% However, this form of abstracting a type is very important when programming in our metatheory---as later we will show inductive type of different constructors locates in different type (universe), and to make later fields inheritable we need to work on an interface of abstracted type. Thus the compilation type is indispensable.

% Looking closely, the sealing judgement helps to model most machinaries about abstraction in plugin implementation.

% Inductive type and recursor and handlers

$$
\Rule[name=Inh]
{\goodInh{\Gamma}{h}{\sigma_1}{\sigma_2}
\quad \goodTerm{\Gamma}{l}{\cL \sigma_1}
}
{\goodTerm{\Gamma}{("inh" \ h \ l)}{\cL \sigma_2}},
\Rule[name=Inh-Id]
{}
{\goodInh{\Gamma}{"inhid"}{\sigma}{\sigma}},
\Rule[name=Inh-Ext]
{\goodInh{\Gamma}{h}{\sigma_1}{\sigma_2}
  \quad \goodTerm{\Gamma, A_2}{t}{T}}
{\goodInh{\Gamma}{"inhext" \ h \ t}{\sigma_1}{(\nu^+ \  \sigma_2\  T)}}
$$

% Inhertance judgement
Currently we can achieve inheritance by projecting out the field, and then add back the projection into the children family. That means the formulation of linkage already captures the concept of \textit{code reuse}. 

However, in the plugin implementation, we use "Family"s to organize the code reuse. They are second class objects like Coq's "Module". What's more, a "Family" (with or without "extends" clause) can be considered as a standalone piece of code trying to \textit{inherit} an existent (but maybe empty) family. Thus we model them using \textit{inheritance judgement} $\goodInh{\Gamma}{h}{\sigma_1}{\sigma_2}$ instead of considering them as first class terms. Inheritance judgement is like ``library function'' upon linkage for achieving inheritance and overriding. The only way to use $h$ is by \ruleref{Inh} and transforming a linkage of signature $\sigma_1$ to another linkage of $\sigma_2$. Inheritance judgement can be inductively constructed via inheriting operations, overriding operations, extending operations and etc, corresponding to "Inherit", "Override" and "Extend" command  in our plugin implementation.


In this way, we will consider the program in our metatheory similar to those vanilla MLTT program, except there are meta variables denoting inheirtance judgement (i.e. "Family"), of which the only usage is to apply \ruleref{Inh}. We don't yet consider non-trivial equality between derivations of inheritance judgements.


\subsection{Syntactic Translation that removes Linkages}
To further provide the intuition why our core calculus is sound, we provide a syntactic translation from our core calculus to the subpart where linkages are absent. This can justify the \textit{consistency} and \textit{canonicity} modulo the quirky (but apparent) formulation of our inductive type. To be complete, we also provide a consistency and canonicity proof for our complete core calculus in the appendix.

\newcommand{\denotesT}[1]{{{\llbracket {#1} \rrbracket}_T}}
\newcommand{\Sigr}[2]{{ "Sig"^r~{#1}~{#2} }}

Thanks to QIIT formulation, syntactic translation becomes a function between QIIT type, which by definition respects all the judgemental equalities. We will now sketch this translation $\denotesT{\_}$, which keeps most parts untouched except for the linkages related. We omit most of the inheritance translation.



\begin{align*}
  \text{We first define a new type }& \goodType{\Gamma}{\Sigr{\Gamma}{n}}{}  \text{ when } \goodCtx{\Gamma}{}  \text{ is well-formed and } n \in \mathbb{N} \\
  \denotesT{\goodSig{\Gamma}{\_}{n}},\ &({\Sigr{\Gamma}{n}}), \text{ and } \ \denotesT{\TyTkg {-}} \text{ are mutually recursive, } \\
  & \text{defined by induction on the signature length} \\  
  \denotesT{\goodSig{\Gamma}{\_}{n}} &= (\goodTerm{\Gamma}{\_}{\Sigr{\Gamma}{n}}) \\ 
  &\text{, and thus } \denotesT{\goodSig{\Gamma}{\sigma}{n}} \text{ defined} \iff \goodTerm{\Gamma}{\denotesT{\sigma}}{\Sigr{\Gamma}{n}} \\ 
  {\Sigr{\Gamma}{(n+1)}} &= 
    \Sigma~(\Sigr{\Gamma}{n})
          ~(\Sigma~\cU
                  ~(\Sigma~(\Pi(\TyTkg {("var"_0[p^1])})(El~"var"_0[p^1]))~\cU)) \\
  \Sigr{\Gamma}{0} &= \top \\
  \denotesT{\goodType{\Gamma}{\TyTkg {\sigma}}{}}, \denotesT{\goodType{\Gamma}{\TyLkg{\sigma}}{}} &\text{ defines upon } \denotesT{\sigma} 
  \text{ and inductively on the signature length} \\
  \denotesT{\goodType{\Gamma}{\TyTkg {\sigma}}{}} &= \denotesT{\goodType{\Gamma}{\TyLkg{\sigma}}{}} = \top \quad
      \text{given } \denotesT{\goodSig{\Gamma}{\sigma}{0}} \\ 
  \denotesT{\goodType{\Gamma}{\TyTkg {\sigma}}{}} &= 
    \Sigma~\denotesT{\TyTkg {(\lsigproj{1}{\sigma})}}~(El \ (\app{("pjr"^3~\denotesT{\sigma})})[(p^1, \app{(\fst{("pjr"^2~{\denotesT{\sigma}})})})]) \\
  \denotesT{\goodType{\Gamma}{\TyLkg{\sigma}}{}} &=
  \denotesT{\TyLkg{(\lsigproj{1}{\sigma})}} \times \Pi(El~(\fst{(\snd{\denotesT{\sigma}})}))(El~(\app{("pjr"^3~\denotesT{\sigma})} )) \\
  &\text{given } \denotesT{\goodSig{\Gamma}{\sigma}{n+1}} \\
  \denotesT{\goodSig{\Gamma}{\LSigAdd{\sigma}{f}{T}}{n+1}} &= (\denotesT{\sigma}, \codety{A}, \lam{f}, \lam{(\codety{T})}) \\ 
  \denotesT{\goodTerm{\Gamma}{\mu^+~m~t}{\TyLkg{(\LSigAdd{\sigma}{s}{T})}}} &= (\denotesT{m}, \lam{t}) \\ 
  \denotesT{\goodTerm{\Gamma}{\Tkg{m}}{\TyTkg {\sigma}}} & \text{ is defined upon } \denotesT{m} \text{ and inductively on the signature length} \\ 
  \denotesT{\goodTerm{\Gamma}{\Tkg {m}}{\TyTkg {\sigma}}} &= 
  (\denotesT{\Tkg {o}}, t[(\SubstExt{p^1}{f})][(\SubstExt{"id"}{\denotesT{\Tkg {o}}})]) \\
  & \text{given } \denotesT{\goodSig{\Gamma}{\sigma}{n+1}}, \text{ where } o = \lsigproj{1}{m}, t = \app{(\snd{\denotesT{m}})}, f = p_f\nu~\sigma \\ 
  \denotesT{\goodTerm{\Gamma}{\Tkg {m}}{\TyTkg {\sigma}}} &= () \quad \text{given } \denotesT{\goodSig{\Gamma}{\sigma}{0}}
\end{align*}
\begin{align*}
  \denotesT{\goodInh{\Gamma}{\_}{\sigma_1}{\sigma_2}} &= \goodTerm{\Gamma, \TyLkg{\sigma_1}}{\_}{\TyLkg{\sigma_2[p^1]}} \\ \text{ and thus } \denotesT{\goodInh{\Gamma}{h}{\sigma_1}{\sigma_2}} &\iff  \goodTerm{\Gamma, \TyLkg{\sigma_1}}{h}{\TyLkg{\sigma_2[p^1]}} \\
  \denotesT{\goodInh{\Gamma}{"inhinh"~h~T~\uparrow^s}{(\LSigAdd {\sigma_1} {s_1}{T})}{(\LSigAdd {\sigma_2} {s_2} {T[(p^1, \uparrow^s)]})}} &= \mu^+~(h[(p^1,\lkgproj{1}{"var"_0})])\\ &\quad \quad \{f_2[p^1]\}~(\lkgproj{2}{"var"_0})[(p^1, \uparrow^s[{p^1}^{\uparrow}])] \\
\end{align*}






\section{Case Studies}
\label{sec:casestudies}
We 

\bigskip 

\cref{sec:coqexample-stlc}
\cref{sec:coqexample-analysis}
\cref{sec:coqexample-parser}

\section{Related Work}\label{sec:related-work}
This section discusses related work.




\textbf{Modular Mechanized Metatheory: Syntactic Approach} We group the work that achieves modular mechanized metatheory by extending surface syntax of proof assistants as \textit{syntactic approach}. 

The closest work to ours might be \citet{boite2004proo}--they try to support \textit{extended} inductive type and related proof reuse in Coq. This biggest difference lies in our (incomplete) adaption of Family Polymorphism -- we support proof reuse and extension by inheritance, but also support implementation swapping using overriding, and we also argue that Family Polymorphism can provide a better and cleaner way of organizing mechanized metatheory development. The other difference lies in the handling of the extension of the inductive type. In our paradigm, the discussion of \textit{extensible} inductive type and partial recursor is unavoidable, which characterizes all possible extension of a given inductive type. In their work, without using Family to organized the code, the extended inductive type can be considered as a different inductive type than the original one, possibly leading to more expression problem~\cite{wadler-ep}.

\textbf{Modular Mechanized Metatheory: Semantic Approach} We group the work using certain algebraic encoding (e.g. Church Encoding, datatype-generic programming) without modify the surface syntax of the proof assistants as \textit{semantic approach}, that includes \citet{delaware2013,forsta2020,liwei2022,schwaab2013modular, keuchel2013generic} and so on. Their advantages are various: this technique is more general and thus can be applied across different (dependent-typed) proof assistants and this technique can be easily distributed like a library and thus doesn't need to alter the core of the proof assistants. The downsides are mainly about the threshold: they are not friendly for the junior Coq development and these indirect encoding of datatype might lead to worse readability and accessibility compared to the direct style like ours.




\textbf{Solution to The Expression Problem}
% Have no idea how to write this :(

\section{Conclusion}
\label{sec:conclusion}

It is hard to write modular, extensible code and proofs.
We have presented a solution that equips a proof assistant with
linguistic facilities for family polymorphism.
The language design ensures that the expressive power brought by family polymorphism
is in harmony with the strictness of a proof assistant, while incurring
low cognitive overhead and allowing an idiomatic programming experience.
We implement the design via a translation to Coq and demonstrate its applicability
using case studies.
A novel dependent type theory formalizes the core aspects of the language mechanism
and is shown to enjoy consistency and canonicity.
Generalizing the mechanism to support nested family polymorphism and
scaling it to sizable proof engineering efforts make exciting directions for
future work.

\setlength{\bibsep}{.8ex}
\bibliographystyle{tex-macros/ACM-Reference-Format.bst}
\bibliography{refs.bib}

\appendix

\section{Metatheory in (pseudo-)Agda}
We make four agda source code files (publicly and anonymously) openly avaliable, and they include the corresponding QIIT syntax definition and models. 

For syntax, please check \\ \href{https://drive.google.com/file/d/1aoG67rmXzP_x1MvZCIN3do0sucyqjwkn/view?usp=sharing}{https://drive.google.com/file/d/1aoG67rmXzP_x1MvZCIN3do0sucyqjwkn/view?usp=sharing}. This is done by mainly following the formulation \citet{altkap2016}. This is also very close to \citet{coquand2018canonicity} and \citet{sterling2019algebraic} but we formulate our syntax in a type-theoretic framework.

For syntactic translation, please check \\ \href{https://drive.google.com/file/d/1pMqn8DS4T4jiCubzk3HgMANjrfKpcGlI/view?usp=sharing}{https://drive.google.com/file/d/1pMqn8DS4T4jiCubzk3HgMANjrfKpcGlI/view?usp=sharing}. The syntactic translation here is to translate most of the novel feature introduced in this paper into a standard MLTT. 
The essence here is to translate linkage into a sigma type  with dependent function inside---and the dependent function is simulating the nature of ``late-binding'' of each field.


This is the first QIIT-model we give. A QIIT-model can be roughly consider as an arbitrary mapping from the QIIT data(our syntax in this case), also respecting the quotient(judgemental equality in this case). Here, the syntactic translation model is mapping to (a subpart of) the original QIIT data (itself). 

For consistency model, please check \\ \href{https://drive.google.com/file/d/1pNhnn125P5byAHDaSIlpbxMvr1F-9FRo/view?usp=sharing}{https://drive.google.com/file/d/1pNhnn125P5byAHDaSIlpbxMvr1F-9FRo/view?usp=sharing}. This is following the standard model from \citet{altkap2016,kaposi2017type,kaposi2019gluing}.

For canonicity model, please check \\ \href{https://drive.google.com/file/d/1R6C7QNfyu8fbl6LE2ruvpqZ_ZDSRVo0c/view?usp=sharing}{https://drive.google.com/file/d/1R6C7QNfyu8fbl6LE2ruvpqZ_ZDSRVo0c/view?usp=sharing}. This is mainly following \citet{coquand2018canonicity,sterling2019algebraic}, but we carried out our argument in a type-theortic framework.  In other references, it is also called reducibility argument, as an instance of proof-relevant logical relation. Our model can be considered as a verbose/more concrete version of \citet{kaposi2019gluing}.

For both consistency and canonicity model, the key part is still about handling the novel feature. Thanks to the insight from syntactic translation, we can always inspired by consistency and canonicity model of sigma type and dependent function in standard MLTT.

We suggest using editors with proper syntax highlighting, e.g. VSCode with agda-code plugins, to read these source files.

\section{Typing Rules in one Figure}
\judgebox{\goodCtx{\Gamma}{i}}
$$ 
\Rule[name=Empty Context]{}{\goodCtx{\cdot}{0}} 
\quad
\Rule[name=Context Extension]
{\goodCtx{\Gamma}{i} \quad \goodType{\Gamma}{A}{j}}
{\goodCtx{\Gamma, A}{i \cup j}}  
$$


\judgebox{ \goodType{\Gamma}{T}{i} }
$$
\Rule[name=Type Universe]
{}
{\goodType{\Gamma}{\cU}{j + 1}}
\quad 
\Rule[name=Boolean]
{}
{\goodType{\Gamma}{\cB}{0}}
\quad 
\Rule[name=Bottom]
{}
{\goodType{\Gamma}{\bot}{0}}
\quad 
\Rule[name=Function]
{\goodType{\Gamma}{A}{j} 
  \quad \goodType{\Gamma, A}{B}{k}}
{\goodType{\Gamma}{\Pi A B}{j \cup k}}
$$

$$
\quad 
\Rule[name=Func/DPair Subst]
{\goodSub{\Gamma}{\gamma}{\Delta}
\quad \goodType{\Delta}{A}{} 
\quad \goodType{\Delta, A}{B}{}
}
{
  \goodType{\Gamma}{(\Pi A B)[\gamma] \equiv \Pi A[\gamma] B[\gamma^\uparrow] }{j \cup k}
  \quad 
  \goodType{\Gamma}{(\Sigma A B)[\gamma] \equiv \Sigma A[\gamma] B[\gamma^\uparrow] }{j \cup k}
}
$$

$$
\Rule[name=Type Subst]
{\goodType{\Delta}{T}{j} 
  \quad {\goodSub{\Gamma}{\gamma}{\Delta}}}
{\goodType{\Gamma}{T[\gamma]}{j}}
\quad
\Rule[name=Base Type Subst]
{\goodSub{\Gamma}{\gamma}{\Delta}}
{\goodType{\Gamma}{\cU[\gamma] \equiv \cU }{j + 1} \quad
  \goodType{\Gamma}{\cB[\gamma] \equiv \cB}{0} \quad 
  \goodType{\Gamma}{\bot[\gamma] \equiv \bot}{0}
}
$$
\judgebox{ \goodTerm{\Gamma}{t}{T} }
$$
\Rule[name=Code/Get Type]
{\goodType{\Gamma}{T}{j}}
{\goodTerm{\Gamma}{"c" \ T}{\cU}
}\quad
\Rule[]
{\goodTerm{\Gamma}{T}{\cU}}
{\goodType{\Gamma}{El \ T}{j}}
\quad 
\Rule[name=Term Subst]
{\goodTerm{\Delta}{t}{T}
  \quad {\goodSub{\Gamma}{\gamma}{\Delta}}}
{\goodTerm{\Gamma}{t[\gamma]}{T[\gamma]}}
$$
$$
\Rule[]
{\goodTerm{\Gamma, A}{t}{B}}
{\goodTerm{\Gamma}{\lambda t}{\Pi A B}}
\quad 
\Rule[]
{\goodTerm{\Gamma}{t}{\Pi A B}}
{\goodTerm{\Gamma, A}{"app"~t}{B}}
\quad 
\Rule[]
{\goodTerm{\Gamma}{u}{A} 
\quad \goodTerm{\Gamma}{v}{B[(id, u)]}}
{\goodTerm{\Gamma}{(u,v)}{\Sigma A B}}
\quad 
\Rule[name=Sigma Elim]
{\goodTerm{\Gamma}{t}{\Sigma A B}}
{\goodTerm{\Gamma}{"pjl" \ t}{A}
\\\\ \goodTerm{\Gamma}{"pjr" \  t}{B[(id, "pjl" \  t)]}
}
$$
$$
\Rule
{}
{\goodTerm{\Gamma}{(\lambda t)[\gamma] \equiv \lambda (t[\gamma^\uparrow])}{\Pi A B}
\quad \goodTerm{\Gamma}{(u,v)[\gamma] \equiv (u[\gamma],v[\gamma])}{\Sigma A B}
\quad \goodType{\Gamma}{El \ (T[\gamma]) \equiv (El \ T) [\gamma]}{}
}
$$

$$
\Rule[name=Bool]
{}
{\goodTerm{\Gamma}{"tt", "ff"}{\cB}}
\quad
\Rule[]
{\goodSub{\Gamma}{\gamma}{\Delta}}
{\goodType{\Gamma}{(El \ T)[\gamma] \equiv El \ (T[\gamma]) }{j} \quad
 \goodTerm{\Gamma}{"tt"[\gamma] \equiv "tt"}{\cB} \quad 
 \goodTerm{\Gamma}{"ff"[\gamma] \equiv "ff"}{\cB} 
}
$$
\judgebox{\goodSub{\Gamma}{\sigma}{\Delta}}
$$
\Rule[name=Ept Subst]
{}{\goodSub{\Gamma}{\epsilon}{\cdot}}
\quad
\Rule[]
{}{\goodSub{\Gamma}{"id"}{\Gamma}}
\quad
\Rule[name=Sub Comp]{
  \goodSub{\Delta}{\delta}{\Theta}
  \quad \goodSub{\Gamma}{\gamma}{\Delta} 
}{\goodSub{\Gamma}{\delta \circ \gamma}{\Theta}}
\Rule[name=Sub Extend]
{\goodSub{\Gamma}{\gamma}{\Delta} \quad \goodTerm{\Gamma}{t}{A[\gamma]}}
{\goodSub{\Gamma}{(\gamma, t)}{(\Delta, A)}}
$$

$$
\Rule[name=Proj Subst-1]
{\goodSub{\Gamma}{\gamma}{(\Delta, A)}}
{\goodSub{\Gamma}{\pi_1 \gamma}{\Delta}}
\quad
\Rule[name=Proj Subst-2]
{\goodSub{\Gamma}{\gamma}{(\Delta, A)}}
{\goodSub{\Gamma}{\pi_2 \gamma}{A[\pi_1 \gamma]}}
\quad
\Rule[name=Proj-Ext]
{}
{\goodSub{\Gamma}{(\pi_1 \gamma, \pi_2 \gamma) \equiv \gamma}{\Delta}}
$$



\judgebox{\goodWSig{\Gamma}{\tau}{n}}
$$
\Rule[name=Emp-WSig]
{}
{\goodWSig{\Gamma}{w\cdot}{0}}
\quad
\Rule[name=WSig-Add]
{\goodWSig{\Gamma}{\tau}{n}
  \quad \goodType{\Gamma}{A}{i}
  \quad \goodType{\Gamma, A}{B}{i}}
{\goodWSig{\Gamma}{w^+ \  \tau \  A \  B}{n+1}}
\quad
\Rule[name=Ind-Univ]
{\goodWSig{\Gamma}{\tau}{n}}
{\goodType{\Gamma}{\bW \tau}{i}
}
$$

$$
\Rule[name=WSig-Proj]
{\goodWSig{\Gamma}{\tau}{n} \quad j < n}
{\goodType{\Gamma}{\pi^j_1 \tau}{i} \quad \goodType{\Gamma, \pi^j_1 \tau}{\pi^j_2  \tau}{i}}
\quad
\Rule[name=Ind-Sig]
{\goodWSig{\Delta}{\tau}{n}
  \quad {\goodSub{\Gamma}{\gamma}{\Delta}}}
{\goodWSig{\Gamma}{\tau[\gamma]}{n}
  \quad \goodType{\Gamma}{W (\tau[\gamma]) \equiv (W \tau)[\gamma]}{i}}
$$

$$
\Rule[name=Ind-Type]
{\goodWSig{\Gamma}{\tau}{n}}
{\goodTerm{\Gamma}{"W" \ \tau}{\cU}
}
\quad 
\Rule[]{}{\goodTerm{\Gamma}{\star_\tau}{\bW \tau}}
\quad
\Rule[name=Ind-Term]
{{\goodWSig{\Gamma}{\tau}{n}}
  \quad \goodTerm{\Gamma}{a}{\pi^j_1\tau}
  \quad \goodTerm{\Gamma, \pi^j_2\tau[(id, a)]}{b}{El~("W"~\tau)}}
{\goodTerm{\Gamma}{"Wsup"~T~a~b}{El~("W"~\tau)} }
$$


\judgebox{\goodSig{\Gamma}{\sigma}{n} }
$$
\Rule[name=Lnkg Type/Compile]
{\goodSig{\Gamma}{\sigma}{n}}
{\goodType{\Gamma}{\cL \sigma}{i}
\quad \goodType{\Gamma}{\cC \sigma}{}}
\quad
\Rule[name=Sig/Lnkg Subst]
{\goodSig{\Delta}{\sigma}{n}
  \quad {\goodSub{\Gamma}{\gamma}{\Delta}}}
{\goodSig{\Gamma}{\sigma[\gamma]}{n}
  \quad \goodType{\Gamma}{\cL (\sigma[\gamma]) \equiv (\cL \sigma)[\gamma]}{i}}
$$

$$
\Rule[name=Ept Sig]
{}
{\goodSig{\Gamma}{\nu\cdot}{0}}
\quad
\Rule[name=Sig Add]
{\goodSig{\Gamma}{\sigma}{n} 
 \quad \goodType{\Gamma}{A}{}
 \quad \goodSeal{\Gamma}{f}{\sigma}{A}
 \quad \goodType{\Gamma, A}{T}{i}}
{\goodSig{\Gamma}{(\nu^+ \ \sigma \ \{f\} \ T)}{n+1}}
$$

$$ 
\Rule[name=Sig Proj]
{\goodSig{\Gamma}{\sigma}{n+1}}
{\goodSig{\Gamma}{p_1\nu \ \sigma}{n}
\\ \goodType{\Gamma}{p_1\nu' \ \sigma}{}
\\ \goodSeal{\Gamma}{p_f\nu \  \sigma}{p_1\nu \  \sigma}{p_1\nu' \sigma}
\\ \goodType{\Gamma, p_1\nu' \ \sigma}{p_2\nu \ \sigma}{}
}
$$

$$ 
\Rule[name=Sig Proj Beta]
{\goodSig{\Gamma}{\sigma}{n} 
\quad \goodType{\Gamma}{A}{}
\quad \goodSeal{\Gamma}{f}{\sigma}{A}
\quad \goodType{\Gamma, A}{T}{i}}
{\goodSig{\Gamma}{p_1\nu \ (\nu^+ \ \sigma \ \{f\} \ T) \equiv \sigma}{n}
\\ \goodType{\Gamma}{p_1\nu' \ (\nu^+ \ \sigma \ \{f\} \ T) \equiv A}{}
\\ \goodSeal{\Gamma}{p_f\nu \  (\nu^+ \ \sigma \ \{f\} \ T) \equiv f}{\sigma}{A}
\\ \goodType{\Gamma, p_1\nu' \ \sigma}{p_2\nu \ (\nu^+ \ \sigma \ \{f\} \ T) \equiv T}{}
}
$$

$$
\Rule[name=Ept Lnkg]
{}
{\goodTerm{\Gamma}{\mu\cdot}{\cL \nu\cdot}}
\quad
\Rule[name=Lnkg Add]
{ \goodTerm{\Gamma}{o}{\cL \sigma} 
\quad \goodType{\Gamma}{A}{}
\quad  \goodSeal{\Gamma}{f}{\sigma}{A} 
 \quad \goodTerm{\Gamma, A}{t}{T}
}
{\goodTerm{\Gamma}{(\mu^+ \ o \ \{f\} \ t)}{\cL(\nu^+ \ \sigma \ \{f\} \ T)}}
$$

$$
\Rule[name=Lnkg Proj]
{\goodTerm{\Gamma}{o}{\cL\sigma}}
{\goodTerm{\Gamma}{p_1\mu \ o}{\cL (p_1\nu \ \sigma)}
\\ \goodTerm{\Gamma, p_1\nu' \ \sigma}{p_2\mu \ o}{p_2\nu \ \sigma}
}
\quad 
\Rule[name=Lnkg Compile]
{ \goodTerm{\Gamma}{o}{\cL \sigma} 
}
{
  \goodTerm{\Gamma}{\cCt o}{\cC \sigma}
}
$$
$$ 
\Rule[name=Lnkg Proj Beta]
{\goodTerm{\Gamma}{o}{\cL \sigma} 
\quad \goodType{\Gamma}{A}{}
\quad  \goodSeal{\Gamma}{f}{\sigma}{A} 
 \quad \goodTerm{\Gamma, A}{t}{T}}
{\goodTerm{\Gamma}{p_1\mu \ (\mu^+ \ o \ \{f\} \ t) \equiv o}{\cL \sigma}
\\ \goodTerm{\Gamma, A}{p_2\mu \  (\mu^+ \ o \ \{f\} \ t) \equiv t}{T}
}
$$
$$
\Rule[name=Compile]
{}
{\goodType{\Gamma}{\cC \nu\cdot \equiv \top}{} 
\\
\goodType{\Gamma}{\cC (\nu^+ \ \sigma \ \{f\} \ T) \equiv 
    \Sigma (\cC \sigma) T["sf" \ f]}{}
\\\\ \goodTerm{\Gamma}{ \cCt \mu\cdot \equiv ()}{\cC \nu\cdot}
\\ \goodTerm{\Gamma}{\cCt \ (\mu^+ \ o \ \{f\}\ t) \equiv ((\cCt \ o), t["sf" \ f][(id, \cCt \ o)]) }{}
}
$$


\begin{figure}[H]

  \judgebox{[\goodSeal{\Gamma}{s}{\sigma}{A}] 
  = [\goodTerm{\Gamma, \cC\sigma}{s}{A[\pi_1]}] }
  $$
  \Rule[name=Seal Subst]
  {\goodSeal{\Delta}{s}{\sigma}{A}
    \quad {\goodSub{\Gamma}{\gamma}{\Delta}}}
  {\goodSeal{\Gamma}{s[\gamma]}{\sigma[\gamma]}{A[\gamma]}}
  =
  \frac
  {\goodTerm{\Delta, \cC\sigma}{s}{A[\pi_1]}
    \quad  \goodSub{\Gamma}{\gamma}{\Delta}  }
  {\goodTerm{\Gamma,\cC\sigma[\gamma]}{s[\gamma^\uparrow]}{A[\pi_1][\gamma^\uparrow]\equiv A[\gamma][\pi_1]}}
  $$
  $$
  \Rule[name=Seal-Id]
  {}
  {\goodSeal{\Gamma}{id_s}{\sigma}{\cC \sigma}}
  = \goodTerm{\Gamma, \cC\sigma}{\pi_2}{\cC\sigma[\pi_1]}
  $$

\medskip

\end{figure}

$$
\Rule[name=Hdler]
{\goodType{\Gamma}{A}{}
\quad \goodType{\Gamma, A}{B}{}
\quad \goodType{\Gamma}{R}{}
}
{\goodType{\Gamma}{"RecSig1"~A~B~R \equiv (\Pi A (\Pi (\Pi B (R[\pi_1\circ\pi_1])) R[\pi_1\circ\pi_1]) )}{}}
\quad 
\Rule[name=Hdlers]
{\goodWSig{\Gamma}{\tau}{n}
\quad \goodType{\Gamma}{R}{}
}
{\goodSig{\Gamma}{"RecSig"~\tau~R}{n}}
$$
$$
\Rule[name=Hdler-Proj]
{ \goodWSig{\Gamma}{\tau}{N}
\quad \goodTerm{\Gamma}{o}{\cL("RecSig"~\tau~R)}
\quad j < N
}
{\goodTerm{\Gamma}{"prjR"~j~o}{("RecSig1"~(\pi_1^j \tau)~(\pi_2^j \tau)~R)[\pi_1]}}
$$
$$
\Rule[name=Rec-Constr]
{ \goodWSig{\Gamma}{\tau}{N}
\quad \goodTerm{\Gamma}{T}{\bW \tau}
\quad \goodType{\Gamma}{R}{}
\quad \goodTerm{\Gamma}{r}{\cL("RecSig"~\tau~R)}
\quad \goodTerm{\Gamma}{w}{El~("W"~T)}
}
{\goodTerm{\Gamma}{"Wrec"~r~w}{R}}
$$


\judgebox{\goodInh{\Gamma}{h}{\sigma}{\tau}}

$$
\Rule[name=Inh-Id]
{}
{\goodInh{\Gamma}{"inhid"}{\sigma}{\sigma}}
\quad
\Rule[name=Inh-Override]
{
\goodInh{\Gamma}{h}{\sigma_1}{\sigma_2}  
\quad \goodType{\Gamma, A_1}{T_1}{}
\quad \goodType{\Gamma, A_2}{T_2}{}
  \quad \goodTerm{\Gamma, A_2}{t}{T_2}}
{\goodInh{\Gamma}{"inhov" \ h \ t}{(\nu^+ \  \sigma_1 \  T_1)}{(\nu^+ \  \sigma_2\  T_2)}}
$$

$$
\Rule[name=Inh-Ext]
{\goodInh{\Gamma}{h}{\sigma_1}{\sigma_2}
  \quad \goodTerm{\Gamma, A_2}{t}{T}}
{\goodInh{\Gamma}{"inhext" \ h \ t}{\sigma_1}{(\nu^+ \  \sigma_2\  T)}}
\quad
\Rule[name=Inh-Inh]
{\goodInh{\Gamma}{h}{\sigma_1}{\sigma_2}
\quad \goodType{\Gamma, A_1}{T}{}
\quad \goodTerm{\Gamma, A_2}{\uparrow^s}{A_1[\pi_1]}
}
{\goodInh{\Gamma}{"inhinh" \ h}{(\nu^+ \  \sigma_1 \  T)}{(\nu^+ \  \sigma_2 \  T[(\pi_1, \uparrow^s)])}}
$$
$$
\Rule[name=Inh]
{\goodInh{\Gamma}{h}{\sigma_1}{\sigma_2}
\quad \goodTerm{\Gamma}{l}{\cL \sigma_1}
}
{\goodTerm{\Gamma}{("inh" \ h \ l)}{\cL \sigma_2}} 
\quad
\Rule[name=Inh+Inh]
{\goodInh{\Gamma}{h}{\sigma_1}{\sigma_2}
\quad \goodTerm{\Gamma, A_2}{\uparrow^s}{A_1[\pi_1]}
\\\\
\goodInh{\Gamma, A_2}{i}{\tau_1[(\pi_1, \uparrow^s)]}{\tau_2}}
{\goodInh{\Gamma}{"inhnest" \ h \ i}{(\nu^+ \  \sigma_1\  \cL\tau_1)}{(\nu^+ \  \sigma_2\  \cL\tau_2)}}
\quad
$$

$$
\Rule[name=Inh-Inh-beta]
{\goodInh{\Gamma}{h}{\sigma_1}{\sigma_2}
  \quad \goodTerm{\Gamma}{m}{\cL \sigma_1}
  \quad \goodTerm{\Gamma, A_1}{t}{T}
  \quad \goodTerm{\Gamma, A_2}{\uparrow^s}{A_1[\pi_1]}
}
{\goodTerm{\Gamma}{"inh" ("inhinh" \ h) (\mu^+ \ m \  \{f_1\} \ t) \newline  \equiv \mu^+ \ ("inh" \ h \ m) \  \{f_2\} \ t[(\pi_1, \uparrow^s)]}{...}} 
$$

\section{Metatheory, in detail}

\label{sec:metatheory}

We start with the formulation of the syntax, which is based on the formalization of a predicative \textit{Martin-Lof Type Theory} (MLTT) given by \citep{coquand2018canonicity}. Thus we will omit a great deal of details about MLTT here but focus on the new facility we introduced. 

Though phrased in Latex, we are following \citet{altkap2016} to use as
our meta-logic a type theoretical framework (like Agda), 
together with \textit{Quotient Inductive Inductive Type} (QIIT). Each
judgement will actually be represented by a QIIT type, which are
mutually recursively defined together. Quotient is used to represent
judgemental(definitional) equality, thanks to which we can have a
concise representation. This way, for all four kinds of judgement we are
using, there are always equality among these data. Doing so, we can omit
some obvious equality judgement. Most of the reasoning can be formulated
using the \textit{algebra} of QIIT (mapping out function from this
quotient data-type). 

We also highlight some special feature in this style of formulation of dependent type theory: 1. this modern formulation of type theory does't have operational semantic but only equality 
%We will recover canonicity afterwards, which implies the computational ability of our theory. 
;
2. instead of using meta-level substitution, we use explicit substitution and de Bruijn indices in our intrinsic formulation. This is also called substitution calculus in the literature and is favoured due to its clear categorical semantic; 
3. we are dealing with quotient data, so we are actually manipulating equivalence classes of terms. 

We will write out the syntax of the whole type theory for completeness.


\newcommand{\denotes}[1]{{\llbracket {#1} \rrbracket}}
\newcommand{\denotesS}[1]{{{\llbracket {#1} \rrbracket}_S}}
\newcommand{\goodCtx}[2]{{ {#1} \ \vdash }}
\newcommand{\goodType}[3]{{ {#1} \vdash {#2} }}
\newcommand{\goodTerm}[3]{{ {#1} \vdash {#2} : {#3} }}
\newcommand{\goodSub}[3]{{ {#1} \vdash {#2} : {#3} }}
\newcommand{\goodSig}[3]{{ {#1} \vdash {#2} \ \  Sig^{#3} }}
\newcommand{\goodWSig}[3]{{ {#1} \vdash {#2} \ \ WSig^{#3} }}
\newcommand{\goodSeal}[4]{{ {#1} \vdash {#2} : {#3} \  |\  {#4} }}
\newcommand{\goodInh}[4]{{ {#1} \vdash {#2} : {#3} \twoheadrightarrow {#4}}}
\newcommand{\nat}{\mathbf{N}}

\newcommand{\cU}{{\mathcal{U}}}
\newcommand{\cB}{{\mathbb{B}}}
\newcommand{\cL}{{\mathcal{L}}}
\newcommand{\cC}{{\mathcal{C}}}
\newcommand{\cCt}{{\mathcal{C}_t}}
\newcommand{\bW}{{\mathbb{W}}}



\subsection{Syntax}




We formalize our language as a deep embedding in a type theory with QIIT types. Every judgement is a (dependent) QIIT \textit{type} and all judgement derivations are well-typed QIIT terms. (Otherwise how can they be encoded in Agda?) There are four standard judgement forms in MLTT,
well-typed contexts $\goodCtx{\Gamma}{l}$, well-typed types
$\goodType{\Gamma}{T}{l}$, well-typed terms $\goodTerm{\Gamma}{t}{T}$, and
well-typed substitutions $\goodSub{\Gamma}{\gamma}{\Delta}$. These four judgements (QIIT types) are
mutually recursively defined together. Our presentation
follows the style of~\citet{kaposi2019gluing}, but for readability we
omit the universe levels. We still working on predicative MLTT.
%\YZ{What's the distinction of judgments vs. sequents?}.\EDJreply{I am not sure about the correct terminology here. Here I want to say "even underivable judgement has to be well-typed". That's the main reason I spell out the fake Agda code there. I actually want the reader to have intrinsic-typed syntax in mind all the time (like read conventional math but acknowledge we are manipulating Agda)}\EDJreply{I rephrase that so there won't be confusion for the future readers. Please check}
Latex-wise, we spell out these \textit{well-formedness rules} of type
judgements and their Agda counterparts to make things more accessible. Of
course, a well-formed judgement is not necessarily derivable, just as
a well-formed QIIT type is not necessarily inhabited. 


Due to our intrinsic setting, \textbf{we can and will omit a lot of
presumptions of the typing rules}. This can work because we are using
QIIT and everything is intrinsically well-formed. For example, we don't
spell out that the rule \ruleref{Type Universe} requires a well-formed
context, even though it does implicitly, \textbf{because it is not
possible to have ill-formed contexts in our intrinsic setting}. This is
a bit unusual for conventional extrinsic setting where they will put
well-formedness of context everywhere in a lot of term judgements. We
consider this as one advantage of using intrinsic setting---this
well-formed context is usually attached to some other presumptions thus
inferrable and can be considered as ``boilerplate code''. Free of
such boilerplate code, our intrinsic setting provides a more concise
representation.

Quotient is used to represent definitional equality (aka judgemental
equality) between QIIT types concisely.
This way, for all four kinds of judgements we are
using, there is always equality among these data. Doing so, we can omit
some obvious equality judgements. Most of the reasoning can be formulated
using the \textit{algebra} of QIIT (mapping out function from this
quotient data-type). 

Now we highlight some other special features that differ from a more
conventional setting: (1) This modern formulation of type theory does not
have an operational semantics but only equality.\footnote{Later we will prove
canonicity which can imply a certain level of computational ability of
our theory}
%We will recover canonicity afterwards, which implies the computational ability of our theory. 
(2) Instead of using meta-level substitution, we use explicit
substitution and de Bruijn indices in our intrinsic formulation. This is
also called substitution calculus in the literature and is favoured due
to its closedness categorical semantic.
(3) We are dealing with quotient data, so we are actually manipulating
equivalence classes of terms.

We quickly review the substitution calculus.
Substitution is very (if not the most) important because function is
first-class citizens, and can be applied (almost) everywhere and can return
(almost) everything, thus we require the resulting substitution to
commute with most (if not all) type/term constructs. 

A term $\goodTerm{\Gamma}{t}{A}$ can be understood as one term $t$ of
type $A$, with a bunch of holes (of type $\Gamma$) inside the term $t$.
A substitution $\goodSub{\Gamma}{\gamma}{\Delta}$ can be similarly
understood in the way that $\gamma$ will output a \textbf{series} of
(dependently-typed related) terms of types $\Delta$, parametrized by
$\Gamma$ (i.e., we can consider there are holes of type $\Gamma$ in $\gamma$).
Thus (1) \ruleref{Sub Extend} looks like concatenating one
series of terms with a new term (the weird looking type of
$\goodTerm{\Gamma}{t}{A[\sigma]}$ is because we need $A$ to be typed in
the context $\Delta$ so that we can use \ruleref{Context Extension} to
create new context); (2) we have \ruleref{Proj Subst-1} to extract a
subpart of a given substitution; (3) \ruleref{Proj-Ext} explains why
substitution really looks like a (deeply-nested) pair. 

Since the $\Gamma$ before the $\vdash$ is always indicating the types of
``holes'', this explains the \ruleref{Term Subst} and \ruleref{Type
Subst}: given a term $\goodTerm{\Delta}{t}{A}$ and a substitution
$\goodSub{\Gamma}{\gamma}{\Delta}$, we will have a new term
$\goodTerm{\Gamma}{t[\gamma]}{A[\gamma]}$ after the substitution---this
substitution can be understood as ``filling all the holes''. Once all
the holes are filled with a given substitution, the new holes all come
from the substitution itself.
Consecutive substitutions can be composed using \ruleref{Sub Comp}, because
all the holes are always from the last substitutions.

Note that, for simplicity, we will often write $\pi_1$ as a
shorthand for ${\pi_1~"id"}$, typed as
$\goodSub{\Gamma,A}{\pi_1~"id"}{\Gamma}$, and $\pi_2$ as a shorthand
for $\pi_2~"id"$, typed as $\goodTerm{\Gamma, A}{\pi_2~"id"}{A[\pi_1]}$.



The core language builds on MLTT and adds four new judgement forms.
Among them, sealing judgement $\goodSeal{\Gamma}{\_}{\sigma}{A}$ is just a
shorthand for term judgement
$\goodTerm{\Gamma , \cC \sigma}{\_}{A[\pi_1]}$ where $A$ is a type.



\newcommand\mathboxtext[1]{
  \fcolorbox{black}{faint-gray}{\ensuremath{#1}}
}

\begin{figure}[!htb]
  \begin{align*}
    &\Gamma~:~"Con"&&\mathboxtext{\goodCtx{\Gamma}{}} &&T~:~"Ty"~\Gamma&&\mathboxtext{\goodType{\Gamma}{T}{}}  &&\gamma~:~"Tms"~\Gamma~\Delta&&\mathboxtext{\goodSub{\Gamma}{\gamma}{\Delta}}  \\ & t~:~"Tm"~\Gamma~T&&\mathboxtext{\goodTerm{\Gamma}{t}{T}} 
    &&\sigma~:~"Sig"~\Gamma~n&&\mathboxtext{\goodSig{\Gamma}{\sigma}{n}}  &&\tau~:~"WSig"~\Gamma~n&&\mathboxtext{\goodWSig{\Gamma}{\tau}{n}} \\ &h~:~"Inh"~\Gamma~\sigma_1~\sigma_2&&\mathboxtext{\goodInh{\Gamma}{h}{\sigma}{\tau}} && f~:~"Seal"~\Gamma~\sigma~T&&\mathboxtext{\goodSeal{\Gamma}{f}{\sigma}{A}}\YZ{A little unfriendly to the reader that there are three different things that sigma can range over. Besides, later you also use gamma to range over substitutions and s to range over W-types, which add to the confusion.}\EDJreply{Fair point. I have fixed fix sigma and tau to be about signatures and W signature, and use gamma as substitutions}
  \end{align*}
\caption{Judgement forms, as Agda types and as presented in this section.}
\end{figure}

\begin{figure}[!htb]
  \begin{minipage}[b]{0.3\linewidth}
      $$
      \Rule[name=Tm]
      {\goodCtx{\Gamma}{i} \quad \goodType{\Gamma}{T}{j}}
      {\goodTerm{\Gamma}{\_}{T}}
      $$
      $$
      \Rule[name=Tms]
      {\goodCtx{\Gamma}{i} \quad \goodCtx{\Delta}{i}}
      {\goodSub{\Gamma}{\_}{\Delta} \text{ is well-formed}}
      $$
      $$
      \Rule[name=Sig]
      {\goodCtx{\Gamma}{i} \quad n \in \nat}
      {\goodSig{\Gamma}{\_}{n} \text{ is well-formed}}
      $$
      $$
      \Rule[name=Seal]
      {\goodCtx{\Gamma}{i} \quad \goodSig{\Gamma}{\sigma}{n} 
      \quad \goodType{\Gamma}{A}{} }
      {\goodSeal{\Gamma}{\_}{\sigma}{A} \text{ is well-formed}}
      $$\YZ{Doesn't the left column show well-formedness rules for ONLY TWO of the four new judgments? Besides, are these rules used anywhere in the following? Seems a bit unusual to have well-formedness rules for judgments per se.}\EDJreply{Yes. Right column show all the well-formedness rule for all the judgement. That is unusual because it is a direct translation from QIIT. }
  \end{minipage}
  \begin{minipage}[b]{0.6\linewidth}
    \begin{minted}[]{agda}
      data Con   : Set 
      data Ty    : Con → Set   
      data Tms   : Con → Con → Set 
      data Tm    : (Γ : Con) → Ty Γ  → Set 
      data Sig  : Con → ℕ → Set
      data WSig : Con → ℕ → Set 
      data Inh  : (Γ : Con) → Sig Γ n → Sig Γ m → Set
      Seal : (Γ : Con) → Sig Γ n → Sig Γ n → Set
    \end{minted}
  \end{minipage}

\caption{Well-formedness rules of the judgements}
\end{figure}





% \begin{minted}[texcomments]{Coq}
%   Γ : Con  (* $\goodCtx{\Gamma}{}$ *) 
% \end{minted}
% T : Ty Γ (* $\goodType{\Gamma}{T}{}$ *)       σ : Tms Γ Δ {- $\goodSub{\Gamma}{\sigma}{\Delta}$ -}        t : Tm Γ T {- $\goodTerm{\Gamma}{t}{T}$ -}





We will have dependent function types
and dependent pair types. Dependent pair type is used to model Coq
module, which is useful when we need to model the ``compilation'' from
linkages to modules. We also have $\beta,\eta$-rule for both dependent
function type and dependent pair type.
The substitution law for these two
require \textit{de Bruijn lifting}: for a given substitution
$\goodTerm{\Gamma}{\sigma}{\Delta}$ and a type
$\goodType{\Delta}{A}{}$, we define its de Bruijn lifting
$\goodTerm{\Gamma,A[\sigma]}{\sigma^\uparrow := (\sigma \circ \pi_1, \pi_2)}{\Delta, A}$ . Intuitively speaking, de Bruijn lifting is to ``fill all the holes except the last one''.

Because in dependent type, function now can return type, of which the type is ``type of type'', \textit{the universe} $\cU$. We follow the style of \textit{type universe à la Tarski}~\cite{hofmann1997syntax} and
distinguish types from their \textit{codes or names}, and thus we use
$El\ T$ as the type given \textit{type name} $T : \cU$.
In this case, (1) $\cU$ is the type including the \textit{codes or
names} of types, and (2) $T$ is not allowed to locate at type position
(i.e., at the rightmost position of a term judgement) but $El \ T$ is.
"c" is the ``inverse'' of $El$, obtaining  the \underline{"c"}ode of a
type, with the following two equations $"c" (El~T) \equiv T$ and $El
("c"~T) \equiv T$.
Recall that we are actually using predicative universe, so we actually
have different levels of universes $\cU_j$.
We omit universe levels for conciseness.


  \label{fig:rules:well-typed-ctx}
\judgebox{\goodCtx{\Gamma}{i}}
$$ 
\Rule[name=Empty Context]{}{\goodCtx{\cdot}{0}} 
\quad
\Rule[name=Context Extension]
{\goodCtx{\Gamma}{i} \quad \goodType{\Gamma}{A}{j}}
{\goodCtx{\Gamma, A}{i \cup j}}  
$$


\judgebox{ \goodType{\Gamma}{T}{i} }
$$
\Rule[name=Type Universe]
{}
{\goodType{\Gamma}{\cU}{j + 1}}
\quad 
\Rule[name=Boolean]
{}
{\goodType{\Gamma}{\cB}{0}}
\quad 
\Rule[name=Bottom]
{}
{\goodType{\Gamma}{\bot}{0}}
\quad 
\Rule[name=Function]
{\goodType{\Gamma}{A}{j} 
  \quad \goodType{\Gamma, A}{B}{k}}
{\goodType{\Gamma}{\Pi A B}{j \cup k}}
$$

$$
\quad 
\Rule[name=Func/DPair Subst]
{\goodSub{\Gamma}{\gamma}{\Delta}
\quad \goodType{\Delta}{A}{} 
\quad \goodType{\Delta, A}{B}{}
}
{
  \goodType{\Gamma}{(\Pi A B)[\gamma] \equiv \Pi A[\gamma] B[\gamma^\uparrow] }{j \cup k}
  \quad 
  \goodType{\Gamma}{(\Sigma A B)[\gamma] \equiv \Sigma A[\gamma] B[\gamma^\uparrow] }{j \cup k}
}
$$
%\YZ{What is the requirement on Delta?}
%\EDJreply{ Delta here is of type Con and thus Delta here should be a well-typed Context. This is another issue I want to talk about, in Agda these kinds of stuff is inferred (so the reader can infer them as well). For example, we know delta is at the place of context, so it is a context, but we don't need to specify that it is a ``well-typed context'' because every context is well-typed in this intrinsic-typing. What's more, I think adding this extra judgement here just cause significant code bloat (like what I complained to you before I know how to use implcit variable in agda. For example, TYPE SUBST, there will be two more judgements (with no more information there) ) (Another reason I prefer fake agda formulation) }\YZreply{I was thinking about 'Delta |- A'. I suppose it is also implicitly enforced via intrinsically typed syntax. Seems to me it is worth stating, however, or otherwise the rule appears to say that Delta can be any context as long as Gamma |- gamma : Delta.}\EDJreply{You are absolutely right! I think Delta |- A, even though inferrable, mentioning it is good, also some information are B is also ok, please check, resolve if ok, thanks!}
$$
\Rule[name=Type Subst]
{\goodType{\Delta}{T}{j} 
  \quad {\goodSub{\Gamma}{\gamma}{\Delta}}}
{\goodType{\Gamma}{T[\gamma]}{j}}
\quad
\Rule[name=Base Type Subst]
{\goodSub{\Gamma}{\gamma}{\Delta}}
{\goodType{\Gamma}{\cU[\gamma] \equiv \cU }{j + 1} \quad
  \goodType{\Gamma}{\cB[\gamma] \equiv \cB}{0} \quad 
  \goodType{\Gamma}{\bot[\gamma] \equiv \bot}{0}
}
$$
\judgebox{ \goodTerm{\Gamma}{t}{T} }
$$
\Rule[]
{\goodType{\Gamma}{T}{j}}
{\goodTerm{\Gamma}{"c" \ T}{\cU}
}\quad
\Rule[]
{\goodTerm{\Gamma}{T}{\cU}}
{\goodType{\Gamma}{El \ T}{j}}
\quad
\Rule
{}
{\goodTerm{\Gamma}{"tt", "ff"}{\cB}}
\quad 
\Rule[name=Term Subst]
{\goodTerm{\Delta}{t}{T}
  \quad {\goodSub{\Gamma}{\gamma}{\Delta}}}
{\goodTerm{\Gamma}{t[\gamma]}{T[\gamma]}}
$$
$$
\Rule[]
{\goodTerm{\Gamma, A}{t}{B}}
{\goodTerm{\Gamma}{\lambda t}{\Pi A B}}
\quad 
\Rule[]
{\goodTerm{\Gamma}{u}{A} 
\quad \goodTerm{\Gamma}{v}{B[(id, u)]}}
{\goodTerm{\Gamma}{(u,v)}{\Sigma A B}}
\quad 
\Rule[]
{\goodTerm{\Gamma}{t}{\Sigma A B}}
{\goodTerm{\Gamma}{"pjl" \ t}{A}
\quad \goodTerm{\Gamma}{"pjr" \  t}{B[(id, "pjl" \  t)]}
}
$$
$$
\Rule
{}
{\goodTerm{\Gamma}{(\lambda t)[\gamma] \equiv \lambda (t[\gamma^\uparrow])}{\Pi A B}
\quad \goodTerm{\Gamma}{(u,v)[\gamma] \equiv (u[\gamma],v[\gamma])}{\Sigma A B}
\quad \goodType{\Gamma}{El \ (T[\gamma]) \equiv (El \ T) [\gamma]}{}
}
$$

$$
\Rule[name=Base Type/Term Subst]
{\goodSub{\Gamma}{\gamma}{\Delta}}
{\goodType{\Gamma}{(El \ T)[\gamma] \equiv El \ (T[\gamma]) }{j} \quad
 \goodTerm{\Gamma}{"tt"[\gamma] \equiv "tt"}{\cB} \quad 
 \goodTerm{\Gamma}{"ff"[\gamma] \equiv "ff"}{\cB} 
}
$$
\judgebox{\goodSub{\Gamma}{\sigma}{\Delta}}
$$
\Rule[name=Ept Subst]
{}{\goodSub{\Gamma}{\epsilon}{\cdot}}
\quad
\Rule[]
{}{\goodSub{\Gamma}{"id"}{\Gamma}}
\quad
\Rule[name=Sub Comp]{
  \goodSub{\Delta}{\delta}{\Theta}
  \quad \goodSub{\Gamma}{\gamma}{\Delta} 
}{\goodSub{\Gamma}{\delta \circ \gamma}{\Theta}}
\Rule[name=Sub Extend]
{\goodSub{\Gamma}{\sigma}{\Delta} \quad \goodTerm{\Gamma}{t}{A[\sigma]}}
{\goodSub{\Gamma}{(\sigma, t)}{(\Delta, A)}}
$$

$$
\Rule[name=Proj Subst-1]
{\goodSub{\Gamma}{\sigma}{(\Delta, A)}}
{\goodSub{\Gamma}{\pi_1 \sigma}{\Delta}}
\quad
\Rule[name=Proj Subst-2]
{\goodSub{\Gamma}{\sigma}{(\Delta, A)}}
{\goodSub{\Gamma}{\pi_2 \sigma}{A[\pi_1 \sigma]}}
\quad
\Rule[name=Proj-Ext]
{}
{\goodSub{\Gamma}{(\pi_1 \sigma, \pi_2 \sigma) \equiv \sigma}{\Delta}}
$$

We have shown some exemplar rules for all four standard types of
judgments. Now we focus on the newly introduced facility. 


We have two kinds of signatures, one for inductive types, the other for linkages.



\judgebox{\goodWSig{\Gamma}{\tau}{n}}
$$
\Rule[name=Emp-WSig]
{}
{\goodWSig{\Gamma}{w\cdot}{0}}
\quad
\Rule[name=WSig-Add]
{\goodWSig{\Gamma}{\tau}{n}
  \quad \goodType{\Gamma}{A}{i}
  \quad \goodType{\Gamma, A}{B}{i}}
{\goodWSig{\Gamma}{w^+ \  \tau \  A \  B}{n+1}}
\quad
\Rule[name=Ind-Univ]
{\goodWSig{\Gamma}{\tau}{n}}
{\goodType{\Gamma}{\bW \tau}{i}}
$$
%\YZ{Is there an introduction rule for terms of type 'bW sigma'?}
%\EDJreply{Yes there is. I omit it because it is a singleton type, please see my newly added explanation below.}

$$
\Rule[name=WSig-Proj]
{\goodWSig{\Gamma}{\tau}{n} \quad j < n}
{\goodType{\Gamma}{\pi^j_1 \tau}{i} \quad \goodType{\Gamma, \pi^j_1 \tau}{\pi^j_2  \tau}{i}}
\quad
\Rule[name=Ind-Sig]
{\goodWSig{\Delta}{\tau}{n}
  \quad {\goodSub{\Gamma}{\gamma}{\Delta}}}
{\goodWSig{\Gamma}{\tau[\gamma]}{n}
  \quad \goodType{\Gamma}{W (\tau[\gamma]) \equiv (W \tau)[\gamma]}{i}}
$$

$$
\Rule[name=Ind-Type]
{\goodTerm{\Gamma}{T}{\bW \tau}}
{\goodTerm{\Gamma}{"W" \ T}{\cU}}
\quad
\Rule[name=Ind-Term]
{\goodTerm{\Gamma}{T}{\bW \tau}
  \quad \goodTerm{\Gamma}{a}{\pi^j_1\tau}
  \quad \goodTerm{\Gamma, \pi^j_2\tau[(id, a)]}{b}{El~("W"~T)}}
{\goodTerm{\Gamma}{"Wsup"~T~a~b}{El~("W"~T)} }
$$


Recall in W-type~\cite{martin1982constructive}, a pair of type $x : A
\vdash B(x)$ decides a W-type. To simulate extensible inductive types, we
have to equip our inductive types with multiple constructors, and for
simplicity, each constructor has a fixed form \mintinline{agda}{cstrᵢ :
(x : Aᵢ) → (Bᵢ x → W) → W}.
Thus, we use a list of pairs ($A_i, B_i$) as the signature of one
inductive type, and given an inductive type signature, we use
\ruleref{WSig-Proj} to extract the corresponding $A_i, B_i$.

Of course we need to put an eye on the substitution law for each piece
of the syntax. Here $\bW \sigma$ is a large singleton
type~\cite{stone2000}, and it only has one ``element'', the inductive
type itself, inside. We formulate in this cumbersome way because we want
to have inductive type to be a field element and exposing the
constructors at the type level for later \textit{pattern-matching
exhaustiveness checking} (i.e. if an inductive type as field element has type
$\cU$, then no information about the inductive type is exposed).
 

The formulation of our recursor is closely related to that of linkages,
so we will postpone their description.




\judgebox{\goodSig{\Gamma}{\sigma}{n} }
$$
\Rule[name=Lnkg Type/Compile]
{\goodSig{\Gamma}{\sigma}{n}}
{\goodType{\Gamma}{\cL \sigma}{i}
\quad \goodType{\Gamma}{\cC \sigma}{}}
\quad
\Rule[name=Sig/Lnkg Subst]
{\goodSig{\Delta}{\sigma}{n}
  \quad {\goodSub{\Gamma}{\gamma}{\Delta}}}
{\goodSig{\Gamma}{\sigma[\gamma]}{n}
  \quad \goodType{\Gamma}{\cL (\sigma[\gamma]) \equiv (\cL \sigma)[\gamma]}{i}}
$$

$$
\Rule[name=Ept Sig]
{}
{\goodSig{\Gamma}{\nu\cdot}{0}}
\quad
\Rule[name=Sig Add]
{\goodSig{\Gamma}{\sigma}{n} 
 \quad \goodSeal{\Gamma}{f}{\sigma}{A}
 \quad \goodType{\Gamma, A}{T}{i}}
{\goodSig{\Gamma}{(\nu^+ \ \sigma \ \{f\} \ T)}{n+1}}
$$

$$ 
\Rule[name=Sig Proj]
{\goodSig{\Gamma}{\sigma}{n+1}}
{\goodSig{\Gamma}{p_1\nu \ \sigma}{n}
\\ \goodType{\Gamma}{p_1\nu' \ \sigma}{}
\\ \goodSeal{\Gamma}{p_f\nu \  \sigma}{p_1\nu \  \sigma}{p_1\nu' \sigma}
\\ \goodType{\Gamma, p_1\nu' \ \sigma}{p_2\nu \ \sigma}{}
}
$$

$$
\Rule[name=Ept Lnkg]
{}
{\goodTerm{\Gamma}{\mu\cdot}{\cL \nu\cdot}}
\quad
\Rule[name=Lnkg Add]
{ \goodTerm{\Gamma}{o}{\cL \sigma} 
\quad  \goodSeal{\Gamma}{f}{\sigma}{A} 
 \quad \goodTerm{\Gamma, A}{t}{T}
}
{\goodTerm{\Gamma}{(\mu^+ \ o \ \{f\} \ t)}{\cL(\nu^+ \ \sigma \ \{f\} \ T)}}
$$

$$
\Rule[name=Lnkg Proj]
{\goodTerm{\Gamma}{o}{\cL\sigma}}
{\goodTerm{\Gamma}{p_1\mu \ o}{\cL (p_1\nu \ \sigma)}
\\ \goodTerm{\Gamma, p_1\nu' \ \sigma}{p_2\mu \ o}{p_2\nu \ \sigma}
}
\quad 
\Rule[name=Lnkg Compile]
{ \goodTerm{\Gamma}{o}{\cL \sigma} 
}
{
  \goodTerm{\Gamma}{\cCt o}{\cC \sigma}
}
$$

$$
\Rule[name=Compile]
{}
{\goodType{\Gamma}{\cC \nu\cdot \equiv \top}{} 
\\ \goodTerm{\Gamma}{ \cCt \mu\cdot \equiv ()}{\cL \nu\cdot}
\\
\goodType{\Gamma}{\cC (\nu^+ \ \sigma \ \{f\} \ T) \equiv 
    \Sigma (\cC \sigma) T["sf" \ f]}{}
\\\\ \goodTerm{\Gamma}{\cCt \ (\mu^+ \ o \ \{f\}\ t) \equiv ((\cCt \ o), t["sf" \ f][(id, \cCt \ o)]) }{}
}
$$
% $$
% \Rule
% { \goodSeal{\Gamma}{f}{\sigma}{A}
% \\ \goodType{\Gamma, A}{T}{}
% \\ \goodTerm{\Gamma}{t}{T}
% }{}
% $$

$\goodSig{\Gamma}{\sigma}{n}$ means $\sigma$ is a well-typed signature
for linkages in the context $\Gamma$ with $n$ fields. We omit the
naming/label information for fields in the meta\-theory. The signature (for the linkage) and the linkage rules are quite similar to
the signature for inductive types.  They are both constructed from the
empty stuff, and respect substitution lemmas. We have
\ruleref{Sig Proj}, \ruleref{Lnkg Proj} projection for each component
due to harmony~\cite{pfenning2009lecture}, so we have local soundness
and local completeness ($\beta,\eta$ rules). We omit them and the
substitution laws for either field addition and projection. For
$\mu^+,\nu^+$ rules, we will omit the sealing $\{f\}$ if possible.\footnote{This
omission can be supported by Agda using Implicit Arguments}

A side note: before we dive in more details of our metatheory for linkage (especially the meaning of sealing and $\cC$), we have to mention that, the challenge of incorporating family polymorphism with inductive type mentioned in \ref{chg:extensible-inductive-type} mutates in metatheory because of de Bruijn indices. (Using surface synax), consider the example of a linkage $\{a~:~\bW \sigma; b_1 : a \to T;..\}$ with a bunch of $b_i$ \textit{using} $a$. The tension we need to solve here is between \textit{exhaustiveness checking} and \textit{inheritance (code reuse)}---on one hand, if we want to make sure $b_i$ is pattern matching correctly, then we have to construct $b_i$ term in the context of $a : \bW\sigma$; on the hand, if we want to let $b_i$ to be inherited into children family with extended inductive type $a~:~\bW \sigma'$, then the old inductive type $a~:~\bW \sigma$ in the context of $b_i$ is stopping that and thus we hope $b_i$ constructed in the context of $a : \cU$ (and thus later this $a$ can be compatible with inductive type of any signature). Our metatheory formulation of linkage is mainly trying to solve this tension.



We use $\cL$ to retrieve the corresponding types of linkage of a given signature, and $\cC$ to get the compiled linkage type (an existential type).
$\cL\sigma$ corresponds to the type of a family term, but $\cC\sigma$, according to \ruleref{Compile}, is a deeply nested dependent pair. Then $\cCt \sigma$ provide the concrete module of type $\cC\sigma$ given a linkage of type $\cL\sigma$. 

The reason we need this additional type construct of $\cC\sigma$ here is
to support \textit{code reuse}, as the rule of linkages hinders its
ability to do abstraction. Concretely speaking, (using surface syntax)
we don't have \\ $\goodTerm{\Gamma, \cL(\{a~:~\bW \sigma; b~:~a \to
T\})}{h}{\cL(\{a~:~\cU; b~:~a \to T\})}$\footnote{If we had one,
consider the case $T$ is bottom type and $\sigma$ is empty record and
thus $a$ is initially bottom type. Now the direct projection using
\ruleref{Sig Proj} $\goodTerm{\Gamma, a : \cU}{h.b}{a \to \bot}$ will
lead to problem because every field in linkage has this implicit
universal quantification, and for this case we can substitute $a$ with
arbitrary type, say $\top$: $\goodTerm{\Gamma}{h.b[a \mapsto \top]}{\top
\to \bot}$ is a contradiction}.
However, this underivable term is important for us to abstract the
concrete inductive type so that the later field can depend on the
``abstract interface'' (i.e. something like $\cL(\{a:\cU~;~b:a \to T\})$) and achieve \textit{code reuse}. 
\YZ{It reads as if you want to have 'h : L\{a : U, b : a -> T\}'. But it is problematic because it leads to inconsistency, as you describe in a footnote.}\EDJreply{Good point. I rephrased it so the introduction of sigma type is smoother. Please check.}
\\ The good news, we don't have to stick with $\cL(\sigma)$ to express the abstract interface. We can also use sigma type, and actually 
we do have $\goodTerm{\Gamma, (\Sigma a : \bW\sigma, a \to T)
}{h}{(\Sigma a : \cU, a \to T)}$ because of the rules for sigma type.\footnote{The intro rule for sigma type decides that there is no such implicit universal quantification happening.}
\YZ{This sentence still speaks of overriding. Explain in terms of projection instead?}\EDJreply{Fixed. Please check.}
Thus to allow abstraction happens, we need $\cC\sigma$ to compile a
linkage to module as a part of the context and we also need
\ruleref{Lnkg Compile} to specify the compilation from linkage to
module.


This is also reflected in the implementation of our plugin : we know every field is defined in the context of the former fields, but our implementation actually make every field dependent on \textbf{the compiled module of the already-defined-part of the family}. Thus we need $\cC\sigma$ in our context.
%\YZ{Why is Csigma needed? How is it different from Lsigma?}\EDJreply{It corresponds to the compilation of the family ``type'', which is part of the implementation detail. Basically we know every field of a family is dependent on the former fields. The implementation detail here is that, every field of a family is dependent on the compiled module of the former family (and thus includes all the former fields)}\YZreply{I don't think you should explain the metatheory from the angle of how things are implemented in your plugin. Rather, when describing your Coq plugin, you can say it matches the metatheory.
%PS: you use two terminologies: 'compile' and 'seal', but I guess they mean very similar things. So avoid saying 'compile' here.
%PPS: Sealing means hiding, so what is being hided? Why do you need to hide it? And how does the 'Compile' and 'Seal' rules perform hiding?}\EDJreply{  Compile is very different from seal.  Compile is making a linkage into a module, depriving the ability to override---if we understand a linkage as a module with a lot of universal quantifier inside. Compile cannot hide things, but only translate a linkage into a module. Seal is really something that does hiding---because existential type can hide stuff. Seal is an term between existential types. \\
%The ultimate reason we have this two notion is that, the introduction rule for linkage fails to provide abstraction mechanism so we have to translate them into module. \\
%Generally speaking Compile : Linkage -> Module, Seal : Module -> Module, we need module here because linkage cannot do abstraction of inductive type. A concrete detail is at the above. \\ 
%I now describe the motivation of compile in the earlier paragraph, right before the "reflecting" section. Note that, this compile is inspired by plugin implementation. This is the reason I say the metatheory is really close and inspired by the implementation. Similarly as I say ``there is no extensible inductive type but only overriding of new inductive type'' because in metatheory it is using overriding to ``extend'' (same as plugin implementation), seal means the same thing, and ``compile'' in metatheory happens in plugin implementation
%}

However, $\cC$ can only make sure we have $\cL(\{a~:~\bW \sigma; b~:~a
\to T\}) \to (\Sigma a:\bW \sigma, a \to T)$, we still need $(\Sigma
a:\bW \sigma, a \to T) \to (\Sigma a:\cU \sigma, a \to T)$ to complete
this shifting of the context. This is sealing's job.



\begin{figure}[H]

  \judgebox{[\goodSeal{\Gamma}{s}{\sigma}{A}] 
  = [\goodTerm{\Gamma, \cC\sigma}{s}{A[\pi_1]}] }
  $$
  \Rule[name=Seal Subst]
  {\goodSeal{\Delta}{s}{\sigma}{A}
    \quad {\goodSub{\Gamma}{\gamma}{\Delta}}}
  {\goodSeal{\Gamma}{s[\gamma]}{\sigma[\gamma]}{A[\gamma]}}
  =
  \frac
  {\goodTerm{\Delta, \cC\sigma}{s}{A[\pi_1]}
    \quad  \goodSub{\Gamma}{\gamma}{\Delta}  }
  {\goodTerm{\Gamma,\cC\sigma[\gamma]}{s[\gamma^\uparrow]}{A[\pi_1][\gamma^\uparrow]\equiv A[\gamma][\pi_1]}}
  $$
  $$
  \Rule[name=Seal-Id]
  {}
  {\goodSeal{\Gamma}{id_s}{\sigma}{\cC \sigma}}
  = \goodTerm{\Gamma, \cC\sigma}{\pi_2}{\cC\sigma[\pi_1]}
  $$

\medskip

\begin{minted}[escapeinside=@@]{agda}
Seal : (Γ : Con) → (σ : Sig Γ n) → (A : Ty Γ) → Set 
Seal Γ σ A = Tm (Γ, @$\mathcal{C}$@σ) A[π₁]

-[-]S : (f : Seal Γ σ A) → (τ : Tms Θ Γ) → Seal Θ σ[τ] A[τ]
f[τ]S = t[@$\tau^\uparrow$@]

idₛ : Seal Γ σ @$\mathcal{C}$@σ
idₛ = π₁
\end{minted}

\caption{Seal Judgement, and its Agda Representation (or What do we mean when we say one judgement is a shorthand for the other)}

\end{figure}


As we said earlier, $\goodSeal{\Gamma}{\_}{\sigma}{A}$ is actually just
a shorthand for term judgement $\goodTerm{\Gamma , \cC
\sigma}{\_}{A[\pi_1]}$,
a special form of \textit{local morphism}~\cite{abbott2003category}.
The substitution law should be understood as a lemma, and it is easily
derivable using $\gamma^\uparrow$. 

For notation clarity, we will sometimes use $"sf'"$ to emphasize
that we consider that seal $\goodSeal{\Gamma}{s}{\sigma}{A}$ as a term $\goodTerm{\Gamma, \cC \sigma}{sf'~s}{A[\pi_1]} $ and "sf" notation (used in \ruleref{Compile}) is defined as $\goodSub{\Gamma, \cC \sigma}{sf~s := (\pi_1, sf'~s)}{\Gamma, A}$.\YZ{What is sf and sf'? sf first appears in Compile rules, so explain sf there.}\EDJreply{Fixed. They both are explained at here now}

Sealing judgement is still about the important term $h$ we mentioned
earlier. Thanks to $\cC$, a field $t_{new}$ that depends on $a$ is now
in the context containing $\cC\{a : \bW \sigma; ..\}$.
But if we want $t_{new}$ to be inherited(reused) when $\sigma$ is
extended, we would like $t_{new}$ to be in the context containing
$\cC\{a : \cU; ..\}$ instead. The sealing is responsible for this
because we have a sealing $\goodTerm{\Gamma, \cC\{a : \bW \sigma;
..\}}{s}{\cC\{a : \cU; ..\}}$ = $\goodTerm{\Gamma, (\Sigma a : \bW
\sigma,..)}{s}{(\Sigma a : \cU,..)}$. Then we can construct
$t_{new}$ in the context containing $(\Sigma a : \cU,..)$, ready for
inheritance.
Apparently $t_{new}$ in this case cannot be doing \textit{exhaustiveness
checking}, but it can freely use any \textit{exhaustiveness-checked}
recursor.
%\YZ{It occurs to me that you use the word "sealing" to mean
%different (though tangentially related) things. Previously you used it
%as in "sealed families".}\EDJreply{They are actually the same thing.
%This sealing in metatheory is much stronger than that of the implemented
%plugin. Downside of the strong sealing is that if the user really use
%this version of sealing, the user will have to annotate what the sealed
%family type is. Now in the plugin, the sealed family type is almost
%fixed (directly inferred using a fixed rules). So basically this
%complete version of sealing (in metatheory) can deduce that of seal (in
%plugin implementation)}


% For example, $ X_1 : \bW
% \sigma_1 \vdash t : T$ in the parent family will generally not be able
% to be inherited to the children once we override the inductive type $\bW
% \sigma_1$ with $\bW \sigma_2$. 
% In
% other words, there is no way to ``transplant'' $t$ into the context $X_2
% : \bW \sigma_2$.
% This makes all the fields after this inductive type
% non-inheritable---which shouldn't be the case because a lot of fields
% might not be directly relying on this concrete $\bW \sigma_1$. 

% Sealing is used to exploit this fact---with sealing, given $X_1 : \bW
% \sigma_1$, we can add well-typed terms $X_1 : \cU \vdash t : T$, into
% the parent family. Later when we want to inherit $t$,
% because $t$ only consider $X_1$ as an arbitrary type, inheritance is
% possible. 

Now, together with sealing, we can read \ruleref{Sig Add} and \ruleref{Lnkg Add} out clearly: given a constructed linkage $o : \cL\sigma$ of signature $\sigma$, the next added field $t : T$ will have to be in the context of the abtraction $A$ of the compilation $\cC\sigma$ (witnessed by the seal $f$). The ``concatenation'' of the $o$ and $t$ leads to a new linkage, of which the signature is the ``concatenation'' of the $\sigma$ and $T$ (we also record the sealing $f$ for simplicity). The construction of linkage in \ruleref{Lnkg Add} requires $t$ to \textbf{only} refer to the (abstraction $A$ of) former part of the defining family $o : \cL\sigma$, this is to avoid inconsistency mentioned in \ref{chg:consistency} in metatheory.

This sealing is the general version of the sealed family implemented in the plugin. Our plugin will deduce a sealed ``interface'' for a sealed family using a fixed set of rules but here sealing in metatheory can be arbitrary local morphism. With the definition of sealing judgement in mind, we can see
\ruleref{Sig Add} and \ruleref{Lnkg Add} work exactly as the
implementation. In the implementation, we will have "self__" all around
when defining new fields in our plugin, and this "self__" is exactly the
$A$ in the context of the two fields---in our implementation, there is only sealing of inductive type happening (where each inductive type we only know \mintinline{Coq}{T : Set}).

In summary, with the help of both sealing and $\cC$, we can shift the context from $\cL(\{a~:~\bW \sigma; ..\})$ to $\Sigma a : \cU,..$ and thus achieve code reuse. However, this is just allow possible \textit{code reuse}, we still need to solve the mutated tension in \ref{chg:extensible-inductive-type}. To make things concrete, we need to introduce the rules for recursors.

% Since our recursor handlers are closely related to linkages, we can finally talk about the rules of recursor now. 

$$
\Rule[name=Hdler]
{\goodType{\Gamma}{A}{}
\quad \goodType{\Gamma, A}{B}{}
\quad \goodType{\Gamma}{R}{}
}
{\goodType{\Gamma}{"RecSig1"~A~B~R \equiv (\Pi A (\Pi (\Pi B (R[\pi_1\circ\pi_1])) R[\pi_1\circ\pi_1]) )}{}}
\quad 
\Rule[name=Hdlers]
{\goodWSig{\Gamma}{\sigma}{n}
\quad \goodType{\Gamma}{R}{}
}
{\goodSig{\Gamma}{"RecSig"~\sigma~R}{n}\YZ{Why does this Sig not have a superscript?}\EDJreply{Fixed now.}}
$$
$$
\Rule[name=Hdler-Proj]
{ \goodWSig{\Gamma}{\sigma}{N}
\quad \goodTerm{\Gamma}{o}{\cL("RecSig"~\sigma~R)}
\quad n < N
}
{\goodTerm{\Gamma}{"prjR"~n~o}{("RecSig1"~(\pi_1^n \sigma)~(\pi_2^n \sigma)~R)[\pi_1]}}
$$
$$
\Rule[name=Rec-Constr]
{ \goodWSig{\Gamma}{\sigma}{N}
\quad \goodType{\Gamma}{T}{\bW \sigma}
\quad \goodType{\Gamma}{R}{}
\quad \goodTerm{\Gamma}{r}{\cL("RecSig"~\sigma~R)}
\quad \goodTerm{\Gamma}{w}{El~("W"~T)}
}
{\goodTerm{\Gamma}{"Wrec"~r~w}{R}}
$$

In our meta-theory, we only formulate non-dependent eliminator for simplicity, i.e., the
\textit{recursor}.\YZ{Do dependent eliminators pose technical difficulties, or do you avoid them solely because they require more verbiage?}\EDJreply{I don't think there is technical difficulties to do that. The formulation of course will be much more horrifying if use dependent eliminator here. } 
Each recursor is constructed by a linkage containing
functions/handlers/pattern-matching cases dealing with each
constructor. Thus code reuse for recursor is delegated to linkage
inheritance---to reuse one particular pattern-matching case,
we just inherit the function/handler. This greatly simplifies the
metatheory development.

Intuitively \ruleref{Hdler} is indicating the type of recursor handler
for one specific constructor, and \ruleref{Hdlers} is the linkage type
of the bundle of all recursor handlers of one given inductive type. We
omit the definition of "RecSig" but \ruleref{Hdler-Proj} indicates its
(apparent) internal definition. "prjR" is also the handler selection
from the handler linkage, and is helpful for the formulation of
$\beta$-rule of "Wrec" (omitted here). Finally \ruleref{Rec-Constr} is
used to construct a concrete recursor given a specific linkage of
handlers. All these rules satisfy substitution laws and omitted here.


Again, we still use \textit{decoupling} of the syntax of the case
handlers and exhaustiveness checking to resolve
\ref{chg:extensible-inductive-type}. We use linkages to define
case handlers and thus they are inheritable like other families.
While any field using "Wsup" and "Wrec" directly relies on the
$\bW\sigma$ (can be spotted from \ruleref{Ind-Term} and
\ruleref{Rec-Constr}) and thus non-inheritable---in these cases, "Wsup"
can be understood as concrete constructors and "Wrec" are
\textit{exhaustiveness checking}. These fields together with inductive
types are the ones sealing is responsible for hiding. After compilation
of linkage types into existential types, we can easily hide the
\textit{concrete} definition of the inductive type while exposing all
the fields using "Wsup" and "Wrec" like an abstract interface.
Apparently, future fields relying on this abstract interface will have no
problem on inheritance.
Of course, since "Wsup" and "Wrec" directly rely on the concrete inductive definition, we will have to override these fields whenever inductive type is overridden. For "Wsup" we only need to override each constructor and for "Wrec" it is just doing exhaustiveness checking inside each children family. The recursor handlers however, can be safely inherited.
\YZ{I suppose the terminology 'sealing' is a reference to ML modules? In
ML, sealing has the connotation of hiding things behind an existential
type. So using the same word may cause the reader to have false
presumptions about what is going on here.}\EDJreply{You are right. Our "Sealing" is not hiding a particular field into an existential type, but our seal is make several fields together into an "existential type". Basically it hides the concrete definition of inductive type so that future fields doesn't have to be defined upon that.}\EDJreply{I will resolve this comment once I add the throughout example and you find it clear---because I haven't changed any text}

There are also cases we don't need to seal anything (i.e., for doing
exhaustiveness checking when using recursor), and thus \ruleref{Seal-Id} is
used. 





Next we talk about inheritance judgement. Note that, there will be a lot of $\sigma_1, A_1$ and $\sigma_2, A_2$, and in these cases, $A_1$ is the result of the sealing of $\sigma_1$, i.e. we assume we have a term $\goodSeal{\Gamma}{f_1}{\sigma_1}{A_1}$, $\goodSeal{\Gamma}{f_2}{\sigma_2}{A_2}$ in the assumption of judgements when needed.


\judgebox{\goodInh{\Gamma}{h}{\sigma}{\tau}}

$$
\Rule[name=Inh-Id]
{}
{\goodInh{\Gamma}{"inhid"}{\sigma}{\sigma}}
\quad
\Rule[name=Inh-Override]
{
\goodInh{\Gamma}{h}{\sigma_1}{\sigma_2}  
\quad \goodType{\Gamma, A_1}{T_1}{}
\quad \goodType{\Gamma, A_2}{T_2}{}
  \quad \goodTerm{\Gamma, A_2}{t}{T_2}}
{\goodInh{\Gamma}{"inhov" \ h \ t}{(\nu^+ \  \sigma_1 \  T_1)}{(\nu^+ \  \sigma_2\  T_2)}}
$$

$$
\Rule[name=Inh-Ext]
{\goodInh{\Gamma}{h}{\sigma_1}{\sigma_2}
  \quad \goodTerm{\Gamma, A_2}{t}{T}}
{\goodInh{\Gamma}{"inhext" \ h \ t}{\sigma_1}{(\nu^+ \  \sigma_2\  T)}}
\quad
\Rule[name=Inh-Inh]
{\goodInh{\Gamma}{h}{\sigma_1}{\sigma_2}
\quad \goodType{\Gamma, A_1}{T}{}
\quad \goodTerm{\Gamma, A_2}{\uparrow^s}{A_1[\pi_1]}
}
{\goodInh{\Gamma}{"inhinh" \ h}{(\nu^+ \  \sigma_1 \  T)}{(\nu^+ \  \sigma_2 \  T[(\pi_1, \uparrow^s)])}}
$$
$$
\Rule[name=Inh+Inh]
{\goodInh{\Gamma}{h}{\sigma_1}{\sigma_2}
\quad \goodTerm{\Gamma, A_2}{\uparrow^s}{A_1[\pi_1]}
\quad 
\goodInh{\Gamma, A_2}{i}{\tau_1[(\pi_1, \uparrow^s)]}{\tau_2}}
{\goodInh{\Gamma}{"inhnest" \ h \ i}{(\nu^+ \  \sigma_1\  \cL\tau_1)}{(\nu^+ \  \sigma_2\  \cL\tau_2)}}
\quad
$$

$$
\Rule[name=Inh]
{\goodInh{\Gamma}{h}{\sigma_1}{\sigma_2}
\quad \goodTerm{\Gamma}{l}{\cL \sigma_1}
}
{\goodTerm{\Gamma}{("inh" \ h \ l)}{\cL \sigma_2}} 
\quad 
\Rule[name=Inh-Inh-beta]
{\goodInh{\Gamma}{h}{\sigma_1}{\sigma_2}
  \quad \goodTerm{\Gamma}{m}{\cL \sigma_1}
  \quad \goodTerm{\Gamma, A_1}{t}{T}
}
{\goodTerm{\Gamma}{"inh" ("inhinh" \ h) (\mu^+ \ m \ t) \equiv \mu^+ \ ("inh" \ h \ m) \ t[(\pi_1, \uparrow^s)]}{}} 
$$

Inheritance are
judgement so naturally second class citizen data (i.e. not something
function can return and cannot be assigned to variables), and it looks
very much like a function that transform a given linkage. Similar to plugin implementation and \ruleref{Lnkg Add}, we resolve \ref{chg:consistency} by making sure corresponding fields are in corresponding context during inheritance, and thus an overriding term can only refer to fields before the overriden field. 

The syntax of the inheritance judgement is very close to how Family is
defined in the surface syntax---a Family with no parent is an
inheritance with empty input; an extended Family is exactly an
inheritance---because the parent of an extended Family can be overridden
thus inheritance should only define upon the ``interface'' instead of a
concrete family. However, since our current formulation doesn't support
nested inheritance (i.e. we cannot do inheritance inside a family but only top level), our
formulation still has some distance to the real family polymorphism. 

"inhid" and "inhext" are simple as expected---the former corresponds to
empty inheritance, and the latter gives the programmer the ability to
add new fields.

"inhinh" is special because it requires user to provide a proof of
``upcasting'' from the context of the child to that of the parent.
In other words, if I want to inherit a specific field from the parent, I
have to convince the type checker that my current context can
``accommodate'' the inherited field. 

Override "inhov" is even more special because in mundane OO programming language, we usually require that the overriding term has the same type/interface as the overriden one.  Here we don't since the very reason we want the same type/interface is that we hope other inherited fields can use this overridden/late-binding field without breaking abstraction, and this reason is already managed by the assumption of "inhinh", the upcasting proof $\uparrow^s$. Thus "inhov" is quite like "inhext", where overriding is just throwing away one parent field. This difference also show one of the distinct feature of our vanilla family inheritance compared to the mainstream OO inheritance---family inheritance don't need to introduce the concept of ``subclassing'' between families and thus no bounded polymorphism yet\footnote{Of course once trait and interface comes in, we can have bounded polymorphism back}, and code reuse is mainly achieved by late binding. Without ``subclassing'', there is no way of confusing inheritance and subtyping. Note that, "inhov" is also responsible for ``extending'' inductive type---the extended inductive type directly override the old ones when using metatheory to program, just like how plugin implementation works.

Finally we need to introduce nested inheritance "inhnest" for dealing with nested families. The rule is mostly direct and we need "upcasting" again when dealing with nested family in different but inheriting context.  We also show how to use an inheritance judgement to derive children family and how the beta rule is defined. We omit most of the other substitution laws here.



\subsection{Exemplar Usage of Meta Theory}
We use an example to better demonstrate different components of the
syntax of our formulation. Our example will be the meta-theory encoding
of the following pseudocode.

We provide one example in its surface syntax as pseudocode, and their
meta-theory encoding. This example is computing the predecessor of a
mundane natural number, but we split the inductive definition into two
parts and making "O" and "S" standing alone.
"pred" will map $S~n \mapsto (n, S~n)$ and $O \mapsto (O, O)$, and thus a predecessor.\YZ{
  Using family polymorphism for adding  a successor constructor is unnatural.
  I suggest simply renaming the constructors and not mentioning the correspondence to natural numbers.
}




\begin{figure}[H]\label{fig:example-pseudocode}
\begin{minipage}[t]{0.5\linewidth}
\begin{minted}[escapeinside=@@]{Coq}
Family A.
 FInductive N := O : N. 
 (* Osig, sig₁, obj₁, sig₂,
    O₁, obj₂, sig₂', sl₁ *)
 Family handler.
   O :self[A].N × self[A].N 
     := (self[A].O, self[A].O). 


 EndFamily. (* pN₁, handler₁, sig₃ *)
 FRecursor pred about self[A].N 
  motive (fun _ => self[A].N × self[A].N)
  using handler.
 (* pred
  : self[A].N → self[A].N × self[A].N *)
  (* predT, pred₁, sig₄, sig₄', sl₄ *) 
 k := self[A].pred self[A].O 
(* pN₂, sig₅ *)
EndFamily.
\end{minted}
  \end{minipage}
  \begin{minipage}[t]{0.45\linewidth}
\begin{minted}[escapeinside=@@]{Coq}
Family A₂ extends A.
 Extend FInductive N := S : N → N.
 (* Nsig, sig₁₁, inh₁, O₂, sig₂₁, inh₂, 
    sig₂₂, sig₂₁', sig₂₂', sl₂₂, inh₃ *)
 Extend Family handler.
   Inherit O. (* inhᵢₙ₀, pN₃ *)
   S : self[A].N × self[A].N
       → self[A].N × self[A].N 
     := fun (x,y) => (y, self[A₂].S y).
 EndFamily. (* inhᵢₙ, sig₃₁, inh₄*)
 Inherit pred.
 


  (* sig₄₁, inh₅, sl₄₁ *)
 Inherits k. 
 (* inh₆ *)
EndFamily.
    \end{minted}
  \end{minipage}\YZ{This figure needs updating; it should use the same syntax as in §3.}\YZreply{Also, this figure should contain an example of accessing a field outside its family.}
  \caption{Demonstrating Pseudo-Code}
\end{figure}

\begin{figure}
  \begin{minipage}{\linewidth}
    \begin{minted}{Coq}
      Inductive N :=
       | O : 1 -> (0 -> N) -> N 
       | S : 1 -> (1 -> N) -> N
      (* For example, Number 1 is encoded using
         1 = S () (fun _ -> O () (fun x -> elim-⊥ x)) *)
    \end{minted}
  \end{minipage}

  \begin{minipage}[t]{0.4\linewidth}
  \small
\begin{align*}
  "Osig" &\coloneqq \goodWSig{\cdot}{w^+\ w\cdot\ \top\ \bot}{1} \\
  "sig"_1 &\coloneqq \goodSig{\cdot}{\nu^+ \ \nu\cdot \ {id_s} \ (\bW \ "Osig"[\pi_1])}{1}  \\
  "obj"_1 &\coloneqq \goodTerm{\cdot}{\mu^+ \ \mu\cdot \ {id_s} \ \star}{\cL"sig"_1}\\
  \cC"sig"_1 &\coloneqq \goodType{\cdot}{\Sigma \ \top \ (\bW \ "Osig"[\pi_1]) }{} \\
  "sig"_2 &\coloneqq  \goodSig{\cdot}{\nu^+ \ "sig"_1 \ {id_s} \ El ("W" ("pjr" \ \pi_2))}{2} \\ 
  "O"_1 &\coloneqq {"Wsup" \ ("pjr" \pi_2) \ () \ ("elim-"\bot \ \pi_2)}  \\ 
  & \text{ and thus }  \goodTerm{\cdot, \cC"sig"_1}{"O"_1}{El ("W" ("pjr" \pi_2))} \\
  "obj"_2 &\coloneqq \goodTerm{\cdot}{\mu^+ \ "obj"_1 \ "O"_1}{\cL"sig"_2} \\
  "sig"_2' &\coloneqq \goodType{\cdot}{\Sigma (\Sigma \top \cU)El ("pjr" \pi_2)}{} \\ 
  \cC"sig"_2 &\coloneqq \goodType{\cdot}{\Sigma (\Sigma \top (\bW \ "Osig"[\pi_1]))El ("W"("pjr" \pi_2)) }{} \\ 
  "sl"_1 &\coloneqq ((\star, "W" ("pjr" ("pjl" \pi_2)) ), "pjr" \pi_2) \\ 
  & \text{ and thus }  \goodSeal{\cdot}{"sl"_1}{"sig"_2}{"sig"_2'} \\
  "pN"_1 &\coloneqq \goodType{\cdot, "sig"_2'}{El ("pjr" ("pjl" \pi_2)) \times El ("pjr" ("pjl" \pi_2))}{} \\
  "handler"_1 &\coloneqq \goodTerm{\cdot, "sig"_2'}{\_}{\cL(\nu^+ \ \nu\cdot \ \{id_s\} "pN"_1)} \\
  "sig"_3 &\coloneqq \goodSig{\cdot}{\nu^+ "sig"_2 \{"sl"_1\} \ \cL(\nu^+ \ \nu\cdot \ \{id_s\} "pN"_1)}{3} \\ 
  "obj"_3 &\coloneqq \goodTerm{\cdot}{\mu^+~"obj"_2~"handler"_1}{\cL"sig"_3}\\
  "predT" &\coloneqq \Pi \ El ("W"("pjr" ("pjl"^2 \pi_2)))\ El ("W" ("pjr" ("pjl"^2  \pi_2)))[\pi_1] \\
  "pred"_1 &\coloneqq \goodTerm{\cdot, \cC "sig"_3}{\lambda ("Wrec" \_)}{"predT"} \\ 
  "sig"_4 &\coloneqq \nu^+ \ "sig"_3 \ \{id_s\} \ "predT" \\ 
  "obj"_4 &\coloneqq \goodTerm{\cdot}{\mu^+~"obj"_3~"pred"_1}{\cL"sig"_4}\\
  "sl"_4 &\coloneqq \goodSeal{\cdot}{\_}{"sig"_4}{"sig"_4'} \\
  "pN"_2 &\coloneqq \goodType{\cdot, "sig"_4'}{El ("pjr" ("pjl" \pi_2)) \times El ("pjr" ("pjl" \pi_2))}{}
\end{align*}
\end{minipage}%
\begin{minipage}[t]{0.4\linewidth}
  \small
\begin{align*}
  "sig"_5 &\coloneqq \nu^+ \ "sig"_4 \ \{"sl"_4\} \  "pN"_2 \\ 
  "obj"_5 &: \goodTerm{\cdot}{\mu^+~"obj"_4~\_}{\cL"sig"_4}\\
  \\ 
  \text{Here we}&\text{ start to construct Family } "A"_2 \\ 
  "Nsig" &\coloneqq \goodWSig{\cdot}{w^+\ "Osig"\ \top \ \top}{2} \\ 
  "sig"_{11} &\coloneqq \goodSig{\cdot}{\nu^+ \ \nu\cdot  \ (\bW \ "Nsig"[\pi_1])}{1} \\
  "inh"_1 &\coloneqq \goodInh{\cdot}{"inhov" \  "inhid" \ \star}{"sig"_1}{"sig"_{11}}\\
  "O"_2 &\coloneqq {"Wsup" \ ("pjr" \pi_2) \ () \ ("elim-"\bot \ \pi_2)}  \\ 
  & \text{ but }  \goodTerm{\cdot, \cC"sig"_{11}}{"O"_2}{El ("W" ("pjr" \pi_2))} \\
  "sig"_{21} &\coloneqq \nu^+ \ "sig"_{11} \ \{id_s\} (El (W (pjr p₂))) \\ 
  "inh"_2 &\coloneqq \goodInh{\cdot}{"inhov" \ "inh"_1 \ "O"_2}{"sig"_2}{"sig"_{21}}\\
  "S"_T &\coloneqq \Pi \ El ("W"("pjr"  ("pjl" \pi_2))) \ El ("W" ("pjr" ("pjl"  \pi_2)))[\pi_1] \\
  "sig"_{22} &\coloneqq \nu^+ \ "sig"_{21} \ \{id_s\} \ "S"_T \\ 
  "sig"_{21}' &\coloneqq \goodType{\cdot}{\Sigma (\Sigma \top \cU) (El ("pjr" \pi_2))}{} \\
  "sig"_{22}' &\coloneqq \Sigma "sig"_{21}' (\Pi \ El ("pjr"  ("pjl" \pi_2)) \  ...) \\
  "sl"_{22} &\coloneqq \goodSeal{\cdot}{\_}{"sig"_{22}}{"sig"_{22}'} \\ 
  "inh"_3 &\coloneqq \goodInh{\cdot}{\_}{"sig"_2}{"sig"_{22}} \\ 
  \uparrow^s &\coloneqq \goodTerm{\cdot, "sig"_{22}'}{\_}{"sig"_2'[\pi_1]} \\ 
  "pN"_3 &\coloneqq "pN"_1[(\pi_1, \uparrow^s)] \\ 
\end{align*}
  \end{minipage}

  \begin{minipage}{0.8\linewidth}
    \small
    \centering
    \begin{align*}
      "inh"_{in0} &\coloneqq \goodInh{\cdot, "sig"_{22}'}{"inhid"}{(\nu^+ \ \nu\cdot \ ("pN"_1[\pi_1]))[(\pi_1, \uparrow^s)]}{(\nu^+ \cdot ("pN"_3[\pi_1]))} \\
      "inh"_{in} &\coloneqq \goodInh{\cdot , "sig"_{22}'}{\_}{(\nu^+ \ \nu\cdot \ "pN"_1[\pi_1])[(\pi_1, \uparrow^s)]}{(\nu^+ \ \cdot \ ("pN"_3[\pi_1]) \ \{id_s\} (\Pi "pN"_3 ("pN"_3[\pi_1])) )[\pi_1]}\\
      "sig"_{31} &\coloneqq  \nu^+ \ "sig"_{22} \ {"sl"_{22}}\  \cL(\nu^+ \cdot ("pN"_3[\pi_1])  (\Pi "pN"_3 ("pN"_3[\pi_1]))[\pi_1]) \\ 
      "inh"_4 &\coloneqq \goodInh{\cdot}{"inhnest" \ "inh"_3 \uparrow^s "inh"_{in}}{"sig"_3}{"sig"_{31}} \\
      "sig"_{41} &\coloneqq {\nu^+ \ "sig"_{31} \ (\Pi (El ("W" ("pjr" ("pjl"^4 \pi_2)))) (El ("W" ("pjr" ("pjl"^4 \pi_2))))[\pi_1])} \\ 
      "inh"_5 &\coloneqq \goodInh{\cdot}{inhov \ \_}{"sig"_4}{"sig"_{41}} \\ 
      "sl"_{41} &\coloneqq \goodSeal{\cdot}{\_}{"sig"_{41}}{"sig"_{41}'} \\ 
      \uparrow^s_2 &\coloneqq \goodTerm{\cdot, "sig"_{41}}{\_}{"sig"_4'[\pi_1]} \\ 
      "inh"_6 &\coloneqq \goodInh{\cdot}{"inhinh" \ "inh"_5 \ \uparrow^s_2}{"sig"_5}{(\nu^+ \ "sig"_{41} \{"sl"_{41}\} \ "pN"_2[(\pi_1, \uparrow^s_2)])}
    \end{align*}  
  \end{minipage}
  \caption{Detailed Construction}\label{fig:example-construction}
\end{figure}





Here we explain how  our detailed construction in
\cref{fig:example-construction}. We first elaborate that
how our inductive type "N" is written in the specific format (the style
of W-type). We use 1, 0 here to indicate unit type and bottom type. Recall from \ruleref{Sigma Elim}, we have "pjl" and "pjr" as two projection from the sigma type (dependent pair). We will also use $"pjl"^n$ as a consecutive composition of "pjl", (e.g. $"pjl"^2~t$ is exactly $("pjl" ("pjl"~t))$ ).

Next, we alias each part of the derivation with a name. Every derivation
here is well-typed term. We also annotate in the original pseudocode the
name of the related derivation. 

We start with the construction of "Osig" which is the signature of a
standalone "O" constructor. With this we will have $"sig"_1, "obj"_1,
"sig"_2, "obj"_2$ as the signature and object of the linkage with that
inductive type, and the signature and the linkage once we export the
concrete constructor $"O"_1$. 

Now sealing comes in to seal these concrete component into an abstract
interface $"sig"_2'$ for decoupling the following fields. To make sure
the sealing exists, we take a look at the resulting $\cC "sig"_2$, and
doing abstraction on it to get $"sl"_1$. 

Now we construct the recursor, with the help of the handler module
$"handler"_1$, with which we can construct the real recursor $"pred"_1$.
Both detailed constructions are omitted. $"obj"_3$ and $"sig"_3$ is the resulting linkage and linkage signature aggregating the nested family $"handler"_1$. $"pred"_1$ is then the constructed recursor over $"sig"_3$ , without any abstraction, for the sake of exhaustiveness checking. We show the example $"sig"_4$ as the result of adding this new recursor. 

Then we again abstract the concrete recursor $"pred"_1$ away using $"sl"_4$ so that the field $"k"$ (of type $"pN"_2$)
can apply the recursor without resorting to any concrete recursor, thus achieving late-binding. Finally we
have $"sig"_5$ as final signature. 

Note that, since in our example there is only one type, the complicated debruijn indices $El("pjr" ("pjl" \pi_2)) $ in $"pN"_2$ and  $El("W"("pjr"("pjl"^2 \pi_2)))$ in "predT" are both referring to the type "N". The distinguishing difference is we have "W" (\ruleref{Ind-Type}) for the latter as "predT" is constructed in the context knowing the concrete definition of the inductive type, thus generally not inheritable. On the contrary,  $"pN"_2$ has as its context only $"N" : \cU$ and manifests its late-binding and its overridability for its former fields. \YZ{
  Maybe it is a good place to point out what El(pjr (pjl pi_2)) in pN_2 stands for
  and what El(W(pjr(pjl^2 pi_2))) in predT stands for.
}\EDJreply{Done.}\YZreply{
  I was actually alluding to how the type of pN_2 manifests late binding
  while the type of predT refers to the concrete signature of the
  inductive type.
}\YZreply{
  It’s good to be consistent in notations.  You later use * for products definition of pN_1, but you also use * as the term of type ($\bW$ $\sigma$) 
}\EDJreply{Notation replaced. First * replaced with $\star$, second * replaced with $\times$}
\EDJreply{Sry. Now I attempt another try. I also change the above paragraph a bit to explain the details on sig₃, sig₄, and etc.
I also copy the comments in our slack to here to remind me about the changes.
Also let me paste the full example here for future reference: \\
% following is the body of pred₁
pred_ = Wrec ...  
  : Tm (⋅ ▹ 𝓒sig₃ ▹ (El (W (pjr (pjl (pjl π₂)))))) (El (W (pjr (pjl (pjl π₂)))))[π₁] \\
% we do lambda abstraction now
pred₁ = λ pred_ :   Tm (⋅ ▹ 𝓒sig₃) (∏ El (W (pjr (pjl (pjl π₂)))) (El (W (pjr (pjl (pjl p₂))))[π₁])) \\
% we can see that, the type of pred₁ is predT 
% now we construct sig4, this should not have any abstraction inside
sig₄ = ∨+ sig₃ id_s (∏ El (W (pjr (pjl (pjl π₂)))) (El (W (pjr (pjl (pjl π₂))))[π₁])) \\ 
% we omit the construction of obj₄
obj₄ : Tm ⋅ 𝓛sig₄ \\
% since pred is define in context 𝓒sig₃, we need to abstract 𝓒sig₄ into sig₄'
% again for later field,
sig₄' = 
  Σ (Σ ((Σ (Σ ⊤ 𝕌) (El (pjr π₂))) 𝓛(∨+ ⋅ (pN[π₁]))[sf sl₁])) 
    (∏ El (pjr (pjl (pjl π₂))) (El (pjr (pjl (pjl π₂)))[π₁]))
% and we have, omitting the detail construction 
sl₄ : Seal ⋅ sig₄ sig₄' \\
% pN₂ now is constructed in the context of the sealed context sig₄'
pN₂ = El (pjr (pjl π₂)) * El (pjr (pjl π₂)) : Ty (⋅ ▹ sig₄') 
% the product type \\ 
sig₅ = ∨+ sig₄ {sl₄} pN₂ \\ % the full example is in my gist last time I sent to you (Attempt 9), if you like it I can send the link again but note that the omission is still there and a lot of naming convention is different.
It seems that as you last night complained, I forget to add sig₄' causing problems.  Do I also need to make a note that those primed variables are always the result of abstraction?
}

Now we construct the inheritance judgement for $"A"_2$. Still, we start
with signature for inductive type, "Nsig". But to ``extend'' the
inductive type, we actually \textit{override} the inductive type with
the enriched inductive type in $"inh"_1$. Because inductive type is
overridden, inheritance on the constructors are not pausible and thus we
need to override the old $"O"_1$ constructor using $"O"_2$ and extend
with new constructor for "S", resulting $"inh"_3$. These newly
overridden constructors are sealed by $"sl"_{22}$. However, look
closely, our sealed abstract interface in the children still has more
fields than the sealed abstract interface in the parent. We still need
$\uparrow^s$ to indicate the ``compatibility'' between two abstract
interfaces so that the inheritance from the parents can work. It can be
understood as ``upcasting'', but this ``upcasting'' is used to upcast
the context of the children. We immediately see its usage in $"pN"_3$
and $"inh"_{in0}$. Upon this, we construct the complete
inner-inheritance $"inh"_{in}$ that is responsible for extending
"handler". Then we continue the construction for $"inh"_4$ that ``nest''
the inner-inheritance into the whole $"inh"_3$, and we re-do
exhaustiveness checking when constructing $"inh"_5$ using $"inhov"$ and
"Wrec". Since it is another recursor requiring concrete inductive
definition, we do another abstraction on it using $"sl"_41$. After that,
$\uparrow^s_2$ again witnesses the compatibility of the abstract
interface, and the two are both used for constructing $"inh"_6$, which
is responsible for inheriting field "k".

The resulting constructed families are
$\goodTerm{\cdot}{"obj"_5}{\cL"sig"_5}$ as "Family A" and \\
$\goodTerm{\cdot}{("inh"~"inh"_6~"obj"_5)}{\cL((\nu^+ \ "sig"_{41}
\{"sl"_{41}\} \ "pN"_2[(\pi_1, \uparrow^s_2)]))}$ as "Family" $"A"_2$.

Finally we comment on how a family is encoded in our metatheory.
A family object will have a signature that exposes the type
of fields and the details of inductive types. This exposure can make
sure inheritance can override and inherit fields correctly. Sealing is
only hiding information \textbf{inside one family}, i.e., hiding
information of the former fields for the later fields, and thus the
later fields can be defined relying on only the abstract interface of
the former fields.
The signature will expose the sealing itself and the inductive type.
(See $"sig"_5$ has all the sealing information and inductive
definition). A former sealing might not be related to any later sealing,
for example, $"sig"_3$ is constructed using $"sl"_1$ but $"sig"_4$ is
using trivial sealing $id_s$.


\subsection{Standard Model for Consistency}

Once we have the syntax of our theory, the first question is if our theory is consistent, i.e., if we can syntactically derive bottom in our theory. We prove the consistency by extending the standard model from \citep{kaposi2017type}, where they formally resort to the concept of \textit{algebra} of QIIT, here we only consider inductively interpreting each syntax piece into Agda components that is also respecting judgemental equalities.\footnote{Of course, it is also possible to interpret to constructive set theory.} 
Here we will only show some interpretation, the complete version please refer to appendix.

\begin{align*}
  ss
\end{align*}


Notice that we interpret the bottom type using empty set, and thus we know it is not possible to derive $\cdot \vdash t : \bot$: otherwise we interpret this derivation in our standard model and we can conclude $\denotes{t} : Data.Empty$, 
% which is basically $\denotes{t} \in \emptyset$ set theoretically, 
and that is a contradiction.


\subsection{Proof Relevant Logical Relation for Canonicity}
We follow the reducibility argument from \citep{coquand2018canonicity,sterling2019algebraic, kaposi2019gluing} to construct the canonicity model. We will first specify the formulation of the most basic canonicity theorem. 

\begin{theorem}[Canonicity]
  For a closed boolean term $\goodTerm{\cdot}{t}{\mathbb{B}}$, we have $\goodTerm{\cdot}{t \equiv tt}{\mathbb{B}}$ or $\goodTerm{\cdot}{t \equiv ff}{\mathbb{B}}$ hold.
\end{theorem}

Canonicity is one of the basic criteria to consider a dependent type theory to be ``a programming language or as a computational foundation for mathematics''\footnote{See nlab explanation: \href{https://ncatlab.org/nlab/show/canonical+form}{Canonical Form}}.
We can even argue that, if this theorem is proven in a constructive meta-logic\footnote{QIIT does have computational content\citep{altenkirch2016type}}, 
then by Curry-Howard Correspondence, this theorem provides a big-step interpreter for closed term of boolean type.

Now we sketch out the big picture of the proof. The canonicity model $"Tm"^C~\Gamma^C~"T"^C$ will map each syntax piece $("t : Tm Γ T") \mapsto "(t, tₚ) :" "Tm"^C~\Gamma^C~"T"^C = \sum~("t : Tm Γ T")."Tm"_p~\Gamma^C~"T"^C~"t"$ to a dependent pair\footnote{This $"Tm"_p$ is usually called dependent model.}, of which the first part will be the same as the input syntax. Using this mapping on closed boolean term $"t : Tm ⋅ "\mathbb{B}$ will give us $"Tm"_p~\cdot^C~\mathbb{B}^C~"t"$, which will unfolds to our final goal $"t" \equiv "tt" + "t" \equiv "ff"$. Roughly, this can read as ``every closed boolean term is reducible''.  Thus once we can ``propagate'' $(\cdot)^C$ into every syntactic piece, and construct this model, we are done. In our (fake) Agda formulation, we use $(\cdot)_2$ to replace the notation of $(\cdot)^C$ because we consider most parts of this model are actually a dependent pair.

Another helpful analogy of the canonicity model is that $"Tm"_p~\Gamma^C~"T"^C~"t"$ can be understood as $"t" \in \{$ Reducible Terms of type $T$ (and $T_p$) in context $\Gamma$ (and $\Gamma_p$) $\}$. The latter is more conventional and the former can be understood as type theoretic encoding of the latter\footnote{We usually use predicate to encode "belong to" relation in type theory}. From this perspective, this part of the canonicity model is quite similar to how conventional (proof-irrelevant) logical relation is carried out\citep{skorstengaard2019introduction} --
using the terminology of the conventional formulation, $\Gamma_p$ will be the set of substitution into closed reducible terms, and we will prove $t[\gamma] \in T_p$, a set of closed reducible term of type $T$ when $\gamma \in \Gamma_p$.

The proof can be found in appendix.

With the logical relation and the help of eta rule, we figure out the following canonical form
\begin{theorem}[Canonical Forms].
  \begin{itemize}
    \item if $\goodTerm{\cdot}{t}{El~("W"~*)}$ with $\goodTerm{\cdot}{*}{\bW~\sigma}$ and $\goodWSig{\cdot}{\sigma}{n}$, then $\goodTerm{\cdot}{t \equiv "Wsup"~j~a~b}{El~("W"~*)}$ for some $\goodTerm{\cdot}{a}{A}$ and $\goodTerm{\cdot, B[(id, a)]}{b}{El~("W"~*)}$ and $j < n$
    \item if $\goodTerm{\cdot}{t}{\mathbb{B}}$ then $\goodTerm{\cdot}{t \equiv "tt"}{\mathbb{B}}$ or $\goodTerm{\cdot}{t \equiv "ff"}{\mathbb{B}}$ 
    \item if $\goodTerm{\cdot}{t}{\cL\sigma}$ with $\goodSig{\cdot}{\sigma}{n}$, then $\goodTerm{\cdot}{t \equiv \mu^+~o~t}{\cL\sigma}$ 
    
      $\quad$ for some $\goodTerm{\cdot}{o}{\cL(p_1\nu~\sigma)}$ and $\goodTerm{\cdot, p_1\nu'~\sigma}{t}{p_2\nu~\sigma}$
    \item if $\goodTerm{\cdot}{t}{\Sigma A B}$ then $\goodTerm{\cdot}{t = (a, b)}{\Sigma A B}$ with $\goodTerm{\cdot}{a}{A}$ and $\goodTerm{\cdot}{b}{B[(id, a)]}$
    \footnote{We emphasize the last one because $\cCt t$ is a dependent pair}
  \end{itemize}
\end{theorem}

% the canonical form of inductive type, linkage and boolean 


\section{Example in Metatheory}\label{sec:coqexample}
We provide two examples here for illustration and evaluation. 

\subsection{Type Safety for STLC}
We start with the very basic type safety proof of STLC, largely adapted from Software Foundation, which is also one of the primal motivating example of this project. 

Our emphasis here is that, we singled out some programming language features from STLC and prove the type safety for each feature separately, following the style of the examples in MTC\citep{delaware2013,forsta2020}. And we use a \textit{not-yet-completely-defined} \textbf{mix-in} feature to mix the semantic and properties of two programming language feature -- product and boolean -- with vanilla STLC. This way, we can say precisely, \textit{a programming language feature itself is a piece of data/inheritance judgement/inherited family}. Compared to MTC, our example uses small-step operational semantic by exploiting the extensible inductive type, and most of the proof is directly adapted from the one in Software Foundation, resulting a more accessible proof. 
% In related work, Coq/Metatheory a la carte/Tion embedding can be emphasized as a more "semantic approach" because they encode the meaning using a special design pattern (for example, open recursive inductive type for extension), compared to our more syntactic approach. Their advantage is the transferability of this technique accross different proof assistants, and their disadvantage is that their approach are less accessible and unfriendly to amateur Coq users -- which can be reflected from the distance of their approach and text-book Software Foundation proof. 

% We have problem on inversion lemma. Check if it is the same problem as MTC
We use `Closing Fact' to state and prove the inversion lemma instead of using the extensible proving mechanism `FTheorem'. The reason is that 1. it introduce much less boilerplate code because the proof for these inversion lemma should be just simple case analysis and we should rely on Coq to auto-generate them; 2. it shouldn't bother us in the future because any extension on the syntax should still satisfy these inversion lemmas; 3. most importantly, we believe this inversion ``lemma'' should be part of the definition of the syntax instead of considering the syntax as a mere concrete inductive type. We should postulate this inversion ``lemma'' like \textit{a constraint} and post-hoc-ly verify that our inductive definition did satisfy the constraint, which is exactly what we expect from `Closing Fact'.

\subsection{Abstract Interpreters for Imp}
The second example is adapted and modified from the Familia\citep{zhang2017familia}
% citation needed here
-- contrary to our first example, we use big-step interpreter and fuel to indicate the operational semantic on an imperative language with side-effect, and we specify the abstract interpretation and prove its soundness, with some of postulation on both computation and property. Then we extend the language feature and we instantiate the postulation on computation for both concrete interpreters and abstract interpreters. Thanks to the compilation, we can directly run the resulting abstract interpreters.