\begin{abstract}

With the growing practice of mechanizing language metatheories,
it has become ever more pressing that interactive theorem provers 
make it easy to write reusable, extensible code and proofs.
%
This paper presents a language design useful for extensible metatheory
mechanization in a proof assistant.
The new design achieves reuse and extensibility via a form of family
polymorphism, a seemingly object-oriented idea, that allows code and
proofs to be polymorphic to their enclosing families.
Our development addresses significant technical challenges that arise
from the underlying language of a proof assistant being simultaneously
functional, dependently typed, a logic, and an interactive tool.
%
Our results include
\begin{enumerate*}
\item a prototypical implementation of the language design as a Coq plugin,
\item a formalization of the language design as a constructive type theory with inductive types,
\item proofs of strong guarantees afforded by the type theory,
and 
\item case studies showing how the new expressiveness naturally addresses real
programming challenges in metatheory mechanization.
\end{enumerate*}
\end{abstract}

\maketitle

\section{Introduction}
\label{sec:intro}

\section{Metatheory}
\label{sec:metatheory}
\subsection{Syntax}
We start with the formulation of the syntax, which is based on the formalization of a predicative \textit{Martin-Lof Type Theory} (MLTT) given by \cite{coquand2018canonicity}. Thus we will omit most of the discussion about MLTT here but focus on the new facility we introduced. Though we formulate in Latex, our thought process is mainly following \cite{kaposi2017type} where we use a type theoretical framework (like Agda) as our meta-logic instead of a the conventional constructive set theory as \cite{coquand2018canonicity}. 

More concretely, we consider the syntax of our type theory as a \textit{Quotient Inductive Inductive Type} (QIIT), and most of the reasoning will be formulated using the \textit{algebra} of QIIT (mapping out function from this quotiented data-type). Each judgement will actually be represented by a QIIT type, which are mutual-recursively defined because those judgements are inter-dependently defined. Quotient is used to represent judgemental(definitional) equality, thanks to which we can have a concise representation.

Compared to the conventional formulation of type theory, we highlight some details of difference in this style of formluation of dependent: 1. this modern formulation of type theory does't rely on any notion of operational semantic but only equality 
%We will recover canonicity afterwards, which implies the computational ability of our theory. 
;
2. instead of using meta-level substitution, we use explicit substitution and debruijn indices. This is also called substitution calculus in the literature and is favoured due to its clear categorical semantic.


\subsection{Standard Model, for Consistency}
Once we have the syntax of our theory, the first question is if our theory is consistent, i.e., if we can syntactically derive bottom in our theory. We prove the consistency by extending the standard model from \citep{kaposi2017type} to interpret our theory. 


Notice that we interpret the bottom type using empty set, and thus we won't have syntactically derived  


\subsection{Canonicity}
 


\section{Conclusion}
\label{sec:conclusion}

\setlength{\bibsep}{.8ex}
\bibliographystyle{tex-macros/ACM-Reference-Format.bst}
\bibliography{refs.bib}
