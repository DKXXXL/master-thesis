Firstly, there are four standard kinds of judgement in MLTT, well-typed context $\goodCtx{\Gamma}{l}$, well-typed type $\goodType{\Gamma}{T}{l}$, well-typed term $\goodTerm{\Gamma}{t}{T}$ and well-typed substitution $\goodSub{\Gamma}{\gamma}{\Delta}$. We are following \citep{kaposi2019gluing} but we will omit the universe level attached to the judgement, to make it more readable. After these, we will have four new kinds of judgement. 

Because we consider each judgement as a (dependent) QIIT type, thus each judgement has a \textit{type} as well (otherwise how can they be encoded in Agda?). Latex-wise, we spell out these \textit{well-formed-ness rule} of type judgements. This \textit{well-formed-ness} might still be confusing thus we spell out their Agda counterpart to make things more understandable. Of course, a sequent is well-formed doesn't mean it is derivable. Every sequent, judgement and data we discuss here will be well-formed because we are working in a type-theoretical framework. Due to our intrinsic setting, we can omit a lot of presumption, for example we don't spell out that "Type-Universe" require a well-typed context, even though it did implicitly.



\begin{figure}[H]
  \begin{minipage}[b]{0.3\linewidth}
      $$
      \Rule[name=Sig]
      {\goodCtx{\Gamma}{i} \quad n \in \nat}
      {\goodSig{\Gamma}{\_}{n} \text{ is well-formed}}$$
      $$
      \Rule[name=Seal]
      {\goodCtx{\Gamma}{i} \quad \goodSig{\Gamma}{\sigma, \tau}{n}}
      {\goodSeal{\Gamma}{\sigma}{\tau} \text{ is well-formed}}
      $$
  \end{minipage}
  \begin{minipage}[b]{0.6\linewidth}
    \begin{minted}[]{agda}
      data Sig  : Con → ℕ → Set
      data WSig : Con i → ℕ → Set 
      data Seal : (Γ : Con) → Sig Γ n → Sig Γ n → Set
      data Inh  : (Γ : Con) → Sig Γ n → Sig Γ m → Set
    \end{minted}
  \end{minipage}
\end{figure}





Now we spell out some of the typing rules, not meant to be comprehensive 
$$ 
\Rule[name=Empty Context]{}{\goodCtx{\cdot}{0}} 
\quad
\Rule[name=Context Extension]
{\goodCtx{\Gamma}{i} \quad \goodType{\Gamma}{A}{j}}
{\goodCtx{\Gamma, A}{i \cup j}}  
$$

$$
\Rule[name=Type Universe]
{}
{\goodType{\Gamma}{\cU_j}{j + 1}}
\quad 
\Rule[name=Boolean]
{}
{\goodType{\Gamma}{\cB}{0}}
\quad 
\Rule[name=Bottom]
{}
{\goodType{\Gamma}{\bot}{0}}
\quad 
\Rule[name=Function]
{\goodType{\Gamma}{A}{j} 
  \quad \goodType{\Gamma, A}{B}{k}}
{\goodType{\Gamma}{\Pi A B}{j \cup k}}
$$

$$
\Rule[name=Type Subst]
{\goodType{\Delta}{T}{j} 
  \quad {\goodSub{\Gamma}{\gamma}{\Delta}}}
{\goodType{\Gamma}{T[\gamma]}{j}}
\quad 
\Rule[name=Function Subst]
{\goodSub{\Gamma}{\gamma}{\Delta}}
{\goodType{\Gamma}{(\Pi A B)[\gamma] \equiv \Pi A[\gamma] B[\gamma \uparrow] }{j \cup k}}
$$
$$
\Rule[name=Base Type Subst]
{\goodSub{\Gamma}{\gamma}{\Delta}}
{\goodType{\Gamma}{\cU_j[\gamma] \equiv \cU_j }{j + 1} \quad
 \goodType{\Gamma}{\cB[\gamma] \equiv \cB}{0} \quad 
 \goodType{\Gamma}{\bot[\gamma] \equiv \bot}{0}
}
$$