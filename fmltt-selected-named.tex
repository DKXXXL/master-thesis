\begin{figure}
\small

\renewcommand*{\arraystretch}{1.25}

\definecolor{new}{HTML}{FFFFFF}

\begin{gather*}
\begin{array}{@{}r@{\hspace{2ex}}c@{\ \ }c@{\ \ }l@{}}
\text{Types} & A, B, T  &\Coloneqq &
    ... \mid
\tikzmarkin[disable rounded corners=true,set fill color=new,set border color=new]{New2}(0.05,-0.15)(-0.10,0.30)
%   \TyWSingle{\wsig} \mid
    \wsigproj{i}{1}{\wsig} \mid \wsigproj{i}{2}{\wsig} \mid
    \TyLkg{\lsig} \mid \TyTkg{\lsig} \mid %\lsigprojT{\lsig} \mid
    \lsigproj{2}{\lsig} \mid \CaseSig{A}{x.B}{T}
\tikzmarkend{New2}
    \\
\text{Variables} & x, x' &\Coloneqq &  ..., \mathit{self}, \mathit{self}_1, \mathit{self}_2, ... \\
\text{Terms} & t, s, \lkg &\Coloneqq &
    ... \mid
\tikzmarkin[disable rounded corners=true,set fill color=new,set border color=new]{New3}(0.05,-0.15)(-0.10,0.30)
    %\wsingle{\wsig} \mid
    \wcode{\wsig} \mid \Wsup{i}{\wsig}{t_1}{x.t_2} \mid \LkgEmp \mid \LkgAdd{\lkg}{x.t} \mid \inh{h}{\lkg} \mid \Wrec{\wsig}{\lkg}{t} \mid
    \\
    &&&
    \lkgproj{1}{\lkg} \mid \lkgproj{2}{\lkg} \mid \lsigproj{s}{\lsig} \mid \Tkg{\lkg} \mid
    \Recproj{i}{\lkg}
\tikzmarkend{New3}
    \\
\tikzmarkin[disable rounded corners=true,set fill color=new,set border color=new]{New1}(0.05,-0.15)(-0.10,0.30)
\text{W-type signatures} & \wsig & \Coloneqq &
    \WSigEmp \mid \WSigAdd{\wsig}{A}{B} \mid \WSigSub{\wsig}
\tikzmarkend{New1}
    \\
\tikzmarkin[disable rounded corners=true,set fill color=new,set border color=new]{New5}(0.05,-0.15)(-0.10,0.30)
\text{Linkage signatures} & \lsig & \Coloneqq &
    \LSigEmp \mid \LSigAdd{\lsig}{x_1.\seal}{x_2.T} \mid \lsigproj{1}{\lsig} \mid \RecSig{\wsig}{T} 
\tikzmarkend{New5}
    \\
\tikzmarkin[disable rounded corners=true,set fill color=new,set border color=new]{New4}(0.05,-0.15)(-0.10,0.30)
\text{Linkage transformers} & h & \Coloneqq &
    \InhId \mid \InhExt{h}{x.t} \mid \InhOv{h}{x.t} \mid \InhInh{h} \mid \InhNest{h}{h'}
\tikzmarkend{New4}
\end{array}
\end{gather*}

\begin{mathpar}
\judgebox{\goodType{\Gm}{T_1 \equiv T_2}{}}

\judgebox{\goodTerm{\Gm}{t}{T}}

\judgebox{\goodTerm{\Gm}{t_1 \equiv t_2}{T}}

\judgebox{\goodWSig{\Gm}{\wsig}{n}}

\judgebox{\goodSig{\Gm}{\lsig}{n} }

\judgebox{\goodInh{\Gm}{h}{\lsig_1}{\lsig_2}}
\\

\Rule[name=tm/w]{
    \goodWSig{\Gm}{\wsig}{n}
}{
%   \goodTerm{\Gm}{\wsingle{\wsig}}{\TyWSingle{\wsig}}
%   \\
    \goodTerm{\Gm}{\wcode{\wsig}}{\cU}
}

\Rule[name=wsig/empty]{
}{
    \goodWSig{\Gm}{\WSigEmp}{0}
}

\Rule[name=wsig/add]{
    \goodWSig{\Gm}{\wsig}{n}
    \\\\
    \goodType{\Gm}{A}{}
    \\
    \goodTerm{\Gm}{B}{A \to \cU}
}{
    \goodWSig{\Gm}{\WSigAdd{\wsig}{A}{B}}{n+1}
}

\Rule[name=tm/wsup]{
    \goodWSig{\Gm}{\wsig}{n}
    \\
    \goodTerm{\Gm}{t_1}{\wsigproj{i}{1}{\wsig}}
    \\\\
    \goodTerm{\Gm, x : \El{\wsigproj{i}{2}{\wsig}(t_1)} }{t_2}{\El{\wcode{\wsig}}}\YZ{Is 'a' a typo?}\EDJreply{Yes. Fixed.}\YZreply{Two x's serving different purposes is a little confusing. The real cause you have to have two x's is that the projection does not return the free variable's name. Write this rule by induction on n (rater than projection)?}\EDJreply{Right. In intrinsic setting returning variable name is not making sense.  \\ I don't know what do you mean by writing by induction on n. Different B all needs to have an 'x' here. I don't see why induction on n is helpful for this. \\ Because this one can be written in debruijn without induction on n, so it also should be able to written in named without induction on n. I will conventionally make x : A (when appearing in B) into xA : B (when appearing in B) if you are okay with this? I changed two places for a look.}
%   \\
%   i \in \{1, ..., n\}
}{
    \goodTerm{\Gm}{\Wsup{i}{\wsig}{t_1}{x.t_2}}{\El{\wcode{\wsig}}}
}

\Rule[name=tm/wrec]{
    \goodTerm{\Gm}{\lkg}{\TyLkg{\RecSig{\wsig}{T}}}
    \\
    \goodTerm{\Gm}{t}{\El{\wcode{\wsig}}}
}{
    \goodTerm{\Gm}{\Wrec{\wsig}{\lkg}{t}}{T}
}

\Rule[name=tyeq/casety]{
    \goodType{\Gm}{A}{}
    \\
    \goodTerm{\Gm}{B}{A \to \cU}
    \\
    \goodType{\Gm}{T}{}
}{
    \goodType{\Gm}{
        \CaseSig{A}{B}{T}
        \equiv
        \nTyPi{x}{A}{(\El{B(x)} \to T) \to T}
        % \TyPi{A}{\TyPi {\TyPi{B}{\sub{T}{\SubstWeak{2}}}} {\sub{T}{\SubstWeak{2}}}}
    }{}
}

\Rule[name=lsig/empty]{
}{
    \goodSig{\Gm}{\LSigEmp}{0}
}
\hfill
\Rule[name=lsig/add]{
    \goodSig{\Gm}{\lsig}{n} 
    \\
    \goodType{\Gm, \mathit{self} : A}{T}{}
    \\\\
    \goodTerm{\Gm, x : \TyTkg{\lsig}}{\seal}{A}
}{
    \goodSig{\Gm}{\LSigAdd{\lsig}{x.\seal}{\mathit{self}.T}}{n+1}
}
\hfill
\Rule[name=l/empty]{
}{
    \goodTerm{\Gm}{\LkgEmp}{\TyLkg{\LSigEmp}}
}
\hfill
\Rule[name=l/add]{
    \goodTerm{\Gm}{\lkg}{\TyLkg {\lsig}}
    \\
    \goodTerm{\Gm, \mathit{self} : A}{t}{T}
    \\\\
    \goodTerm{\Gm,x : \TyTkg{\lsig}}{\seal}{A}
}{
    \goodTerm{\Gm}{\LkgAdd{\lkg}{\mathit{self}.t}}{\TyLkg{\LSigAdd{\lsig}{x.\seal}{\mathit{self}.T}}}
}

%\Rule[name=l/proj/2]{
%    \goodSig{\Gm}{\lsig}{n+1}
%    \\
%    \goodTerm{\Gm}{\lkg}{\TyLkg{\lsig}}
%}{
%    \goodTerm{\Gm, \TyTkg{\lsigproj{1}{\lsig}}}{\lkgproj{2}{\lkg}}{\lsigproj{2}{\lsig}}
%}
%
%\Rule[name=tyeq/pk/empty]{
%}{
%    \goodType{\Gm}{\TyTkg{\LSigEmp} \equiv \top}{}
%}
%
%\Rule[name=tmeq/pk/empty]{
%}{
%    \goodTerm{\Gm}{
%        \Tkg{\LkgEmp}
%        \equiv
%        \unit
%    }{\TyTkg{\LSigEmp}}
%}
%
\Rule[name=tyeq/pk/add]{
    \goodSig{\Gm}{\lsig}{n} 
    \\
    \goodTerm{\Gm,x : \TyTkg{\lsig}}{\seal}{A}
    \\
    \goodType{\Gm, \mathit{self} : A}{T}{}
}{
    \goodType{\Gm}{
        \TyTkg {\LSigAdd  {\lsig} {\seal} {T}}
        \equiv 
        \nTySigma{x}{\TyTkg {\lsig}}{\nsub{T}{\mathit{self}}{s}}
    }{}
}

\Rule[name=tmeq/pk/add]{
    \goodTerm{\Gm}{\lkg}{\TyLkg {\lsig}} 
    \\
    \goodTerm{\Gm,x : \TyTkg{\lsig}}{\seal}{A}
    \\
    \goodTerm{\Gm, \mathit{self}: A}{t}{T}
}{
    \goodTerm{\Gm}{
        \Tkg{\LkgAdd{\lkg}{\mathit{self}.t}}
        \equiv
        \pair{\Tkg{\lkg}}{
            \nsub{t}{\mathit{self}}{\nsub{s}{x}{\Tkg{\lkg}}}
        }
    }{
        \TyTkg {\LSigAdd {\lsig}{\seal}{T}}
    }
}

\Rule[name=tm/inh]{
    \goodInh{\Gm}{h}{\lsig_1}{\lsig_2}
    \\
    \goodTerm{\Gm}{\lkg}{\TyLkg{\lsig_1}}
}{
    \goodTerm{\Gm}{\inh{h}{\lkg}}{\TyLkg{\lsig_2}}
}

\Rule[name=tmeq/ov/beta]{
    \goodInh{\Gm}{h}{\lsig_1}{\lsig_2}  
    \\
    \goodTerm{\Gm}{\lkg}{\TyLkg{\lsig_1}}
    \\\\
    \goodTerm{\Gm, \mathit{self}_1 : A_1}{t_1}{T_1}
    \\
    \goodTerm{\Gm, x_1 : \TyTkg{\lsig_1}}{\seal_1}{{A_1}}
    \\
    \goodTerm{\Gm, \mathit{self}_2 : A_2}{t_2}{T_2}
    \\
    \goodTerm{\Gm, x_2 : \TyTkg{\lsig_2}}{\seal_2}{{A_2}}
}{
    \goodTerm{\Gm}{
        \inh{\InhOv{h}{\mathit{self}_2 . t_2}}{\LkgAdd{\lkg}{\mathit{self}_1 . t_1}}
        \equiv 
        \LkgAdd{\inh{h}{\lkg}}{\mathit{self}_2 . t_2}
    }{
        \TyLkg{\LSigAdd{\lsig_2}{x_2.\seal_2}{\mathit{self}_2.T_2}}
    }
}

\end{mathpar}

\caption{Syntax and selected typing rules of \TT.}
\label{fig:fmltt-selected}
\end{figure}