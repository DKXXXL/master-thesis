We implement our language design as a Coq plugin.
%
It works by translating programs in \Lang syntax into
programs that can be checked and evaluated by Coq.
The translation is compatible with interactive theorem proving (\ref{chg:tooling}), in that
a family is translated piece by piece, allowing each field to be defined
and checked separately.
The translation is modular and efficient\EDJ{I am very hesitant to say efficient :( }, in that
code compiled for fields of a base family
can be shared with derived families without having to be rechecked.

\noindentparagraph{Explicit self parameterization.}

The spirit of the translation is to take ``family polymorphism''
literally: every field is translated into a Coq definition that is
polymorphic to (i.e., universally quantified over) a representation of
its enclosing family.
While this universal quantification has been implicit with the \Lang
syntax, it has to be made explicit in the translated Coq code.

\cref{fig:stlc-compiled,fig:stlcfix-compiled} illustrate the translation of the
\lsti{STLC} and \lsti{STLCFix} families from \cref{fig:stlc-mechanized}.
Fields of a family are translated into parameterized Coq modules
(or parameterized module types).

As an example,
consider field \lsti{env} in family \lsti{STLC}.
It is translated into a top-level module named \lsti{STLC\_env}.
This module has a parameter called \lsti{self} representing the enclosing
family: fields of the current family in the context of \lsti{env} can be
referenced through \lsti{self}.
In particular, \lsti{env} is defined as \lsti{id}\,\lsti{->}\,\lsti{option ty},
where \lsti{ty} is a late-bound reference to the \lsti{ty} field of the
enclosing family.
Hence, this reference to \lsti{ty} is translated to \lsti{self.ty},
which is manifestly polymorphic to the enclosing family.
This translation of the \lsti{env} field can be shared
with a derived family even if it extends \lsti{ty} (e.g., \lsti{STLCProd})---%
no recompilation is needed because \lsti{self.ty} is not tied to any
concrete definition of \lsti{ty}.

The type of \lsti{STLC\_env}'s \lsti{self} parameter is \lsti{STLC\_env\_Ctx},
a module type constructed from \lsti{STLC\_ty}
(i.e., the translation of the field before \lsti{env})
and its context \lsti{STLC\_ty\_Ctx}.
In turn, \lsti{STLC\_ty\_Ctx} (not shown in \cref{fig:stlc-compiled}) is
constructed from \lsti{STLC\_subst},
the translation of the field before \lsti{ty},
and its context \lsti{STLC\_subst\_Ctx}.
Thus, the \lsti{self} parameter can be used to reference those and only
those fields in the current field's typing context, which echoes the
discussion in \cref{sec:sound}.

\begin{figure}
\vspace{-5pt}
\begin{minipage}{\textwidth}
\begin{multicols}{2}

\lstset{
  basicstyle=\fontsize{8}{8.2}\ttfamily,
% numbers=left,
}

\colorlet{codecomment-color}{dkgreen}

\newcommand{\innersep}{\vspace{6pt}}
\newcommand{\outersep}{\vspace{5pt}}
\newcommand{\outdent}{\!\!\!}

\begin{lstlisting}
[@\outdent@][@\codecomment{Code emitted upon definition of \texttt{tm} in family \texttt{STLC}}@][@\tanc{c-STLC-tm-start}@]
[@\outdent@]Module Type STLC\_tm\_Ctx.
[@\outdent@]End STLC\_tm\_Ctx.[@\innersep@]
[@\outdent@]Module Type STLC\_tm (self : STLC\_tm\_Ctx).
Axiom tm : Set.
Axiom tm_unit : tm.[@\hfill@]Axiom tm_var : id -> tm.
Axiom tm_abs : id -> tm -> tm.
Axiom tm_app : tm -> tm -> tm.
Axiom tm_prect_STLC : [@\dadada@].
Axiom [@\textls[-30]{tm\_unit\_eq\_STLC}@][@\;:\;@][@\dadada@]. Axiom [@\textls[-30]{tm\_abs\_eq\_STLC}@][@\;:\;@][@\dadada@].
Axiom [@\textls[-30]{tm\_abs\_eq\_STLC}@][@\;:\;@][@\dadada@].  Axiom [@\textls[-30]{tm\_app\_eq\_STLC}@][@\;:\;@][@\dadada@].
[@\outdent@]End STLC\_tm.[@\tanc{c-STLC-tm-end}@][@\outersep@]
[@\outdent@][@\codecomment{Code emitted upon definition of \texttt{subst} in family \texttt{STLC}}@][@\tanc{c-STLC-subst-start}@]
[@\outdent@]Module Type STLC\_subst\_Cases\_Ctx.
Include STLC\_tm\_Ctx.  Include STLC\_tm.
[@\outdent@]End STLC\_subst\_Cases\_Ctx.[@\innersep@]
[@\outdent@]Module [@\textls[-30]{STLC}@]\_[@\textls[-30]{subst}@]\_[@\textls[-30]{Cases}@] (self[@\,@]:[@\,@][@\textls[-30]{STLC}@]\_[@\textls[-30]{subst}@]\_[@\textls[-30]{Cases}@]\_[@\textls[-30]{Ctx}@]).
Def subst\_tm_unit :=
  \lambda (x : id) (t : self.tm), self.tm_unit.
Def subst\_tm_var := [@\dadada@].
Def subst\_tm_abs := [@\dadada@].  Def subst\_tm_app := [@\dadada@].
[@\outdent@]End STLC\_subst\_Cases.[@\innersep@]
[@\outdent@]Module Type STLC\_subst\_Ctx.
Include STLC\_subst\_Cases\_Ctx.
Include STLC\_subst\_Cases.
[@\outdent@]End STLC\_subst\_Ctx.[@\innersep@]
[@\outdent@]Module Type STLC\_subst (self : STLC\_subst\_Ctx).
Axiom subst : self.tm[@\,@]->[@\,@]id[@\,@]->[@\,@]self.tm[@\,@]->[@\,@]self.tm.
Axiom subst_tm_unit_eq :
  \forall x t, self.subst (self.tm_unit) x t =
  self.subst\_tm_unit x t.
Axiom subst_tm_var_eq : [@\dadada@].
Axiom subst_tm_abs_eq : [@\dadada@].
Axiom subst_tm_app_eq : [@\dadada@].
[@\outdent@]End STLC\_subst.[@\tanc{c-STLC-subst-end}@][@\columnbreak@]
[@\outdent@][@\codecomment{Code emitted upon definition of \texttt{ty} in family \texttt{STLC}}@] [@\tanc{c-STLC-ty-start}@]
[@\dadada@][@\tanc{c-STLC-ty-end}@][@\outersep@]
[@\outdent@][@\codecomment{Code emitted upon definition of \texttt{env} in family \texttt{STLC}}@][@\tanc{c-STLC-env-start}@]
[@\outdent@]Module Type STLC\_env\_Ctx.
Include STLC\_ty\_Ctx.  Include STLC\_ty.
[@\outdent@]End STLC\_env\_Ctx.[@\innersep@]
[@\outdent@]Module STLC\_env (self : STLC\_env\_Ctx).
Def env : Type := id -> option self.ty.
[@\outdent@]End STLC\_env.[@\tanc{c-STLC-env-end}@][@\outersep@]
[@\outdent@][@\codecomment{Code emitted for other fields defined in family \texttt{STLC}}@][@\tanc{c-STLC-rest-start}@]
[@\dadada@][@\tanc{c-STLC-rest-end}@][@\outersep@]
[@\outdent@][@\codecomment{Code emitted upon conclusion of family \texttt{STLC}}@][@\tanc{c-STLC-start}@]
[@\outdent@]Module STLC.[@\innersep@]
Inductive tm\_ : Set :=[@\tanc{c-STLC-d-tm-start}@]
| tm_unit\_[@\,@]:[@\,@]tm\_ | tm_var\_[@\,@]:[@\,@]id -> tm\_
| tm_abs\_[@\,@]:[@\,@]id->tm\_->tm\_ | tm_app\_[@\,@]:[@\,@]tm\_->tm\_->tm\_.
[@\codecomment{Instantiate \texttt{tm} \& its constructors}@]
Def tm := tm\_.
Def tm_unit := tm_unit\_. Def tm_var := tm_var\_.
Def tm_abs  := tm_abs\_.  Def tm_app := tm_app\_.
[@\codecomment{Instantiate \texttt{tm} partial recursor \& its equalities}@]
Def tm_prect_STLC := \lambda P, tm\__rect (\lambda[@\,@]t, option (P t)).
Fact[@\;@][@\textls[-20]{tm\_unit\_eq\_STLC}@][@\,@]:[@\,@][@\dadada@].[@\,@][@\textls[-40]{{\color{dkblue}reflexivity}.}@][@\,@]Qed.[@\hfill@]Fact [@\dadada@][@\tanc{c-STLC-d-tm-end}@][@\innersep@]
Include STLC\_subst\_Cases.[@\tanc{c-STLC-d-subst-start}@]
[@\codecomment{Instantiate \texttt{subst} \& its equalities}@]
Def subst[@\,@] := tm\__rect _ subst\_tm_unit subst\_tm_var
                        subst\_tm_abs  subst\_tm_app.
Fact[@\;@][@\textls[-40]{subst\_tm\_unit\_eq}@][@\,@]:[@\,@][@\dadada@].[@\,@][@\textls[-40]{{\color{dkblue}reflexivity}.}@][@\,@]Qed.[@\hfill@]Fact [@\dadada@][@\tanc{c-STLC-d-subst-end}@][@\tanc{c-STLC-firstcol-end}@][@\innersep@]
[@\codecomment{Instantiate \texttt{ty}, its constructors, partial recusor \& eq.}@] [@\dadada@][@\tanc{c-STLC-d-ty}@][@\tanc{c-STLC-sndcol-start}@][@\innersep@]
Include STLC\_env.[@\hfill\codecomment{Include \texttt{env}}@][@\tanc{c-STLC-d-env}@][@\innersep@]
[@\codecomment{Include/Instantiate other fields of family \texttt{STLC}}@] [@\dadada@][@\tanc{c-STLC-d-rest}@][@\innersep@]
[@\outdent@]End STLC.[@\tanc{c-STLC-end}@]
\end{lstlisting}

\makeline[-1pt]{c-STLC-tm-start}{c-STLC-tm-end}[codecomment-color!50][1.8pt]
\makeline[-1pt]{c-STLC-subst-start}{c-STLC-subst-end}[codecomment-color!50][1.8pt]
\makeline[.5\textwidth+8pt]{c-STLC-ty-start}{c-STLC-ty-end}[codecomment-color!50][1.8pt]
\makeline[.5\textwidth+8pt]{c-STLC-env-start}{c-STLC-env-end}[codecomment-color!50][1.8pt]
\makeline[.5\textwidth+8pt]{c-STLC-rest-start}{c-STLC-rest-end}[codecomment-color!50][1.8pt]
\makeline[.5\textwidth+8pt]{c-STLC-start}{c-STLC-end}[codecomment-color!50][1.8pt]
\makeline[.5\textwidth+12pt]{c-STLC-d-tm-start}{c-STLC-d-tm-end}[codecomment-color!35][1.0pt]
\makeline[.5\textwidth+12pt]{c-STLC-d-subst-start}{c-STLC-d-subst-end}[codecomment-color!35][1.0pt]
\makeline[.5\textwidth+12pt]{c-STLC-d-ty}{c-STLC-d-ty}[codecomment-color!35][1.0pt]
\makeline[.5\textwidth+12pt]{c-STLC-d-env}{c-STLC-d-env}[codecomment-color!35][1.0pt]
\makeline[.5\textwidth+12pt]{c-STLC-d-rest}{c-STLC-d-rest}[codecomment-color!35][1.0pt]

\end{multicols}
\end{minipage}

\vspace{-12pt}

\caption{Compilation of family \lsti!STLC! (\cref{fig:stlc-mechanized}).}
\label{fig:stlc-compiled}
\end{figure}
\begin{figure}
\vspace{-5pt}
\begin{minipage}{\textwidth}
\begin{multicols}{2}

\lstset{
  basicstyle=\fontsize{8}{8.2}\ttfamily,
% numbers=left,
}

\colorlet{codecomment-color}{dkgreen}

\newcommand{\innersep}{\vspace{6pt}}
\newcommand{\outersep}{\vspace{5pt}}
\newcommand{\outdent}{\!\!\!}

\colorlet{codecomment-color}{orange}

\begin{lstlisting}
[@\outdent@][@\codecomment{Code emitted upon definition of \texttt{tm} in \texttt{STLCFix}}@][@\tanc{c-STLCFix-tm-start}@]
[@\outdent@]Module Type STLCFix\_tm\_Ctx.
[@\outdent@]End STLCFix\_tm\_Ctx.[@\innersep@]
[@\outdent@]Module Type STLCFix\_tm (self : STLCFix\_tm\_Ctx).
Include STLC\_tm(self).
Axiom tm_fix : id -> tm -> tm.
Axiom tm_fix_eq_STLC : \forall[@\,@][@\dadada@],[@\,@]tm_prect_STLC [@\dadada@][@\;@]=[@\;@]None.
Axiom tm_prect_STLCFix : [@\dadada@].
Axiom tm_fix_eq_STLCFix : [@\dadada@].
[@\outdent@]End STLCFix\_tm.[@\tanc{c-STLCFix-tm-end}@][@\outersep@]
[@\outdent@][@\codecomment{Code emitted upon definition of \texttt{subst} in \texttt{STLCFix}}@][@\tanc{c-STLCFix-subst-start}@]
[@\outdent@]Module Type STLCFix\_subst\_Cases\_Ctx.
Include STLCFix\_tm\_Ctx. Include STLCFix\_tm.
[@\outdent@]End STLCFix\_subst\_Cases\_Ctx.[@\innersep@]
[@\outdent@]Module STLCFix\_subst\_Cases
[@\outdent@](self : STLCFix\_subst\_Cases\_Ctx).
Include STLC\_subst\_Cases(self).[@\hfill\codecomment{reuse}@]
Def subst\_tm_fix := [@\dadada@].[@\hfill\codecomment{translation of new case}@]
[@\outdent@]End STLCFix\_subst\_Cases.[@\innersep@]
[@\outdent@]Module Type STLCFix\_subst\_Ctx.
Include STLCFix\_subst\_Cases\_Ctx.
Include STLCFix\_subst\_Cases.
[@\outdent@]End STLCFix\_subst\_Ctx.[@\innersep@]
[@\outdent@]Module Type STLCFix\_subst (self[@\,@]:[@\,@][@\textls[-30]{STLCFix}@]\_[@\textls[-30]{subst}@]\_[@\textls[-30]{Ctx}@]).
Include STLC\_subst(self).
Axiom subst_tm_fix_eq : [@\dadada@].
[@\outdent@]End STLCFix\_subst.[@\tanc{c-STLCFix-subst-end}@][@\outersep@]
[@\outdent@][@\codecomment{Code emitted for other fields defined in \texttt{STLCFix}}@] [@\dadada@][@\tanc{c-STLCFix-rest}@][@\outersep@]
[@\outdent\codecomment{Code emitted upon conclusion of \texttt{STLCFix}}@][@\tanc{c-STLCFix-start}@]
[@\outdent@]Module STLCFix.[@\innersep@]
[@\codecomment{Instantiate \texttt{tm} \& its constructors}@][@\tanc{c-STLCFix-d-tm-start}@]
Inductive tm : Set :=
| tm_unit | tm_var (v : id)
| tm_abs (v : id) (b : tm) | tm_app (a b : tm)
| tm_fix (v : id) (b : tm).
[@\codecomment{Inst. \texttt{tm} partial recursors \& their comp.\ behaviors}@] [@\dadada@][@\tanc{c-STLCFix-d-tm-end}@][@\innersep@]
Include STLCFix\_subst\_Cases. [@\tanc{c-STLCFix-d-subst-start}@]
[@\codecomment{Instantiate \texttt{subst} \& its computational behaviors}@]
Def subst := tm_rect _ subst\_tm_unit
subst\_tm_var subst\_tm_abs subst\_tm_app[@\hfill\codecomment{reuse}@]
subst\_tm_fix.
Fact [@\textls[-0]{subst\_tm\_unit\_eq}@] : [@\dadada@]. reflexivity. Qed.
Fact [@\dadada@][@\tanc{c-STLCFix-d-subst-end}@][@\innersep@]
[@\codecomment{\textls[-10]{Include\;\texttt{ty}, its constructors, partial recursor, etc.}}@] [@\dadada@][@\tanc{c-STLCFix-d-ty}@][@\innersep@]
Include STLC\_env.[@\hfill\codecomment{reuse}@][@\tanc{c-STLCFix-d-env}@][@\innersep@]
[@\codecomment{Include/Instantiate other fields of \texttt{STLCFix}}@] [@\dadada@][@\tanc{c-STLCFix-d-rest}@][@\innersep@]
Include STLC\_typesafe.[@\hfill\codecomment{reuse}@][@\tanc{c-STLCFix-d-typesafe}@][@\innersep@]
[@\outdent@]End STLCFix.[@\tanc{c-STLCFix-end}@]
\end{lstlisting}

\makeline[-1pt]{c-STLCFix-tm-start}{c-STLCFix-tm-end}[codecomment-color!50][1.8pt]
\makeline[-1pt]{c-STLCFix-subst-start}{c-STLCFix-subst-end}[codecomment-color!50][1.8pt]
\makeline[.5\textwidth+8pt]{c-STLCFix-rest}{c-STLCFix-rest}[codecomment-color!50][1.8pt]
\makeline[.5\textwidth+8pt]{c-STLCFix-start}{c-STLCFix-end}[codecomment-color!50][1.8pt]
\makeline[.5\textwidth+12pt]{c-STLCFix-d-tm-start}{c-STLCFix-d-tm-end}[codecomment-color!35][1.0pt]
\makeline[.5\textwidth+12pt]{c-STLCFix-d-subst-start}{c-STLCFix-d-subst-end}[codecomment-color!35][1.0pt]
\makeline[.5\textwidth+12pt]{c-STLCFix-d-ty}{c-STLCFix-d-ty}[codecomment-color!35][1.0pt]
\makeline[.5\textwidth+12pt]{c-STLCFix-d-env}{c-STLCFix-d-env}[codecomment-color!35][1.0pt]
\makeline[.5\textwidth+12pt]{c-STLCFix-d-rest}{c-STLCFix-d-rest}[codecomment-color!35][1.0pt]
\makeline[.5\textwidth+12pt]{c-STLCFix-d-typesafe}{c-STLCFix-d-typesafe}[codecomment-color!35][1.0pt]

\vspace{-3pt}

\definecolor{codecomment-color}{HTML}{000080}

\begin{lstlisting}
[@\outdent@][@\codecomment{Code emitted upon command \texttt{\textls[-35]{Check STLCFix.typesafe}}}@][@\tanc{c-Check-STLCFix-typesafe-start}@]
[@\outdent@]Check STLCFix.typesafe.[@\tanc{c-Check-STLCFix-typesafe-end}@]
\end{lstlisting}

\makeline[.5\textwidth+8pt]{c-Check-STLCFix-typesafe-start}{c-Check-STLCFix-typesafe-end}[codecomment-color!50][1.8pt]

\end{multicols}
\end{minipage}

\vspace{-12pt}

\caption{Compilation of family \lsti!STLCFix! and the final \lsti!Check! command (\cref{fig:stlc-mechanized}).}
\label{fig:stlcfix-compiled}
\end{figure}

\noindentparagraph{Translating extensible inductive types.}

An \lsti{FInductive} definition is translated to a parameterized module type.
Consider the inductive type \lsti{tm}.
In \cref{fig:stlc-compiled}, it is translated to a top-level module type
\lsti{STLC\_tm} that declares a \lsti{tm} type,
four functions standing for the constructors,
%\EDJ{I am not sure if it is better we emphasize in the following way :`` that declares a tm type, its constructors (e.g. tmabs) as a normal function,.. '' i.e. emphasize it is a normal function in the module type},
a partial recursor (\lsti{tm_prect_STLC}),
and the computational behaviors of the partial recursor
(e.g., \lsti{tm_abs_eq_STLC}). %\cref{sec:prec}.
%\EDJ{I propose to change this sentence to ``and the prop eq about computational behaviour on the partial recursor''. And thus in the other parts of the paper, we can use the word ``computational behaviour'' instead of prop eq. Because just saying prop eq seems a bit of lack of context/details.}

Importantly,
\lsti{STLC\_tm} merely declares the existence of these names and their types;
it does not specify their definitions.
Having these names and their types available through the context parameters (\lsti{self})
suffices for the translations of the subsequent fields to be type-checked by Coq.
Leaving the definitions unspecified enables \lsti{STLC} and \lsti{STLCFix}
to instantiate \lsti{tm} differently upon \lsti{End}~\lsti{STLC} and upon \lsti{End}~\lsti{STLCFix}.
In particular, non-exhaustive pattern matching is prevented because
an ordinary recursor like \lsti{tm_rect} is not available.

The command \lsti{FInductive tm : Set += tm_fix : }\dadada~in family
\lsti{STLCFix} is again translated to a module type \lsti{STLCFix\_tm} (\cref{fig:stlcfix-compiled}).
It includes all the names declared by \lsti{STLC\_tm}
via command \lsti!Include STLC\_tm(self)!,
and additionally declares \lsti{tm_fix}, a partial recursor, and related equalities.


\noindentparagraph{Translating recursion and induction.}

An \lsti{FRecursion} definition is translated in two parts:
first a module containing the definitions of all the case handlers,
and then a module type declaring the existence of the recursive function as well as
its computational behaviors.

Consider the translation of \lsti{subst} in family \lsti{STLC}.
First, a module named \lsti{STLC\_subst\_Cases} is generated on the fly.
%every time the programmer completes a \lsti{Case}, the case handler (e.g., \lsti{Def subst\_tm_unit})
%is type-checked and added to the module. %\rD{This is also confusing-- why is the programmer involved in the translation? Is it that modules are being generated in the background, while commands are being processed?}\EDJreply{changed to mention "type-checked" explicitly. please review}
Importantly, programming remains interactive, as the programmer need not wait until the
entire \lsti{FRecursion} definition is completed to have a \lsti{Case} command
type-checked.

Upon \lsti{End}~\lsti{subst}, a module type named \lsti{STLC\_subst} is generated.
%
As discussed in \cref{sec:latebind}, \lsti{subst} can be further bound,
% \EDJ{If we don't allow further binding case handler(mentioned in your last reply), how do the user further bind the subst? I mean in the surface syntax}\YZreply{The position is that individual case handlers cannot be further bound, but the recursive function itself can be further bound by acquiring new case handlers.}
so its definition is not exposed to the fields that come after it.
Accordingly, the translation \lsti{STLC\_subst} merely declares the types of \lsti{subst}
and the equalities about its computational behaviors, leaving \lsti{subst} undefined
and the equalities unproven.
%
The equalities are stated in terms of the case handlers,
whose definitions \emph{are} available through the \lsti{self} parameter.
So Coq can simplify, for example, the type of \lsti{subst_tm_unit_eq} to
\lsti[basicstyle=\fontsize{8.5}{9},basewidth=.95ex]{\forall x t, self.subst self.tm_unit x t = self.tm_unit}.
These equalities about the computational behaviors of \lsti{subst}
will be included and available for use in the translations of the
subsequent fields through their \lsti{self} parameters.

Importantly, code generated for the case handlers is shared with derived families.
In \cref{fig:stlcfix-compiled}, module \lsti{STLCFix\_subst\_Cases}
reuses---without rechecking---all the case handlers in
\lsti{STLC\_subst\_Cases} via command \lsti{Include STLC\_subst\_Cases(self)}.

The translation of \lsti{FInduction} is similar, except that there is no need to
register computational behaviors, as \lsti{FInduction} proofs are considered opaque.


\noindentparagraph{Translation of further-bindables vs.\ non-further-bindables.}

In family \lsti{STLC},
field \lsti{env} and the case handlers for \lsti{subst}
are not further-bindable by derived families.
In contrast, \lsti{tm}, \lsti{subst}, and the related equalities
can be further bound.
The distinction is reflected in the translations.
The further-bindable fields are translated to module types that export only types of the fields.
The non-further-bindable fields are translated to modules that export definitional equalities
on the fields.
Opaque fields in \Lang can be further bound (\cref{sec:override});
they are translated to Coq modules that export opaque fields.

%\footnotetext{%
%Axioms in Coq module types are more like parameters than real axioms.
%They become part of the signature described by the module type.
%}


\noindentparagraph{Eliminating \lsti{self} by aggregation.}

Upon the conclusion of a family definition,
a representation of the family is created.
For example, module \lsti{STLC} in \cref{fig:stlc-compiled} is generated
upon \lsti{End}~\lsti{STLC}.
%
This module can be viewed as the ``fixed point'' of the
\lsti{self}-parameterized translations.
The ``fixed point'' is taken step by step, by adding the translation of
each field to this module in the same order as they appear in the family
definition.

For the non-further-bindables, the translated modules are directly included
(e.g., \lsti{Include STLC\_env} and \lsti{Include STLC\_subst\_Cases} in \cref{fig:stlc-compiled}).
%
The instantiation of \lsti{self}s for these modules is implicit, thanks to a Coq nicety:
when including a higher-order module, Coq automatically instantiates its
parameter with the current interactive module environment.
For instance, command \lsti{Include STLC\_subst\_Cases} is successfully executed,
because Coq automatically instantiates the \lsti{self} parameter using the current
module environment, which by construction contains all the fields required by \lsti{STLC\_subst\_Cases\_Ctx}.

For the further-bindables, \lsti{Axiom}s declared in the module types must be instantiated.
%
\begin{itemize}
  [align=left,itemsep=2pt,labelsep=*,leftmargin=*]

\item
In \cref{fig:stlc-compiled},
an inductive type \lsti{tm} is generated, instantiating the axiomatized
\lsti{tm} type and its constructors.
%
The partial recursor \lsti{tm_prect_STLC} is defined
%to \lsti{\lambda}\,\lsti{P, tm_rect (\lambda}\,\lsti{t, option (P t))},
with the help of \lsti{tm_rect}, the recursor Coq generates for \lsti{tm}.
The computational behaviors of \lsti{tm_prect_STLC} are immediate, by \lsti{reflexivity}.

\item
Similarly, \lsti{subst} is instantiated by applying the recursor \lsti{tm_rect}
to the (already included) case handlers. %(\lsti{subst\_tm_unit}, etc.).
The computational behaviors of \lsti{subst} are then immediate, by \lsti{reflexivity}.

\end{itemize}

Module \lsti{STLCFix} in \cref{fig:stlcfix-compiled} is emitted upon
\lsti{End}~\lsti{STLCFix}, in the same way as described above for \lsti{STLC}.
The translation makes sharing evident.
In particular, case handlers compiled for \lsti{STLC}
are reused to instantiate \lsti{subst}, \lsti{substlem}, and alike.
\lsti{STLC\_env} and \lsti{STLC\_typesafe}
are also reused in the construction of module \lsti{STLCFix}.
%
One may argue that since the first four constructors of \lsti{tm} are
repeated in \lsti{STLCFix}, the translation does not satisfy the
modular compilation requirement.
We could address this concern by using wrapper types,
but we consider restating constructors a small price to pay in return
for the clarity and concision of implementation.
We emphasize that compiled case handlers, such as \lsti!subst\_tm_abs!,
are entirely reusable without rechecking, even with restated constructors.
%
Finally, the reference \lsti{STLCFix.typesafe} (where \lsti{STLCFix} is a family)
can simply be translated to \lsti{STLCFix.typesafe} (where \lsti{STLCFix} is a Coq module),
as the last line of \cref{fig:stlcfix-compiled} shows.

\noindentparagraph{Trusted base.}

Rather than modifying the Coq kernel to extend its core theory, a translation to
Coq conveniently reduces the trusted base of any development using \Lang to Coq.
%and improves compatibility with different versions of Coq.
In particular, once a family is closed, \lsti{Print Assumptions} can be used
to verify that there are no lingering \lsti{Axiom}s introduced by the translation.
Ramifications of possible bugs in the \Lang implementation are limited to
the usability of the plugin.
%Finally, we emphasize that our compilation can make sure the trusted code base is restricted to the Coq's kernel, as all the feature supported by our plugin has been compiled into Coq's surface syntax/Vernacular commands.


\section{Limitations}
\label{sec:limitations}

The \Lang implementation currently does not yet bring extensibility to Coq's full
facility for inductive types.
Mutually inductive types and parameterized inductive types are not yet extensible,
though indexed inductive types are supported and can be used to encode parameterized ones.
Partial recursors are automatically generated only for non-indexed inductive types.
These features are not exercised by our case studies (\cref{sec:casestudies})
but may be useful for modeling other languages.
We believe that they do not pose conceptual challenges and can be
addressed with more engineering effort of the same kind that went into \Lang.

%We list some of the current limitation of our implementation. We believe the following limitation can be rememdied by more engineering effort:
%\begin{itemize}
%  \item We don't support the full power of inductive facility in Coq for extensible inductive type---concretely, we don't support mutual inductive type, syntax extensions for inductive type and parameters in the inductive type (we only support indexed inductive type right now)\YZ{Can parameterized inductive types always be encoded as indexed inductive types?}\EDJreply{Yes. but practically the dependent elimination principle is different when using encoding.}
%  \item We don't support mutual fixpoint definition
%  \item We don't support incorporating "Hint" Vernacular commands, and thus the users cannot specify the database hints
%  \item Unicode (i.e. "Require Import Coq.Unicode.Utf8.") is not compatible with our plugin 
%  \item We only support automatic generation of partial recursor and the computation information for non-indexed inductive type 
%\end{itemize}

What seems to require more thought from a language-design perspective is the
restriction that recursion and induction (on extensible types) cannot be nested.
%and must be at the top level through the \lsti{FRecursion} and \lsti{FInduction}
%commands.
A possible solution is to make the plugin generate proof obligations for
nested pattern matches when the inductive types acquire new constructors.
%
Also interesting for future research is the support for automatically converting
terms to propositionally equally typed forms using generated propositional equalities.
The lack of this automation currently may cause inconveniences for
developments using intrinsically typed syntax.

%\begin{itemize}
%  \item We didn't considered the compatibility of universe polymorphism mechanism of Coq with our plugin except we know our plugin elaborate families into modules. Thus our implementation is supposed to be as compatible with universe polymorphism as Coq's Module.\YZ{Is this really a limitation at all? Do we have to mention it?}
%  \item The current recursion mechanism supported by "FRecursion" and "FInduction" only works at top level. Further research efforts are necessary to enable the appearance of recursion handlers at any location while facilitating extensibility of the handlers at said locations. Additionally, a suitable Coq tactic is required to accompany this functionality.
%  \item We don't support normalization of the open term using extensible inductive type. When applying recursor $Rec$ to open terms $t$ using extensible inductive type, the computation cannot be done by the Coq's kernel, thus the computation result of $Rec(t)$ is only propositionally equal (but not judgementally equal) to $Rec(t)$. It will complicate typechecking of the dependently typed terms. 
%\end{itemize}

