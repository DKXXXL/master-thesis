After considering the pros and cons, we decide to implement a Coq Plugin in which we can add new Vernacular command and translate each command into a bunch of Coq commands (the surface Vernacular command) instead of modify the code-base of Coq. 

Doing so, Despite of the possible difficulty of the future maintenance, this approach has various advantages: 1. it is the most easy and accessible way to prototype; 2. we have a clear definition of trusted-base -- the whole Coq; 3. it is easy to debug -- we just need to check the translated commands; 4. it can be well-incorporated with the VSCoq (and proof general I believe); 5. it is more accessible for the the interested audience who can easily adapt our plugin and give a try -- otherwise they have to download and re-build the whole customized Coq; 6. it is more stable because the surface syntax of Coq should be stable across different versions; 7. still, this plugin can capture the key ingredients of implementing family polymorphism inside Coq and act as a reference for guiding an appropriate implementation of Family Polymorphism inside all sorts of proof assistants.


Propositional Partial Recursor can prove injection and discrimination of the constructors.
Propositional Partial Recursor is good enough because it and its computational axiom can prove that the vanilla inductive type can "embed" into the extensible inductive type. 
(i.e. there will be an left inversion of that injection, which is also witnessing the fact that those "constructors" can really act as constructors) 
Thus 1. every future extension can support this partial recursion; 2. every type support this partial recursor with its computational axiom at least "support these constructors" (because of the embedding).