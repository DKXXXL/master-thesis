
We start with the formulation of the syntax, which is based on the formalization of a predicative \textit{Martin-Lof Type Theory} (MLTT) given by~\citet{coquand2018canonicity}. Thus we will omit a great deal of details about MLTT here but focus on the new facility we introduced. 

Though phrased in Latex, we are following~\citet{altkap2016} to use as
our meta-logic a type theoretical framework (like Agda), 
together with \textit{Quotient Inductive Inductive Type} (QIIT). Our work can be seen as following the practice ``type theory should eat itself''\cite{dybjer1995internal, chapman2009type}. We base our formulation of the syntax in this type theory instead of a more conventional and informal representation, becase (1) we hope our formulation can be precise on the mathematical object we are using (2) it is easier to validate a proof because proof checking is simply type checking in type theory (3) when using QIIT to formulate syntax, the induction principle used in reasoning about syntax is itself justified (4) we hope our formulation can be closer to a real theorem prover so that the reader can have an idea about how mechanized formalization can be carried out.

In our syntax, each judgement will actually be represented by a QIIT type, which are
mutually recursively defined together. Quotient is used to represent
judgemental(definitional) equality, thanks to which we can have a
concise representation. This way, for all four kinds of judgement we are
using, there are always equality among these data. Doing so, we can omit
some obvious equality judgement. Most of the reasoning can be formulated
using the \textit{algebra} of QIIT (mapping out function from this
quotient data-type). 

We also highlight some special feature in this style of formulation of dependent type theory: 1. this modern formulation of type theory does't have operational semantic but only equality 
%We will recover canonicity afterwards, which implies the computational ability of our theory. 
;
2. instead of using meta-level substitution, we use explicit substitution and de Bruijn indices in our intrinsic formulation. This is also called substitution calculus in the literature and is favoured due to its clear categorical semantic; 
3. we are dealing with quotient data, so we are actually manipulating equivalence classes of terms. 

We will write out the syntax of the whole type theory for completeness.






\subsection{Syntax}


Firstly, there are four standard judgement forms in MLTT,
well-typed contexts $\goodCtx{\Gamma}{l}$, well-typed types
$\goodType{\Gamma}{T}{l}$, well-typed terms $\goodTerm{\Gamma}{t}{T}$, and
well-typed substitutions $\goodSub{\Gamma}{\gamma}{\Delta}$. Our presentation
follows the style of \citep{kaposi2019gluing}, but for readability we
omit universe levels attached to judgements.
After these,
we will have four new kinds of judgement to model our linkage.

Because we formalize our language as a deep embedding in a type theory with QIIT types, every judgement is a (dependent) QIIT \textit{type} and all judgement derivations are well-typed QIIT terms (otherwise how can they be encoded in Agda?)
%\YZ{What's the distinction of judgments vs. sequents?}.\EDJreply{I am not sure about the correct terminology here. Here I want to say "even underivable judgement has to be well-typed". That's the main reason I spell out the fake Agda code there. I actually want the reader to have intrinsic-typed syntax in mind all the time (like read conventional math but acknowledge we are manipulating Agda)}\EDJreply{I rephrase that so there won't be confusion for the future readers. Please check}
Latex-wise, we spell out these \textit{well-formedness rule} of type
judgements and their Agda counterpart to make things more accessible. Of
course, a well-formed judgement is not necessarily derivable, just like
a well-typed QIIT type is not necessarily inhabited.

Due to our intrinsic setting, \textbf{we can and will omit a lot of presumptions of the typing rules}, for example we don't spell out that the rule "Type Universe" requires a well-typed context, even though it did implicitly -- because it is not possible to have "not-well-formed" context in our setting.

Among them, sealing judgement $\goodSeal{\Gamma}{\_}{\sigma}{A}$ is the most special one because it is actually just a short hand of term judgement $\goodTerm{\Gamma , \cL \sigma}{\_}{A[\pi_1]}$. 

\newcommand\mathboxtext[1]{
  \fcolorbox{black}{faint-gray}{\ensuremath{#1}}
}

\begin{figure}[H]
  \begin{minipage}[b]{0.3\linewidth}
      $$
      \Rule[name=Tm]
      {\goodCtx{\Gamma}{i} \quad \goodType{\Gamma}{T}{j}}
      {\goodTerm{\Gamma}{\_}{T}}
      $$
      $$
      \Rule[name=Tms]
      {\goodCtx{\Gamma}{i} \quad \goodCtx{\Delta}{i}}
      {\goodSub{\Gamma}{\_}{\Delta} \text{ is well-formed}}
      $$
      $$
      \Rule[name=Sig]
      {\goodCtx{\Gamma}{i} \quad n \in \nat}
      {\goodSig{\Gamma}{\_}{n} \text{ is well-formed}}
      $$
      $$
      \Rule[name=Seal]
      {\goodCtx{\Gamma}{i} \quad \goodSig{\Gamma}{\sigma}{n} 
      \quad \goodType{\Gamma}{A}{} }
      {\goodSeal{\Gamma}{\_}{\sigma}{A} \text{ is well-formed}}
      $$
  \end{minipage}
  \begin{minipage}[b]{0.6\linewidth}
    \begin{minted}[]{agda}
      data Con   : Set 
      data Ty    : Con → Set   
      data Tms   : Con → Con → Set 
      data Tm    : (Γ : Con) → Ty Γ  → Set 
      data Sig  : Con → ℕ → Set
      data WSig : Con → ℕ → Set 
      data Inh  : (Γ : Con) → Sig Γ n → Sig Γ m → Set
      Seal : (Γ : Con) → Sig Γ n → Sig Γ n → Set
    \end{minted}
  \end{minipage}
\begin{align*}
  &\Gamma~:~"Con"&&\mathboxtext{\goodCtx{\Gamma}{}} &&T~:~"Ty"~\Gamma&&\mathboxtext{\goodType{\Gamma}{T}{}}  &&\sigma~:~"Tms"~\Gamma~\Delta&&\mathboxtext{\goodSub{\Gamma}{\sigma}{\Delta}}  \\ & t~:~"Tm"~\Gamma~T&&\mathboxtext{\goodTerm{\Gamma}{t}{T}} 
  &&\sigma~:~"Sig"~\Gamma~n&&\mathboxtext{\goodSig{\Gamma}{\sigma}{n}}  &&\sigma~:~"WSig"~\Gamma~n&&\mathboxtext{\goodWSig{\Gamma}{\sigma}{n}} \\ &h~:~"Inh"~\Gamma~\sigma~\tau&&\mathboxtext{\goodInh{\Gamma}{h}{\sigma}{\tau}} && f~:~"Seal"~\Gamma~\sigma~A&&\mathboxtext{\goodSeal{\Gamma}{f}{\sigma}{A}}
\end{align*}
\caption{Well-formed-ness of Four New Judgements, and Agda-encoding-Judgements correspondence}
\end{figure}

% \begin{minted}[texcomments]{Coq}
%   Γ : Con  (* $\goodCtx{\Gamma}{}$ *) 
% \end{minted}
% T : Ty Γ (* $\goodType{\Gamma}{T}{}$ *)       σ : Tms Γ Δ {- $\goodSub{\Gamma}{\sigma}{\Delta}$ -}        t : Tm Γ T {- $\goodTerm{\Gamma}{t}{T}$ -}





Now we show case some of the typing rules, we will have dependent function type and dependent pair type. Dependent pair type is used to model Coq module, which is useful when we need to model the ``compilation'' from linkages to modules. We also have $\beta,\eta$-rule for both dependent function type and dependent pair type.

  \label{fig:rules:well-typed-ctx}
\judgebox{\goodCtx{\Gamma}{i}}
$$ 
\Rule[name=Empty Context]{}{\goodCtx{\cdot}{0}} 
\quad
\Rule[name=Context Extension]
{\goodCtx{\Gamma}{i} \quad \goodType{\Gamma}{A}{j}}
{\goodCtx{\Gamma, A}{i \cup j}}  
$$


\judgebox{ \goodType{\Gamma}{T}{i} }
$$
\Rule[name=Type Universe]
{}
{\goodType{\Gamma}{\cU_j}{j + 1}}
\quad 
\Rule[name=Boolean]
{}
{\goodType{\Gamma}{\cB}{0}}
\quad 
\Rule[name=Bottom]
{}
{\goodType{\Gamma}{\bot}{0}}
\quad 
\Rule[name=Function]
{\goodType{\Gamma}{A}{j} 
  \quad \goodType{\Gamma, A}{B}{k}}
{\goodType{\Gamma}{\Pi A B}{j \cup k}}
$$

$$
\quad 
\Rule[name=Func/DPair Subst]
{\goodSub{\Gamma}{\gamma}{\Delta}
\quad \goodType{\Delta}{A}{} 
\quad \goodType{\Delta, A}{B}{}
}
{
  \goodType{\Gamma}{(\Pi A B)[\gamma] \equiv \Pi A[\gamma] B[\gamma^\uparrow] }{j \cup k}
  \quad 
  \goodType{\Gamma}{(\Sigma A B)[\gamma] \equiv \Sigma A[\gamma] B[\gamma^\uparrow] }{j \cup k}
}
$$
%\YZ{What is the requirement on Delta?}
%\EDJreply{ Delta here is of type Con and thus Delta here should be a well-typed Context. This is another issue I want to talk about, in Agda these kinds of stuff is inferred (so the reader can infer them as well). For example, we know delta is at the place of context, so it is a context, but we don't need to specify that it is a ``well-typed context'' because every context is well-typed in this intrinsic-typing. What's more, I think adding this extra judgement here just cause significant code bloat (like what I complained to you before I know how to use implcit variable in agda. For example, TYPE SUBST, there will be two more judgements (with no more information there) ) (Another reason I prefer fake agda formulation) }\YZreply{I was thinking about 'Delta |- A'. I suppose it is also implicitly enforced via intrinsically typed syntax. Seems to me it is worth stating, however, or otherwise the rule appears to say that Delta can be any context as long as Gamma |- gamma : Delta.}\EDJreply{You are absolutely right! I think Delta |- A, even though inferrable, mentioning it is good, also some information are B is also ok, please check, resolve if ok, thanks!}
$$
\Rule[name=Type Subst]
{\goodType{\Delta}{T}{j} 
  \quad {\goodSub{\Gamma}{\gamma}{\Delta}}}
{\goodType{\Gamma}{T[\gamma]}{j}}
\quad
\Rule[name=Base Type Subst]
{\goodSub{\Gamma}{\gamma}{\Delta}}
{\goodType{\Gamma}{\cU_j[\gamma] \equiv \cU_j }{j + 1} \quad
  \goodType{\Gamma}{\cB[\gamma] \equiv \cB}{0} \quad 
  \goodType{\Gamma}{\bot[\gamma] \equiv \bot}{0}
}
$$
\judgebox{ \goodTerm{\Gamma}{t}{T} }
$$
\Rule[name=Univ-1]
{\goodType{\Gamma}{T}{j}}
{\goodTerm{\Gamma}{"c" \ T}{\cU_j}
}\quad
\Rule[name=Univ-2]
{\goodTerm{\Gamma}{T}{\cU_j}}
{\goodType{\Gamma}{El \ T}{j}}
\quad
\Rule
{}
{\goodTerm{\Gamma}{"tt", "ff"}{\cB}}
\quad 
\Rule[name=Term Subst]
{\goodTerm{\Delta}{t}{T}
  \quad {\goodSub{\Gamma}{\gamma}{\Delta}}}
{\goodTerm{\Gamma}{t[\gamma]}{T[\gamma]}}
$$
$$
\Rule[]
{\goodTerm{\Gamma, A}{t}{B}}
{\goodTerm{\Gamma}{\lambda t}{\Pi A B}}
\quad 
\Rule[]
{\goodTerm{\Gamma}{u}{A} 
\quad \goodTerm{\Gamma}{v}{B[(id, u)]}}
{\goodTerm{\Gamma}{(u,v)}{\Sigma A B}}
\quad 
\Rule[]
{\goodTerm{\Gamma}{t}{\Sigma A B}}
{\goodTerm{\Gamma}{"pjl" \ t}{A}
\quad \goodTerm{\Gamma}{"pjr" \  t}{B[(id, "pjl" \  t)]}
}
$$
$$
\Rule
{}
{\goodTerm{\Gamma}{(\lambda t)[\gamma] \equiv \lambda (t[\gamma^\uparrow])}{\Pi A B}
\quad \goodTerm{\Gamma}{(u,v)[\gamma] \equiv (u[\gamma],v[\gamma])}{\Sigma A B}
\quad \goodType{\Gamma}{El \ (T[\gamma]) \equiv (El \ T) [\gamma]}{}
}
$$

$$
\Rule[name=Base Type/Term Subst]
{\goodSub{\Gamma}{\gamma}{\Delta}}
{\goodType{\Gamma}{(El \ T)[\gamma] \equiv El \ (T[\gamma]) }{j} \quad
 \goodTerm{\Gamma}{"tt"[\gamma] \equiv "tt"}{\cB} \quad 
 \goodTerm{\Gamma}{"ff"[\gamma] \equiv "ff"}{\cB} 
}
$$
\judgebox{\goodSub{\Gamma}{\sigma}{\Delta}}
$$
\Rule[name=Empty Subst]
{}{\goodSub{\Gamma}{\epsilon}{\cdot}}
\quad
\Rule[name=Id Subst]
{}{\goodSub{\Gamma}{"id"}{\Gamma}}
\quad
\Rule[name=Subst Extension]
{\goodSub{\Gamma}{\sigma}{\Delta} \quad \goodTerm{\Gamma}{t}{A[\sigma]}}
{\goodSub{\Gamma}{(\sigma, t)}{(\Delta, A)}}
$$

$$
\Rule[name=Proj Subst-1]
{\goodSub{\Gamma}{\sigma}{(\Delta, A)}}
{\goodSub{\Gamma}{\pi_1 \sigma}{\Delta}}
\quad
\Rule[name=Proj Subst-2]
{\goodSub{\Gamma}{\sigma}{(\Delta, A)}}
{\goodSub{\Gamma}{\pi_2 \sigma}{A[\pi_1 \sigma]}}
\quad
\Rule[name=Proj-Ext]
{}
{\goodSub{\Gamma}{(\pi_1 \sigma, \pi_2 \sigma) \equiv \sigma}{\Delta}}
$$

We have shown case some exemplar rules for all four standard type
judgments, and now we focus on the newly introduced facility. 


We have two kinds of signatures, one for inductive type, the other for linkages.



\judgebox{\goodWSig{\Gamma}{\sigma}{n}}
$$
\Rule[name=Emp WSig]
{}
{\goodWSig{\Gamma}{w\cdot}{0}}
\quad
\Rule[name=WSig Add]
{\goodWSig{\Gamma}{s}{n}
  \quad \goodType{\Gamma}{A}{i}
  \quad \goodType{\Gamma, A}{B}{i}}
{\goodWSig{\Gamma}{w^+ \  s \  A \  B}{n+1}}
\quad
\Rule[name=Ind Univ]
{\goodWSig{\Gamma}{\sigma}{n}}
{\goodType{\Gamma}{\bW \sigma}{i}}
$$
%\YZ{Is there an introduction rule for terms of type 'bW sigma'?}
%\EDJreply{Yes there is. I omit it because it is a singleton type, please see my newly added explanation below.}

$$
\Rule[name=WSig Proj]
{\goodWSig{\Gamma}{s}{n} \quad j < n}
{\goodType{\Gamma}{\pi^j_1 s}{i} \quad \goodType{\Gamma, \pi^j_1 s}{\pi^j_2  s}{i}}
\quad
\Rule[name=Ind Sig]
{\goodWSig{\Delta}{\sigma}{n}
  \quad {\goodSub{\Gamma}{\gamma}{\Delta}}}
{\goodWSig{\Gamma}{\sigma[\gamma]}{n}
  \quad \goodType{\Gamma}{W (\sigma[\gamma]) \equiv (W \sigma)[\gamma]}{i}}
$$

$$
\Rule[name=Ind Type]
{\goodTerm{\Gamma}{T}{\bW \sigma}}
{\goodTerm{\Gamma}{"W" \ T}{\cU}}
\quad
\Rule[name=Ind Term]
{\goodTerm{\Gamma}{T}{\bW \sigma}
  \quad \goodTerm{\Gamma}{a}{\pi^j_1\sigma}
  \quad \goodTerm{\Gamma, \pi^j_1\sigma}{b}{\pi^j_2\sigma}}
{\goodTerm{\Gamma}{"Wsup" \ T \ a \ b}{El\ ("W" \ T)} }
$$


Recall in W-type \citep{martin1982constructive}, a pair of type $x : A \vdash B(x)$ decides a W-type. To simulate extensible inductive type, we have to equip our inductive type with multiple constructors, and for simplicity, each constructor has a fixed form \mintinline{agda}{cstrᵢ : (x : Aᵢ) → (Bᵢ x → W) → W}. Thus we use a list of pairs of such $A_i, B_i$ as the signature for one inductive type, and given an inductive type signature, we use \ruleref{WSig Proj} to extract and corresponding $A_i, B_i$.

Of course we need to put an eye on the substitution law for each piece of the syntax. Here $\bW \sigma$ is a large singleton type \cite{stone2000}, and it only has one "element", the inductive type itself, inside. We formulate in this cumbersome way because we want to have inductive type to be a field element and exposing the constructors at the type level for later \textit{pattern-matching coverage checking} (i.e. if an inductive type as field element has type $\cU$, then no information about the inductive type is exposed).\YZ{What is the distinction of 'W T' vs 'El (W T)'?}\EDJreply{(W T) is a term inside the type universe, El (W T) is really a type. They are basically the same thing but (W T) cannot be at the type position in a judgement, i.e. k |- q : (W T) is not allowed, we only have k |- q : El (W T). This is kind of standard notation (at least in both Sterling and Kaposi) so I will avoid explaining in the main text if you think ok?}\YZreply{I think explanation is warranted; readers unfamiliar with that line of work may find it confusing. BTW, is the elimination rule for "El (W T)" omitted?}\EDJreply{Please check the following paragraph for the explanation for El, and also Univ-1 is the elimination rule for "El T" in general. I just didn't write out the two equations c (El T) = T, El (c T) = T}. We follow the style of \textit{type universe à la Tarski} and distinguish types from their \textit{codes or names}, and thus we use $El\ T$ as the type given type name $T : \cU$ -- in these cases, $T$ is not allowed to locate at type position (i.e. at the right-most position of a term judgement) but $El \ T$ is.  

In our meta-theory, we only formulate non-dependent eliminator, i.e. the \textit{recursor}. Each recursor is constructed by a linkage contained with functions/handlers/pattern-matching cases dealing with each constructor. Thus code-reuse for recursor is delegated by the inheritance of linkage -- to reuse one particular pattern-matching case, we just inherit the function/handler. This greatly simplifies the meta\-theory development. 


\judgebox{\goodSig{\Gamma}{\sigma}{n} }
$$
\Rule[name=Lnkg Type/Compile]
{\goodSig{\Gamma}{\sigma}{n}}
{\goodType{\Gamma}{\cL \sigma}{i}
\quad \goodType{\Gamma}{\cC \sigma}{}}
\quad
\Rule[name=Sig/Lnkg Subst]
{\goodSig{\Delta}{\sigma}{n}
  \quad {\goodSub{\Gamma}{\gamma}{\Delta}}}
{\goodSig{\Gamma}{\sigma[\gamma]}{n}
  \quad \goodType{\Gamma}{\cL (\sigma[\gamma]) \equiv (\cL \sigma)[\gamma]}{i}}
$$

$$
\Rule[name=Ept Sig]
{}
{\goodSig{\Gamma}{\nu\cdot}{0}}
\quad
\Rule[name=Sig Add]
{\goodSig{\Gamma}{\sigma}{n} 
 \quad \goodSeal{\Gamma}{f}{\sigma}{A}
 \quad \goodType{\Gamma, A}{T}{i}}
{\goodSig{\Gamma}{(\nu^+ \ \sigma \ \{f\} \ T)}{n+1}}
$$

$$ 
\Rule[name=Sig Proj]
{\goodSig{\Gamma}{\sigma}{n+1}}
{\goodSig{\Gamma}{p_1\nu \ \sigma}{n}
\\ \goodType{\Gamma}{p_1\nu' \ \sigma}{}
\\ \goodSeal{\Gamma}{p_f\nu \  \sigma}{p_1\nu \  \sigma}{p_1\nu' \sigma}
\\ \goodType{\Gamma, p_1\nu' \ \sigma}{p_2\nu \ \sigma}{}
}
$$

$$
\Rule[name=Ept Lnkg]
{}
{\goodTerm{\Gamma}{\mu\cdot}{\cL \nu\cdot}}
\quad
\Rule[name=Lnkg Add]
{ \goodTerm{\Gamma}{o}{\cL \sigma} 
\quad  \goodSeal{\Gamma}{f}{\sigma}{A} 
 \quad \goodTerm{\Gamma, A}{t}{T}
}
{\goodTerm{\Gamma}{(\mu^+ \ o \ \{f\} \ t)}{\cL(\nu^+ \ \sigma \ \{f\} \ T)}}
$$

$$
\Rule[name=Lnkg Proj]
{\goodTerm{\Gamma}{o}{\cL\sigma}}
{\goodTerm{\Gamma}{p_1\mu \ o}{\cL (p_1\nu \ \sigma)}
\\ \goodTerm{\Gamma, p_1\nu' \ \sigma}{p_2\mu \ o}{p_2\nu \ \sigma}
}
\quad 
\Rule[name=Lnkg Compile]
{ \goodTerm{\Gamma}{o}{\cL \sigma} 
}
{
  \goodTerm{\Gamma}{\cCt o}{\cC \sigma}
}
$$

$$
\Rule[name=Compile]
{}
{\goodType{\Gamma}{\cC \nu\cdot \equiv \top}{} 
\\ \goodTerm{\Gamma}{ \cCt \mu\cdot \equiv ()}{\cL \nu\cdot}
\\
\goodType{\Gamma}{\cC (\nu^+ \ \sigma \ \{f\} \ T) \equiv 
    \Sigma (\cC \sigma) T[sf \ f]}{}
\\\\ \goodTerm{\Gamma}{\cCt \ (\mu^+ \ o \ \{f\}\ t) \equiv ((\cCt \ o), t[sf \ f][(id, \cCt \ o)]) }{}
}
$$
% $$
% \Rule
% { \goodSeal{\Gamma}{f}{\sigma}{A}
% \\ \goodType{\Gamma, A}{T}{}
% \\ \goodTerm{\Gamma}{t}{T}
% }{}
% $$

$\goodSig{\Gamma}{\sigma}{n}$ means $\sigma$ is a well-typed signature for linkages in the context $\Gamma$ with $n$ fields. We omit the naming/label information for fields in the meta-theory. We can use $\cL$ to retrieve the corresponding types of linkage of a given signature, and $\cC$ to get the compiled linkage type (an existential type). 


The signature (for the linkage) and the linkage rules are quite similar the signature for inductive type.  They are both constructed from the empty stuff, and respect substitution lemma for sure. We have \ruleref{Sig Proj}, \ruleref{Lnkg Proj} projection for each componenet due to harmony \citep{pfenning2009lecture} thus we have local soundness and local completeness ($\beta,\eta$ rules). We omit them and the substitution laws for either field addition and projection. For $\mu^+,\nu^+$ rules, we will omit $\{f\}$ if possible\footnote{Interestingly, this omission can be supported by Agda using Implicit Arguments}.


Intuitively, both of the $\mu^+, \nu^+$ rules should be motivated by the implementation of our linkage, so an obvious question is what is this sealing judgement for?

\judgebox{[\goodSeal{\Gamma}{s}{\sigma}{A}] 
= [\goodTerm{\Gamma, \cC\sigma}{s}{A[\pi_1]}] }
$$
\Rule[name=Seal Subst]
{\goodSeal{\Delta}{s}{\sigma}{A}
  \quad {\goodSub{\Gamma}{\gamma}{\Delta}}}
{\goodSeal{\Gamma}{s[\gamma]}{\sigma[\gamma]}{A[\gamma]}}
=
\frac
{\goodTerm{\Delta, \cC\sigma}{s}{A[\pi_1]}
  \quad  \goodSub{\Gamma}{\gamma}{\Delta}  }
{\goodTerm{\Gamma,\cC\sigma[\gamma]}{s[\gamma^\uparrow]}{A[\pi_1][\gamma^\uparrow]\equiv A[\gamma][\pi_1]}}
$$
$$
\Rule[name=Seal-Id]
{}
{\goodSeal{\Gamma}{id_s}{\sigma}{\cC \sigma}}
= \goodTerm{\Gamma, \cC\sigma}{\pi_2}{\cC\sigma[\pi_1]}
$$
\begin{figure}[H]

\centering
\captionsetup{justification=centering}

\caption{Seal Judgement, and its Agda Representation \\ (or What do we mean when we say one judgement is a short hand of the other)}

\begin{minted}[]{agda}
Seal : (Γ : Con) → (σ : Sig Γ n) → (A : Ty Γ) → Set 
Seal Γ A τ = Tm (Γ, 𝓛σ) A[p₁]

-[-]S : (f : Seal Γ σ A) → (τ : Tms Θ Γ) → Seal Θ σ[τ] A[τ]
f[τ]S = t[τ ↑ A]

idₛ : Seal Γ σ 𝓒σ
idₛ = π₁
\end{minted}

\end{figure}




As we said earlier, $\goodSeal{\Gamma}{\_}{\sigma}{A}$ is actually just a short hand of term judgement $\goodTerm{\Gamma , \cC \sigma}{\_}{A[\pi_1]}$. The substitution law should understand as a lemma, and easily derivable by using $\gamma^\uparrow$. 

The motivation for sealing judgement is acting like a syntactic sugar helpful for inheritance -- due to the presence of inductive type in the context, not everything is inheritable. For example, $ X_1 : \bW \sigma_1 \vdash t : T$ in the parent family will generally not be able to be inherited to the children once we override the inductive type. In other words, there is no way ``transplanting'' $t$ into the context $X_2 : \bW \sigma_2$. This makes all the fields after this inductive type non-inheritable -- which shouldn't be the case because a lot of fields might not be directly relying on this concrete $\bW \sigma_1$. 

Sealing is used to exploit this fact -- with sealing, we can add well-typed terms $X_1 : \cU \vdash t : T$, into the parent family, in the context of $ X_1 : \bW \sigma_1$. Later when we want to inherit $t$, because $t$ only consider $X_1$ as an arbitrary type, inheritance is plausible.

Of course, any fields using "Wsup" and "Wrec" directly relies on the $\bW(\sigma)$ and thus non-inheritable -- in these cases, "Wsup" can be understood as concrete constructors and "Wrec" are \textit{exhaustiveness checking}. These fields together with inductive types are the one sealing responsible for hiding. After compilation of linkage type into existential type, we can easily hide the \textit{concrete} definition of the inductive type while exposing all the fields using "Wsup" and "Wrec" like an abstract interface. Apparently, future fields relies on this abstract interface will have no problem on inheritance. \YZ{Does coverage checking mean exhaustiveness checking?}\EDJreply{Yes. The one make sure every case is handled in pattern-matching. Do you want me to replace the terminology using exhaustiveness checking? }\YZreply{Yes, I think exhaustiveness is standard parlance.}
Of course, since "Wsup" and "Wrec" directly rely on the concrete inductive definition, we will have to override these fields whenever inductive type is overridden. For "Wsup" we only need to override each constructor and for "Wrec" it is just doing exhaustiveness checking inside each children family. The recursor handlers however, can be safely inherited.
\YZ{I suppose the terminology 'sealing' is a reference to ML modules? In
ML, sealing has the connotation of hiding things behind an existential
type. So using the same word may cause the reader to have false
presumptions about what is going on here.}\EDJreply{You are right. Our "Sealing" is not hiding a particular field into an existential type, but our seal is make several fields together into an "existential type". Basically it hides the concrete definition of inductive type so that future fields doesn't have to be defined upon that.}\EDJreply{I will resolve this comment once I add the throughout example and you find it clear -- because I haven't changed any text}

\YZ{Can we have a more concrete example of how the judgment is useful?}\EDJreply{Working on it, don't resolve}

And these are also the cases we don't need to seal anything (i.e. doing exhaustiveness checking when using recursor), and thus the "Seal-Id" is used. For notation clearance, we will sometime use $sf'$ to emphasize that we consider that seal $s$ as a term $(sf' \ s)$. 

With the definition of sealing judgement in mind, we can see \ruleref{Sig Add} and \ruleref{Lnkg Add} work exactly as the implementation. In the implementation, we will have "self__" all around when defining new fields in our plugin, and this "self__" is exactly the $A$ in the context of the two fields -- especially when we don't do any sealing, this $A = \cC \sigma$, which is exactly how the implementation expose "self__" as a module parameter. 

Next we talk about inheritance judgement. Note that, there will be a lot of $\sigma_1, A_1$ and $\sigma_2, A_2$, and in these cases, $A_1$ is the result of the sealing of $\sigma_1$, i.e. we assume we have a term $\goodSeal{\Gamma}{f_1}{\sigma_1}{A_1}$, $\goodSeal{\Gamma}{f_2}{\sigma_2}{A_2}$ in the assumption of judgements when needed.

\judgebox{\goodInh{\Gamma}{h}{\sigma}{\tau}}

$$
\Rule[name=Inh-Id]
{}
{\goodInh{\Gamma}{"inhid"}{\sigma}{\sigma}}
\quad
\Rule[name=Inh-Override]
{
\goodInh{\Gamma}{h}{\sigma_1}{\sigma_2}  
\quad \goodType{\Gamma, A_1}{T_1}{}
\quad \goodType{\Gamma, A_2}{T_2}{}
  \quad \goodTerm{\Gamma, \cL \sigma_2'}{t}{T_2}}
{\goodInh{\Gamma}{"inhov" \ h \ t}{(\nu^+ \  \sigma_1 \  T_1)}{(\nu^+ \  \sigma_2\  T_2)}}
$$

$$
\Rule[name=Inh-Ext]
{\goodInh{\Gamma}{h}{\sigma_1}{\sigma_2}
  \quad \goodTerm{\Gamma, \cL \sigma_2'}{t}{T}}
{\goodInh{\Gamma}{"inhext" \ h \ t}{\sigma_1}{(\nu^+ \  \sigma_2\  T)}}
\quad
\Rule[name=Inh-Inh]
{\goodInh{\Gamma}{h}{\sigma_1}{\sigma_2}
\quad \goodType{\Gamma, A_1}{T}{}
\quad \goodTerm{\Gamma, A_2}{\uparrow^s}{A_1[\pi_1]}
}
{\goodInh{\Gamma}{"inhinh" \ h}{(\nu^+ \  \sigma_1 \  T)}{(\nu^+ \  \sigma_2 \  T[(\pi_1, \uparrow^s)])}}
$$
$$
\Rule[name=Inh+Inh]
{\goodInh{\Gamma}{h}{\sigma_1}{\sigma_2}
\quad \goodTerm{\Gamma, A_2}{\uparrow^s}{A_1[\pi_1]}
\quad 
\goodInh{\Gamma, A_2}{i}{\tau_1[(\pi_1, \uparrow^s)]}{\tau_2}}
{\goodInh{\Gamma}{"inhnest" \ h \ i}{(\nu^+ \  \sigma_1\  \cL\tau_1)}{(\nu^+ \  \sigma_2\  \cL\tau_2)}}
\quad
$$

$$
\Rule[name=Inh]
{\goodInh{\Gamma}{h}{\sigma_1}{\sigma_2}
\quad \goodTerm{\Gamma}{l}{\cL \sigma_1}
}
{\goodTerm{\Gamma}{("inh" \ h \ l)}{\cL \sigma_2}} 
\quad 
\Rule[name=Inh-Inh-beta]
{\goodInh{\Gamma}{h}{\sigma_1}{\sigma_2}
  \quad \goodTerm{\Gamma}{m}{\cL \sigma_1}
  \quad \goodTerm{\Gamma, A_1}{t}{T}
}
{\goodTerm{\Gamma}{"inh" ("inhinh" \ h) (\mu^+ \ m \ t) \equiv \mu^+ \ ("inh" \ h \ m) \ t[(\pi_1, \uparrow^s)]}{}} 
$$

Finally we introduce the inheritance judgement. Inheritance are judgement so naturally second class citizen data (i.e. not something function can return and cannot be assigned to variables), and it looks very much alike a function that transform a given linkage. 

The syntax of the inheritance judgement is very close to how Family is defined in the surface syntax -- a Family with no parent is an inheritance with empty input; an extended Family is exactly an inheritance -- because the parent of an extended Family can be overridden thus inheritance should only define upon the ``interface'' instead of a concrete family. However, since our current formulation doesn't support further-binding (i.e. we cannot do inheritance inside a family), our formulation still have some distance to the real family polymorphism. 

"inhid" and "inhext" are simple as expected -- the former corresponds to empty inheritance, and the latter gives the programmer the ability to add new field. 

"inhinh" is special because it requires user to provide a proof of ``upcasting'' from the context of the children to that of the parent. In other words, if I want to inherit a specific field from the parent, I have to convince the type checker that my current context can ``accommodate'' the inherited field. 

Override "inhov" is even more special because in mundane OO programming language, we usually require that the overriding term has the same type/interface as the overriden one.  Here we don't since the very reason we want the same type/interface is that we hope other inherited fields can use this overridden/late-binding field without breaking abstraction, and this reason is already managed by the assumption of "inhinh", the upcasting proof $\uparrow^s$. Thus "inhov" is quite like "inhext", where overriding is just throwing away one parent field. This difference also show one of the distinct feature of our vanilla family inheritance compared to the mainstream OO inheritance -- family inheritance don't need to introduce the concept of ``subclassing'' between families and thus no bounded polymorphism yet\footnote{Of course once trait and interface comes in, we can have bounded polymorphism back}, and code reuse is mainly achieved by late binding. Without ``subclassing'', there is no way of confusing inheritance and subtyping.

Finally we need to introduce nested inheritance "inhnest" for dealing with nested families. The rule is mostly direct and we need "upcasting" again when dealing with nested family in different but inheriting context.  We also show how to use an inheritance judgement to derive children family and how the beta rule is defined. We omit most of the other substitution laws here.



\subsection{Exemplar Usage of Meta Theory}
We use an example to better demonstrate different components of the
syntax of our formulation. Our example will be the meta-theory encoding
of the following pseudocode.

We provide one example in its surface syntax as pseudocode, and their
meta-theory encoding. This example is computing the predecessor of a
mundane natural number, but we split the inductive definition into two
parts and making "O" and "S" standing alone.
"pred" will map $S~n \mapsto (n, S~n)$ and $O \mapsto (O, O)$, and thus a predecessor.




\begin{figure}[H]\label{fig:example-pseudocode}
\begin{minipage}[t]{0.5\linewidth}
\begin{minted}[escapeinside=@@]{Coq}
Family A.
 FInductive N := O : N. 
 (* Osig, sig₁, obj₁, sig₂,
    O₁, obj₂, sig₂', sl₁ *)
 Family handler.
   O :self[A].N × self[A].N 
     := (self[A].O, self[A].O). 


 EndFamily. (* pN₁, handler₁, sig₃ *)
 FRecursor pred about self[A].N 
  motive (fun _ => self[A].N × self[A].N)
  using handler.
 (* pred
  : self[A].N → self[A].N × self[A].N *)
  (* predT, pred₁, sig₄, sig₄', sl₄ *) 
 k := self[A].pred self[A].O 
(* pN₂, sig₅ *)
EndFamily.
\end{minted}
  \end{minipage}
  \begin{minipage}[t]{0.45\linewidth}
\begin{minted}[escapeinside=@@]{Coq}
Family A₂ extends A.
 Extend FInductive N := S : N → N.
 (* Nsig, sig₁₁, inh₁, O₂, sig₂₁, inh₂, 
    sig₂₂, sig₂₁', sig₂₂', sl₂₂, inh₃ *)
 Extend Family handler.
   Inherit O. (* inhᵢₙ₀, pN₃ *)
   S : self[A].N × self[A].N
       → self[A].N × self[A].N 
     := fun (x,y) => (y, self[A₂].S y).
 EndFamily. (* inhᵢₙ, sig₃₁, inh₄*)
 Inherit pred.
 


  (* sig₄₁, inh₅, sl₄₁ *)
 Inherits k. 
 (* inh₆ *)
EndFamily.
    \end{minted}
  \end{minipage}
  \caption{Demonstrating Pseudo-Code}
\end{figure}

\begin{figure}
  \begin{minipage}{\linewidth}
    \begin{minted}{Coq}
      Inductive N :=
       | O : 1 -> (0 -> N) -> N 
       | S : 1 -> (1 -> N) -> N
      (* For example, Number 1 is encoded using
         1 = S () (fun _ -> O () (fun x -> elim-⊥ x)) *)
    \end{minted}
  \end{minipage}

  \begin{minipage}[t]{0.4\linewidth}
  \small
\begin{align*}
  "Osig" &\coloneqq \goodWSig{\cdot}{w^+\ w\cdot\ \top\ \bot}{1} \\
  "sig"_1 &\coloneqq \goodSig{\cdot}{\nu^+ \ \nu\cdot \ {id_s} \ (\bW \ "Osig"[\pi_1])}{1}  \\
  "obj"_1 &\coloneqq \goodTerm{\cdot}{\mu^+ \ \mu\cdot \ {id_s} \ *}{\cL"sig"_1}\\
  \cC"sig"_1 &\coloneqq \goodType{\cdot}{\Sigma \ \top \ (\bW \ "Osig"[\pi_1]) }{} \\
  "sig"_2 &\coloneqq  \goodSig{\cdot}{\nu^+ \ "sig"_1 \ {id_s} \ El ("W" ("pjr" \ \pi_2))}{2} \\ 
  "O"_1 &\coloneqq {"Wsup" \ ("pjr" \pi_2) \ () \ ("elim-"\bot \ \pi_2)}  \\ 
  & \text{ and thus }  \goodTerm{\cdot, \cC"sig"_1}{"O"_1}{El ("W" ("pjr" \pi_2))} \\
  "obj"_2 &\coloneqq \goodTerm{\cdot}{\mu^+ \ "obj"_1 \ "O"_1}{\cL"sig"_2} \\
  "sig"_2' &\coloneqq \goodType{\cdot}{\Sigma (\Sigma \top \cU)El ("pjr" \pi_2)}{} \\ 
  \cC"sig"_2 &\coloneqq \goodType{\cdot}{\Sigma (\Sigma \top (\bW \ "Osig"[\pi_1]))El ("W"("pjr" \pi_2)) }{} \\ 
  "sl"_1 &\coloneqq ((*, "W" ("pjr" ("pjl" \pi_2)) ), "pjr" \pi_2) \\ 
  & \text{ and thus }  \goodSeal{\cdot}{"sl"_1}{"sig"_2}{"sig"_2'} \\
  "pN"_1 &\coloneqq \goodType{\cdot, "sig"_2'}{El ("pjr" ("pjl" \pi_2)) * El ("pjr" ("pjl" \pi_2))}{} \\
  "handler"_1 &\coloneqq \goodTerm{\cdot, "sig"_2'}{\_}{\cL(\nu^+ \ \nu\cdot \ \{id_s\} "pN"_1)} \\
  "sig"_3 &\coloneqq \goodSig{\cdot}{\nu^+ "sig"_2 \{"sl"_1\} \ \cL(\nu^+ \ \nu\cdot \ \{id_s\} "pN"_1)}{3} \\ 
  "obj"_3 &\coloneqq \goodTerm{\cdot}{\mu^+~"obj"_2~"handler"_1}{\cL"sig"_3}\\
  "predT" &\coloneqq \Pi \ El ("W"("pjr" ("pjl"^2 \pi_2)))\ El ("W" ("pjr" ("pjl"^2  \pi_2)))[\pi_1] \\
  "pred"_1 &\coloneqq \goodTerm{\cdot, \cC "sig"_3}{\lambda ("Wrec" \_)}{"predT"} \\ 
  "sig"_4 &\coloneqq \nu^+ \ "sig"_3 \ \{id_s\} \ "predT" \\ 
  "obj"_4 &\coloneqq \goodTerm{\cdot}{\mu^+~"obj"_3~"pred"_1}{\cL"sig"_4}\\
  "sl"_4 &\coloneqq \goodSeal{\cdot}{\_}{"sig"_4}{"sig"_4'} \\
  "pN"_2 &\coloneqq \goodType{\cdot, "sig"_4'}{El ("pjr" ("pjl" \pi_2)) * El ("pjr" ("pjl" \pi_2))}{}
\end{align*}
\end{minipage}%
\begin{minipage}[t]{0.4\linewidth}
  \small
\begin{align*}
  "sig"_5 &\coloneqq \nu^+ \ "sig"_4 \ \{"sl"_4\} \  "pN"_2 \\ 
  "obj"_5 &: \goodTerm{\cdot}{\mu^+~"obj"_4~\_}{\cL"sig"_4}\\
  \\ 
  \text{Here we}&\text{ start to construct Family } "A"_2 \\ 
  "Nsig" &\coloneqq \goodWSig{\cdot}{w^+\ "Osig"\ \top \ \top}{2} \\ 
  "sig"_{11} &\coloneqq \goodSig{\cdot}{\nu^+ \ \nu\cdot  \ (\bW \ "Nsig"[\pi_1])}{1} \\
  "inh"_1 &\coloneqq \goodInh{\cdot}{"inhov" \  "inhid" \ *}{"sig"_1}{"sig"_{11}}\\
  "O"_2 &\coloneqq {"Wsup" \ ("pjr" \pi_2) \ () \ ("elim-"\bot \ \pi_2)}  \\ 
  & \text{ but }  \goodTerm{\cdot, \cC"sig"_{11}}{"O"_2}{El ("W" ("pjr" \pi_2))} \\
  "sig"_{21} &\coloneqq \nu^+ \ "sig"_{11} \ \{id_s\} (El (W (pjr p₂))) \\ 
  "inh"_2 &\coloneqq \goodInh{\cdot}{"inhov" \ "inh"_1 \ "O"_2}{"sig"_2}{"sig"_{21}}\\
  "S"_T &\coloneqq \Pi \ El ("W"("pjr"  ("pjl" \pi_2))) \ El ("W" ("pjr" ("pjl"  \pi_2)))[\pi_1] \\
  "sig"_{22} &\coloneqq \nu^+ \ "sig"_{21} \ \{id_s\} \ "S"_T \\ 
  "sig"_{21}' &\coloneqq \goodType{\cdot}{\Sigma (\Sigma \top \cU) (El ("pjr" \pi_2))}{} \\
  "sig"_{22}' &\coloneqq \Sigma "sig"_{21}' (\Pi \ El ("pjr"  ("pjl" \pi_2)) \  ...) \\
  "sl"_{22} &\coloneqq \goodSeal{\cdot}{\_}{"sig"_{22}}{"sig"_{22}'} \\ 
  "inh"_3 &\coloneqq \goodInh{\cdot}{\_}{"sig"_2}{"sig"_{22}} \\ 
  \uparrow^s &\coloneqq \goodTerm{\cdot, "sig"_{22}'}{\_}{"sig"_2'[\pi_1]} \\ 
  "pN"_3 &\coloneqq "pN"_1[(\pi_1, \uparrow^s)] \\ 
\end{align*}
  \end{minipage}

  \begin{minipage}{0.8\linewidth}
    \small
    \centering
    \begin{align*}
      "inh"_{in0} &\coloneqq \goodInh{\cdot, "sig"_{22}'}{"inhid"}{(\nu^+ \ \nu\cdot \ ("pN"_1[\pi_1]))[(\pi_1, \uparrow^s)]}{(\nu^+ \cdot ("pN"_3[\pi_1]))} \\
      "inh"_{in} &\coloneqq \goodInh{\cdot , "sig"_{22}'}{\_}{(\nu^+ \ \nu\cdot \ "pN"_1[\pi_1])[(\pi_1, \uparrow^s)]}{(\nu^+ \ \cdot \ ("pN"_3[\pi_1]) \ \{id_s\} (\Pi "pN"_3 ("pN"_3[\pi_1])) )[\pi_1]}\\
      "sig"_{31} &\coloneqq  \nu^+ \ "sig"_{22} \ {"sl"_{22}}\  \cL(\nu^+ \cdot ("pN"_3[\pi_1])  (\Pi "pN"_3 ("pN"_3[\pi_1]))[\pi_1]) \\ 
      "inh"_4 &\coloneqq \goodInh{\cdot}{"inhnest" \ "inh"_3 \uparrow^s "inh"_{in}}{"sig"_3}{"sig"_{31}} \\
      "sig"_{41} &\coloneqq {\nu^+ \ "sig"_{31} \ (\Pi (El ("W" ("pjr" ("pjl"^4 \pi_2)))) (El ("W" ("pjr" ("pjl"^4 \pi_2))))[\pi_1])} \\ 
      "inh"_5 &\coloneqq \goodInh{\cdot}{inhov \ \_}{"sig"_4}{"sig"_{41}} \\ 
      "sl"_{41} &\coloneqq \goodSeal{\cdot}{\_}{"sig"_{41}}{"sig"_{41}'} \\ 
      \uparrow^s_2 &\coloneqq \goodTerm{\cdot, "sig"_{41}}{\_}{"sig"_4'[\pi_1]} \\ 
      "inh"_6 &\coloneqq \goodInh{\cdot}{"inhinh" \ "inh"_5 \ \uparrow^s_2}{"sig"_5}{(\nu^+ \ "sig"_{41} \{"sl"_{41}\} \ "pN"_2[(\pi_1, \uparrow^s_2)])}
    \end{align*}  
  \end{minipage}
  \caption{Detailed Construction}\label{fig:example-construction}
\end{figure}





Here we explain how  our detailed construction in
\cref{fig:example-construction}. We first elaborate that
how our inductive type "N" is written in the specific format (the style
of W-type). We use 1, 0 here to indicate unit type and bottom type. 

Next, we alias each part of the derivation with a name. Every derivation
here is well-typed term. We also annotate in the original pseudocode the
name of the related derivation. 

We start with the construction of "Osig" which is the signature of a
standalone "O" constructor. With this we will have $"sig"_1, "obj"_1,
"sig"_2, "obj"_2$ as the signature and object of the linkage with that
inductive type, and the signature and the linkage once we export the
concrete constructor $"O"_1$. 

Now sealing comes in to seal these concrete component into an abstract
interface $"sig"_2'$ for decoupling the following fields. To make sure
the sealing exists, we take a look at the resulting $\cC "sig"_2$, and
doing abstraction on it to get $"sl"_1$. 

Now we construct the recursor, with the help of the handler module
$"handler"_1$, with which we can construct the real recursor $"pred"_1$.
Both detailed constructions are omitted. Then we again abstract the
concrete recursor $"pred"_1$ away using $"sl"_4$ so that the field $"k"$
can apply it without resorting to the concrete recursor. Finally we
have $"sig"_5$ as final signature.\YZ{
  Maybe it is a good place to point out what El(pjr (pjl pi_2)) in pN_2 stands for
  and what El(W(pjr(pjl^2 pi_2))) in predT stands for.
}

Now we construct the inheritance judgement for $"A"_2$. Still, we start
with signature for inductive type, "Nsig". But to ``extend'' the
inductive type, we actually \textit{override} the inductive type with
the enriched inductive type in $"inh"_1$. Because inductive type is
overridden, inheritance on the constructors are not pausible and thus we
need to override the old $"O"_1$ constructor using $"O"_2$ and extend
with new constructor for "S", resulting $"inh"_3$. These newly
overridden constructors are sealed by $"sl"_{22}$. However, look
closely, our sealed abstract interface in the children still has more
fields than the sealed abstract interface in the parent. We still need
$\uparrow^s$ to indicate the ``compatibility'' between two abstract
interfaces so that the inheritance from the parents can work. It can be
understood as ``upcasting'', but this ``upcasting'' is used to upcast
the context of the children. We immediately see its usage in $"pN"_3$
and $"inh"_{in0}$. Upon this, we construct the complete
inner-inheritance $"inh"_{in}$ that is responsible for extending
"handler". Then we continue the construction for $"inh"_4$ that ``nest''
the inner-inheritance into the whole $"inh"_3$, and we re-do
exhaustiveness checking when constructing $"inh"_5$ using $"inhov"$ and
"Wrec". Since it is another recursor requiring concrete inductive
definition, we do another abstraction on it using $"sl"_41$. After that,
$\uparrow^s_2$ again witnesses the compatibility of the abstract
interface, and the two are both used for constructing $"inh"_6$, which
is responsible for inheriting field "k". 

The resulting constructed families are
$\goodTerm{\cdot}{"obj"_5}{\cL"sig"_5}$ as "Family A" and \\
$\goodTerm{\cdot}{("inh"~"inh"_6~"obj"_5)}{\cL((\nu^+ \ "sig"_{41}
\{"sl"_{41}\} \ "pN"_2[(\pi_1, \uparrow^s_2)]))}$ as "Family" $"A"_2$.

Finally we comment on how a family is encoded in our metatheory.
A family object will have a signature that exposes the type
of fields and the details of inductive types. This exposure can make
sure inheritance can override and inherit fields correctly. Sealing is
only hiding information \textbf{inside one family}, i.e., hiding
information of the former fields for the later fields, and thus the
later fields can be defined relying on only the abstract interface of
the former fields.
The signature will expose the sealing itself and the inductive type.
(See $"sig"_5$ has all the sealing information and inductive
definition). A former sealing might not be related to any later sealing,
for example, $"sig"_3$ is constructed using $"sl"_1$ but $"sig"_4$ is
using trivial sealing $id_s$.


\subsection{Implementation Formalized as Syntactical Translation}

The implementation described in \cref{sec:coqimpl} has hinted the fact that
linkage can be translated into dependent tuple type.
The described compilation strategy can be considered as
a \textit{type preserving syntactic translation}, a function mapping from our syntactic
model(typed Abstract Syntax Tree(AST)) to a ``subpart''\footnote{Not exactly subpart---the only
difference lies in "Wrec". In our metatheory, it is constructed
using linkage; in ``subpart'', it will be using $\Sigma$-types (translation
of the linkage).} of our syntactic model(typed AST) \textit{without
linkages}.

This ``subpart'' is the standard MLTT with an
(unconventional) formulation of W-types,
and recursors that use $\Sigma$-types to aggregate handlers.

Basically, we can compile away (the derivation of) judgement $\goodSig{\Gamma}{v}{n}$,
the type $\cL$ and $\cC$, and related terms, by transforming them into
corresponding $\Sigma$-types.

 

% \newcommand{\denotesT}[1]{{{\llbracket {#1} \rrbracket}_T}}
% \newcommand{\Sigr}[2]{{ "Sig"^r~{#1}~{#2} }}

We use $\denotesT{}$ as the translation. This translation behaves like identity
\YZ{What does "invariant" mean?}\EDJreply{behaves like identity. Sorry I should have said "subpart is invariant under this translation" instead}
under the subpart. Since we are in an intrinsic setting, everything here
is well-typed (well-formed)including this translation. Thus
$\denotesT{T}$ means the same as $\denotesT{\goodType{\Gamma}{T}{}}$
(i.e. every type is tracking its context). In other words, every
derivation of any judgement will track its ``context''. 


\begin{align*}
  \text{We first define a new type }& \goodType{\Gamma}{\Sigr{\Gamma}{n}}{}  \text{ when } \goodCtx{\Gamma}{}  \text{ is well-formed and } n \in \mathbb{N} \\
  \text{such that } &\goodTerm{\Gm}{\denotesT{\goodSig{\Gamma}{\lsig}{n}}}{\Sigr{\Gm}{n}} \\
  \text{We make } & ({\Sigr{\Gamma}{n}}), \text{ and } \ \denotesT{\TyTkg {-}} \text{ mutually recursive, } \\
  & \text{defined by induction on $n$ and the signature length} \\  
  % \denotesT{\goodSig{\Gamma}{\_}{n}} &= (\goodTerm{\Gamma}{\_}{\Sigr{\Gamma}{n}}) \\ 
  % &\text{, and thus } \denotesT{\goodSig{\Gamma}{\sigma}{n}} \text{ defined} \iff \goodTerm{\Gamma}{\denotesT{\sigma}}{\Sigr{\Gamma}{n}} \\ 
  {\Sigr{\Gamma}{(n+1)}} &= 
    \Sigma~(\Sigr{\Gamma}{n})
          ~(\Sigma~\cU
                  ~(\Sigma~(\Pi(\denotesT{\TyTkg {("var"_0[p^1])}})(El~"var"_0[p^1]))~\cU)) \\
  \Sigr{\Gamma}{0} &= \top \\
  \denotesT{\goodType{\Gamma}{\TyTkg {\sigma}}{}}, \denotesT{\goodType{\Gamma}{\TyLkg{\sigma}}{}} &\text{ defines upon } \denotesT{\sigma} 
  \text{ and inductively on the signature length} \\
  \denotesT{\goodType{\Gamma}{\TyTkg {\sigma}}{}} &= \denotesT{\goodType{\Gamma}{\TyLkg{\sigma}}{}} = \top \quad
      \text{given } \denotesT{\goodSig{\Gamma}{\sigma}{0}} \\ 
  \denotesT{\goodType{\Gamma}{\TyTkg {\sigma}}{}} &= 
    \Sigma~\denotesT{\TyTkg {(\lsigproj{1}{\sigma})}}~(El \ (\app{("pjr"^3~\denotesT{\sigma})})[(p^1, \app{(\fst{("pjr"^2~{\denotesT{\sigma}})})})]) \\
  \denotesT{\goodType{\Gamma}{\TyLkg{\sigma}}{}} &=
  \denotesT{\TyLkg{(\lsigproj{1}{\sigma})}} \times \Pi(El~(\fst{(\snd{\denotesT{\sigma}})}))(El~(\app{("pjr"^3~\denotesT{\sigma})} )) \\
  &\text{given } \denotesT{\goodSig{\Gamma}{\sigma}{n+1}} \\
  \denotesT{\goodSig{\Gamma}{\LSigEmp}{0}} &= \unit \\ 
  \denotesT{\goodSig{\Gamma}{\LSigAdd{\sigma}{f}{T}}{n+1}} &= (\denotesT{\sigma}, \codety{A}, \lam{f}, \lam{(\codety{T})}) \\ 
  \denotesT{\goodTerm{\Gamma}{\LkgEmp}{\TyLkg{\LSigEmp}}} &= \unit \\ 
  \denotesT{\goodTerm{\Gamma}{\mu^+~m~t}{\TyLkg{(\LSigAdd{\sigma}{s}{T})}}} &= (\denotesT{m}, \lam{t}) \\ 
  \denotesT{\goodTerm{\Gamma}{\Tkg{m}}{\TyTkg {\sigma}}} & \text{ is defined upon } \denotesT{m} \text{ and inductively on the signature length} \\ 
  \denotesT{\goodTerm{\Gamma}{\Tkg {m}}{\TyTkg {\sigma}}} &= () \quad \text{given } \denotesT{\goodSig{\Gamma}{\sigma}{0}} \\
  \denotesT{\goodTerm{\Gamma}{\Tkg {m}}{\TyTkg {\sigma}}} &= 
  (\denotesT{\Tkg {o}}, t[(\SubstExt{p^1}{f})][(\SubstExt{"id"}{\denotesT{\Tkg {o}}})]) \\
  & \text{given } \denotesT{\goodSig{\Gamma}{\sigma}{n+1}}, \text{ where } o = \lsigproj{1}{m}, t = \app{(\snd{\denotesT{m}})}, f = p_f\nu~\sigma 
\end{align*}
% \begin{align*}
%   \denotesT{\goodInh{\Gamma}{\_}{\sigma_1}{\sigma_2}} &= \goodTerm{\Gamma, \TyLkg{\sigma_1}}{\_}{\TyLkg{\sigma_2[p^1]}} \\ \text{ and thus } \denotesT{\goodInh{\Gamma}{h}{\sigma_1}{\sigma_2}} &\iff  \goodTerm{\Gamma, \TyLkg{\sigma_1}}{h}{\TyLkg{\sigma_2[p^1]}} \\
%   \denotesT{\goodInh{\Gamma}{"inhinh"~h~T~\uparrow^s}{(\LSigAdd {\sigma_1} {s_1}{T})}{(\LSigAdd {\sigma_2} {s_2} {T[(p^1, \uparrow^s)]})}} &= \mu^+~(h[(p^1,\lkgproj{1}{"var"_0})])\\ &\quad \quad \{f_2[p^1]\}~(\lkgproj{2}{"var"_0})[(p^1, \uparrow^s[{p^1}^{\uparrow}])] \\
% \end{align*}



The main idea of this syntactic translation is that, we make sure
$\goodSig{\Gamma}{\nu^+ \_ }{n+1}$ and $\goodSig{\Gamma}{\mu^+ \_}{n+1}$
wrapped into a dependent pair. These two definitions decide the
syntactic translation for $\goodSig{\Gamma}{\_}{n}$ and $\cL$. The
syntactic translation for $\cP$ are decided by their
$\beta$-rules. 

% We use the local morphism\cite{abbott2003category} (the same one for encoding the Sealing
% judgements) to translate inheritance judgement. This is mainly due to
% the existence of \ruleref{Inh} rule. 
% Here we only typeset the overriding
% and inheritance rule, but all of the translation for inh judgment are
% induced by the $\beta$-rule of \ruleref{Inh}.

We omit all the equational rules ($\beta,\eta$ and substitution) here.
However, for coherence, when we mutual recursively define
$\denotesT{\goodSig{\Gamma}{\_}{n}}$, ${\Sigr{\Gamma}{n}} $, and
$\denotesT{\cP}$ above, we actually have to prove the substitution laws
$\sub{(\TyTkg{\lsig})}{\gm}\equiv \TyTkg{\sub{\lsig}{\gm}}$ and $\sub{(\Sigr{\Delta}{n})}{\gm}
\equiv \Sigr{\Gamma}{n}$ together. 
% The details (of all the equational rules) can be found in the appendices written in (pseudo-)Agda form. We encourage the reader to read that (pseudo-)Agda-style proof because it is more friendly to type theorists, easy to do type checking and
% validate the proof and it is clearer on the mathematical object each
% translation uses.


Ultimately, we have constructed a model for our syntax using only a
``subpart'' of our syntax -- without these linkages. It is also
possible to compile our $\goodWSig{\Gamma}{w}{n}$ and W-type into a more
conventional formulation, but the recursor will always require a sigma
type as handler and thus the elimination rule for our W-type is still
non-conventional.

This translation justifies the intuition -- a linkage is actually a
module \textbf{with universal quantification wrapping its fields to
achieve late binding}. 

After syntactic translation, we can directly have consistency and
canonicity \textit{if we assume the consistency and canonicity of the
``subpart''}. However, due to our non-conventional formulation of the
inductive type, it is better we don't impose such assumption and
directly give a consistency and canonicity proof.




\subsection{Standard Model for Consistency}

Once we have the syntax of our theory, the first question is if our theory is consistent, i.e., if we can syntactically derive bottom in our theory. We prove the consistency by extending the standard model from \citep{kaposi2017type}, where they formally resort to the concept of \textit{algebra} of QIIT, here we only consider inductively interpreting each syntax piece into naive set theory that is also respecting judgemental equalities for accessibility\footnote{In our appendix, we still interpret into Agda component to be more precise and less error-prone, also we need the inductive facility from Agda.} 

Here we will only show some interpretation and also omit the universe level to sketch out the idea, please refer to appendix for the complete version.


\newcommand{\goodTypeS}[3]{{ {#1} \models_S {#2} }}
\newcommand{\goodTermS}[3]{{ {#1} \models_S {#2} : {#3} }}
\newcommand{\goodSubS}[3]{{ {#1} \models_S {#2} : {#3} }}
\newcommand{\goodSigS}[3]{{ {#1} \models_S {#2} \ \  Sig^{#3} }}
\newcommand{\goodWSigS}[3]{{ {#1} \models_S {#2} \ \ WSig^{#3} }}
\newcommand{\goodSealS}[4]{{ {#1} \models_S {#2} : {#3} \  |\  {#4} }}
\newcommand{\goodInhS}[4]{{ {#1} \models_S {#2} : {#3} \twoheadrightarrow {#4}}}

\begin{align*}
  \denotesS{\goodCtx{\Gamma}{i}} & \text{ is a Set} \\
  \denotesS{\goodType{\Gamma}{T}{i}} & \text{ is a family of sets indexed by } \denotesS{\goodCtx{\Gamma}{i}} \\
  & \text{ and we define } \goodTypeS{\Gamma}{T}{i} : \iff \denotesS{\goodType{\Gamma}{T}{i}} \text{ is defined } \\
  \denotesS{\goodSub{\Gamma}{\_}{\Delta}} & \text{ is a Set and we define }  {\goodSubS{\Gamma}{\gamma}{\Delta}} : \iff \denotesS{\gamma} \in \denotesS{\goodSub{\Gamma}{\_}{\Delta}} \\
  \denotesS{\goodSub{\Gamma}{\_}{\Delta}} & = \denotesS{\goodCtx{\Gamma}{i}} \rightarrow \denotesS{\goodCtx{\Delta}{j}} \\
  \denotesS{\goodTerm{\Gamma}{\_}{T}} & \text{ is a Set (of dependent function) where }   {\goodTermS{\Gamma}{t}{T}} : \iff \denotesS{t} \in \denotesS{\goodTerm{\Gamma}{\_}{T}} \\
  \denotesS{\goodTerm{\Gamma}{\_}{T}} & = \prod_{\gamma \in \denotesS{\goodCtx{\Gamma}{i}}}\denotesS{T}(\gamma) \\
  % 
  \denotesS{\goodSig{\Gamma}{\_}{n}} & \text{ is a Set  and we define  } {\goodSigS{\Gamma}{\sigma}{n}} :\iff \denotesS{\sigma} \in  \denotesS{\goodSig{\Gamma}{\_}{n}} \\
  \denotesS{\goodType{\Gamma}{\cL \sigma}{i}} & \text{ is a family of sets, dependent on } \denotesS{\sigma}  \\
  \denotesS{\goodSig{\Gamma}{\_}{n+1}} &= \{
    (\denotesS{\sigma}, \denotesS{\sigma'}, \denotesS{f}, \denotesS{T}) :
      {\goodSigS{\Gamma}{\sigma}{n}}
      \land  {\goodSigS{\Gamma}{\sigma'}{n}} \\  
      & \quad \quad \quad \land  {\goodSealS{\Gamma}{f}{\sigma}{\sigma'}}
      \land  {\goodTypeS{\Gamma, \cL \sigma'}{T}{i}} 
  \} \\ 
  \denotesS{\goodSig{\Gamma}{\_}{0}} &= \{\star\} \quad \text{ a specific singleton set} \\
  \denotesS{\goodType{\Gamma}{\cL \sigma}{i}} (\gamma)&= \{(m, f) :  m \in \denotesS{\cL \tau}(\gamma) \land f \in \prod_{m' \in \denotesS{\cL \tau'}(\gamma)}\denotesS{T}(\gamma, m')  \}  \\
  & \quad \text{when } \denotesS{\sigma} = (\denotesS{\tau}, \denotesS{\tau'}, \_ , \denotesS{T}) \in \denotesS{\goodSig{\Gamma}{\_}{n+1}} \\
  \denotesS{\goodType{\Gamma}{\cL \sigma}{i}} (\gamma)&= \{\star\} \text{ when } \denotesS{\sigma} \in \denotesS{\goodSig{\Gamma}{\_}{0}} \\
\end{align*}


Notice that we interpret the bottom type using empty set, and thus we know it is not possible to derive $\cdot \vdash t : \bot$: otherwise we have  $\denotesS{t}() \in \denotesS{\bot}() = \emptyset$ 
and that is a contradiction.


\subsection{Proof Relevant Logical Relation for Canonicity}
Now we prove canonicity (and consistency) for \TT using a logical-relations model.
We follow the reducibility argument of
\citet{kaposi2019gluing}, \citet{coquand2018canonicity}, and \citet{sterling2019algebraic}
to construct our model.
The meta\-languages of these prior models are based on QIITs, categories with
families \cite{dybjer1995internal}, and the generalized algebraic theory
\cite{cartmell1986generalised}, respectively.
Without exposing the reader to too many technical details, our meta\-language
should be understood as an instance of any of the above logical frameworks---the
difference is that quotienting is manual in our formulation, whereas it is
automatic with the logical frameworks.
%Our model works directly on the earlier introduced syntax for readability\footnote{We still
%take quotient, so ultimately our syntax should be considered as an instance of
%any of the above logical framework, without exposing the reader too many
%technical details}.

We state the canonicity theorem first:

\begin{theorem}[Canonicity]
\label{thm:canonicity-appendix}
  If $\goodTerm{\cdot}{t}{\cB}$, then either $\goodTerm{\cdot}{t \equiv \true}{\cB}$ or $\goodTerm{\cdot}{t \equiv \false}{\cB}$.
\end{theorem}

\noindent
Canonicity is a key criterion for a dependent type theory to be considered as
a programming language or as a computational foundation for mathematics.
%\footnote{See nlab explanation: \href{https://ncatlab.org/nlab/show/canonical+form}{Canonical Form}}
% We can even argue that, if this theorem is proven in a computable meta\-logic, 
% then by the Curry–Howard correspondence, the canonicity theorem provides a big-step
% interpreter for closed terms of the boolean type.
%\YZ{Can we say something about
%strong/weak normalization?}\EDJreply{
%  No. \\
%  1. In the declarative system, we cannot talk about strong/weak normalization (because we don't have operational semantic, if you recall the definition of strong/weak normalization, it based on reduction/path of steps/op semantic). So any review says strong/weak normalization to you, please stop them. \\ 
%  2. In declarative system, we only say ``normalization'' directly, which means, we construct a function that \\
%      for arbitrary class of (judgemental-equivalent) terms, it will return one syntax representation (called normal-form) for each class \\
%    (apparently most people will make the normal form  without beta redex, however, interestingly, they may do eta expansion instead of eta reduction because theoretically easier) \\ 
%  3. So translation won't work because it compiles away the normal form of linkage (the $\mu^+$). This part requires a direct proof which construct normal form and neutral form for our calculus, and a normalization model from which we extract a runnable interpreter for open terms \\
%  4. Of course we can conjecture we have that normalization function \\
%  5. Also we do have close term normalization, which is canonicity. Or more operationally speaking, we have termination (a runnable interpreter) for close term (since QIIT is runnable, the canoncity model itself is the runnable interpreter), \\
%  but we don't have termination (a runnable interpreter) for open terms.  \\ 
%  So earlier you say ``normalize'' a recursion on an inductive type, that is no doable -- if the programmer wants to program on meta-theory, they will have to manually write out the a chain of judgemental equality (with a bunch of beta rules inside) by themselves. 
%}

First, we need the mathematical setup to interpret universe levels, following \citet{sterling2019algebraic}:

\newcommand{\Set}[1]{\ensuremath{\texttt{Set}_{#1}}}
\newtheorem{assumption}{Assumption}[section]

\begin{assumption}[Set-theoretic Universe Assumption] We assume an infinite hierarchy of Grothendieck universes $\Set{i}$ for $i \in \mathbb{N}$ in our ambient meta\-logic.
\end{assumption}

We can roughly consider each Grothendieck universe $\Set{i}$ as the $\Set{i}$ in Agda: 
\begin{itemize}
  \item Each $\Set{i}$ is closed under dependent function types and dependent
  pair types. For example, later, for our interpretation of dependent function
  types, when we have $\denotesC{A}, \denotesC{B} \in \Set{i}$, we will have
  $\denotesC{\TyPi{A}{B}} \in \Set{i}$.
  \item The universe hierarchy is cumulative, as $\Set{i} \in \Set{i+1}$ and $\Set{i} \subseteq \Set{i+1}$.
  \item Thus, if $\denotesC{A} \in \Set{i}, \denotesC{B} \in \Set{j}$, we will
  have $\denotesC{\TyPi{A}{B}} \in \Set{i} \cup \Set{j} = \Set{i \lcup j}$.
\end{itemize}


Like most logical-relations proofs, we interpret each judgment and inductively
interpret each syntax piece. We are working in an intrinsic setting; thus, even if
we omit contexts for brevity, the syntax piece is still well-typed. However,
unlike most logical-relations proofs, our logical-relations model is proof-relevant,
which is essential for modeling universes
properly~\cite{coquand2018canonicity}.
Our canonicity model for the base MLTT fragment is close to the ones
in \citet{coquand2018canonicity} and \citet{sterling2019algebraic}.
% We will omit the treatment of the universe and the verification of most equality for simplicity (and the treatment of the universe will not affect our model). Interested audience can refer to \citet{kaposi2019gluing, coquand2018canonicity,sterling2019algebraic} for a complete technical treatements.

\begin{align*}
  \denotesC{\goodCtx{\Gm}{k}} \text{ is a function }&: \setc{\gm}{\goodSub{\cdot}{\gm}{\Gm}} \to \Set{k} \quad \text{ (i.e., sets indexed by closed substitution)}\\
  % &\text{and we define } \goodCtxS{\Gm}{i} :\iff \denotesC{\goodCtx{\Gm}{i}} \text{ is defined} \\
  \denotesC{\goodType{\Gm}{T}{j}} \text{ is a dependent function }&: \prod_{\goodSub{\cdot}{\gm}{\Gm}}\prod_{\gm' \in \denotesC{\goodCtx{\Gm}{}}(\gm)}\setc{t}{\goodTerm{\cdot}{t}{\sub{T}{\gm}}}\to\Set{j} \\
  % &\text{and we define } \goodTypeS{\Gm}{T}{i} : \iff \denotesC{\goodType{\Gm}{T}{i}} \text{ is defined } \\ 
  \denotesC{\goodSub{\Gm}{\delta}{\Delta}} \text{ is a dependent function }&: \prod_{\goodSub{\cdot}{\gm}{\Gm}} \prod_{\gm' \in \denotesC{\goodCtx{\Gm}{}}(\gm)} \denotesC{\goodCtx{\Dl}{}}(\delta \circ \gm) \\
  % &\text{and we define } \goodSubC{\Gm}{\gm}{\Delta} :\iff \denotesC{\goodSub{\Gm}{\gm}{\Delta}} \text{ is defined} \\
  \denotesC{\goodTerm{\Gm}{t}{T}} \text{ is a dependent function}&: \prod_{\goodSub{\cdot}{\gm}{\Gm}} \prod_{\gm' \in \denotesC{\goodCtx{\Gm}{}}(\gm)}\denotesC{\goodType{\Gm}{T}{}}(\gm)(\gm')(\sub{t}{\gm})\\
  \denotesC{\goodType{\Gm}{\sub{T}{\sigma}}{}}(\gm)(\gm')(t) &= \denotesC{T}(\sigma \circ \gm)(\denotes{\sigma}(\gm)(\gm'))(t)\\
  \denotesC{\goodType{\Gm}{\top}{}}(\gm)(\gm')(t) &= \{\star \} \\
  \denotesC{\goodType{\Gm}{\bot}{}}(\gm)(\gm')(t) &= \emptyset \\
  \denotesC{\goodType{\Gm}{\cB}{}}(\gm)(\gm')(t) &= 
    %\YZ{Is this more accurate than 'disjoint union'?}\EDJreply{It is accurate. It is actually the way Coquand write it. But I find this way weird type-theoretically---at the very end, when we want to prove a closed term t is either true or false, we will use this cB. But defining in this Coquand way, type-theoretically, we cannot extract a proof term t = true. (i.e. there is no proof term stored here!). I know my above sentence is a bit awkward but I just find this Coquand way of saying very...counter-intuitive.}
    \begin{cases}
      \{\star^1\} & \text{if } t \equiv \true \\
      \{\star^2\} & \text{if } t \equiv \false \\
      \emptyset & \text{otherwise }
    \end{cases}\\
  \denotesC{\goodType{\Gm}{\TyId{a}{b}}{}}(\gm)(\gm')(t) &= 
  \begin{cases}
    \{\star\} & \text{if } t \equiv \eqrefl{\sub{a}{\gm}} \text{ and } \sub{a}{\gm} \equiv \sub{b}{\gm}\\
    \emptyset & \text{otherwise }
  \end{cases}\\
  % \setc{\star}{\text{ when } t \equiv \eqrefl{\sub{a}{\gm}} \text{ and } \sub{a}{\gm} \equiv \sub{b}{\gm}} %\YZ{What if the conditions do not hold? Empty set?}\EDJreply{Yes. It is a family of set indexed by t (or just a function maps t to a set), and if t is neither of the above, then the function will map to empty set.}
  \denotesC{\goodType{\Gm}{\TyPi{A}{B}}{}}(\gm)(\gm')(t) &= \prod_{\goodTerm{\cdot}{u}{\sub{A}{\gm}}} \ \ \prod_{u' \in \denotesC{A}(\gm)(\gm')(u)} \denotesC{B}(\SubstExt{\gm}{u})((\gm', u'))(\sub{\app{t}}{\SubstExt{\SubstId}{u}})\\
  \denotesC{\goodType{\Gm}{\TySigma{A}{B}}{}}(\gm)(\gm')(t) &= \sum_{u' \in \denotesC{A}(\gm)(\gm')(\fst{t})} \denotesC{B}(\SubstExt{\gm}{\fst{t}})((\gm', u'))(\snd{t}) \\
  \denotesC{\goodCtx{\cdot}{}}(\gm) &= \{\star\} \quad \text{  a singleton set } \\ 
  \denotesC{\goodCtx{\Gm, T}{}}(\gm_t) &= \setc{ (\gamma' , t') }{ \gm' \in \denotesC{\goodCtx{\Gm}{}}(\pi_1 \gm_t), t' \in \denotesC{\goodType{\cdot}{\sub{T}{\gm}}{}}(\pi_1 \gm_t)(\gm')(\pi_2 \gm_t)  } \\
  \denotesC{\goodTerm{\Gm}{\sub{t}{\sigma}}{\sub{T}{\sigma}}}(\gm)(\gm') &= \denotesC{t}(\sigma \circ \gm)(\denotesC{\sigma}(\gm)(\gm')) \\
  \denotesC{\goodTerm{\Gm}{\unit}{\top}}(\gm)(\gm') &= \star\\
  \denotesC{\goodTerm{\Gm}{\true}{\cB}}(\gm)(\gm') &= \star^1 \\
  \denotesC{\goodTerm{\Gm}{\false}{\cB}}(\gm)(\gm') &= \star^2 \\
  \denotesC{\goodTerm{\Gm}{\ifb{c}{a}{b}}{T}}(\gm)(\gm') &= 
  \begin{cases}
    \denotesC{a}(\gm)(\gm') \quad \text{ if } \denotes{c}(\gm)(\gm') = \star^1 \\
    \denotesC{b}(\gm)(\gm') \quad \text{ if } \denotes{c}(\gm)(\gm') = \star^2
  \end{cases}\\
  % \denotesC{a}(\gm)(\gm') \quad \text{ if } \denotes{c}(\gm)(\gm') = \star^1 \\
  % \denotesC{\goodTerm{\Gm}{\ifb{c}{a}{b}}{T}}(\gm)(\gm') &= \denotesC{b}(\gm)(\gm') \quad \text{ if } \denotes{c}(\gm)(\gm') = \star^2 \\
  \denotesC{\goodTerm{\Gm}{\eqrefl{t}}{\TyId{t}{t}}}(\gm)(\gm') &= \star\\
  \denotesC{\goodTerm{\Gm}{\lam{t}}{\TyPi{A}{B}}}(\gm)(\gm') &= \lambda u \lambda u' . \denotesC{t}(\SubstExt{\gm}{u}) (\gm' , u') \\
  \denotesC{\goodTerm{\Gm}{\app{t}}{B}}(\gm)(\gm') &= \denotesC{t}(\pi_1 \gm)(\gm'[0])(\pi_2 \gm)(\gm'[1]) \\
  \denotesC{\goodTerm{\Gm}{\pair{a}{b}}{\TySigma{A}{B}}}(\gm)(\gm') &= (\denotesC{a}(\gm)(\gm') , \denotesC{b}(\gm)(\gm') )\\
  \denotesC{\goodTerm{\Gm}{\fst{t}}{T}}(\gm)(\gm') &= \denotesC{t}(\gm)(\gm')[0] \\
  \denotesC{\goodTerm{\Gm}{\snd{t}}{T}}(\gm)(\gm') &= \denotesC{t}(\gm)(\gm')[1]\\
  \denotesC{\goodType{\Gm}{\cU_j}{j+1}}(\gm)(\gm')(T) &= \setc{t} { \goodTerm{\cdot}{t}{\El{T}} }  \to \Set{j} \\
  \denotesC{\goodTerm{\Gm}{\codety{T}}{\cU_j}}(\gm)(\gm') &= \denotesC{T}(\gm)(\gm') \\ 
  \denotesC{\goodType{\Gm}{\El{T}}{j}}(\gm)(\gm')(t) &= \denotesC{T}(\gm)(\gm')(t) \\
\end{align*}

\newcommand{\Glued}[1]{\ensuremath{{#1}^\bullet}}
\newcommand{\GluedPi}[2]{\ensuremath{\Pi^\bullet({#1},{#2})}}
\newcommand{\GSubstExt}[2]{\ensuremath{{#1},^\bullet{#2}}}
\newcommand{\Gpair}[2]{\ensuremath{({#1},^\bullet{#2})}}
\newcommand{\Gfst}[1]{\ensuremath{\texttt{fst}^\bullet~{#1}}}
\newcommand{\Gsnd}[1]{\ensuremath{\texttt{snd}^\bullet~{#1}}}
\newcommand{\Gsub}[2]{\ensuremath{{#1}\!\left[{#2}\right]^\bullet}}
\newcommand{\Glam}[1]{\ensuremath{\lambda^\bullet({#1})}}
\newcommand{\Gapp}[1]{\ensuremath{\texttt{app}^\bullet({#1})}}
\newcommand{\GSubstWeak}[1]{\ensuremath{(\texttt{p}^\bullet)^{#1}}}
\newcommand{\GLSigAdd}[3]{\ensuremath{\nu^{+\bullet}({#1},{#2},{#3})}}
\newcommand{\GCaseSig}[3]{\ensuremath{\texttt{CaseTy}^\bullet({#1},{#2},{#3})}}
\newcommand{\Gmodel}[1]{\ensuremath{{({#1})}^c}}
\newcommand{\GEl}[1]{\ensuremath{\Glued{\texttt{El}}({#1})}}
\newcommand{\Gwcode}[1]{\ensuremath{\Glued{\texttt{W}}({#1})}}
\newcommand{\GSubstId}{\ensuremath{\GSubstWeak{0}}}
\newcommand{\Gwsigproj}[3]{\ensuremath{\Glued{{\texttt{w}\pi^{#1}_{\texttt{#2}}}}({#3})}}
\newcommand{\GRecproj}[2]{\ensuremath{\Glued{{\texttt{R}\pi^{#1}}}({#2})}}



% What is perhaps unintuitive about the proof-relevant logical-relations model is the
% treatment of propositions.
%The model treats propositions in a special way : \EDJ{Without this first sentence, this paragraph seems unmotivated.}\YZreply{Why is it 'difficult'? Is there a more appropriate adjective than 'difficult'?}\EDJreply{Don't you find difficult? At first sight, I cannot understand why this sentence, proposed by Sterling in two of his paper, can be valid mathematically. It only makes sense to me until I am in a type-theoretic world (by Kaposi, since Eq is an inductive type now) (or maybe topos-theoretically where categorical logic will encode equality using data like equalizer). What about changing the phrase to "eccentric" or "weird"?}
% A proposition is encoded as a sub\-singleton. For
% example, $\setc{\star^1}{\text{ when } t \equiv \true} \uplus \setc{\star^2}{\text{ when } t \equiv \false}$ represents a disjoint union of two
% sub\-singleton sets, where the first one is either a singleton if we have a proof of
% $t \equiv \true $ or an empty set otherwise.
% In other words,
% $\star^1$ witnesses a proof that $t \equiv \true$.
Here, $\star$, $\star^1$, $\star^2$ are just some arbitrary fixed elements.  


Given the above model for MLTT, there should be a function ${\Pi}^c$ such
that ${\Pi}^c({\denotesC{A}},{\denotesC{B}}) = \denotesC{\TyPi{A}{B}}$. 
Type-theoretically speaking, this ${\Pi}^c$ uses the \emph{internal dependent
function type} of
the above model. We hope to
use this function when defining the logical-relations model for the rest of \TT.
However, such a function $\Pi^c$ is not yet possible because the definition
$\denotesC{\TyPi{A}{B}}$ is not based solely on $\denotesC{A}$ and
$\denotesC{B}$,
but also on the syntax $\goodType{\Gm}{A}{}$ and $\goodType{\Gm, A}{B}{}$. 

Thus, we define a new denotation $\denotesCC{S} \coloneq (S, \denotesC{S})$
that also returns the syntax piece $S$.\footnote{This is also called
glued interpretation in \citet{sterling2019algebraic}.}
Thus, we can have a
function $\Glued{\Pi}$ such that $\GluedPi{\denotesCC{A}}{\denotesCC{B}} =
\denotesCC{\TyPi{A}{B}}$ now that the syntax is available.
Similarly, there are functions $\Glued{\Sigma}$, $\Gpair{a}{b}$, and
$\GSubstExt{\gm}{t}$ for dependent pair types, dependent pairs, and
substitution extension (and more for other constructions).
Furthermore, given $\Glued{S}$ (i.e., the syntax and its semantic interpretation),
we use $\Gmodel{\Glued{S}}$ to mean the latter of the two.%\YZ{why not just $\denotesC{S}$?}\EDJreply{Using superscript C is shorter and more precise -- we are not "reconstructing the C model from the syntax, but taken out the C model directly from the pair". Look at the canonicity model where superscript c is used, if change to $\denotesC{-}$ on them, it is not really well-founded recursion, technically speaking (only equivalent to one).}
% These will be useful in our \TT model because as syntactic translation hinted, \TT reuses a lot from MLTT. What's more, during the construction of the model for \TT, we will use $\Glued{(\cdot)}$ when we can.

\newcommand{\GTy}[2]{\ensuremath{\Glued{\texttt{Ty}}_{#1}~{#2}}}
\newcommand{\GCon}[1]{\ensuremath{\Glued{\texttt{Con}}_{#1}}}
\newcommand{\GTm}[2]{\ensuremath{\Glued{\texttt{Tm}}~{#1}~{#2}}}
\newcommand{\GSub}[2]{\ensuremath{\Glued{\texttt{Sub}}~{#1}~{#2}}}
\newcommand{\GMWSig}[3]{\ensuremath{{\texttt{WSig}^C}_{#1}^{#3}~{#2}}}
\newcommand{\GWSig}[3]{\ensuremath{{\Glued{\texttt{WSig}}}_{#1}^{#3}~{#2}}}


We need more \emph{internal} type-theoretic constructions:

We define $\GCon{k} \coloneq \sum_{\goodCtx{\Gm}{k}} \{\gm : \goodSub{\cdot}{\gm}{\Gm} \} \to \Set{k}$ 


and $\GTy{j}{\Glued{\Gm}} \coloneq \sum_{\goodType{\Gm}{T}{j}} \prod_{\goodSub{\cdot}{\gm}{\Gm}}\prod_{\gm' \in \Gmodel{\Glued{\Gm}}(\gm)}\{t : \goodTerm{\cdot}{t}{\sub{T}{\gm}}\}\to\Set{j} $ for $\Glued{\Gm} \in \GCon{k}$

and $\GTm{\Glued{\Gm}}{\Glued{T}} \coloneq \sum_{\goodTerm{\Gm}{t}{T}} \prod_{\goodSub{\cdot}{\gm}{\Gm}} \prod_{\gm' \in\Gmodel{\Glued{\Gm}}} \Gmodel{\Glued{T}}(\gm)(\gm')(\sub{t}{\gm})$

and $\GSub{\Glued{\Gm}}{\Glued{\Dl}} \coloneq \sum_{\goodSub{\Gm}{\delta}{\Dl}} \prod_{\goodSub{\cdot}{\gm}{\Gm}} \prod_{\gm' \in \Gmodel{\Glued{\Gm}}(\gm)} \Gmodel{\Glued{\Dl}}(\delta \circ \gm) $

These four sets are collecting the glued interpretation.
For each well-formed type $\goodType{\Gm}{T}{j}$, we have its denotation $\denotesCC{T} \in \GTy{j}{\denotesCC{\Gm}}$;
for each well-typed term $\goodTerm{\Gm}{t}{T}$, we have its denotation $\denotesCC{t} \in \GTm{\denotesCC{\Gm}}{\denotesCC{T}}$;
etc.


Notice that
$\prod_{\goodSub{\cdot}{\gm}{\Gm}} \prod_{\gm' \in \Gmodel{\Glued{\Gm}}(\gm)}$
is part of $\GTy{\!}{\!}$, $\GTm{\!}{\!}$, and $\GSub{\!}{\!}$.
An interesting fact about them is : given a pair of arbitrary $\goodSub{\cdot}{\gm}{\Gm}$
and $\gm' \in \Gmodel{\Glued{\Gm}}(\gm)$, we can consider $(\gm, \gm')$ as an
element of $\GSub{\Glued{\cdot}}{\Glued{\Gm}}$.
Thus, we can consider the pair $(\gm,
\gm')$ as an element $\Glued{\gm} \in \GSub{\Glued{\cdot}}{\Glued{\Gm}}$.


Then we define $\GMWSig{j}{\Glued{\Gm}}{n} \coloneq "Vector"^n~\sum_{\Glued{A}
\in \GTy{j}{\Glued{\Gm}}} \GTy{j}{(\GSubstExt{\Glued{\Gm}}{\Glued{A}})}$ for
$\Glued{\Gm} \in \GCon{k}$, a length-$n$ list of pairs of types.
As before, we can define glued interpretation $\GWSig{j}{\Glued{\Gm}}{n}
\coloneq \{ \wsig : \goodWSig{\cdot}{\wsig}{n} \} \times
\GMWSig{j}{\Glued{\Gm}}{n}$. This will be useful when we interpret judgments for
W-type signatures.


Now we can extend the model above to include \TT constructs.
%We still start with the judgments.
%Since linkage transformers are just syntactic sugar, we will not interpret it.
The key idea of this extension is similar to that of the syntactic translation:
we interpret linkage types using the canonicity model of sigma types (developed in the prior work).
We use the inductive facility of the ambient meta\-logic to justify our W-types.

\newcommand{\CWmodel}{\ensuremath{\mathit{W}^C}}
\newcommand{\CWsup}{\ensuremath{\mathit{W^Csup}}}
\newcommand{\CWrec}{\ensuremath{\mathit{W^Crec}}}

\begin{align*}
  \denotesC{\goodSig{\Gm}{\lsig}{n}} & \text{ is a list of 3-tuple of length $n$  } \\
  % & \text{ and we define  } {\goodSigC{\Gm}{\lsig}{n}} :\iff \denotesC{\goodSig{\Gm}{\lsig}{n}} \text{ is defined} \\ 
  \denotesC{\goodWSigU{j}{\Gm}{\wsig}{n}} &: \GMWSig{j}{\denotesCC{\Gm}}{n} \\ 
    &\text{ i.e., a list of 2-tuple of length $n$ } \\
  % & \text{ and we define } \goodWSigC{\Gm}{\wsig}{n} :\iff \denotesC{\goodWSig{\Gm}{\tau}{n}} \text{ is defined} \\
  & \text{we define }\TyLkg{}, \TyTkg{} \text{ inductively on its input signature} \\ 
  \mathit{L}("nil") &=  \denotesC{\top}  \\
  \mathit{L}((A,s,T) :: tl)(\gm)(\gm')(t) &= {\mathit{L}(tl)}(\gm)(\gm')(\lkgproj{1}{t}) \times \Gmodel{\GluedPi{A}{T}}(\gm)(\gm')(\lam{\lkgproj{2}{t}})   \\
  \Glued{\mathit{P}}("nil") &= ( \TyTkg{\LSigEmp} , \denotesC{\top} )\\
  \Glued{\mathit{P}}((A,s,T) :: tl) &= \Glued{\Sigma}(\Glued{\mathit{P}}(tl), \Gsub{T}{\GSubstExt{\Glued{\pi_1}}{s}}) \quad \text{(doing substitution on $T$)} \\
  \denotesC{\goodType{\Gm}{\TyLkg{\lsig}}{}} &= {\mathit{L}(\denotesC{\lsig})} \\
  \denotesC{\goodType{\Gm}{\TyTkg{\lsig}}{}} &= \Gmodel{\Glued{\mathit{P}}(\denotesC{\lsig})} \quad \text{ discard syntax info}  \\
  % 
  \denotesC{\goodSig{\Gm}{\LSigEmp}{0}} &= "nil"\\ 
  \denotesC{\goodSig{\Gm}{\LSigAdd{\lsig}{s}{T}}{n+1}} &= (\denotesCC{A}, \denotesCC{s}, \denotesCC{T})::\denotesC{\lsig}\\ 
  \denotesC{\goodSig{\Gm}{\sub{\lsig}{\gm}}{n}} & \text{ is done by point-wise/component-wise substitution} \\
  \denotesC{\goodSig{\Gm}{\lsigproj{1}{\lsig}}{n}} &= "tl"~\denotesC{\lsig}\\ 
  \denotesC{\goodType{\Gm}{\lsigprojT{\lsig}}{}} &= \Gmodel{("hd"~\denotesC{\lsig})[0]} \quad \text{ take the first element in the tuple,...}\\ 
  \denotesC{\goodTerm{\Gm, \TyTkg{\lsigproj{1}{\lsig}}}{\lsigproj{s}{\lsig}}{\lsigprojT{\lsig}}} &= \Gmodel{("hd"~\denotesC{\lsig})[1]} \quad \text{  ... and discard syntax}\\ 
  \denotesC{\goodType{\Gm, \lsigprojT{\lsig}}{\lsigproj{2}{\lsig}}{}} &= \Gmodel{("hd"~\denotesC{\lsig})[2]}\\
  \denotesC{\goodTerm{\Gm}{\LkgEmp}{\TyLkg{\LSigEmp}}} &= \denotesC{\unit} \\
  \denotesC{\goodTerm{\Gm}{\LkgAdd{\lkg}{t}}{\TyLkg{\LSigAdd{\lsig}{s}{T}}}}(\gm)(\gm') &= (\denotesC{\lkg}(\gm)(\gm') , \Gmodel{\Glam{\denotesCC{t}}}(\gm)(\gm')) \\
  \denotesC{\goodTerm{\Gm}{\Tkg{\lkg}}{\TyTkg{\lsig}}} &= \denotesC{\unit}
  \quad \text{ when \goodSig{\Gm}{\lsig}{0}} \\
  \denotesC{\goodTerm{\Gm}{\Tkg{\lkg}}{\TyTkg{\lsig}}} &= \Gmodel{\Gpair{\denotesC{\Tkg{\lkgproj{1}{\lkg}}}}{\Gsub{\Gsub{\denotesC{\lkgproj{2}{\lkg}}}{\GSubstExt{\Glued{\GSubstWeak{1}}}{s}}}{\GSubstExt{\Glued{(\SubstId)}}{\denotesC{\Tkg{\lkgproj{1}{\lkg}}}}}}}
  \\ &\text{ when} \denotesC{\goodSig{\Gm}{\lsig}{n+1}} = (A, s, T)::\_ \\
  \denotesC{\goodTerm{\Gm}{\lkgproj{1}{\lkg}}{\TyLkg{\lsig}}}(\gm)(\gm') &= \denotesC{\ell}(\gm)(\gm')[0] \quad \text{ take the first element in the tuple} \\
  \denotesC{\goodTerm{\Gm, \lsigprojT{\lsig}}{\lkgproj{2}{\lkg}}{T}}(\gm_+)(\gm_+') &= ((\denotesC{\ell}(\pi_1{\gm_+})(\gm_+'[0]))[1])(\pi_2{\gm_+})(\gm_+'[1]) \\
  \denotesC{\goodWSig{\Gm}{\WSigEmp}{0}} &= "nil" \\
  \denotesC{\goodWSig{\Gm}{\WSigAdd{\wsig}{A}{B}}{n+1}} &= (\denotesCC{A} , \denotesCC{B}) :: \denotesC{\wsig}\\
  \denotesC{\goodWSig{\Gm}{\sub{\wsig}{\gm}}{n}} & \text{ is done by point-wise/component-wise substitution} \\
  \denotesC{\goodType{\Gm}{\wsigproj{j}{1}{\wsig}}{}} &= \Gmodel{(\text{ $j$-th element of }~\denotesC{\wsig})[0]}\\
  \denotesC{\goodType{\Gm, \wsigproj{j}{1}{\wsig}}{\wsigproj{j}{2}{\wsig}}{}} &= \Gmodel{(\text{ $j$-th element of }~\denotesC{\wsig})[1]}\\
  \denotesC{\goodWSig{\Gm}{\WSigSub{\wsig}}{n}} &= "tl"~\denotesC{\wsig}\\
  \denotesC{\goodTerm{\Gm}{\wcode{\wsig}}{\cU}}(\gm)(\gm')(t) &= \CWmodel~(\Gsub{\denotesCC{\wsig}}{\Glued{\gm}})~t \\
  \denotesC{\goodTerm{\Gm}{\Wsup{i}{\wsig}{a}{b}}{\El{\wcode{\wsig}}}}(\gm)(\gm') &= \CWsup~i~(\Gsub{\denotesCC{\wsig}}{\Glued{\gm}})~(\Gsub{\denotesCC{a}}{\Glued{\gm}})~(\Gsub{\denotesCC{b}}{{\Glued{\gm^\uparrow}}}) \\
  % CsTy~A~B~R &= {\GluedPi{A}{\GluedPi {\GluedPi{B}{\Gsub{R}{\GSubstWeak{2}}}} {\Gsub{R}{\GSubstWeak{2}}}}} \\
  \denotesC{\goodType{\Gm}{\CaseSig{A}{B}{R}}{}} &= 
  \Gmodel{\GluedPi{\denotesC{A}}{\GluedPi {\GluedPi{\denotesC{B}}{\Gsub{\denotesC{R}}{\GSubstWeak{2}}}} {\Gsub{\denotesC{R}}{\GSubstWeak{2}}}}} \\
  % CsTy~\denotesC{A}~\denotesC{B}~\denotesC{R}\\
  \text{we define } \RecSig{}{} &\text{ by induction on the signature} \\
  RS~"nil"~R &= \denotesCC{\LSigEmp} \\
  RS~((A, B) :: tl)~R &= \GLSigAdd{RS~tl~R}{\Glued{\pi_2}}{\GCaseSig{A}{B}{R}} \\
  \denotesC{\goodType{\Gm}{\RecSig{\wsig}{"R"}}{}} &= \Gmodel{RS~\denotesC{\tau}~\denotesC{"R"}} \\
  \denotesC{\Recproj{j}{\lkg}} &= \text{ take the $j$-th field from } \lkg \\
  \denotesC{\goodTerm{\Gm}{\Wrec{\wsig}{\lkg}{t}}{T}}(\gm)(\gm') &= \CWrec~(\lambda w. \denotesC{"R"}(\gm)(\gm')(\Wrec{\wsig}{\sub{\lkg}{\gm'}}{w}))~f^r~\sub{t}{\gm'}~(\denotesC{t}(\gm)(\gm')) \\
  \text{ where } \quad f^r~j~\Glued{a}~b~b^C &= 
  \texttt{let}~\Glued{\rho}  \in \GTm{(\Glued{\cdot} \Glued{,} \Gsub{\Glued{B}}{\GSubstExt{\Glued{"id"}}{\Glued{a}}})}{(\Gsub{\Gsub{\Glued{R}}{\Glued{\gm}}}{\GSubstWeak{1}})} 
  \\
  & \ \texttt{ s.t.}~\Glued{\rho} \coloneq (\Wrec{\wsig}{\sub{\lkg}{\SubstComp{\gm}{\SubstWeak{1}}}}{b},b^C)~\texttt{ in } \\
  & \Gmodel{\Gsub{\Gapp{\Gsub{\Gapp{\GRecproj{j}{\sub{\denotesCC{\lkg}}{\Glued{\gm}}}}}{\GSubstExt{\GSubstId}{\Glued{a}}}}}{\GSubstExt{\GSubstId}{\Glam{\Glued{\rho}}}}}(\epsilon)(\star) \\
  & \quad 
   \text{ and also } \goodTerm{\Gm}{\lkg}{\RecSig{\wsig}{"R"}} \quad ,\Glued{R} = \denotesCC{"R"}, \Glued{B} = \Gwsigproj{j}{2}{\Glued{\wsig}}
\end{align*}

The semantic interpretations above are 
mutually recursively defined with the following inductively defined indexed set $\CWmodel$ (with only one constructor $\CWsup$)
%
\begin{align*}
  "Inductive"~\CWmodel &: (\Glued{\wsig} \in \GWSig{i}{\denotesCC{\cdot}}{N}) \to \{ t : \goodTerm{\cdot}{t}{\El{\wcode{\wsig}}} \} \to \Set{i + 1} \quad "where" \\
    \CWsup &: j < N \to \Glued{a} \in \GTm{\Glued{\cdot}}{\wsigproj{j}{1}{\Glued{\wsig}}} \\
    & \to \Glued{b} \in (\GTm{(\Glued{\cdot} \Glued{,} \Gsub{\wsigproj{j}{2}{\Glued{\wsig}}}{\SubstExt{\GSubstId}{\Glued{a}}}) }{\Gsub{\GEl{\Gwcode{\Glued{\wsig}}}}{\GSubstWeak{1}}}) \to \CWmodel~\Glued{\wsig}~\Wsup{j}{\wsig}{a}{b}
\end{align*}
and its eliminator $\CWrec$
\begin{align*}
  \CWrec \  &: (\Glued{\wsig} \in \GWSig{i}{\denotesCC{\cdot}}{N}) \to 
    (P : \{ t : \goodTerm{\cdot}{t}{\El{\wcode{\wsig}}} \} \to \Set{k})  \\
    & \to [j < N \to  \Glued{a} \in \GTm{\Glued{\cdot}}{\Gwsigproj{j}{1}{\Glued{\wsig}}} \\ 
    & \quad \quad \to 
    \goodTerm{({\cdot} {,} \sub{\wsigproj{j}{2}{{\wsig}}}{\SubstExt{\SubstId}{{a}}}) }{b}{\sub{\El{\wcode{\wsig}}}{\SubstWeak{1}}} \\ 
    & \quad \quad \to b^C : (\Glued{\gm} \in (\Glued{\cdot} \Glued{,} \sub{\Gwsigproj{j}{2}{\Glued{\wsig}}}{\SubstExt{\SubstId}{\Glued{a}}}) \to P~\sub{b}{\Gmodel{\Glued{\gm}}}) \\
    & \quad \quad \to P~(\Wsup{j}{\wsig}{a}{b}) ] \quad \text{ (square bracket here just to make things readable)} \\
    & \to \goodTerm{\cdot}{t}{\El{\wcode{\wsig}}} \to \CWmodel~\Glued{\wsig}~t \to P~t \\
  \CWrec \  & \Glued{\wsig}~P~f~t~(\CWsup~\Glued{a}~\Glued{b}) = f~\Glued{a}~b~(\lambda \Glued{\gm}. \CWrec~P~f~(\sub{b}{\Gmodel{\Glued{\gm}}})~(b^C~\Glued{\gm}))
\end{align*}

Note that in $\CWmodel$, the $\Glued{b}$ uses the definition of
$\denotesC{\wcode{\wsig}}$, which after unfolding, recursively references $\CWmodel$
in a strictly positive position. %So it is a well-founded definition.
We do not distinguish $(b, b^C)$ and $\Glued{b}$ for simplicity. Again, we omit validating the equational rules ($\beta$, $\eta$, and substitution) here.

We state the fundamental property of the logical-relations model.
%in our proof relevant model, it is admitting that our model is really a model,
%and our syntax is the initial model so there is map from our syntax to our
%canonicity model:
\begin{theorem}[Fundamental Property]
  If\/ $\goodTerm{\Gm}{t}{T}$, then its semantic interpretation is a dependent function such that
  $\denotesC{t} : \prod_{\goodSub{\cdot}{\gm}{\Gm}} \prod_{\gm' \in \denotesC{\goodCtx{\Gm}{}}(\gm)}\denotesC{\goodType{\Gm}{T}{}}(\gm)(\gm')(\sub{t}{\gm})$.
\end{theorem}

The first consequence of this model is the consistency of \TT---we cannot derive
$\goodTerm{\cdot}{t}{\bot}$. Otherwise, we would have an element in the empty set,
$\denotesC{\goodTerm{\cdot}{t}{\bot}}(\SubstEmp)(\star) \in
\denotesC{\bot}(\SubstEmp)(\star)(\sub{t}{\gm}) = \emptyset$, a contradiction.

\begin{theorem}[Consistency]
  The typing judgment $\goodTerm{\cdot}{t}{\bot}$ is not derivable for any term~$t$.
\end{theorem}

Next, with the logical-relations model, we can map an arbitrary closed boolean
term $\goodTerm{\cdot}{t}{\cB}$ to get the result
$\denotesC{\goodTerm{\cdot}{t}{\cB}}(\SubstEmp)(\star) = \star^1$ or $\star^2$,
witnessing the proof of $t \equiv \true \text{ or } t \equiv\false$ by the definition of our model, arriving at
\cref{thm:canonicity-appendix}.


Further, with the help of eta rules, we have the following canonical forms.
\begin{theorem}[Canonical Forms]\hfill
  \begin{itemize}
    \item If $\goodTerm{\cdot}{t}{\El{\wcode{\tau}}}$ and $\goodWSig{\cdot}{\tau}{n}$, then $\goodTerm{\cdot}{t \equiv \Wsup{j}{\tau}{a}{b}}{\El{\wcode{\tau}}}$ for some $\goodTerm{\cdot}{a}{A}$, $\goodTerm{B[("id", a)]}{b}{\El{\wcode{\tau}}}$, and $j < n$
    \item If $\goodTerm{\cdot}{t}{\mathbb{B}}$ then $\goodTerm{\cdot}{t \equiv "tt"}{\mathbb{B}}$ or $\goodTerm{\cdot}{t \equiv "ff"}{\mathbb{B}}$ 
    \item If $\goodTerm{\cdot}{t}{\TyLkg{\lsig}}$ with $\goodSig{\cdot}{\sigma}{n}$, then $\goodTerm{\cdot}{t \equiv \LkgAdd{o}{t}}{\TyLkg{\lsig}}$ 
      for some $\goodTerm{\cdot}{o}{\TyLkg{\lkgproj{1}{\sigma}}}$ and $\goodTerm{\lsigproj{2}{\sigma}}{t}{\lkgproj{2}{\sigma}}$
    \item If $\goodTerm{\cdot}{t}{\TySigma A B}$ then $\goodTerm{\cdot}{t \equiv (a, b)}{\TySigma A B}$ with $\goodTerm{\cdot}{a}{A}$ and $\goodTerm{\cdot}{b}{B[("id", a)]}$
    \footnote{We emphasize the last one because $\Tkg{\ell}$ is a dependent pair}
  \end{itemize}
\end{theorem}

% the canonical form of inductive type, linkage and boolean 
