% mention expression problem and family polymorphism solving it
% proof engineering, not being focused, fampoly on proof engineering
% contribution

There is a growing trend among programming languages researchers
to use proof assistants to mechanize meta\-theories.
%
However, the programmer runs into an old problem
in the new setting of proof engineering:
the Expression Problem (EP) \cite{wadler-ep}.

The EP is a programming challenge that
epitomizes the difficulty of writing type-safe, extensible code.
To define an expression language that can be reused for future extensions,
the programmer faces a fundamental tension \cite{reynolds1975} between
adding new constructors of a data type (e.g., new abstract syntax) and
adding new functions over the data type (e.g., new compiler passes).

The EP is well studied in the conventional setting of functional
programming and object-oriented (OO) programming.
Modern languages, such as Scala \cite{scala-oopsla05}, have a good
supply of linguistic features that offer expressive power to solve the
EP.

In contrast, proof assistants offer few linguistic solutions that
address the EP.
Yet the challenge of writing extensible, type-safe code (and proofs!) is
as real, especially for metatheory mechanization.
As an example, consider \cref{fig:STLC-example}, the simply typed
lambda calculus (STLC) mechanized in Coq.
A programming challenge there is to define distinct extensions of this
STLC formalization to support distinct features,
and then selectively compose these extensions to form new STLC variants.

In Coq, inductive types, as well as functions and theorems over
inductive types, are closed to extension.
So to reuse mechanized metatheories,
the common practice is still to copy and modify them for each extension.
But having to maintain multiple copies is highly non-modular and
antithetical to good software engineering.
%
The programmer could also turn to design patterns~\cite{delaware2011,delaware2013}.
But they require heavy lifting from the programmer to make code
extensible, often leading to verbose, non-idiomatic programming styles.

A question thus arises: can linguistic solutions to the EP, as studied
in the conventional setting, be adapted to metatheory mechanization in
a proof assistant?

At the core of many linguistic solutions to the EP is \emph{inheritance}.
Inheritance is sometimes interpreted narrowly as a subclass'
inheriting methods and fields from a superclass.
But its language-theoretic essence is more general:
inheritance is a linguistic approach to incrementally
modifying \emph{recursive, record-like definitions}~\cite{cook1990inheritance}.

Language mechanisms including
mixins~\cite{mixin-1990},
virtual classes~\cite{virtualclasses-1989,vc-calculus-2006},
virtual types~\cite{thorup97} and associated types~\cite{ckj05},
extensible cases~\cite{bac2006},
and so on, are all based on the essential idea of inheritance.
%
In particular, when a language mechanism enables inheritance over
recursive definitions containing related types and terms,
it is said to support \emph{family polymorphism}~\cite{ernst2001family}:
types and terms are polymorphic to a family they are nested within.

\paragraph{Main ideas.}

We contribute a language design that integrates family polymorphism into
a proof assistant.
Because code and proofs are polymorphic to a family they are nested
within, they can be inherited and reused by a derived family.
%
Family polymorphism enables extensible metatheory mechanization.
As \cref{fig:?} shows,
different extensions to STLC can all inherit from the base STLC family:
they reuse mechanized metatheories from syntax to the type safety
theorem, only adding new constructors to inductive types
and new cases to the induction proofs as needed by an extension.

Integrating family polymorphism into a dependent type theory with
inductive types poses significant technical challenges, however.
As we analyze, inductive types, definitional equalities, and logical
consistency, are all inimical to the kind of family polymorphism
found in existing language designs.
Thus, our core contributions include design recipes for dealing with
these issues and strong metatheoretical guarantees on consistency and
canonicity.

\paragraph{Contributions} To our knowledge, we offer the first
language design that enables extensible metatheory mechanization
in a higher-order, dependent type theory with inductive types.

\begin{itemize}[leftmargin=3.5ex]

\item We show how to integrate family polymorphism into a proof
assistant (\cref{sec:lang-design}).
The language design reconciles the expressiveness enabled by
family polymorphism with the exactness of a proof assistant,
while retaining an idiomatic programming style.

\item We contribute a prototypical implementation of our language
mechanism as a Coq plugin (\cref{sec:coqimpl}). The plugin works by
compiling surface-language terms into Gallina terms parameterized by
extensibility hooks.

\item We capture the new language mechanism formally by extending
Martin–Löf type theory with family polymorphism and extensible inductive
types (\cref{sec:metatheory}).
We derive strong metatheoretical results including on consistency and
canonicity.
We also formalize the translation implemented by the Coq plugin.

\item We present case studies of using our Coq plugin to mechanize
language metatheories.
They show how our linguistic approach naturally solves the EP and
enables a high degree of reusability and extensibility
for proof engineering.

\end{itemize}

%\paragraph{Structure of the paper} We will quickly introduce Family
%Polymorphism and the challenges to adapt it into dependent type setting
%in \cref{sec:background+challenge}. After that, we will talk about the
%language design of family polymorphism in dependent type setting and the
%implementation of our Coq plugin in \cref{sec:coqimpl}. Then, we
%consider the meta-theory of incorporating family polymorphism into
%predicative MLTT, and deriving consistency and canonicity results in
%\cref{sec:metatheory}. \ref{sec:related-work} discusses related works
%and \ref{sec:conclusion} concludes.


