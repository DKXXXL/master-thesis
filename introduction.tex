% mention expression problem and family polymorphism solving it
% proof engineering, not being focused, fampoly on proof engineering
% contribution

There is a growing trend among semantics researchers to
use proof assistants to mechanize meta\-theories.
Consequently, it has become ever more important that proof assistants
make it easy to write reusable, extensible code and proofs.

However, the programmer runs into an old problem,
the Expression Problem (EP) \cite{wadler-ep},
that has deep roots in language design.
%
The EP is a programming challenge
epitomizing the difficulty of writing type-safe, extensible code.
To define an expression language that can be reused for future extension,
the programmer faces a fundamental tension \cite{reynolds1975} between
adding new constructors of a data type (e.g., new abstract syntax) and
adding new functions over the data type (e.g., new compiler passes).

The EP is well studied in the conventional setting of functional and
object-oriented (OO) languages.
Modern languages, such as Scala \cite{scala-oopsla05}, now have a good
supply of linguistic features that offer expressive power to solve the
EP.

In contrast, proof assistants offer few linguistic solutions to
programmers who want to mechanize their metatheories.
Yet the challenge of writing extensible, type-safe code (and proofs!) is
as real.
As an example, consider \cref{fig:STLC-example}, the simply typed
lambda calculus (STLC) mechanized in Coq.
A programming challenge is to define distinct extensions of this STLC
formalization to support distinct features,
and then selectively compose these extensions to form new STLC variants.

Because in Coq, code and proofs are closed to extension, the programmer
cannot easily reuse them.
The common practice is still to copy the existing mechanization and
modify it for each extension.
But having to maintain multiple copies is highly non-modular and
inimical to good software engineering.
%
The programmer could also turn to design patterns~\cite{delaware2011,delaware2013}.
But they require heavy lifting from the programmer to make code
extensible, often leading to complex, non-idiomatic programming styles.

A question naturally arises: can linguistic solutions to EP, as studied
in the conventional setting, be adapted to metatheory mechanization in
a proof assistant?

At the core of many linguistic solutions to EP is \emph{inheritance}.
Inheritance is sometimes interpreted narrowly as a subclass'
inheriting methods and fields from a superclass.
But its language-theoretic essence is more general:
inheritance is a linguistic approach to incrementally
modifying \emph{recursive, record-like structures}~\cite{cook1990inheritance}.

Language mechanisms including
mixins~\cite{mixin-1990},
virtual classes~\cite{virtualclasses-1989,vc-calculus-2006},
virtual types~\cite{thorup97} and associated types~\cite{ckj05},
extensible cases~\cite{bac2006},
and so on, are all based on the idea of inheritance.

In particular, when a language mechanism applies inheritance to
recursive, record-like definitions containing related types and terms,
it is said to support \emph{family polymorphism}~\cite{ernst2001family}:
code is polymorphic to a family it is nested within.

\paragraph{Main ideas.}

We contribute a language design that integrates family polymorphism into
a proof assistant.
Because code and proofs are polymorphic to a family it is nested
within, they can be inherited and reused by a derived family.
Extensible metatheory mechanization is thus made possible.

Integrating family polymorphism into a dependent type theory poses
significant technical challenges, however.
\TODO

\paragraph{Structure of the paper} 
We will quickly introduce Family Polymorphism and the challenges to adapt it into dependent type setting in \cref{sec:background+challenge}. After that, we will talk about the language design of family polymorphism in dependent type setting and the implementation of our Coq plugin in \cref{sec:coqimpl}. Then, we consider the meta-theory of incorporating family polymorphism into predicative MLTT, and deriving consistency and canonicity results in \cref{sec:metatheory}. \ref{sec:related-work} discusses related works and \ref{sec:conclusion} concludes.


\paragraph{Contributions} include the following
\begin{itemize}
  \item To our knowledge, this is the first attempt incorporating family polymorphism with theorem proving and higher order logic
  \item We contribute a prototypical Coq plugin for the interested users and developers 
  \item We provide some first-step meta-theoretical results, including consistency and canonicity
\end{itemize}
