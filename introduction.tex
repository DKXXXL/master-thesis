% mention expression problem and family polymorphism solving it
% proof engineering, not being focused, fampoly on proof engineering
% contribution
It is hard to write modular, extensible code.
It is even harder to write modular, extensible proofs.
While there exist many linguistic solutions to code reuse and
extensibility in object-oriented and functional languages,
linguistic solutions are lacking in proof assistants.

Meanwhile, language designers and researchers are increasingly
using proof assistants to mechanize meta\-theories, building on each
other's work.
Therefore, it has become ever more important that proof assistants
make it easy to write reusable, extensible code and proofs.



\large{\textbf{Structure of the paper}} We will quickly introduce Family Polymorphism and the challenges to adapt it into dependent type setting in \cref{sec:background+challenge}. After that, we will talk about the language design of family polymorphism in dependent type setting and the implementation of our Coq plugin in \cref{sec:coqimpl}. Then, we consider the meta-theory of incorporating family polymorphism into predicative MLTT, and deriving consistency and canonicity results in \cref{sec:metatheory}. \ref{sec:related-work} discusses related works and \ref{sec:conclusion} concludes.


\large{\textbf{Contributions}} include the following
\begin{itemize}
  \item To our knowledge, this is the first attempt incorporating family polymorphism with theorem proving and higher order logic
  \item We contribute a prototypical Coq plugin for the interested users and developers 
  \item We provide some first-step meta-theoretical results, including consistency and canonicity
\end{itemize}
