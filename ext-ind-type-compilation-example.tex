\begin{figure}
\lstset{
    basicstyle=\fontsize{8}{8.5}\ttfamily,
% numbers=left,
}

\begin{minipage}{\textwidth}
\begin{multicols}{2}



\definecolor{codecomment-color}{HTML}{0DA3FF}

\begin{lstlisting}
(* Type Checking of Inductive Type *) [@\tanc{Tyck-IndTy-start}@]
(* Tyck happens during interactively creating families*)
Module Type STLC_Ctx1. 
(* The Context including Inductive Type *)
  Parameter (tm : Set).
  Parameter (tm_var : id → tm).
  Parameter (tm_abs : id → tm → tm).
  (* [@\dadada@] and all the fields for inductive type [@\dadada@] *)
End STLC_Ctx1.
Module Type tm_prec_64 (self: STLC_Ctx1).[@\YZ{tm_6? Do you mean STLC_Ctx1 instead?}\EDJreply{Yes Thanks.}@]
Parameter (* Type for partial recursor *)
(tm_prec : ∀ t, ∀ P : self.tm -> Type,
  (∀ n, option (P (self.tm_var n))) -> 
  [@\dadada@], option (P t)).
End tm_prec_64.
Module Type STLC_Ctx2.  
  Include STLC_Ctx1.
  Include tm_prec_64. (* New Ctx include prec *)
End STLC_Ctx2.
(* [@\dadada@] etc, everytime a new field is added, a new 
module type is created as the ctx with the new field *)
(* Collect eliminator TYPE for later type-check *)
Module tm_rec_type (self: STLC_Ctx1).
  Definition _TYPE_tm_rec :=
    ∀ P : self.tm -> Set,
    ( ∀ n, P (self.tm_var n)) ->
    ( ∀ n i, P i -> P (self.tm_abs n i)) ->
    [@\dadada@] -> ∀ i : self.tm, P i.
End tm_rec_type.  [@\tanc{Tyck-IndTy-end}@]
\end{lstlisting}

\makeline[0pt]{Tyck-IndTy-start}{Tyck-IndTy-end}[codecomment-color!50]


\definecolor{codecomment-color}{HTML}{0DA30F}

\begin{lstlisting}
(* Type checking of Recursion *) [@\tanc{Tyck-RecTy-start}@]
(* Typecheck each handler, here STLC.subst.tm_var *)
(* Also an example of fields in nested family [@\dadada@]*) 
Module tm_var_handle (STLC: STLC_Ctx1)[@\YZ{(1) Why doesn't the module name contain "subst"? (2) Should it be STLC_Ctx2 instead of STLC_Ctx1?}\EDJreply{I rename to ctx so that I think it is easier to understand. Feel Free to change to the way you want. I think name them all into "_Ctx" suffix is easier to emphasize the fact they are the result of compilation of the context. (2) The fact is, no, it also will need to include tm_prec's computational axiom and ty, which I also forgot to mention. Also ty has its prec and computational axiom. So at least 6. Do you want me to write doesn STLC_Ctx(1,2,3,4,5,6)? It is unnecessary complication and the space is not enough. I proposed to move this part of information into the figure 5 toy example in the slack, what do you think? }@]
    (self: EmptySig). (* [@\dadada@] thus with two contexts *)[@\YZ{(3) Are nested families really needed?}\EDJreply{Not really but if consider handlers as one family, it is nested family. What's more, the audience needs to know how nested family is compiled at some point.}@]
  Definition tm_var s x t := 
    if eqb x s then t else STLC.tm_var s.
End tm_var_handle. (* [@\dadada@] other handlers are alike *)
(* Assembling tycked handlers in STLC.subst *)
Module subst_internal_410 (M: STLC_Ctx1). 
 Module subst_internal.
 Include tm_var_handle M. Include tm_abs_handle M.
 Include tm_app_handle M. Include tm_unit_handle M.
 End subst_internal.
End subst_internal_410.
(* Intermediate Module solely for exhausitivity checking *) 
Module do_tc (self: STLC_Ctx[@\dadada@]).
  Include tm_rec_type self. 
  Include subst_internal_410 self.
  Parameter (for_tyck : _TYPE_tm_rec).
  Definition term_for_type_checking :=
    for_tyck  ( λ _  ⇒ (id → self.tm → self.tm))
      subst_internal.tm_var [@\dadada@] .
End do_tc.[@\tanc{Tyck-RecTy-end}@]
\end{lstlisting}

\makeline[0pt]{Tyck-RecTy-start}{Tyck-RecTy-end}[codecomment-color!50]

\columnbreak



\begin{lstlisting}
(* Exposing Computational Axiom into Ctx *) [@\tanc{Tyck-Recty-cont-start}@]
Module Type substtm_139 (self: STLC_Ctx[@\dadada@]).[@\YZ{What is "..."?}\EDJreply{My intension is STLC_Ctx_n for some n where at this point the audience should able to see what the context of that STLC_Ctx is. Apparently I failed.}@]
Parameter
  (subst_on_tm_unit :
	 (self.subst (self.tm_unit)) =[@\YZ{Where is this "subst" added to the context?}\EDJreply{The reason is above. Currently the space is already not enough.}@]
      self.subst_internal.tm_unit).
(* [@\dadada@] subst equality on other constructors *)
End substtm_139. [@\tanc{Tyck-Recty-cont-end}@]
\end{lstlisting}



\makeline[.5\textwidth+9pt]{Tyck-Recty-cont-start}{Tyck-Recty-cont-end}[codecomment-color!50]


\definecolor{codecomment-color}{HTML}{5D030F}

\begin{lstlisting}
(* Compilation, happens when closing a top-level family *) [@\tanc{Cmp-Result-start}@]
(* Ctx that expose concrete inductive type *)
Module Type STLC_CCtx.
Inductive __internal_tm : Set := [@\dadada@] 
Include subst_internal_410. (* [@\dadada@] and more inclusion *)
End STLC_CCtx.
(* Both the compilation of recursor and[@\dadada@] *)
Module subst_128 (self: STLC_CCtx).
Include __motive_of_subst_107 self.
Definition subst :=
  self.__internal_tm_rec __motiveTsubst
	self.subst_internal.tm_var [@\dadada@] .
End subst_128.
(* [@\dadada@]the compilation of partial recursor 
require concrete inductive type exposed in the ctx *)
Module tm_prec_72 (self: STLC_CCtx[@\dadada@]).
Definition tm_prec :
  ∀ t, ∀ P : self.tm -> Type,
  (∀ n, option (P (self.tm_var n))) ->
  [@\dadada@], option (P t) := [@\dadada@]
End tm_prec_72. 
Module STLC. (* Assembling/Compiling the whole STLC *)
Include tm_5. Include tm_prec_72. 
Include tm_prectm_81. (*Justify comp axiom, mostly eq_refl*)
Include subst_internal_410. Include subst_128.
End v_129.
(* Assembling STLC_bool.subst_internal via reusing 
    most of the former compilation results *)
(* Notice how handlers don't require explicitly
   exposing the inductive type *)
Module subst_internal_248 (self: STLC_Ctx1).
(* The New Handlers *)
Include tm_true_197 self. Include [@\dadada@]
(* The Reused Compilation result *)
Include tm_var_handle self. Include [@\dadada@]
End v_248. 
(* Regenerated FRecursion Compilation *)
Module subst_242 (self: [@\dadada@]).
Include __motive_of_subst_107 self.
Definition subst :=
  self.__internal_tm_rec __motiveTsubst
	self.subst_internal.tm_var [@\dadada@]
    self.subst_internal.tm_true [@\dadada@] .
End subst_242. [@\tanc{Cmp-Result-end}@]
\end{lstlisting}



\makeline[.5\textwidth+9pt]{Cmp-Result-start}{Cmp-Result-end}[codecomment-color!50]

\end{multicols}
\end{minipage}

\caption{Exemplar Compilation about Inductive Type, for STLC example}\label{fig:plugin-example2}
\end{figure}