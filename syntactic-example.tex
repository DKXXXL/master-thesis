We use an example to better demonstrate different components of the
syntax of our formulation. Our example will be the meta-theory encoding
of the following pseudocode.

We provide one example in its surface syntax as pseudocode, and their
meta-theory encoding. This example is computing the predecessor of a
mundane natural number, but we split the inductive definition into two
parts and making "O" and "S" standing alone.
"pred" will map $S~n \mapsto (n, S~n)$ and $O \mapsto (O, O)$, and thus a predecessor.\YZ{
  Using family polymorphism for adding  a successor constructor is unnatural.
  I suggest simply renaming the constructors and not mentioning the correspondence to natural numbers.
}\EDJreply{I don't have a good idea for the renaming. What about you?}

\begin{figure}[!htb]\label{fig:example-pseudocode}
  \lstset{
      basicstyle=\fontsize{8}{8.5}\ttfamily,
  % numbers=left,
  }
  
  \begin{minipage}{\textwidth}
  \begin{multicols}{2}
  
  
  
  \definecolor{codecomment-color}{HTML}{0DA3FF}
  
  \begin{lstlisting} 
  Family A. [@\tanc{Coq-Example1-start}@]
  FInductive N := O : N. 
  (* [@ \tiny{Osig, sig₁, obj₁, sig₂,
     O₁, obj₂, sig₂', sl₁} @] *)

  FRecursion pred on N
    motive (fun _ => N × N).
  Case O := (O, O). 
  End pred. (* [@ \tiny{pN₁, handler₁, sig₃} @] *)
  (* pred : self[A].N → self[A].N × self[A].N *)
   (* [@ \tiny{predT, pred₁, sig₄, sig₄', sl₄} @] *) 
  k := pred O 
  (* [@ \tiny{pN₂, sig₅} @] *)  
  End A.[@\tanc{Coq-Example1-end}@]
  \end{lstlisting}
  
  \makeline[0pt]{Coq-Example1-start}{Coq-Example1-end}[codecomment-color!50]
  
  
  \columnbreak
  
  
  
  \definecolor{codecomment-color}{HTML}{5D030F}
  
  \begin{lstlisting}
  Family [@ \tiny{"A"$_2$} @] extends A. [@\tanc{Coq-Example2-start}@]
  FInductive N += S : N → N.
  (* [@ \tiny{Nsig, sig₁₁, inh₁, O₂, sig₂₁, inh₂, 
      sig₂₂, sig₂₁', sig₂₂', sl₂₂, inh₃} @] *)
  FRecursion pred.
  Inherit O. (* [@ \tiny{inhᵢₙ₀, pN₃} @] *)
  Case S := fun (x,y) => (y, S y).
  EndFamily. (* [@ \tiny{inhᵢₙ, sig₃₁, inh₄} @] *)
  Inherit pred.
    (* [@ \tiny{sig₄₁, inh₅, sl₄₁} @] *)
  Inherits k. 
  (* inh₆ *)
  End [@\tiny{"A"$_2$}@][@\tanc{Coq-Example2-end}@]
  \end{lstlisting}
  
  
  
  \makeline[.5\textwidth+9pt]{Coq-Example2-start}{Coq-Example2-end}[codecomment-color!50]
  

  \end{multicols}
  \end{minipage}
  
  \caption{Exemplar Compilation about Inductive Type, for STLC example}\label{fig:plugin-example2}
  \end{figure}

\YZreply{Also, this figure should contain an example of accessing a field outside its family.}
\begin{figure}
  \begin{minipage}{\linewidth}
    \begin{minted}{Coq}
      Inductive N :=
       | O : 1 -> (0 -> N) -> N 
       | S : 1 -> (1 -> N) -> N
      (* For example, Number 1 is encoded using
         1 = S () (fun _ -> O () (fun x -> elim-⊥ x)) *)
    \end{minted}
  \end{minipage}

  \begin{minipage}[t]{0.4\linewidth}
  \small
\begin{align*}
  "Osig" &\coloneqq \goodWSig{\cdot}{w^+\ w\cdot\ \top\ \bot}{1} \\
  "sig"_1 &\coloneqq \goodSig{\cdot}{\nu^+ \ \nu\cdot \ {id_s} \ (\bW \ "Osig"[\pi_1])}{1}  \\
  "obj"_1 &\coloneqq \goodTerm{\cdot}{\mu^+ \ \mu\cdot \ {id_s} \ \star}{\cL"sig"_1}\\
  \cC"sig"_1 &\coloneqq \goodType{\cdot}{\Sigma \ \top \ (\bW \ "Osig"[\pi_1]) }{} \\
  "sig"_2 &\coloneqq  \goodSig{\cdot}{\nu^+ \ "sig"_1 \ {id_s} \ El ("W" ("pjr" \ \pi_2))}{2} \\ 
  "O"_1 &\coloneqq {"Wsup" \ ("pjr" \pi_2) \ () \ ("elim-"\bot \ \pi_2)}  \\ 
  & \text{ and thus }  \goodTerm{\cdot, \cC"sig"_1}{"O"_1}{El ("W" ("pjr" \pi_2))} \\
  "obj"_2 &\coloneqq \goodTerm{\cdot}{\mu^+ \ "obj"_1 \ "O"_1}{\cL"sig"_2} \\
  "sig"_2' &\coloneqq \goodType{\cdot}{\Sigma (\Sigma \top \cU)El ("pjr" \pi_2)}{} \\ 
  \cC"sig"_2 &\coloneqq \goodType{\cdot}{\Sigma (\Sigma \top (\bW \ "Osig"[\pi_1]))El ("W"("pjr" \pi_2)) }{} \\ 
  "sl"_1 &\coloneqq ((\star, "W" ("pjr" ("pjl" \pi_2)) ), "pjr" \pi_2) \\ 
  & \text{ and thus }  \goodSeal{\cdot}{"sl"_1}{"sig"_2}{"sig"_2'} \\
  "pN"_1 &\coloneqq \goodType{\cdot, "sig"_2'}{El ("pjr" ("pjl" \pi_2)) \times El ("pjr" ("pjl" \pi_2))}{} \\
  "handler"_1 &\coloneqq \goodTerm{\cdot, "sig"_2'}{\_}{\cL(\nu^+ \ \nu\cdot \ \{id_s\} "pN"_1)} \\
  "sig"_3 &\coloneqq \goodSig{\cdot}{\nu^+ "sig"_2 \{"sl"_1\} \ \cL(\nu^+ \ \nu\cdot \ \{id_s\} "pN"_1)}{3} \\ 
  "obj"_3 &\coloneqq \goodTerm{\cdot}{\mu^+~"obj"_2~"handler"_1}{\cL"sig"_3}\\
  "predT" &\coloneqq \Pi \ El ("W"("pjr" ("pjl"^2 \pi_2)))\ El ("W" ("pjr" ("pjl"^2  \pi_2)))[\pi_1] \\
  "pred"_1 &\coloneqq \goodTerm{\cdot, \cC "sig"_3}{\lambda ("Wrec" \_)}{"predT"} \\ 
  "sig"_4 &\coloneqq \nu^+ \ "sig"_3 \ \{id_s\} \ "predT" \\ 
  "obj"_4 &\coloneqq \goodTerm{\cdot}{\mu^+~"obj"_3~"pred"_1}{\cL"sig"_4}\\
  "sl"_4 &\coloneqq \goodSeal{\cdot}{\_}{"sig"_4}{"sig"_4'} \\
  "pN"_2 &\coloneqq \goodType{\cdot, "sig"_4'}{El ("pjr" ("pjl" \pi_2)) \times El ("pjr" ("pjl" \pi_2))}{}
\end{align*}
\end{minipage}%
\begin{minipage}[t]{0.4\linewidth}
  \small
\begin{align*}
  "sig"_5 &\coloneqq \nu^+ \ "sig"_4 \ \{"sl"_4\} \  "pN"_2 \\ 
  "obj"_5 &: \goodTerm{\cdot}{\mu^+~"obj"_4~\_}{\cL"sig"_4}\\
  \\ 
  \text{Here we}&\text{ start to construct Family } "A"_2 \\ 
  "Nsig" &\coloneqq \goodWSig{\cdot}{w^+\ "Osig"\ \top \ \top}{2} \\ 
  "sig"_{11} &\coloneqq \goodSig{\cdot}{\nu^+ \ \nu\cdot  \ (\bW \ "Nsig"[\pi_1])}{1} \\
  "inh"_1 &\coloneqq \goodInh{\cdot}{"inhov" \  "inhid" \ \star}{"sig"_1}{"sig"_{11}}\\
  "O"_2 &\coloneqq {"Wsup" \ ("pjr" \pi_2) \ () \ ("elim-"\bot \ \pi_2)}  \\ 
  & \text{ but }  \goodTerm{\cdot, \cC"sig"_{11}}{"O"_2}{El ("W" ("pjr" \pi_2))} \\
  "sig"_{21} &\coloneqq \nu^+ \ "sig"_{11} \ \{id_s\} (El (W (pjr p₂))) \\ 
  "inh"_2 &\coloneqq \goodInh{\cdot}{"inhov" \ "inh"_1 \ "O"_2}{"sig"_2}{"sig"_{21}}\\
  "S"_T &\coloneqq \Pi \ El ("W"("pjr"  ("pjl" \pi_2))) \ El ("W" ("pjr" ("pjl"  \pi_2)))[\pi_1] \\
  "sig"_{22} &\coloneqq \nu^+ \ "sig"_{21} \ \{id_s\} \ "S"_T \\ 
  "sig"_{21}' &\coloneqq \goodType{\cdot}{\Sigma (\Sigma \top \cU) (El ("pjr" \pi_2))}{} \\
  "sig"_{22}' &\coloneqq \Sigma "sig"_{21}' (\Pi \ El ("pjr"  ("pjl" \pi_2)) \  ...) \\
  "sl"_{22} &\coloneqq \goodSeal{\cdot}{\_}{"sig"_{22}}{"sig"_{22}'} \\ 
  "inh"_3 &\coloneqq \goodInh{\cdot}{\_}{"sig"_2}{"sig"_{22}} \\ 
  \uparrow^s &\coloneqq \goodTerm{\cdot, "sig"_{22}'}{\_}{"sig"_2'[\pi_1]} \\ 
  "pN"_3 &\coloneqq "pN"_1[(\pi_1, \uparrow^s)] \\ 
\end{align*}
  \end{minipage}

  \begin{minipage}{0.8\linewidth}
    \small
    \centering
    \begin{align*}
      "inh"_{in0} &\coloneqq \goodInh{\cdot, "sig"_{22}'}{"inhid"}{(\nu^+ \ \nu\cdot \ ("pN"_1[\pi_1]))[(\pi_1, \uparrow^s)]}{(\nu^+ \cdot ("pN"_3[\pi_1]))} \\
      "inh"_{in} &\coloneqq \goodInh{\cdot , "sig"_{22}'}{\_}{(\nu^+ \ \nu\cdot \ "pN"_1[\pi_1])[(\pi_1, \uparrow^s)]}{(\nu^+ \ \cdot \ ("pN"_3[\pi_1]) \ \{id_s\} (\Pi "pN"_3 ("pN"_3[\pi_1])) )[\pi_1]}\\
      "sig"_{31} &\coloneqq  \nu^+ \ "sig"_{22} \ {"sl"_{22}}\  \cL(\nu^+ \cdot ("pN"_3[\pi_1])  (\Pi "pN"_3 ("pN"_3[\pi_1]))[\pi_1]) \\ 
      "inh"_4 &\coloneqq \goodInh{\cdot}{"inhnest" \ "inh"_3 \uparrow^s "inh"_{in}}{"sig"_3}{"sig"_{31}} \\
      "sig"_{41} &\coloneqq {\nu^+ \ "sig"_{31} \ (\Pi (El ("W" ("pjr" ("pjl"^4 \pi_2)))) (El ("W" ("pjr" ("pjl"^4 \pi_2))))[\pi_1])} \\ 
      "inh"_5 &\coloneqq \goodInh{\cdot}{inhov \ \_}{"sig"_4}{"sig"_{41}} \\ 
      "sl"_{41} &\coloneqq \goodSeal{\cdot}{\_}{"sig"_{41}}{"sig"_{41}'} \\ 
      \uparrow^s_2 &\coloneqq \goodTerm{\cdot, "sig"_{41}}{\_}{"sig"_4'[\pi_1]} \\ 
      "inh"_6 &\coloneqq \goodInh{\cdot}{"inhinh" \ "inh"_5 \ \uparrow^s_2}{"sig"_5}{(\nu^+ \ "sig"_{41} \{"sl"_{41}\} \ "pN"_2[(\pi_1, \uparrow^s_2)])}
    \end{align*}  
  \end{minipage}
  \caption{Detailed Construction}\label{fig:example-construction}
\end{figure}





Here we explain how  our detailed construction in
\cref{fig:example-construction}. We first elaborate that
how our inductive type "N" is written in the specific format (the style
of W-type). We use 1, 0 here to indicate unit type and bottom type. Recall from \ruleref{Sigma Elim}, we have "pjl" and "pjr" as two projection from the sigma type (dependent pair). We will also use $"pjl"^n$ as a consecutive composition of "pjl", (e.g. $"pjl"^2~t$ is exactly $("pjl" ("pjl"~t))$ ).

Next, we alias each part of the derivation with a name. Every derivation
here is well-typed term. We also annotate in the original pseudocode the
name of the related derivation. 

We start with the construction of "Osig" which is the signature of a
standalone "O" constructor. With this we will have $"sig"_1, "obj"_1,
"sig"_2, "obj"_2$ as the signature and object of the linkage with that
inductive type, and the signature and the linkage once we export the
concrete constructor $"O"_1$. 

Now sealing comes in to seal these concrete component into an abstract
interface $"sig"_2'$ for decoupling the following fields. To make sure
the sealing exists, we take a look at the resulting $\cC "sig"_2$, and
doing abstraction on it to get $"sl"_1$. 

Now we construct the recursor, with the help of the handler module
$"handler"_1$, with which we can construct the real recursor $"pred"_1$.
Both detailed constructions are omitted. $"obj"_3$ and $"sig"_3$ is the resulting linkage and linkage signature aggregating the nested family $"handler"_1$. $"pred"_1$ is then the constructed recursor over $"sig"_3$ , without any abstraction, for the sake of exhaustiveness checking. We show the example $"sig"_4$ as the result of adding this new recursor. 

Then we again abstract the concrete recursor $"pred"_1$ away using $"sl"_4$ so that the field $"k"$ (of type $"pN"_2$)
can apply the recursor without resorting to any concrete recursor, thus achieving late-binding. Finally we
have $"sig"_5$ as final signature. 

Note that, since in our example there is only one type, the complicated debruijn indices $El("pjr" ("pjl" \pi_2)) $ in $"pN"_2$ and  $El("W"("pjr"("pjl"^2 \pi_2)))$ in "predT" are both referring to the type "N". The distinguishing difference is we have "W" (\ruleref{Ind-Type}) for the latter as "predT" is constructed in the context knowing the concrete definition of the inductive type, thus generally not inheritable. On the contrary,  $"pN"_2$ has as its context only $"N" : \cU$ and manifests its late-binding and its overridability for its former fields. \YZ{
  Maybe it is a good place to point out what El(pjr (pjl pi_2)) in pN_2 stands for
  and what El(W(pjr(pjl^2 pi_2))) in predT stands for.
}\EDJreply{Done.}\YZreply{
  I was actually alluding to how the type of pN_2 manifests late binding
  while the type of predT refers to the concrete signature of the
  inductive type.
}\YZreply{
  It’s good to be consistent in notations.  You later use * for products definition of pN_1, but you also use * as the term of type ($\bW$ $\sigma$) 
}\EDJreply{Notation replaced. First * replaced with $\star$, second * replaced with $\times$}
\EDJreply{Sry. Now I attempt another try. I also change the above paragraph a bit to explain the details on sig₃, sig₄, and etc.
I also copy the comments in our slack to here to remind me about the changes.
Also let me paste the full example here for future reference: \\
% following is the body of pred₁
pred_ = Wrec ...  
  : Tm (⋅ ▹ 𝓒sig₃ ▹ (El (W (pjr (pjl (pjl π₂)))))) (El (W (pjr (pjl (pjl π₂)))))[π₁] \\
% we do lambda abstraction now
pred₁ = λ pred_ :   Tm (⋅ ▹ 𝓒sig₃) (∏ El (W (pjr (pjl (pjl π₂)))) (El (W (pjr (pjl (pjl p₂))))[π₁])) \\
% we can see that, the type of pred₁ is predT 
% now we construct sig4, this should not have any abstraction inside
sig₄ = ∨+ sig₃ id_s (∏ El (W (pjr (pjl (pjl π₂)))) (El (W (pjr (pjl (pjl π₂))))[π₁])) \\ 
% we omit the construction of obj₄
obj₄ : Tm ⋅ 𝓛sig₄ \\
% since pred is define in context 𝓒sig₃, we need to abstract 𝓒sig₄ into sig₄'
% again for later field,
sig₄' = 
  Σ (Σ ((Σ (Σ ⊤ 𝕌) (El (pjr π₂))) 𝓛(∨+ ⋅ (pN[π₁]))[sf sl₁])) 
    (∏ El (pjr (pjl (pjl π₂))) (El (pjr (pjl (pjl π₂)))[π₁]))
% and we have, omitting the detail construction 
sl₄ : Seal ⋅ sig₄ sig₄' \\
% pN₂ now is constructed in the context of the sealed context sig₄'
pN₂ = El (pjr (pjl π₂)) * El (pjr (pjl π₂)) : Ty (⋅ ▹ sig₄') 
% the product type \\ 
sig₅ = ∨+ sig₄ {sl₄} pN₂ \\ % the full example is in my gist last time I sent to you (Attempt 9), if you like it I can send the link again but note that the omission is still there and a lot of naming convention is different.
It seems that as you last night complained, I forget to add sig₄' causing problems.  Do I also need to make a note that those primed variables are always the result of abstraction?
}

Now we construct the inheritance judgement for $"A"_2$. Still, we start
with signature for inductive type, "Nsig". But to ``extend'' the
inductive type, we actually \textit{override} the inductive type with
the enriched inductive type in $"inh"_1$. Because inductive type is
overridden, inheritance on the constructors are not pausible and thus we
need to override the old $"O"_1$ constructor using $"O"_2$ and extend
with new constructor for "S", resulting $"inh"_3$. These newly
overridden constructors are sealed by $"sl"_{22}$. However, look
closely, our sealed abstract interface in the children still has more
fields than the sealed abstract interface in the parent. We still need
$\uparrow^s$ to indicate the ``compatibility'' between two abstract
interfaces so that the inheritance from the parents can work. It can be
understood as ``upcasting'', but this ``upcasting'' is used to upcast
the context of the children. We immediately see its usage in $"pN"_3$
and $"inh"_{in0}$. Upon this, we construct the complete
inner-inheritance $"inh"_{in}$ that is responsible for extending
"handler". Then we continue the construction for $"inh"_4$ that ``nest''
the inner-inheritance into the whole $"inh"_3$, and we re-do
exhaustiveness checking when constructing $"inh"_5$ using $"inhov"$ and
"Wrec". Since it is another recursor requiring concrete inductive
definition, we do another abstraction on it using $"sl"_41$. After that,
$\uparrow^s_2$ again witnesses the compatibility of the abstract
interface, and the two are both used for constructing $"inh"_6$, which
is responsible for inheriting field "k".

The resulting constructed families are
$\goodTerm{\cdot}{"obj"_5}{\cL"sig"_5}$ as "Family A" and \\
$\goodTerm{\cdot}{("inh"~"inh"_6~"obj"_5)}{\cL((\nu^+ \ "sig"_{41}
\{"sl"_{41}\} \ "pN"_2[(\pi_1, \uparrow^s_2)]))}$ as "Family" $"A"_2$.

Finally we comment on how a family is encoded in our metatheory.
A family object will have a signature that exposes the type
of fields and the details of inductive types. This exposure can make
sure inheritance can override and inherit fields correctly. Sealing is
only hiding information \textbf{inside one family}, i.e., hiding
information of the former fields for the later fields, and thus the
later fields can be defined relying on only the abstract interface of
the former fields.
The signature will expose the sealing itself and the inductive type.
(See $"sig"_5$ has all the sealing information and inductive
definition). A former sealing might not be related to any later sealing,
for example, $"sig"_3$ is constructed using $"sl"_1$ but $"sig"_4$ is
using trivial sealing $id_s$.